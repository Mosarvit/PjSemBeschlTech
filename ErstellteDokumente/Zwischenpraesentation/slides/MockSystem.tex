\begin{frame}
\frametitle{Das Mock-System}

\onslide<2->{
\begin{itemize}
	\onslide<1->{\item Wird genutzt, wenn mit Geräten kommuniziert wird:	
	\begin{itemize}
		\item \texttt{mock\_system.write\_to\_AWG}
		\item \texttt{mock\_system.read\_from\_DSO}
	\end{itemize}		
 	}
	\onslide<3->{\item Simuliert das Verhalten des Messaufbaus nach dem Hammerstein Model}
\end{itemize}
	
}


\onslide<4->{Vorteile:
\onslide<5->{
\begin{itemize}
	\onslide<5->{\item Ermöglicht:	
	\begin{itemize}
			\onslide<5->{\item Unit Tests von Bausteinen, in den Gerätekommunikation stattfindet }
			\onslide<6->{\item System Tests }	
			\onslide<7->{\item Testen von Randfällen}
	\end{itemize}		
 	}
	\onslide<8->{\item Hilft das System besser zu verstehen }
\end{itemize}
}
} 

\onslide<9->{Nachteile:
\onslide<9->{
\begin{itemize}
	\onslide<9->{\item Extra Aufwand: mehr Code zu debuggen}
\end{itemize}
}
}
 
 

\end{frame}
