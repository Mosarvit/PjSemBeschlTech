\begin{frame}{Ausblick}



%	\begin{itemize}
%	\uncover<2->{
%		\item Einbindung des RF-Data-Tools zur Beurteilung der Qualität des Ausgangssignals
%		}
%		\uncover<3->{
%		\item \textit{Zusatz:} Konvertierung der Funktionalitäten von Python in die TEMF RF-Data-Tools
%		}
%	\end{itemize}
%\end{frame}
%
%\begin{frame}{Ausblick}
%	
%	\begin{itemize}
%		\item Optimierung der linearen Übertragungsfunktion: %mittels Auswertung der erwarteten und gemessenen Ausgangssignale $U_{out}$: 
%		\begin{align*}
%			H^{i+1} \left( \omega \right) = H^{i} \left( \omega \right) \left( 1+ \sigma_H \cdot \left( \frac{U}_{out,\mathrm{mess}}^{i} \left( \omega \right) }{U_{out,\mathrm{ideal}}^{i} \left( \omega \right) } -1 \right) \right) 
%		\end{align*}
%		mit $ \sigma_H $ als Schrittweite. %der jeweiligen Iteration
%		% würde Notation als Vektoreintrag vorschlagen H_{...} [i] statt Darstellung als Funktion
%		
%	\end{itemize}
%
%	
%	\begin{picture}(100,70)
%		\put(15,-10){
%			\includegraphics[scale=1.0]{slides/ResultCode/Slide1.eps} 
%		}
%	\end{picture}
%\end{frame}
%	
%%%%%%%% weitere Kommentare: %%%%%%%%%%%%%%
%%Frequenzverhalten der Kennlinie? Funktioniert Optimierung wie angesprochen überhaupt sinnvoll?
%% mit welcher Iteration? Zuerst H optimieren, dann neue Werte generieren (messen) und dann K? Oder beides nacheinander und dann erst neu messen?
%\begin{frame}{Ausblick}
%	
%	\begin{itemize}
%		\item Optimierung der nichtlinearen Kennlinie: %mittels Vergleich der Differenz der erwarteten und gemessenen Spannungssignale $U_{quest}$ und der Faktoren $a$ der polynomialen Kennlinie:
%		\begin{align*}
%			\Delta U_{?} = U_{?, \mathrm{mess}} - U_{?, \mathrm{berechnet}} = \sum_n \tilde{a}_n U_{in}^n
%			&&
%			a_n^{\mathrm{neu}} = a_n^{\mathrm{alt}} + \sigma_a \cdot \tilde{a}_n
%		\end{align*}
%		mit $\sigma_a$ als Schrittweite. %der jeweiligen Iteration
%	\end{itemize}
%	
%	\begin{picture}(100,70)
%		\put(15,-10){
%			\includegraphics[scale=1.0]{slides/ResultCode/Slide1.eps} 
%		}
%	\end{picture}



\end{frame}





