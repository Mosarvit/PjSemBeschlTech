\begin{frame}{Ausblick}

	\begin{itemize}
		\item Iterative Optimierung der linearen Übertragungsfunktion mittels Auswertung der erwarteten und gemessenen Ausgangssignale $U_{out}$: 
		\begin{align*}
			\underline{H}^{\mathrm{neu}} \left( \omega \right) = \underline{H}^{\mathrm{alt}} \left( \omega \right) \cdot \frac{\underline{U}_{out,\mathrm{ideal}} \left( \omega \right) 	}{\underline{U}_{out,\mathrm{mess}} \left( \omega \right)} \cdot \sigma_H
		\end{align*}
		mit $ \sigma_H $ als Schrittweite der jeweiligen Iteration
		% würde Notation als Vektoreintrag vorschlagen H_{...} [i] statt Darstellung als Funktion
		
		\item Optimierung der nichtlinearen Kennlinie mittels Vergleich der Differenz der erwarteten und gemessenen Spannungssignale $U_{quest}$ und der Faktoren $a$ der polynomialen Kennlinie:
		\begin{align*}
			\Delta U_{?} = U_{?, \mathrm{mess}} - U_{?, \mathrm{berechnet}} = \sum_n \tilde{a}_n U_{in}^n
			&&
			a_n^{\mathrm{neu}} = a_n^{\mathrm{alt}} + \sigma_a \cdot \tilde{a}_n
		\end{align*}
		mit $\sigma_a$ als Schrittweite der jeweiligen Iteration
	\end{itemize}
	
	
%%%%%%% weitere Kommentare: %%%%%%%%%%%%%%
%Frequenzverhalten der Kennlinie? Funktioniert Optimierung wie angesprochen überhaupt sinnvoll?
% mit welcher Iteration? Zuerst H optimieren, dann neue Werte generieren (messen) und dann K? Oder beides nacheinander und dann erst neu messen?


\end{frame}



