\begin{frame}{Offene Fragen}

\begin{itemize}
	\item Reihenfolge der Optimierung: Parallele Iteration $ \Leftrightarrow $ alternierende Iteration von H und K
	\item Einfluss von K auf das Spektrum von $U_?$ und damit auf Optimierung von H durch Oberschwingungen bei Potenzierung des Eingangssignals
	\item Bewertung der Qualität des Ausgangssignals nach einem Iterationsschritt
\end{itemize}

%%%%%%% weitere Kommentare: %%%%%%%%%%%%%%
%Frequenzverhalten der Kennlinie? Funktioniert Optimierung wie angesprochen überhaupt sinnvoll?
% mit welcher Iteration? Zuerst H optimieren, dann neue Werte generieren (messen) und dann K? Oder beides nacheinander und dann erst neu messen?
% Literatur zu iterativer Optimierung eines Hammerstein-Modells?
%macht Formel zu Iteration linear überhaupt so Sinn?


\end{frame}



