\documentclass[../Report.tex]{subfiles}


\begin{document}


\chapter{Ausblick}
\label{chap:ausb}
---- In diesem Kapitel wird auf offene Fragen / neue Probleme / Anstöße für weitere Arbeiten eingegangen. Dabei sollte es um eher inhaltliche Aspekte gehen (u. U. wenig to dos für Code-Design) gegebenenfalls darf hier bei vielem auf die Erfahrungen aus den vorigen Kapiteln verwiesen werden und damit einen Übersichts-Charakter haben (erleichtert nachfolgenden Projekten die Arbeit) --- 

\section{Impressionen zum Code}
\label{sec:ausb.code}
In Bezug auf das Design des Codes verbleiben eine Reihe Anregungen und Gedanken, die nicht realisiert wurden, je nach Ausführung aber interessant zu bedenken sein könnten.

\begin{itemize}
	\item	Die Geräte AWG und DSO als Klassen einzubinden könnte den Vorteil bieten, alle SCPI Commands an einer Stelle zu bündeln und den Zugriff darauf allgemeingültig zu halten.
	
	\item	Eine Einbindung des momentanen Programms in die RF-Tools des GSI-Standards könnte in zwei Teilen erfolgen. Die bestehende Funktionalität zur Berechnung von $K$ und $\Hcompl$ kann unabhängig der Optimierungsalgorithmen genutzt werden. Dabei ist insbesondere zu prüfen, ob oder wie die Aspekte im \nameref{chap:code} beibehalten werden. 
	
	\item 	Die momentane Version ist in erster Linie über eine Python-IDE ausführbar. Die Ausführung über die Kommandozeile wurde nicht fokussiert.
	\item Für die Erstellung eines idealen Barrier-Bucket Ausgangssignals wird momentan noch eine selbst implementierte Methode verwendet. Dafür kann auch das vorhandene RF-Tool verwendet werden.
	
\end{itemize}





\section{Ausblick: Optimierung}
\label{sec:ausb.opti}

- (offene Punkte K-Optimierung?)


Auf den Ergebnissen aus \nameref{chap:opt} aufbauend, verbleiben eine Reihe von offenen Fragestellungen:

\begin{enumerate}
	\item 	Wie wirkt sich die Optimierung von $K$ aufgrund ihrer Nichtlinearität auf die Übertragungsfunktion $\Hcompl$ aus und muss dies in der Optimierung berücksichtigt werden, etwa in der Reihenfolge der Iterationsschritte?
	
	\item	Wie wird mit der Tatsache umgegangen, dass im Frequenzbereich unabhängig der Qualität der Messung nur etwa halb so viele Daten für die Anpassung zur Verfügung stehen, wie in $\Hcompl$ selbst vorliegen? Eine erste Verbesserung könnte das zero-padding genannte Hinzufügen von zusätzlichen $0$-Werten an die Zeitsignale sein, womit sich für die Berechnung der FFT der Frequenzabstand zweier Werte verringern lässt. 
	
	\item	In welcher Reihenfolge wird die Iteration durchgeführt? Wird zuerst $\Hcompl$ in mehreren Durchgängen angepasst und danach $K$? Oder im Wechsel je eine Iteration?
	
	\item 	Wie wird die Auswahl der Schrittweiten $\sigma_H$ und $\sigma_a$ vorgenommen? Werden diese pauschal einmal festgesetzt zu Beginn des Algorithmus oder ist eine dynamische Anpassung, etwa durch die Qualität des letzten gemessenen Signals oder in Abhängigkeit des Iterationsschritts vorzuziehen? 
	
	\item 	Wie lässt sich der Einfluss von zufälligem Rauschen auf die Optimierung reduzieren? Insbesondere die Auflösung im höherfrequenten Betragsspektrum ist hier problematisch. Und im Falle des Ignorieren von Korrekturen an als fehlerhaft befundenen Frequenzen: Auf welchen Wert wird der Korrekturterm bei diesen Frequenzen gesetzt?
	
	\item	Wie lassen sich die im idealen Spektrum des Einzelsinus enthaltenen Nulldurchgänge in der Optimierung von $\Hcompl$ berücksichtigen, um interpolationsbedingt große Fehlerterme zu vermeiden? Ist das einfache Ignorieren dieser Frequenzen für die Anpassung eine Möglichkeit? Und in welchem kleinen Frequenzbereich um einen Nulldurchgang müssten dann die Werte ignoriert werden?
	
	\item 	Wie - insofern überhaupt - ist eine Optimierung der Phase von $\Hcompl$ zu gestalten?
	
	\item	Nach welchem Qualitätsmerkmal wird das Signal bewertet und wie wirkt sich dies auf den Algorithmus aus?
	
	\item	Gibt es eine sinnvolle Abbruchbedingung, mit der die Iteration versehen werden sollte? Etwa, dass sich die Qualität im Vergleich zu den vorherigen Iterationen nicht mehr mit ähnlicher Rate verbessert hat? Der Trade-Off liegt zwischen Laufzeit und Signal-Qualität.
	
	\item	Wie bestimmt man die Grenzen, in denen $K$ genutzt werden kann? Wie kann man garantieren, dass $K$ in den Bereichen, aus denen die Daten zur Berechnung von $a_n$ benutzten werden, bijektiv ist?
	
	\item	Gibt es eine Möglichkeit die Grenzen von $K$ bei der Optimierung zu erweitern? Ein möglicher Indikator: Kann man die Rückgabe von \lstinline{numpy.linalg.lstsq(LGS)} aus der Berechnung von $a_n$ nutzen, um eine Aussage über die Abweichung der Werte zu treffen?
\end{enumerate}





\section{--- Gerätekomm---}
\label{sec:ausb.geraete}
--- in dieser Section werden weitere Punkte der Geräte-Komm aufgegriffen, etwa - die (geringe) Auflösung des AWG im Kontext der Optimierung (ggf. in \ref{sec:ausb.opti} besser?) , die Einbindung des neuen Oszis oder die Idee der Klassen-Implementierung 
%TODO: prüfe Redundanz mit Abschnitt Gerätekommunikation! nur eine Einbindung, wo ist sie sinnvoller? 

%TODO: Überlegung ausführen, ob RF-Tools einbinden hier sinnvoller ist als in "Offenen Fragen" im Abschnitt Code-Design?






\end{document}