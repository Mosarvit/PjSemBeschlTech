\documentclass[../Report.tex]{subfiles}


\begin{document}


\chapter{Ausblick}
\label{chap:ausb}
---- In diesem Kapitel wird auf offene Fragen / neue Probleme / Anstöße für weitere Arbeiten eingegangen. Dabei sollte es um eher inhaltliche Aspekte gehen (u. U. wenig to dos für Code-Design) gegebenenfalls darf hier bei vielem auf die Erfahrungen aus den vorigen Kapiteln verwiesen werden und damit einen Übersichts-Charakter haben (erleichtert nachfolgenden Projekten die Arbeit) --- 

\section{---- Optimierung ----}
\label{sec:ausb.opti}
--- in dieser section werden die offenen Fragen / Anregungen in Bezug auf den Optimierungs-algorithmus dargelegt, etwa: \\
- Iterations-Reihenfolge mit Messungen: abwechselnd K / H optimieren und dann messen oder zwischen zwei Messungen beide optimieren? Oder erst das eine mit mehreren Messungen optimieren und dann das andere ?
\\
- sinnhaftigkeit / Grenzen des genutzten Algorithmus (Idee Jens) erfragen?
\\
- Problem / Frage nach Phasen-Optimierung des H-Optimierers
\\
- Umgang mit Rauschen, Null-Durchgängen o. ä. im H-Optimierer
\\
- Möglichkeiten der Optimierung: Durchlaufen lassen mit direkter Anpassung an steigende Amplituden? 
\\
- (offene Punkte K-Optimierung?)

\section{--- Gerätekomm---}
\label{sec:ausb.geraete}
--- in dieser Section werden weitere Punkte der Geräte-Komm aufgegriffen, etwa - die (geringe) Auflösung des AWG im Kontext der Optimierung (ggf. in \ref{sec:ausb.opti} besser?) , die Einbindung des neuen Oszis oder die Idee der Klassen-Implementierung 
%TODO: prüfe Redundanz mit Abschnitt Gerätekommunikation! nur eine Einbindung, wo ist sie sinnvoller? 

%TODO: Überlegung ausführen, ob RF-Tools einbinden hier sinnvoller ist als in "Offenen Fragen" im Abschnitt Code-Design?






\end{document}