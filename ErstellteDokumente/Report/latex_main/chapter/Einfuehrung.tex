\documentclass[../Report.tex]{subfiles}


\begin{document}


\chapter{Einführung}
\label{chap:einfuehrung}
---- In diesem Kapitel werden die Rahmen / Start-Bedingungen des Seminars vorgestellt. --- 

\section[--- name in inhaltsverzeichnis ---]{---- Modell BB ----}
\label{sec:einf.modell_BB}
--- in dieser section wird das BB-Signal nochmals kurz erläutert und eine allgemeine Erläuterung gegeben ---
--- Ziel des Abschnitts: Leser hat grobe Vorstellung, in welchem Kontext unser Programm entstanden ist und eingesetzt wird ---

\subsection{ ----- whatever necessary ----}
\label{subsec:einf.modell_BB.name}
--- falls notwendig ---

\section{ --- Motivation PJSem ---- }
\label{sec:einf.motivation}
--- in dieser section wird der Kontext konkretisiert, u. U. auf die Vorarbeit eingegangen --- 
--- je nach Ausführung kann diese section mit der folgenden zusammengelegt werden --- 

\section{ --- Problem / Aufgabenstellung PJSem ---- }
\label{sec:einf.problem}
--- in dieser section wird die konkrete Problemstellung erläutert und damit die Zielsetzung formuliert, 
auf die im Fazit zurückgekommen wird. Das Ziel darf damit auch als \glqq benefit\grqq des Programms im oben beschriebenen Kontext angesehen werden. --- 
\\





\end{document}