\documentclass[../Report.tex]{subfiles}


\begin{document}

\bibliographystyle{abbrv}

\begin{thebibliography}{9}
%TODO: füge unsere Quellen hinzu und passe Darstellung an, siehe dazu auch
% https://en.wikibooks.org/wiki/LaTeX/Bibliography_Management
%TODO: überprüfe Darstellung im Layout und überprüfe Position in der Reihenfolge der Kapitel (hier sinnvoll? Oder in den Anhang direkt?) und wird das Verzeichnis im Inhaltsverzeichnis geführt? 

  
\bibitem{PJS_Denys}
	Denys Bast, Armin Galetzka, 
	\textit{Projektseminar Beschleunigertechnik},
	2017.
 
\bibitem{harzheim}
	Jens Harzheim \textit{et al.}, 
	\textit{Input Signal Generation For Barrier Bucket RF Systems At GSI}, In Proceedings of IPAC2017 in Copenhagen, Dänemark. 
	pp. 3948 - 3950
	2017.

\bibitem{harzheim18}
	Jens Harzheim,
	----- Idee iterative Optimierung ----
	2018.
	
\bibitem{gross17}
	Kerstin Gross \textit{et al.},
	\textit{Test Setup For Automated Barrier Bucket Signal Generation},
	In Proceedings of IPAC2017 in Copenhagen, Dänemark. 
	pp. 3948 - 3950
	2017.
	
\bibitem{keysHand15}
	Keysight Technologies,
	\textit{Keysight Trueform Series Operating and Service Guide},
	2015.

\bibitem{keysData14}
	Keysight Technologies,
	\textit{33600A Series Trueform Waveform Generators - Data Sheet},
	2014.
	
\bibitem{troeser13}
	C. Tröser, 
	\textit{Application Note: Top Ten SCPI Programming Tips for Signal Generators},
	Rohde \& Schwarz,
	2013.
	
\bibitem{numpyfft}
	The SciPy community,
	https://docs.scipy.org/\-doc/numpy/reference/\-routines.fft.html
	- abgerufen am 11.08.2018.
	
\bibitem{cooley65}
	James Cooley, John Tukey,
	\textit{An algorithm for the machine calculation of complex Fourier series},
	Math. Comput. 19,
	1965.
	
\bibitem{lerch10}
	Reinhard Lerch,
	\textit{Elektrische Messtechnik, Analoge, digitale und computergestützte Verfahren}
	Springer,
	2010.

\bibitem{zeropad}
	---- Reference for zero-padding ---

	
\bibitem{Sommerville}
	I. Sommerville,
	\textit{Software Engineering, 9th edition}
	
\bibitem{helper_class}
	\url{https://en.wikipedia.org/wiki/Helper_class}
	
\bibitem{mcConnell}
	Code Complete 2 by Steve McConnell
	
\bibitem{ttd}
	\url{https://www.testingexcellence.com/pros-cons-test-driven-development/}
	
\bibitem{mock}
	\url{https://en.wikipedia.org/wiki/Mock_object}
	
\bibitem{system_testing}
	\url{https://en.wikipedia.org/wiki/System_testing}	

\end{thebibliography}

\chapter{Anhang}
\label{chap:anhang}
\begin{table}[H]
\centering 
\begin{tabular}[t]{| >{\texttt\bgroup}m{3.5cm}<{\egroup}|m{8cm}|m{2cm}|m{2cm}|} 
  \hline
    \textrm{\textbf{Name}} & \textbf{Funktion} & \textbf{Input} & \textbf{Output} \\ 
  \hline \hline
    \multicolumn{4}{|c|}{blocks} \\
  \hline \hline
  adjust\_a & Anpassung der Parameter für die Kennlinie & $a_{old}$ \newline $\Uin$ \newline $\Uquest_{,\textrm{ideal}}$ \newline $\Uquest_{,\textrm{meas}}$ \newline $\sigma_a$ & $a_{new}$\\
  \hline
  adjust\_H & Anpassung der Übertragungsfunktion & $H_{old}$ \newline $\Uout_{,\textrm{ideal}}$ \newline $\Uout_{,\textrm{meas}}$ \newline $\sigma_H$ & $H_{new}$\\
  \hline
  compute\_a & Berechnung der Parameter für die Kennlinie & $\Uin$ \newline $\Uquest$ \newline $N$ & $a$\\
  \hline
  compute\_K & Aufstellen der Look-Up Tabelle für die Kennlinie & $a$ & $K$\\
  \hline
  compute\_Uin & Berechnung des nichtlinear vorverzerrten Eingangssignals & $\Uquest$ \newline $K$ & $\Uin$\\
  \hline
  compute\_Uquest & Berechnung es linear vorverzerrten Eingangssignals & $\Uout$ \newline $H$ & $\Uquest$\\
  \hline
  determine\_a & Bestimmung der Parameter für die erste Kennlinie & $H$ \newline $\Uout_{,\textrm{ideal}}$ \newline $f_{S,\textrm{DSO}}$ & $a_0$\\
  \hline
  determine\_H & Bestimmung der Übertragungsfunktion über ein Pseudorauschen nach \cite{PJS_Denys} & load\_csv \newline save\_csv  & $H_0$\\
  \hline
  generate\_BB\_Signal & Erstellen eines idealen Barrier-Bucket Ausgangssignal & $f_{rep}$ \newline $f_{BB}$ \newline $V_{PP}$ \newline $f_{S,\textrm{AWG}}$ \newline save\_csv & $\Uout_{,\textrm{ideal}}$\\
  \hline
  get\_H & Funktion um Übertragungsfunktion zu Messen & $f_{max}$ \newline $V_{PP}$ \newline bits & $[f]$ \newline $[|\Hcompl|$ \newline $[\arg{\Hcompl}]$\\
  \hline
  loop\_adust\_a & Ein Iterationsschritt in der Anpassung von $K$ & $a_{old}$ \newline $K_0$ \newline $\Uout_{,\textrm{ideal}}$ \newline Iterationen \newline $f_{S,\textrm{DSO}}$ & $\Uout_{,\textrm{meas}}$ \newline$[Q_i]$ \newline $[K_i]$\\
  \hline
  loop\_adust\_H & Ein Iterationsschritt in der Anpassung von $\Hcompl$ & $H_0$ \newline $K$ \newline $\Uout_{,\textrm{ideal}}$ \newline Iterationen \newline $f_{S,\textrm{DSO}}$ & $[\Hcompl_i]$ \newline $[Q_i]$ \newline $\Uout_{,\textrm{meas}}$\\
  \hline
  measure\_Uout & Messsignal aus DSO auslesen & $a_{old}$ \newline $\Uin$ \newline $f_{S,\textrm{DSO}}$ & $\Uout$\\
  \hline
\end{tabular}
\caption{Liste aller Funktionen in blocks}
\label{tab:anhang.Funktionen.blocks}
\end{table}
\begin{table}[H]
\centering 
\begin{tabular}[t]{| >{\texttt\bgroup}m{3.5cm}<{\egroup}|m{8cm}|m{2cm}|m{2cm}|} 
  \hline
    \textrm{\textbf{Name}} & \textbf{Funktion} & \textbf{Input} & \textbf{Output} \\ 
  \hline \hline
    \multicolumn{4}{|c|}{classes} \\
  \hline \hline
  signal & Abstract Data Type für Signale & $[time]$ \newline $[U]$  & $U$\\
  \hline
  transfer\_function & Abstract Data Type für Übertragungsfunktion & $H_{old}$ \newline $\Uout_{,\textrm{ideal}}$ \newline $\Uout_{,\textrm{meas}}$ \newline $\sigma_H$ & $H_{new}$\\
  \hline \hline
    \multicolumn{4}{|c|}{helpers} \\
  \hline \hline
  apply\_ \newline transfer\_function & Anwenden einer Übertragungsfunktion auf ein Signal & $Uout$ \newline $\Hcompl$  & $\Uquest$\\
  \hline
  MLBS & Pseudorauschsignal bestimmen & bits & output \newline seedRandom\\
  \hline
  overlay & Zwei Signale übereinanderlegen mit Kreuzkorrelation & $U_1$ \newline $U_2$ & $U_{1,\textrm{shifted}}$\\
  \hline
  read\_from\_DSO & Auslesen eines Messsignals aus dem DSO, der Parameter measure\_H muss auf 1 gesetzt werden, damit das Signal für die Übertragungsfunktion richtig gemessen wird (diese Funktion sollte dringend nochmal überarbeitet werden & $f_{S,\textrm{DSO}}$ \newline $V_{PP,\textrm{ch1}}$ \newline $V_{PP,\textrm{out}}$ \newline $f_{max}$ \newline $U$ \newline measure\_H & $[time]$ \newline $[U_{in}]$ \newline $[U_{out}]$\\
  \hline
  read\_from\_DSO\_ \newline resolution & Verwendet die vorherige Funktion, um eine bessere Auflösung zu erreichen & $f_{S,\textrm{DSO}}$ \newline $V_{PP,\textrm{ch1}}$ \newline $f_{max}$ \newline $U$ \newline measure\_H & $[time]$ \newline $[U_{in}]$ \newline $[U_{out}]$\\
  \hline
  signal\_evaluate & Bestimmung der Güte eines Ausgangssignales mit dem RF-Tool & Uout\_file \newline Result\_file & $[Q]$\\
  \hline
  write\_to\_AWG & Signal an das AWG übergeben & $U$ \newline $V_{PP,\textrm{AWG}}$ $f_{rep}$ \newline $f_{S,\textrm{AWG}}$ &  \\
  \hline
\end{tabular}
\caption{Liste aller Funktionen und Klassen aus classes und helpers}
\label{tab:anhang.Funktionen.class.helpers}
\end{table}

\begin{table}[H]
\centering 
\begin{tabular}[t]{| >{\texttt\bgroup}m{6cm}<{\egroup}|m{8cm}|} 
  \hline
    \textrm{\textbf{Routine}} & \textbf{Funktion} \\ 
  \hline \hline
  evaluate\_adjust\_a & Ist die Implementierung des ersten Optimierungsansatzes für $K$ \\
  \hline
  evaluate\_adjust\_H & Ist die Implementierung des ersten Optimierungsansatzes für $\Hcompl$ \\
  \hline 
  evaluate\_connect\_Devices & Testet unterschiedliche Befehle aus der Gerätekommunikation \\
  \hline
  evaluate\_K & Testet mehrere Bereiche einer zuvor berechneten Kennlinie \\
  \hline
  evaluate\_with\_BBSignal & Berechnung des ersten nichtlinear vorverzerrten Eingangssignals \\
  \hline
\end{tabular}
\caption{Liste aller verwendeten Routinen}
\label{tab:anhang.Funktionen.routine}
\end{table}

\end{document}