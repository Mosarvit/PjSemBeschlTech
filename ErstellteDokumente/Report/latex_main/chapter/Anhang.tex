\documentclass[../Report.tex]{subfiles}


\begin{document}

\bibliographystyle{abbrv}

\begin{thebibliography}{9}
%TODO: füge unsere Quellen hinzu und passe Darstellung an, siehe dazu auch
% https://en.wikibooks.org/wiki/LaTeX/Bibliography_Management
%TODO: überprüfe Darstellung im Layout und überprüfe Position in der Reihenfolge der Kapitel (hier sinnvoll? Oder in den Anhang direkt?) und wird das Verzeichnis im Inhaltsverzeichnis geführt? 

\bibitem{lamport94}
  Leslie Lamport,
  \textit{\LaTeX: a document preparation system},
  Addison Wesley, Massachusetts,
  2nd edition,
  1994.
  
\bibitem{PJS_Denys}
	Denys Bast, Armin Galetzka, 
	\textit{Projektseminar Beschleunigertechnik},
	2017.
 
\bibitem{harzheim}
	Jens Harzheim \textit{et al.}, 
	\textit{Input Signal Generation For Barrier Bucket RF Systems At GSI}, In Proceedings of IPAC2017 in Copenhagen, Dänemark. 
	pp. 3948 - 3950
	2017.
	
\bibitem{gross17}
	Kerstin Gross \textit{et al.},
	\textit{Test Setup For Automated Barrier Bucket Signal Generation},
	In Proceedings of IPAC2017 in Copenhagen, Dänemark. 
	pp. 3948 - 3950
	2017.
	
\bibitem{keysHand15}
	Keysight Technologies,
	\textit{Keysight Trueform Series Operating and Service Guide},
	2015.

\bibitem{keysData14}
	Keysight Technologies,
	\textit{33600A Series Trueform Waveform Generators - Data Sheet},
	2014.
	
\bibitem{troeser13}
	C. Tröser, 
	\textit{Application Note: Top Ten SCPI Programming Tips for Signal Generators},
	Rohde \& Schwarz,
	2013.
	
\bibitem{Sommerville}
	I. Sommerville,
	\textit{Software Engineering, 9th edition}
	
\bibitem{helper_class}
	\url{https://en.wikipedia.org/wiki/Helper_class}
	
\bibitem{mcConnell}
	Code Complete 2 by Steve McConnell
	
\bibitem{ttd}
	\url{https://www.testingexcellence.com/pros-cons-test-driven-development/}
	
\bibitem{mock}
	\url{https://en.wikipedia.org/wiki/Mock_object}
	
\bibitem{system_testing}
	\url{https://en.wikipedia.org/wiki/System_testing}	

\end{thebibliography}

\chapter{Anhang}
\label{chap:anhang}


\begin{table}[H]
\centering 
\begin{tabular}[t]{| >{\texttt\bgroup}r<{\egroup} | l | l | l |} 
  \hline
    \textrm{\textbf{Ordner}} & \textbf{Funktion} & \textbf{Input} & \textbf{Output} \\ 
  \hline \hline
    \multicolumn{4}{|c|}{blocks} \\
  \hline \hline
  adjust\_a & Anpassung der Parameter der Kennlinie 	& $a_{old}$ & $a_{new}$ \\
  & & $\Uin$ & \\
  & & $\Uquest_{,\textrm{ideal}}$ & \\
  & & $\Uquest_{,\textrm{meas}}$ & \\
  & & $\sigma_a$ & \\
  \hline
  adjust\_H & Anpassung der Übertragungsfunktion 	& $H_{old}$ & $H_{new}$ \\
  & & $\Uout_{,\textrm{ideal}}$ & \\
  & & $\Uout{,\textrm{meas}}$ & \\
  & & $\sigma_H$ & \\
  \hline
  compute\_a & Berechnung der Parameter der Kennlinie 	& $\Uin$ & $a_n$ \\
  & & $\Uquest$ & \\
  & & $N$ & \\
  \hline
  compute\_K & Berechnung der Parameter der Kennlinie 	& $a_n$ & $K$ \\
  \hline
  compute\_Uin & Berechnung des Eingangssignals		 	& $\Uquest$ & $\Uin$ \\
  & & $K$ & $\Uquest_{,\textrm{adapted}}$ \\
  \hline
  compute\_Uquest & Berechnung der virtuellen Spannung $\Uquest$ & $\Uout$ & $\Uquest$ \\
  & & $H$ & \\
  \hline
  
  
  
  \hline \hline
    classes & ADTs, siehe \ref{subsec:code.adt} \\  
  \hline \hline
    data & Alle Ergebnisse \\ 
  \hline
    helpers & Hilfsfunktionen nach \cite{helper_class} 	\\
  \hline 
    tests & Alle Klassen mit Unit Tests \\  
  \hline 
    tools & RF-Tools \\ 
  \hline
\end{tabular}
\caption{Liste aller Funktionen}
\label{tab:anhang.Funktionen}
\end{table}



\end{document}