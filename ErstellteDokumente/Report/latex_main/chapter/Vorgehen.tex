\documentclass[../Report.tex]{subfiles}


\begin{document}


\chapter{Vorgehen}
\label{chap:vorgehen}
---- In diesem Kapitel soll die Übernahme des Matlab-Codes, die Dokumentation (samt Konventionen und Dokumentationskonzept) und die Gerätekommunikation dargestellt werden 
Ggf. ist es sinnvoll, Teile davon in eigene Chapter auszulagern oder den Titel anzupassen--- 

\section[---setup ---]{---- Start und bereits vorhanden  ----}
\label{sec:vorg.setup}
--- in dieser section wird auf die anfangs gegebene Code-Struktur, die vorgegebenen Methoden und die parallelen Strukturen zwischen matlab und python eingegangen und ggf. die projektarbeit von Denys und Armin in Hinblick auf konkretere Aspekte referenziert ---
--- Ziel des Abschnitts: Leser hat (gute) Vorstellung, mit welchen Voraussetzungen, welchen Daten- und Programmstrukturen wir gestartet sind und auf welchen wir aufbauen---


\section{ --- Konventionen / Dokumentation ---- }
\label{sec:vorg.doku}
--- in dieser section wird unser Konzept der Dokumentation in Code wie auch in Dokumenten erläutert und ggf. auf die Sammlung von Handbüchern eingegangen --- 
--- es könnte sinnvoll sein, diese section nach der folgenden erst zu setzen --- 

\section{ --- Gerätekommunikation ---- }
\label{sec:vorg.geraete}
--- in dieser section wird ausführlich erläutert, wie wir an die Geräte herangegangen sind, ggf. nochmals wie wir die Arbeit damit dokumentiert haben oder wie wir versucht haben, die Routinen auszulagern / zu separieren bzw. welche Verbesserungen wir (etwa bzgl. der Achsen-Anpassung o.ä.) vorgenommen haben   --- 
\\

\subsection{--- offene Punkte---}
\label{subsec:vorg.geraete.open}
-- hier könnten punkte wie etwa die Kommunikation mit dem neuen Oszi oder die evtl. Möglichkeit, Geräte künftig eher als Klassen zu implrementieren, in denen die Visa-Commands gehandhabt werden --- 





\end{document}