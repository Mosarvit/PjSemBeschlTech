\documentclass[../Report.tex]{subfiles}


\begin{document}


\chapter{Code-Design}
\label{chap:code}
Die Aufgabe des Projektseminars war ein bestehendes Tool weiter zu entwickeln. Es wurde nach einer Methode gesucht an die vorhandene Funktionalität anzuschließen und fortzusetzen. Gleiches wird wahrscheinlich mit dieser Projektarbeit passieren. Deshalb wird in diesem Kapitel die Namenskonvention und die allgemeine Struktur unseres Design vorgestellt.

\section{Motivation}
\label{sec:code.motivation}
Ziel ist es ein Design aufzustellen, das mit seinen Funktionen und seiner gesamten Dokumentation für sich spricht. Des weiteren wird eine lose Kopplung der einzelnen Bestandteile angestrebt, damit Bestandteile ohne die Logik anderer Funktionen verstanden und bearbeitet werden können. Dies verbessert zum einen die Arbeitsaufteilung zum anderen auch die Verständlichkeit des gesamten Programms.

\section{Aufbau}
\label{sec:code.aufbau}
Der Aufbau des Programms orientiert sich an dem vorgegebenen Modell aus \figref{fig:Hammerstein}.

\subsection{Namenskonvention}
\label{subsec:code.namen}
Einfache und selbsterklärende Präfixe von Funktionen \lstinline{adjust_}


 

\subsection{Ordnerstruktur}
\label{subsec:code.ordner}

\subsection{Abstact Data Types}
\label{subsec:code.adt}

\section{Methodik}
\label{subsec:code.methodik}

\subsection{Test Driven Development}
\label{subsec:code.tdd}

\subsection{Mock-System}
\label{subsec:code.mock}

\end{document}