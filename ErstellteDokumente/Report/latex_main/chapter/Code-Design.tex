\documentclass[../Report.tex]{subfiles}


\begin{document}


\chapter{Code-Design}
\label{chap:code}
---- In diesem Kapitel wird auf das neue Code-Design (Refactoring), das Konzept und die Motivation im Kontext eines Mess-Konzepts eingegangen ggf. auch auf Versionenverfolgung Git?--- 

\section{---- Motivation / Begründung ----}
\label{sec:code.mot}
--- hier kann auf die Sinnhaftigkeit unseres Code-Designs (rückwärts gedacht ;) ) eingegangen werden: Welche Ideen finden sich nachher wieder? Warum also wurde die folgende Umsetzung gewählt? Ggf. auch kurzes Eingehen auf Entwicklung des Designs hier??? Keine Ausführlichen Erklärungen, eher allgemeine Ideen und Aspekte des Software-Engineering ausführen. Konkrete Punkte nachfolgend erst ---


\section{ --- Aspekte des Desings ---- }
\label{sec:code.asp}
--- in dieser section werden die einzelnen Punkte weiter ausgeführt. Aus Gründen der Dokument-Hierarchie erscheint es sinnvoller, die einzelnen Punkte in sub-sections zu setzen. Hier also nur eine Kurze Zusammenfassung? --- 

\subsection{--- Struktur -- ?}
\label{subsec:code.asp.struc}
--- hier kann ein kleiner Überblick über die Programmstruktur gegeben werden, also die Aufteilung nach Ordnern (routinen und Funktionalität) und Nomenklatur??? Unbedingt auf Redundanz mit Motivation prüfen!!! Insb. Einführung von Klassen und Helpers-System erläutern
---

\subsection{--- TDD und Mock-System --- }
\label{subsec:code.asp.tdd}
--- Ausführen des Test-Driven-Developments und des Mock-Systems. Wie wurde implementiert?---

\subsection{Offene Punkte}
\label{subsec:code.asp.open}
--- hier ist m.E. eine Ausführung noch offener Aspekte angebracht, etwa: Typprüfung bei Übergaben ausweiten (! und an RF-Tool Konventionen anpassen?), Test für falsche übergabe-Parameter noch als Möglichkeit nennen, Erweiterbarkeit des Mock-Systems erklären --- 




\end{document}