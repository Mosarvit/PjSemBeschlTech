\documentclass[../Report.tex]{subfiles}


\begin{document}


\chapter{Optimierung}
\label{chap:opt}
---- In diesem Kapitel werden die Aspekte der durchgeführten Optimierungs-Algorithmen erläutert --- 

\section{---- Theorie ----}
\label{sec:opt.theo}
--- in dieser section werden die von Jens vorgeschlagenen Ideen ausgeführt, rein theoretischer Natur also. Ebenso unsere Anpassungen.  ---
%TODO: überprüfe Referenz zu Jens, da Ideen nicht publiziert wurden! Wie wird das eingebunden?

\section{ --- Opti Übertragungsfkt --}
\label{sec:opt.H}
--- evaluate-Aufruf, Schleife, Speicher? Laufzeit? . insbesondere gemessene Daten, ohne jedwede Anpassung /Limitierung der Faktoren, Erfahrungen mit Phase, mit RMS-Cutting, mit Prozentualem Ausschnitt aus FFTs Hier (oder in \ref{sec:opt.theo} ) kann auf die Einbindung des neuen Qualitäts-Tools eingegangen werden --- 

\section{--- Opti Kennlinie --}
\label{sec:opt.K}
%Sofern Erfahrungen vorhanden!

\end{document}