\documentclass[../Report.tex]{subfiles}


\begin{document}

\chapter{Iterative Optimierung des Hammerstein-Modells}
\label{chap:opt}
---- In diesem Kapitel werden die Aspekte der durchgeführten Optimierungs-Algorithmen erläutert --- \\
%TODO: überprüfe Referenz zu Jens, da Ideen nicht publiziert wurden! Wie wird das eingebunden?
Ziel der Optimierung von Übertragungsfunktion $\Hcompl$ und Kennlinie $K$ mit ihren Parametern $a$ ist die Minimierung des Fehlers zwischen idealem und gemessenem Ausgangssignal, $\Uout_{, \mathrm{id}}$ und $\Uout_{, \mathrm{meas}}$ . Die Minimierung des relativen Fehlers ist also gegeben durch
\begin{align}
\label{eq:opt.relFehler}
	\min \; \mathit{f} \oft = \min \left( \frac{\Uout_{, \mathrm{meas}}  - \Uout_{, \mathrm{id}} }{ \Uout_{, \mathrm{id}} } \right) 
	= \min \left( \frac{ \Uout_{, \mathrm{meas}}}{ \Uout_{, \mathrm{id}}} -1 \right) 
	\; .
\end{align}
Für das verwendete Hammerstein-Modell liegt die in \cite{----Jens---} %TODO: Referenz Idee iterative Optimierung Jens
vorgeschlagene getrennte, iterative Optimierung von $\Hcompl$ und $K$ nahe. 
Die Auswertung der Qualität des Einzelsinus erfolgt dabei durch das RF-Tool von --- Zitat RF-Tool --- mit Entwicklungsstand vom --- --- unter Verwendung des als \lstinline{QGesamt1} geführten Qualitätswerts \footnote{\label{foot:opt.H.quality}Hierauf beziehen sich alle weiteren Angaben zur Qualität des Signals. Eine intensivere Befassung mit dem Tool hat nicht stattgefunden.}.
%TODO: Auffüllen Daten!




\section{Optimierung der linearen Übertragungsfunktion $H$}
\label{sec:opt.H}
--- evaluate-Aufruf, Schleife, Speicher? Laufzeit? . insbesondere gemessene Daten, ohne jedwede Anpassung /Limitierung der Faktoren,kann auf die Einbindung des neuen Qualitäts-Tools eingegangen werden --- \\

Die Optimierung von $\Hcompl \ofomega$ beruht auf der Annahme, dass sich \eqref{eq:opt.relFehler} auf die Betragsspektren des berechneten und des gemessenen Ausgangssignals, $\Uoutc_{, \mathrm{id}} \ofomega $ und $\Uoutc_{, \mathrm{meas}} \ofomega $ fortsetzen lässt mit 

\begin{align}
\label{eq:opt.ratio}
	\fabs \ofomega :=  
				\frac{\mathrm{abs} \left( \Uoutc_{, \mathrm{meas}} \ofomega \right)}{\mathrm{abs} \left(\Uoutc_{, \mathrm{id}} \ofomega \right)} -1
				\; .
\end{align} 

Ist im Betragsspektrum des gemessenen Signals eine Frequenz mit halbem Betrag verglichen mit dem idealen Signal vertreten, wird dies entsprechend der Linearität der Übertragunsfunktion dahingehend gedeutet, dass die Verstärkung von $\Hcompl$ bei dieser Frequenz um einen Faktor $2$ zu gering ist.
Iterativ mit einer Schrittweite $\sigma_H$ ausgeführt, folgt für den $i$-ten Schritt

\begin{align}
\label{eq:opt.Hnew}
	\mathrm{abs} \left( \Hcompl^{i+1} \right)
		=\mathrm{abs} \left( \Hcompl^{i}  \right) \cdot
		\left( 1 - \sigma_H^i \: \fabs^{i}	\right)					 
\end{align}

für $\sigma_H^i \in \left[ 0 , 1 \right]$ und $\Uoutc_{, \mathrm{meas}}^{i}$ in $\fabs^{i}$ als gemessenem Ausgangssignal für das mit $\Hcompl^{i}$ berechnete Eingangssignal \footnote{Nachfolgend wird aus Gründen der Übersichtlichkeit $\fabs$ statt dem länglichen Bruch genutzt}.
Würde allerdings \eqref{eq:opt.Hnew} mit komplexen Zahlen und nicht allein den Beträgen ausgeführt, würde auch die Phase der $-1$ beachtet und folglich die durch $\sigma_H$ skalierte komplexe Zahl wesentlich verändert. Also muss für das Phasenspektrum eine andere Optimierung erfolgen.
Eine Möglichkeit hierfür wäre die simple Anpassung der Phase $ \mathrm{arg} \left( \Hcompl \right) = \varphi_H$ mit 
\begin{align}
\label{eq:opt.HnewPhase}
	\varphi_H^{i+1} = \varphi_H^{i} - \sigma_{\varphi}^{i} 
			\left( \: \mathrm{arg} \left( \Uoutc_{, \mathrm{meas}} \right)
					- \mathrm{arg} \left( \Uoutc_{, \mathrm{id}} \right) \: \right)
\end{align}
mit $\sigma_{\varphi}^i \in \left[ 0 , 1 \right]$. Diese Anpassung der Phase wurde jedoch nur kurzen Tests unterzogen und anschließend nicht weiter verfolgt. Es hat sich die Signalform des Ausgangssignals unproportional stärker verändert, als dies nur im Falle der Betrags-Anpassung der Fall war. Vermutlich %%%% TODO: kann man vermutlich hier nutzen???
liegt dies an dem aus dem Ausgangssignal gewonnenen Phasengang, der in wesentlich größerem Maße vom idealen Phasengang abweicht als im Betragsspektrum. In \figref{fig:opt.spektrum_BB_signal} sind Betrag und Phase der durch FFT erhaltenen Spektren für gemessenes und ideales Ausgangssignal vor Durchführung einer Optimierung dargestellt. Insbesondere illustriert \figref{fig:opt.tzeum} die bei gemessenem Signal auftretende Streuung der Phase. 
\\


\pgfplotstableread[col sep = comma] {opt_spect_ideal_abs.csv} \absSpectIdeal 
\pgfplotstableread[col sep = comma] {opt_spect_meas_abs.csv} \absSpectMeas
\pgfplotstableread[col sep = comma] {opt_spect_ideal_angle.csv} \angleSpectIdeal 
\pgfplotstableread[col sep = comma] {opt_spect_meas_angle.csv} \angleSpectMeas 

\begin{figure}[htb]
\begin{subfigure}{0.5 \textwidth}
\centering
    \begin{tikzpicture}
\begin{axis}[
		legend entries = {Ideales Signal, Gemessenes Signal},
		legend pos = north east,
		xlabel={Frequenz},
		ylabel={Spektraldichte },
		%xtick distance = 10000000,
%		xminorgrids,
%		xmajorgrids,
		minor x tick num =3,
		xtick pos = lower,
		ytick pos = left,
		xtick align = outside,
		ytick align = outside,
		scaled x ticks = base 10:-6,
		xtick scale label code/.code={\si{\MHz}},
		xmin = -5000000,
		xmax = 85000000,
		ymin = 0,
		ymax = 0.03,
		]
		
		\addplot[blue, mark size=3.5pt] table [ x index =0, y index=1] {\absSpectIdeal};	% plot des Idealen Betragsspektrums
		\addplot[green, mark size=3.5pt] table [ x index =0, y index=1] {\absSpectMeas};	% Plot des gemessenen Betragsspektrums
\end{axis}
\end{tikzpicture}
\caption{Betragsspektren}
	\label{fig:opt.abs_spektrum}
\end{subfigure}
\begin{subfigure}{0.5 \textwidth}
\centering
    \begin{tikzpicture}
\begin{axis}[
		legend entries = {Ideales Signal, Gemessenes Signal},
		legend pos = north east,
		xlabel={Frequenz},
		ylabel={Phase},
		y label style={at={(axis description cs:-0.1,.5)},anchor=south},
%		%xtick distance = 10000000,
%		xminorgrids,
%		xmajorgrids,
		minor x tick num =3,
		xtick pos = lower,
		ytick pos = left,
		xtick align = outside,
		ytick align = outside,
		scaled x ticks = base 10:-6,
		xtick scale label code/.code={\si{\MHz}},
		xmin = -5000000,
		xmax = 85000000,
		scaled y ticks={real:3.1415},
		ytick scale label code/.code={$\si{\radian}$},
%		ymin = 0,
%		ymax = 0.03,
		]
		
		\addplot[blue, only marks, mark size=1pt] table [ x index =0, y index=1] {\angleSpectIdeal};	% plot des Idealen Betragsspektrums
		\addplot[green, only marks, mark size=1pt] table [ x index =0, y index=1] {\angleSpectMeas};	% Plot des gemessenen Betragsspektrums
\end{axis}
\end{tikzpicture}
\caption{Phasenspektren}
	\label{fig:opt.angle_spektrum}
\end{subfigure}
\caption{Spektrum des Einzelsinus-Signals, berechnet und gemessen mit je $109$ Punkten}
\label{fig:opt.spektrum_BB_signal}
\end{figure}

Eine Aufstellung der rein auf \eqref{eq:opt.Hnew} beruhenden Anpassung der Übertragungsfunktion über mehrere Iterationsschritte findet sich in --- Abb ---. 
Neben den für kontinuierliche Funktionen problemlos definierbaren iterativen Zuweisungen ergeben sich in Messung und diskreter Ausführung jedoch Fehlerquellen. Problematisch sind insbesondere solche, die in \eqref{eq:opt.Hnew} durch das Betragsverhältnis der Ausgangssignale verstärkt werden. 
Unterscheiden sich die Spektren hier um einen großen Faktor, resultiert dies in einer großen Anpassung der Übertragungsfunktion für die betreffende Frequenz. Dies ist folglich insbesondere bei kleinen Beträgen der Spektren problematisch, wenn Ungenauigkeiten und Störeinflüsse betrachtet werden. 
\\
Besondere Störeinflüsse ergeben sich also durch
\begin{itemize}
	\item Rauschen: Weißes Rauschen macht sich in allen Frequenzen bemerkbar mit kritischem Einfluss bei geringer Spektraldichte des Signals.
	
	\item Diskretisierungsfehler: Die FFT bedingt eine begrenzte Auflösung, in den Spektren von $\Hcompl$ und den gemessenen Signalen und liegt insbesondere im Allgemeinen an unterschiedlichen Frequenzen und mit unterschiedlich vielen Punkten vor.
	
	\item Interpolationsfehler: Die (hier lineare) Interpolation der Spektren zur Auswertung von $\fabs$ an den Frequenzen von $\Hcompl$ kann insbesondere den Einfluss oben genannter Punkte verstärken.
\end{itemize}

Weiterhin zeigt sich auch in der geringen Stützstellenzahl in \figref{fig:opt.spektrum_BB_signal} bereits eine erste konzeptuelle Problematik des Vorgehens. Der Frequenzabstand zwischen zwei Werten der FFT ist stets mit der Wiederholfrequenz $f_{rep}$ gegeben und lässt sich somit nicht durch eine höhere Auflösung der Messgeräte verbessern, der betrachtete Bereich bis $\SI{80}{\MHz}$ nicht besser auflösen.
Folglich setzt sich diese Ungenauigkeit auch auf die Optimierung der Kennlinie fort. Insbesondere relevant wird dies, da die Kennlinie mit nahezu der doppelten Anzahl an Werten erstellt wird und somit der Interpolationsfehler ungleich größer wird als bei ähnlicher Anzahl Stützstellen.


\subsubsection*{Ignorieren kleiner Beträge im Spektrum}
\label{subsubsec:opt.H.prom}

Um Rauscheinflüsse und Probleme durch Nulldurchgänge zu dämpfen, wurde einer erster intuitiver Ansatz vorgenommen: Bei den Betragsspektren der in $\fabs$ eingehenden Signale, des gemessenen und idealisierten Spannungssignals, wurden alle Anteile, die verglichen mit dem Maximalwert des betreffenden Spektrums besonders klein sind, auf einen vorgegebenen Wert, im Folgenden Default-Wert genannt, gesetzt. Dies führt an den betroffenen Frequenzen zu $\fabs = 0$ und damit keiner Änderung von $\Hcompl$.
Dies bedeutet also, dass alle Einträge des Betragsspektrums von $\Uout_{,\mathrm{ideal}}$ mit weniger als zum Beispiel $5 \, \promille $ der maximalen Amplitude auf den Default-Wert gesetzt werden. Insbesondere werden auch die Einträge an den Frequenzen zurückgesetzt, die im Spektrum von $\Uout_{,\mathrm{meas}}$ klein gegen das zugehörige Maximum sind.
\\
Zu beachten bei letzterem Punkt ist die notwendige Rundung, wenn die Einträge der FFT an unterschiedlichen Frequenzen vorliegen. 
\\
\\
\indent
Mit Beschränkung auf $5 \, \promille$ und dem globalen Minimum beider Spektren als Default-Wert ergeben sich die angepassten Betragsspektren wie in ---- ABB --- zu sehen, der Übersichtlichkeit halber mit kleinem vertikalen Ausschnitt.
Mit diesem Schritt ergibt sich über drei Iterationen ein Verlauf des Korrekturterms $\fabs$ und der Übertragungsfunktion für eine Schrittweite $\sigma_H = \nicefrac{1}{2}$ wie in ---- ABB --- dargestellt. 


\subsubsection*{Ignorieren großer Korrektur-Terme}
\label{subsubsec:opt.H.RMS}

Ein zweiter, sehr grober Ansatz liegt in der Beschränkung von $\fabs$ auf Werte unterhalb einer vorgegebenen Schwelle. Zugrunde liegt die Annahme, dass die gerade an Nulldurchgängen des Spektrums sowie bei vielen hohen Frequenzen auftretenden großen Werte durch die in obiger Aufzählung genannten Fehlerquellen entstehen. Hier bedeutet dies insbesondere, dass die Diskretisierung die Nulldurchgänge nicht korrekt darstellen kann. Die Interpolation auf Frequenzen von $\Hcompl$ ist dann aufgrund der großen Sprünge von Werten in direkter Umgebung der problematischen Frequenzen mit großer Ungenauickgeit behaftet. Dies kann zu den beschriebenen, großen Korrektur-Termen in $\fabs$ führen.

\lstset{language=Python}
\begin{lstlisting}[caption={Pseudocode zur Veranschaulichung der Anpassung des Korrekturterms}, label=code:opt.H.pseudoRMS, numbers=none]
	rms_orig = root_mean_square( f_abs )
	f_abs_to_use = f_abs[ where( abs(f_abs) >= 0.02 * rms_orig ] 
	rms_mod = root_mean_square( f_abs_to_use )
	idx_to_clear = f_abs[ where( abs(f_abs) >= rms_mod ] 
	f_abs[ ix_to_clear ] = 0
\end{lstlisting}

Vereinfacht bedeutet der verfolgte Ansatz, ausnehmend große Werte von $\fabs$ als unrealistisch abzutun. Eine Pseudo-Implementierung findet sich in \ref{code:opt.H.pseudoRMS}, um die nachfolgende Erläuterung zu illustrieren. In der vorgenommenen Implementierung wurde $\fabs$ an den ausgewählten Frequenzen auf $0$ gesetzt. Als Grenze genutzt wurde ein modifizierter Effektivwert, nachfolgend mit RMS (Root Mean Square) bezeichnet. 
Der reine RMS von $\fabs$ unterliegt der Problematik, eine unproportional große Gewichtung von kleinen Einträgen zu enthalten.
\\
Idealerweise enthält $\fabs$ mit jeder Iteration kleinere Einträge als zuvor. Es würden also bei Nutzung des reinen RMS unter Umständen mit zunehmender Schrittzahl zunehmend mehr Werte in $\fabs$ ignoriert - was der Optimierung entsprechende Grenzen setzt. 
In Kombination mit den im vorigen Abschnitt erläuterten Anpassungen wäre die Problematik unumgänglich, da Frequenzen, die explizit nicht bei der Anpassung berücksichtig werden sollen, den reinen RMS-Wert beeinflussen.
Folglich muss der RMS modifiziert werden. Hier wurden zur Berechnung des modifizierten RMS nur die Werte einbezogen, die mehr als beliebig gewählte $2 \, \%$ des reinen RMS betragen. Es handelt sich also bei der vorgenommenen Anpassung um eine sehr grobe und größtenteils willkürliche Wahl der Parameter, die zu Zwecken der Illustration jedoch brauchbare Ergebnisse liefert.
 

In \figref{fig:opt.H.iteration} ist die Entwicklung von Übertragungsfunktion und $\fabs$ über mehrere Iterationen aufgetragen. Der Einfluss des RMS-Cutters macht sich dabei verglichen mit ---- ABB oben, rein iteriert --- bemerkbar, es treten weniger starke Änderungen auf. 
Gleichzeitig zeigt sich, dass nicht in jedem Schritt an exakt den gleichen Stellen am jeweiligen RMS geschnitten werden muss. Dies belegt die Zufälligkeit des Fehlers und erläutert die prinzipielle Berechtigung der Methodik.
Es zeigt sich, dass die vorgenommene Anpassung keinen großartigen Einfluss auf die Qualität des Signals hat.
Dies ist insofern beachtenswert, als dass die Übertragungsfunktion auch an einigen Stellen mit massiver Verstärkung stark angepasst wird, vergleiche hierzu \figref{fig:opt.H.RMS_H} bei etwa $\SI{25}{\MHz}$.
Die Qualität des Signals bewegt sich zwischen einem Wert von ---- ---- und ---- ---- und zeigt vor allem rauschbedingte Schwankungen.
%TODO: check!!! Mehr Durchläufe ausprobieren, unbedingt!

\pgfplotstableread[col sep = comma] {f_abs_orig.csv} \fabsOrig 
\pgfplotstableread[col sep = comma] {f_abs_RMS_1.csv} \fabsRMSone 
\pgfplotstableread[col sep = comma] {f_abs_RMS_2.csv} \fabsRMStwo
\pgfplotstableread[col sep = comma] {f_abs_RMS_3.csv} \fabsRMSthree
\pgfplotstableread[col sep = comma] {H_RMS_orig.csv} \Horig
\pgfplotstableread[col sep = comma] {H_RMS_1.csv} \HrmsOne
\pgfplotstableread[col sep = comma] {H_RMS_2.csv} \HrmsTwo
\pgfplotstableread[col sep = comma] {H_RMS_3.csv} \HrmsThree
 

\begin{figure}[htb]
%\begin{center}
\begin{subfigure}{\textwidth}
%	\begin{center}
    \begin{tikzpicture}
\begin{axis}[
		scale only axis,
		width = 0.75 \textwidth,
		height = 0.2 \textheight,
		legend entries = {original, Step 1, Step 2, Step 3},
		legend pos = north east,
		legend columns = 3,
		xlabel={Frequenz},
		ylabel={Betrag Übertragung $\Hcompl$},
		y label style={at={(axis description cs:-0.1,.5)},anchor=south},
		%xtick distance = 10000000,
%		xminorgrids,
%		xmajorgrids,
%		minor x tick num =3,
		xtick pos = lower,
		ytick pos = left,
		xtick align = outside,
		ytick align = outside,
		scaled x ticks = base 10:-6,
		xtick scale label code/.code={[MHz]},
		xmin = 0,
		xmax = 80000000,
%		ymin = 0,
		%ymax = 0.03,
		]
		\addplot [black, sharp plot, mark size = 3pt] table [x index = 0, y index = 1] {\Horig};
		\addplot [blue, sharp plot, mark size =3pt] table [x index = 0, y index = 1] {\HrmsOne};
		\addplot [red, sharp plot, mark size =3pt] table [x index = 0, y index = 1] {\HrmsTwo};
%		\addplot [red, sharp plot, mark size =3pt] table [x index = 0, y index = 1] {\HrmsThree};
\end{axis}
\end{tikzpicture}
\caption{Entwicklung des Betrags der Übertragungsfunktion über mehrere Iterationen und im Anfangszustand}
	\label{fig:opt.H.RMS_H}
%	\end{center}
\end{subfigure}
\\
\begin{subfigure}{\textwidth}
%\begin{center}
    \begin{tikzpicture}
\begin{axis}[
		scale only axis,
		width = 0.75 \textwidth,
		height = 0.2 \textheight,
		legend entries = {original, Step 1, Step 2, Step 3},
		legend pos = south west,
		legend columns = 3,
		xlabel={Frequenz},
		ylabel={Korrekturterm $\fabs$},
		y label style={at={(axis description cs:-0.1,.5)},anchor=south},
		%xtick distance = 10000000,
%		xminorgrids,
%%		xmajorgrids,
%		minor x tick num =3,
		xtick pos = lower,
		ytick pos = left,
		xtick align = outside,
		ytick align = outside,
		scaled x ticks = base 10:-6,
		xtick scale label code/.code={[MHz]},
		extra y ticks={0.5697153886590458, -0.5697153886590458
						},
			extra y tick labels={+RMS, -RMS},
		extra y tick style={grid=major, ytick pos=right, ytick align=outside, ticklabel pos=right},		
		xmin = 0,
		xmax = 80000000,
		ymin = -0.75,
		ymax = 0.75,
		]
		
		\addplot [black, sharp plot, mark size = 3pt] table [x index = 0, y index = 1] {\fabsOrig};
		\addplot [blue, sharp plot, mark size =3pt] table [x index = 0, y index = 1] {\fabsRMSone};
		\addplot [red, sharp plot, mark size =3pt] table [x index = 0, y index = 1] {\fabsRMStwo};
%		\addplot [blue, sharp plot, mark size =3pt] table [x index = 0, y index = 1] {\fabsRMSthree};
\end{axis}
\end{tikzpicture}
\caption{Entwicklung des Korrekturterms in angepasster Form über mehrere Iterationen und in initialer, nicht angepasster Form - RMS aus Step 1 zum Vergleich}
\label{fig:opt.H.RMS_fabs}
%\end{center}
\end{subfigure}
%\end{center}

\caption[Ignorieren großer Korrektur-Terme]{Entwicklung von Übertragungsfunktion und Korrekturterm bei Beschränkung von $\fabs$ mit angepasstem RMS-Wert und Schrittweite $\sigma_H = \frac12$}
\label{fig:opt.H.iteration}
\end{figure}



\section{Optimierung der nichtlinearen Kennlinie $K$}
\label{sec:opt.K}
Der Unterschied zur Optimierung von $\Hcompl$ ist, dass diese Optimierung im Zeitbereich statt findet. Deshalb kann \ref{eq:opt.relFehler} zu
\begin{align}
	\label{eq:opt.deltaUquest}
	\Delta \Uquest \oft = \Uquest_{, \mathrm{meas}} \oft - \Uquest_{, \mathrm{ideal}} \oft
	\qquad
	\Uquest_{, \mathrm{meas}} \oft = \mathscr{F}^{-1} \left\{ \Hcompl^{-1} \ofomega \cdot \Uoutc_{, \mathrm{meas}} \ofomega \right\}
\end{align}
geändert werden. Bei den Funktionen $\Uquest_{, \mathrm{meas}} \oft$ und $\Uquest_{, \mathrm{ideal}} \oft$ handelt es sich um Polynome gleichen Grades deshalb lässt sich die Differenz ebenfalls als ein Polynom mit Grad $N$ darstellen
\begin{align}
	\Delta \Uquest \oft = \sum_{n=1}^N \, \tilde{a}_n \, \left[ U_{in}(t) \right]^n	
\end{align}
Die Berechnung der Koeffizienten $\tilde{a}_n$ stellt ebenso ein lineares Optimierungsproblem dar wie schon die Berechnung der Koeffizienten $a_n$ in -Gleichung- \ref{eq:Uquest} siehe ---Paper Kerstin---. Dabei werden $M$ Samples von ${\Delta \Uquest_{,i} = \Delta \Uquest (i \cdot \Delta t)}$ mit zugehörigen Samples des Eingangssignals ${\Uin_{,i} = \Uin (i \cdot \Delta t)}$ verglichen. Mit der Potenzreihe aus -Gleichung- \ref{eq:opt.deltaUquest} ergibt sich folgendes Gleichungssystem
\begin{align}
	\left( 
	\begin{matrix}
	 	\Uin_{,1} & \Uin_{,1}^2 & \dots & \Uin_{,1}^N \\
		\Uin_{,2} & \Uin_{,2}^2 & \dots & \Uin_{,2}^N \\
		\vdots & \vdots & \ddots & \vdots \\
		\Uin_{,M} & \Uin_{,M}^2 & \dots & \Uin_{,M}^N \\
	\end{matrix}
	\right)
	\cdot
	\left(
	\begin{matrix}
		\tilde{a}_1 \\
		\tilde{a}_2 \\
		\vdots \\
		\tilde{a}_N \\	 
	\end{matrix}
	\right) = \left( 
	\begin{matrix}
		\Delta \Uquest_{,1} \\
		\Delta \Uquest_{,2} \\
		\vdots \\
		\Delta \Uquest_{,M} \\	 
	\end{matrix}
	\right)
	\label{eq:Uquest.Gleichungssystem}
\end{align}
Dieses Gleichungssystem ist mit normalerweise $M>N$ überbestimmt und wird mit der Methode der kleinsten Quadrate gelöst. Die Koeffizienten $\tilde{a}_n$ werden nun wie folgt zur Anpassung der Koeffizienten $a_n$ verwendet
\begin{align}
	\label{eq:opt.adjusta}
	a_n^{i+1} = a_n^{i} + \sigma_{a}^{i} \tilde{a}_n^{i}
\end{align}
Für die Schrittweite gilt $\sigma_{a}^i \in \left[ 0 , 1 \right]$.

\subsubsection*{Erste Ergebnisse}
\label{subsubsec:opt.adjusta.results}
Für die Berechnung der ersten Kennlinie $K_0$ wurde das ideal Ausgangssignal $\Uout_{, \mathrm{ideal}}$ über $\Hcompl^{-1}$ zurückgerechnet und als Eingangssignal verwendet $\Uin_{, \mathrm{ideal}} = \Uquest_{, \mathrm{ideal}}$. Dabei wurde $V_{PP} = \SI{600}{\mV}$ gesetzt, um $K_0$ in einen größeren Bereich berechnen zu können.\\
Wenn man jetzt $\Uquest_{, \mathrm{ideal}}$ mit $V_{PP} = \SI{600}{\mV}$ über $K_0$ zurückrechnet, um das erste nichtlinear vorverzerrtes Eingangsignal zu erhalten, so stellt man fest, dass die Grenzen, in denen $K_0$ invertiert werden kann, zu klein sind. Wenn also $\Uquest_{, \mathrm{ideal}}$ über die Grenzen von $K_0$ geht, so wäre ein möglicher Ansatz $V_{PP}$ auf den maximal von $K_0$ zulässigen Wert zu setzen.\\
Als andere Möglichkeit die Kennlinie anzupassen könnte man $V_{PP}$ von $\Uquest_{, \mathrm{ideal}}$ verkleinern siehe \figref{Kennlinie}
\begin{figure}[H]
\begin{subfigure}{0.5 \textwidth}
    \newlength\figureheight
	\newlength\figurewidth
	\setlength\figureheight{8cm}
	\setlength\figurewidth{8cm}
	% This file was created by matplotlib2tikz v0.6.17.
\begin{tikzpicture}

\definecolor{color0}{rgb}{0.12156862745098,0.466666666666667,0.705882352941177}
\definecolor{color1}{rgb}{1,0.498039215686275,0.0549019607843137}

\begin{axis}[
xlabel={$U_{in}$ in \si{\milli \volt}},
ylabel={$U_{?}$ in \si{\milli \volt}},
xmin=-330, xmax=330,
ymin=-703.9464559176, ymax=610.6443744396,
width=\figurewidth,
height=\figureheight,
tick align=outside,
tick pos=left,
x grid style={white!69.01960784313725!black},
y grid style={white!69.01960784313725!black},
legend cell align={left},
legend style={at={(0.03,0.97)}, anchor=north west, draw=white!80.0!black},
legend entries={{$K_0$},{$K_1$}}
]
\addlegendimage{no markers, color0}
\addlegendimage{no markers, color1}
\addplot [semithick, color0]
table {%
-300 -355.073268737
-299 -355.413336321
-298 -355.737870278
-297 -356.046924387
-296 -356.340552425
-295 -356.618808173
-294 -356.881745408
-293 -357.129417909
-292 -357.361879456
-291 -357.579183826
-290 -357.781384799
-289 -357.968536153
-288 -358.140691667
-287 -358.29790512
-286 -358.44023029
-285 -358.567720957
-284 -358.680430898
-283 -358.778413893
-282 -358.861723721
-281 -358.93041416
-280 -358.984538988
-279 -359.024151986
-278 -359.04930693
-277 -359.060057601
-276 -359.056457777
-275 -359.038561236
-274 -359.006421757
-273 -358.96009312
-272 -358.899629102
-271 -358.825083483
-270 -358.736510041
-269 -358.633962555
-268 -358.517494804
-267 -358.387160566
-266 -358.24301362
-265 -358.085107746
-264 -357.913496721
-263 -357.728234324
-262 -357.529374334
-261 -357.316970531
-260 -357.091076691
-259 -356.851746596
-258 -356.599034022
-257 -356.332992749
-256 -356.053676556
-255 -355.761139221
-254 -355.455434522
-253 -355.13661624
-252 -354.804738152
-251 -354.459854038
-250 -354.102017675
-249 -353.731282843
-248 -353.34770332
-247 -352.951332886
-246 -352.542225318
-245 -352.120434396
-244 -351.686013898
-243 -351.239017604
-242 -350.779499291
-241 -350.307512739
-240 -349.823111726
-239 -349.326350031
-238 -348.817281433
-237 -348.29595971
-236 -347.762438642
-235 -347.216772006
-234 -346.659013583
-233 -346.089217149
-232 -345.507436485
-231 -344.913725369
-230 -344.308137579
-229 -343.690726895
-228 -343.061547095
-227 -342.420651958
-226 -341.768095262
-225 -341.103930787
-224 -340.42821231
-223 -339.740993612
-222 -339.04232847
-221 -338.332270663
-220 -337.61087397
-219 -336.87819217
-218 -336.134279042
-217 -335.379188363
-216 -334.612973914
-215 -333.835689472
-214 -333.047388817
-213 -332.248125726
-212 -331.43795398
-211 -330.616927356
-210 -329.785099634
-209 -328.942524591
-208 -328.089256008
-207 -327.225347662
-206 -326.350853332
-205 -325.465826797
-204 -324.570321836
-203 -323.664392228
-202 -322.748091751
-201 -321.821474183
-200 -320.884593305
-199 -319.937502893
-198 -318.980256728
-197 -318.012908588
-196 -317.035512251
-195 -316.048121497
-194 -315.050790104
-193 -314.043571851
-192 -313.026520516
-191 -311.999689879
-190 -310.963133717
-189 -309.916905811
-188 -308.861059938
-187 -307.795649877
-186 -306.720729407
-185 -305.636352307
-184 -304.542572355
-183 -303.439443331
-182 -302.327019012
-181 -301.205353178
-180 -300.074499607
-179 -298.934512079
-178 -297.785444371
-177 -296.627350263
-176 -295.460283534
-175 -294.284297961
-174 -293.099447324
-173 -291.905785402
-172 -290.703365973
-171 -289.492242816
-170 -288.27246971
-169 -287.044100433
-168 -285.807188764
-167 -284.561788482
-166 -283.307953366
-165 -282.045737195
-164 -280.775193746
-163 -279.4963768
-162 -278.209340134
-161 -276.914137527
-160 -275.610822758
-159 -274.299449607
-158 -272.980071851
-157 -271.652743269
-156 -270.31751764
-155 -268.974448743
-154 -267.623590356
-153 -266.264996259
-152 -264.898720229
-151 -263.524816046
-150 -262.143337489
-149 -260.754338336
-148 -259.357872366
-147 -257.953993357
-146 -256.542755089
-145 -255.12421134
-144 -253.698415889
-143 -252.265422514
-142 -250.825284994
-141 -249.378057109
-140 -247.923792637
-139 -246.462545356
-138 -244.994369045
-137 -243.519317483
-136 -242.037444449
-135 -240.548803721
-134 -239.053449079
-133 -237.5514343
-132 -236.042813164
-131 -234.52763945
-130 -233.005966936
-129 -231.4778494
-128 -229.943340622
-127 -228.402494381
-126 -226.855364454
-125 -225.302004622
-124 -223.742468662
-123 -222.176810353
-122 -220.605083474
-121 -219.027341804
-120 -217.443639121
-119 -215.854029205
-118 -214.258565833
-117 -212.657302786
-116 -211.05029384
-115 -209.437592776
-114 -207.819253372
-113 -206.195329406
-112 -204.565874658
-111 -202.930942906
-110 -201.290587928
-109 -199.644863504
-108 -197.993823413
-107 -196.337521432
-106 -194.676011342
-105 -193.009346919
-104 -191.337581944
-103 -189.660770195
-102 -187.978965451
-101 -186.29222149
-100 -184.600592091
-99 -182.904131034
-98 -181.202892096
-97 -179.496929056
-96 -177.786295693
-95 -176.071045786
-94 -174.351233114
-93 -172.626911455
-92 -170.898134588
-91 -169.164956292
-90 -167.427430346
-89 -165.685610528
-88 -163.939550616
-87 -162.189304391
-86 -160.43492563
-85 -158.676468112
-84 -156.913985616
-83 -155.147531921
-82 -153.377160805
-81 -151.602926047
-80 -149.824881427
-79 -148.043080721
-78 -146.25757771
-77 -144.468426173
-76 -142.675679887
-75 -140.879392631
-74 -139.079618185
-73 -137.276410327
-72 -135.469822836
-71 -133.65990949
-70 -131.846724068
-69 -130.030320349
-68 -128.210752112
-67 -126.388073136
-66 -124.562337199
-65 -122.733598079
-64 -120.901909556
-63 -119.067325409
-62 -117.229899415
-61 -115.389685355
-60 -113.546737006
-59 -111.701108147
-58 -109.852852557
-57 -108.002024016
-56 -106.1486763
-55 -104.29286319
-54 -102.434638464
-53 -100.574055901
-52 -98.7111692787
-51 -96.8460323769
-50 -94.9786989741
-49 -93.1092228489
-48 -91.2376577802
-47 -89.3640575465
-46 -87.4884759267
-45 -85.6109666995
-44 -83.7315836437
-43 -81.8503805379
-42 -79.9674111609
-41 -78.0827292914
-40 -76.1963887082
-39 -74.30844319
-38 -72.4189465156
-37 -70.5279524636
-36 -68.6355148129
-35 -66.7416873421
-34 -64.84652383
-33 -62.9500780553
-32 -61.0524037967
-31 -59.153554833
-30 -57.253584943
-29 -55.3525479053
-28 -53.4504974987
-27 -51.5474875019
-26 -49.6435716937
-25 -47.7388038527
-24 -45.8332377578
-23 -43.9269271877
-22 -42.0199259211
-21 -40.1122877367
-20 -38.2040664132
-19 -36.2953157295
-18 -34.3860894642
-17 -32.4764413961
-16 -30.5664253039
-15 -28.6560949664
-14 -26.7455041622
-13 -24.8347066701
-12 -22.9237562689
-11 -21.0127067373
-10 -19.101611854
-9 -17.1905253977
-8 -15.2795011472
-7 -13.3685928813
-6 -11.4578543786
-5 -9.54733941788
-4 -7.63710177791
-3 -5.7271952374
-2 -3.81767357509
-1 -1.90859056971
0 0
1 1.90804435531
2 3.81548871748
3 5.72227930778
4 7.62836234747
5 9.53368405782
6 11.4381906601
7 13.3418283756
8 15.2445434255
9 17.1462820311
10 19.0469904137
11 20.9466147946
12 22.845101395
13 24.7423964361
14 26.6384461393
15 28.5331967258
16 30.4265944169
17 32.3185854338
18 34.2091159979
19 36.0981323303
20 37.9855806523
21 39.8714071852
22 41.7555581503
23 43.6379797688
24 45.5186182621
25 47.3974198512
26 49.2743307576
27 51.1492972025
28 53.0222654072
29 54.8931815929
30 56.7619919808
31 58.6286427923
32 60.4930802486
33 62.3552505711
34 64.2150999808
35 66.0725746992
36 67.9276209474
37 69.7801849468
38 71.6302129186
39 73.477651084
40 75.3224456644
41 77.164542881
42 79.0038889551
43 80.8404301078
44 82.6741125606
45 84.5048825347
46 86.3326862513
47 88.1574699316
48 89.9791797971
49 91.7977620688
50 93.6131629681
51 95.4253287163
52 97.2342055346
53 99.0397396443
54 100.841877267
55 102.640564623
56 104.435747934
57 106.227373422
58 108.015387308
59 109.799735812
60 111.580365157
61 113.357221563
62 115.130251252
63 116.899400446
64 118.664615364
65 120.425842229
66 122.183027262
67 123.936116684
68 125.685056716
69 127.42979358
70 129.170273496
71 130.906442687
72 132.638247374
73 134.365633777
74 136.088548118
75 137.806936618
76 139.520745499
77 141.229920981
78 142.934409286
79 144.634156636
80 146.329109251
81 148.019213353
82 149.704415163
83 151.384660903
84 153.059896793
85 154.730069055
86 156.39512391
87 158.055007579
88 159.709666284
89 161.359046246
90 163.003093687
91 164.641754826
92 166.274975887
93 167.902703089
94 169.524882655
95 171.141460805
96 172.752383761
97 174.357597744
98 175.957048975
99 177.550683676
100 179.138448068
101 180.720288371
102 182.296150809
103 183.8659816
104 185.429726968
105 186.987333133
106 188.538746316
107 190.083912739
108 191.622778623
109 193.15529019
110 194.681393659
111 196.201035254
112 197.714161194
113 199.220717702
114 200.720650998
115 202.213907304
116 203.700432842
117 205.180173831
118 206.653076494
119 208.119087053
120 209.578151727
121 211.030216738
122 212.475228309
123 213.913132659
124 215.34387601
125 216.767404584
126 218.183664602
127 219.592602285
128 220.994163854
129 222.38829553
130 223.774943535
131 225.154054091
132 226.525573417
133 227.889447736
134 229.245623269
135 230.594046238
136 231.934662862
137 233.267419364
138 234.592261966
139 235.909136887
140 237.21799035
141 238.518768575
142 239.811417785
143 241.095884199
144 242.372114041
145 243.64005353
146 244.899648888
147 246.150846336
148 247.393592096
149 248.627832389
150 249.853513435
151 251.070581458
152 252.278982676
153 253.478663313
154 254.669569589
155 255.851647725
156 257.024843943
157 258.189104464
158 259.344375509
159 260.4906033
160 261.627734057
161 262.755714003
162 263.874489357
163 264.984006342
164 266.08421118
165 267.17505009
166 268.256469294
167 269.328415014
168 270.390833471
169 271.443670886
170 272.48687348
171 273.520387476
172 274.544159093
173 275.558134553
174 276.562260077
175 277.556481888
176 278.540746205
177 279.514999251
178 280.479187246
179 281.433256412
180 282.37715297
181 283.310823141
182 284.234213147
183 285.147269209
184 286.049937548
185 286.942164385
186 287.823895942
187 288.69507844
188 289.5556581
189 290.405581143
190 291.244793791
191 292.073242265
192 292.890872787
193 293.697631576
194 294.493464856
195 295.278318846
196 296.052139769
197 296.814873846
198 297.566467297
199 298.306866344
200 299.036017209
201 299.753866112
202 300.460359275
203 301.15544292
204 301.839063266
205 302.511166537
206 303.171698952
207 303.820606734
208 304.457836103
209 305.083333281
210 305.697044488
211 306.298915947
212 306.888893879
213 307.466924504
214 308.032954045
215 308.586928722
216 309.128794756
217 309.658498369
218 310.175985783
219 310.681203217
220 311.174096895
221 311.654613036
222 312.122697862
223 312.578297595
224 313.021358456
225 313.451826666
226 313.869648446
227 314.274770017
228 314.667137601
229 315.046697419
230 315.413395693
231 315.767178643
232 316.107992491
233 316.435783458
234 316.750497765
235 317.052081634
236 317.340481286
237 317.615642942
238 317.877512824
239 318.126037152
240 318.361162148
241 318.582834033
242 318.790999029
243 318.985603357
244 319.166593237
245 319.333914892
246 319.487514543
247 319.62733841
248 319.753332715
249 319.86544368
250 319.963617525
251 320.047800473
252 320.117938743
253 320.173978558
254 320.215866138
255 320.243547705
256 320.256969481
257 320.256077686
258 320.240818541
259 320.211138269
260 320.16698309
261 320.108299225
262 320.035032897
263 319.947130325
264 319.844537731
265 319.727201338
266 319.595067365
267 319.448082034
268 319.286191566
269 319.109342184
270 318.917480107
271 318.710551557
272 318.488502756
273 318.251279924
274 317.998829283
275 317.731097055
276 317.44802946
277 317.14957272
278 316.835673056
279 316.506276689
280 316.161329841
281 315.800778733
282 315.424569585
283 315.03264862
284 314.624962059
285 314.201456122
286 313.762077032
287 313.306771009
288 312.835484275
289 312.348163051
290 311.844753558
291 311.325202017
292 310.78945465
293 310.237457678
294 309.669157323
295 309.084499805
296 308.483431346
297 307.865898166
298 307.231846489
299 306.581222533
300 305.913972522
};
\addplot [semithick, color1]
table {%
-300 -644.192327265
-299 -641.714228035
-298 -639.238926493
-297 -636.76641675
-296 -634.296692915
-295 -631.8297491
-294 -629.365579415
-293 -626.90417797
-292 -624.445538876
-291 -621.989656242
-290 -619.53652418
-289 -617.086136799
-288 -614.63848821
-287 -612.193572524
-286 -609.751383851
-285 -607.3119163
-284 -604.875163983
-283 -602.44112101
-282 -600.009781491
-281 -597.581139537
-280 -595.155189258
-279 -592.731924764
-278 -590.311340166
-277 -587.893429574
-276 -585.478187099
-275 -583.06560685
-274 -580.655682938
-273 -578.248409475
-272 -575.843780569
-271 -573.441790331
-270 -571.042432872
-269 -568.645702303
-268 -566.251592732
-267 -563.860098272
-266 -561.471213032
-265 -559.084931122
-264 -556.701246654
-263 -554.320153736
-262 -551.941646481
-261 -549.565718997
-260 -547.192365396
-259 -544.821579788
-258 -542.453356283
-257 -540.087688991
-256 -537.724572024
-255 -535.36399949
-254 -533.005965502
-253 -530.650464168
-252 -528.2974896
-251 -525.947035908
-250 -523.599097202
-249 -521.253667592
-248 -518.91074119
-247 -516.570312104
-246 -514.232374447
-245 -511.896922327
-244 -509.563949856
-243 -507.233451144
-242 -504.9054203
-241 -502.579851437
-240 -500.256738663
-239 -497.936076089
-238 -495.617857827
-237 -493.302077985
-236 -490.988730674
-235 -488.677810006
-234 -486.369310089
-233 -484.063225035
-232 -481.759548954
-231 -479.458275956
-230 -477.159400152
-229 -474.862915652
-228 -472.568816566
-227 -470.277097005
-226 -467.987751079
-225 -465.700772899
-224 -463.416156574
-223 -461.133896216
-222 -458.853985934
-221 -456.57641984
-220 -454.301192042
-219 -452.028296653
-218 -449.757727781
-217 -447.489479538
-216 -445.223546034
-215 -442.959921379
-214 -440.698599684
-213 -438.439575059
-212 -436.182841614
-211 -433.92839346
-210 -431.676224707
-209 -429.426329465
-208 -427.178701845
-207 -424.933335958
-206 -422.690225913
-205 -420.449365821
-204 -418.210749792
-203 -415.974371937
-202 -413.740226366
-201 -411.50830719
-200 -409.278608518
-199 -407.051124462
-198 -404.825849131
-197 -402.602776636
-196 -400.381901087
-195 -398.163216595
-194 -395.946717271
-193 -393.732397223
-192 -391.520250564
-191 -389.310271402
-190 -387.102453849
-189 -384.896792015
-188 -382.693280011
-187 -380.491911946
-186 -378.292681931
-185 -376.095584077
-184 -373.900612493
-183 -371.707761291
-182 -369.51702458
-181 -367.328396471
-180 -365.141871074
-179 -362.9574425
-178 -360.775104859
-177 -358.594852261
-176 -356.416678817
-175 -354.240578637
-174 -352.066545832
-173 -349.894574512
-172 -347.724658787
-171 -345.556792767
-170 -343.390970564
-169 -341.227186287
-168 -339.065434047
-167 -336.905707954
-166 -334.748002118
-165 -332.592310651
-164 -330.438627661
-163 -328.28694726
-162 -326.137263559
-161 -323.989570666
-160 -321.843862693
-159 -319.700133751
-158 -317.558377949
-157 -315.418589398
-156 -313.280762208
-155 -311.144890489
-154 -309.010968353
-153 -306.878989909
-152 -304.748949268
-151 -302.620840539
-150 -300.494657835
-149 -298.370395264
-148 -296.248046937
-147 -294.127606965
-146 -292.009069458
-145 -289.892428526
-144 -287.77767828
-143 -285.66481283
-142 -283.553826287
-141 -281.44471276
-140 -279.337466361
-139 -277.232081199
-138 -275.128551385
-137 -273.026871029
-136 -270.927034242
-135 -268.829035135
-134 -266.732867816
-133 -264.638526398
-132 -262.546004989
-131 -260.455297701
-130 -258.366398645
-129 -256.279301929
-128 -254.194001665
-127 -252.110491963
-126 -250.028766934
-125 -247.948820687
-124 -245.870647334
-123 -243.794240984
-122 -241.719595748
-121 -239.646705736
-120 -237.575565059
-119 -235.506167827
-118 -233.438508151
-117 -231.37258014
-116 -229.308377905
-115 -227.245895557
-114 -225.185127206
-113 -223.126066962
-112 -221.068708936
-111 -219.013047238
-110 -216.959075978
-109 -214.906789267
-108 -212.856181215
-107 -210.807245932
-106 -208.75997753
-105 -206.714370117
-104 -204.670417806
-103 -202.628114705
-102 -200.587454926
-101 -198.548432578
-100 -196.511041773
-99 -194.47527662
-98 -192.44113123
-97 -190.408599713
-96 -188.37767618
-95 -186.348354741
-94 -184.320629506
-93 -182.294494586
-92 -180.269944091
-91 -178.246972132
-90 -176.225572818
-89 -174.205740261
-88 -172.18746857
-87 -170.170751856
-86 -168.155584229
-85 -166.141959801
-84 -164.12987268
-83 -162.119316978
-82 -160.110286804
-81 -158.10277627
-80 -156.096779486
-79 -154.092290561
-78 -152.089303607
-77 -150.087812733
-76 -148.087812051
-75 -146.089295669
-74 -144.0922577
-73 -142.096692253
-72 -140.102593439
-71 -138.109955367
-70 -136.118772149
-69 -134.129037894
-68 -132.140746714
-67 -130.153892718
-66 -128.168470017
-65 -126.18447272
-64 -124.20189494
-63 -122.220730785
-62 -120.240974367
-61 -118.262619795
-60 -116.285661181
-59 -114.310092634
-58 -112.335908264
-57 -110.363102183
-56 -108.3916685
-55 -106.421601327
-54 -104.452894772
-53 -102.485542947
-52 -100.519539962
-51 -98.554879928
-50 -96.5915569544
-49 -94.6295651519
-48 -92.6688986309
-47 -90.7095515018
-46 -88.751517875
-45 -86.7947918608
-44 -84.8393675696
-43 -82.8852391118
-42 -80.9324005978
-41 -78.9808461379
-40 -77.0305698425
-39 -75.0815658219
-38 -73.1338281867
-37 -71.187351047
-36 -69.2421285133
-35 -67.298154696
-34 -65.3554237055
-33 -63.413929652
-32 -61.4736666461
-31 -59.534628798
-30 -57.5968102181
-29 -55.6602050169
-28 -53.7248073046
-27 -51.7906111916
-26 -49.8576107884
-25 -47.9258002053
-24 -45.9951735527
-23 -44.0657249409
-22 -42.1374484803
-21 -40.2103382813
-20 -38.2843884543
-19 -36.3595931096
-18 -34.4359463576
-17 -32.5134423087
-16 -30.5920750732
-15 -28.6718387616
-14 -26.7527274842
-13 -24.8347353513
-12 -22.9178564734
-11 -21.0020849608
-10 -19.0874149239
-9 -17.1738404731
-8 -15.2613557187
-7 -13.3499547711
-6 -11.4396317407
-5 -9.53038073784
-4 -7.62219587292
-3 -5.71507125631
-2 -3.80900099836
-1 -1.90397920947
0 0
1 1.90294251968
2 3.80485423919
3 5.70574104816
4 7.60560883621
5 9.50446349298
6 11.4023109081
7 13.2991569712
8 15.1950075719
9 17.0898685997
10 18.9837459445
11 20.8766454957
12 22.768573143
13 24.6595347761
14 26.5495362845
15 28.4385835579
16 30.3266824859
17 32.2138389582
18 34.1000588643
19 35.9853480939
20 37.8697125366
21 39.7531580821
22 41.6356906199
23 43.5173160398
24 45.3980402312
25 47.2778690839
26 49.1568084876
27 51.0348643317
28 52.9120425059
29 54.7883488999
30 56.6637894033
31 58.5383699058
32 60.4120962968
33 62.2849744662
34 64.1570103034
35 66.0282096982
36 67.8985785401
37 69.7681227187
38 71.6368481238
39 73.504760645
40 75.3718661718
41 77.2381705938
42 79.1036798008
43 80.9683996823
44 82.832336128
45 84.6954950275
46 86.5578822705
47 88.4195037464
48 90.2803653451
49 92.140472956
50 93.9998324689
51 95.8584497733
52 97.7163307589
53 99.5734813153
54 101.429907332
55 103.285614699
56 105.140609306
57 106.994897042
58 108.848483797
59 110.70137546
60 112.553577922
61 114.405097071
62 116.255938798
63 118.106108992
64 119.955613543
65 121.80445834
66 123.652649273
67 125.500192232
68 127.347093105
69 129.193357784
70 131.038992157
71 132.884002115
72 134.728393546
73 136.57217234
74 138.415344387
75 140.257915577
76 142.099891799
77 143.941278943
78 145.782082899
79 147.622309555
80 149.461964803
81 151.30105453
82 153.139584628
83 154.977560985
84 156.814989492
85 158.651876038
86 160.488226511
87 162.324046804
88 164.159342803
89 165.994120401
90 167.828385485
91 169.662143946
92 171.495401673
93 173.328164556
94 175.160438485
95 176.992229348
96 178.823543037
97 180.65438544
98 182.484762447
99 184.314679947
100 186.144143831
101 187.973159988
102 189.801734307
103 191.629872679
104 193.457580992
105 195.284865136
106 197.111731002
107 198.938184479
108 200.764231455
109 202.589877822
110 204.415129468
111 206.239992283
112 208.064472157
113 209.88857498
114 211.712306641
115 213.535673029
116 215.358680035
117 217.181333547
118 219.003639456
119 220.825603652
120 222.647232023
121 224.468530459
122 226.289504851
123 228.110161087
124 229.930505058
125 231.750542653
126 233.570279761
127 235.389722273
128 237.208876077
129 239.027747064
130 240.846341123
131 242.664664143
132 244.482722015
133 246.300520628
134 248.118065872
135 249.935363635
136 251.752419809
137 253.569240282
138 255.385830944
139 257.202197685
140 259.018346394
141 260.834282962
142 262.650013277
143 264.465543229
144 266.280878708
145 268.096025603
146 269.910989805
147 271.725777202
148 273.540393685
149 275.354845143
150 277.169137465
151 278.983276542
152 280.797268262
153 282.611118516
154 284.424833194
155 286.238418184
156 288.051879376
157 289.86522266
158 291.678453926
159 293.491579064
160 295.304603962
161 297.117534511
162 298.9303766
163 300.743136118
164 302.555818956
165 304.368431004
166 306.180978149
167 307.993466283
168 309.805901295
169 311.618289075
170 313.430635512
171 315.242946495
172 317.055227915
173 318.867485661
174 320.679725623
175 322.49195369
176 324.304175752
177 326.116397699
178 327.928625419
179 329.740864804
180 331.553121742
181 333.365402123
182 335.177711837
183 336.990056773
184 338.802442821
185 340.61487587
186 342.427361811
187 344.239906533
188 346.052515925
189 347.865195877
190 349.677952279
191 351.49079102
192 353.30371799
193 355.116739079
194 356.929860176
195 358.743087171
196 360.556425953
197 362.369882413
198 364.183462439
199 365.997171922
200 367.81101675
201 369.625002814
202 371.439136004
203 373.253422208
204 375.067867317
205 376.882477219
206 378.697257806
207 380.512214966
208 382.327354589
209 384.142682565
210 385.958204782
211 387.773927132
212 389.589855503
213 391.405995786
214 393.222353869
215 395.038935643
216 396.855746996
217 398.672793819
218 400.490082002
219 402.307617433
220 404.125406003
221 405.943453601
222 407.761766117
223 409.58034944
224 411.39920946
225 413.218352067
226 415.037783151
227 416.8575086
228 418.677534304
229 420.497866154
230 422.318510039
231 424.139471848
232 425.960757471
233 427.782372798
234 429.604323718
235 431.426616121
236 433.249255897
237 435.072248934
238 436.895601124
239 438.719318355
240 440.543406517
241 442.3678715
242 444.192719193
243 446.017955486
244 447.843586269
245 449.66961743
246 451.496054861
247 453.32290445
248 455.150172087
249 456.977863662
250 458.805985064
251 460.634542184
252 462.463540909
253 464.292987131
254 466.122886739
255 467.953245623
256 469.784069671
257 471.615364774
258 473.447136822
259 475.279391703
260 477.112135308
261 478.945373527
262 480.779112248
263 482.613357361
264 484.448114757
265 486.283390325
266 488.119189954
267 489.955519533
268 491.792384954
269 493.629792105
270 495.467746875
271 497.306255155
272 499.145322835
273 500.984955803
274 502.825159949
275 504.665941164
276 506.507305336
277 508.349258355
278 510.191806111
279 512.034954494
280 513.878709393
281 515.723076697
282 517.568062298
283 519.413672083
284 521.259911942
285 523.106787766
286 524.954305444
287 526.802470866
288 528.65128992
289 530.500768498
290 532.350912488
291 534.20172778
292 536.053220263
293 537.905395828
294 539.758260364
295 541.61181976
296 543.466079907
297 545.321046693
298 547.176726009
299 549.033123744
300 550.890245787
};
\end{axis}

\end{tikzpicture}	
\subcaption{Kennlinie}
	\label{fig:K}
\end{subfigure}
\begin{subfigure}{0.5 \textwidth}
	\setlength\figureheight{8cm}
	\setlength\figurewidth{8cm}
    %% This file was created by matplotlib2tikz v0.6.17.
\begin{tikzpicture}

\definecolor{color0}{rgb}{0.12156862745098,0.466666666666667,0.705882352941177}
\definecolor{color1}{rgb}{1,0.498039215686275,0.0549019607843137}

\begin{axis}[
xlabel={$t$ in \si{\micro \second}},
ylabel={$U_{in}$ in \si{\milli \volt}},
xmin=-0.055552, xmax=1.166592,
ymin=-80.519635, ymax=95.236935,
width=\figurewidth,
height=\figureheight,
tick align=outside,
tick pos=left,
x grid style={white!69.01960784313725!black},
y grid style={white!69.01960784313725!black},
legend cell align={left},
legend entries={{$U_{in,1}$},{$U_{in,2}$}},
legend style={draw=white!80.0!black}
]
\addlegendimage{no markers, color0}
\addlegendimage{no markers, color1}
\addplot [semithick, color0]
table {%
0 -1.6708
0.00016 -2.2028
0.00032 -2.7265
0.00048 -3.1255
0.00064 -3.2585
0.0008 -3.1006
0.00096 -2.6725
0.00112 -2.0283
0.00128 -1.2635
0.00144 -0.4779688
0.0016 0.2078125
0.00176 0.7273438
0.00192 1.0183
0.00208 1.0806
0.00224 0.914375
0.0024 0.6192813
0.00256 0.3075625
0.00272 0.0748125
0.00288 -0.0041563
0.00304 0.0581875
0.0032 0.2701562
0.00336 0.56525
0.00352 0.847875
0.00368 1.0391
0.00384 1.0432
0.004 0.8852813
0.00416 0.615125
0.00432 0.3117188
0.00448 0.049875
0.00464 -0.1454688
0.0048 -0.23275
0.00496 -0.2036563
0.00512 -0.1122188
0.00528 -0.016625
0.00544 0.0041563
0.0056 -0.0789688
0.00576 -0.2784688
0.00592 -0.581875
0.00608 -0.9600938
0.00624 -1.3633
0.0064 -1.7332
0.00656 -2.0116
0.00672 -2.1529
0.00688 -2.1488
0.00704 -2.0158
0.0072 -1.8412
0.00736 -1.7332
0.00752 -1.7747
0.00768 -2.0283
0.00784 -2.4813
0.008 -3.059
0.00816 -3.645
0.00832 -4.0856
0.00848 -4.2602
0.00864 -4.0856
0.0088 -3.6409
0.00896 -3.0715
0.00912 -2.5312
0.00928 -2.1737
0.00944 -2.0781
0.0096 -2.2485
0.00976 -2.581
0.00992 -2.9343
0.01008 -3.1629
0.01024 -3.1671
0.0104 -2.9634
0.01056 -2.6434
0.01072 -2.3275
0.01088 -2.1405
0.01104 -2.1571
0.0112 -2.3649
0.01136 -2.6808
0.01152 -2.9883
0.01168 -3.1712
0.01184 -3.1463
0.012 -2.9302
0.01216 -2.6143
0.01232 -2.315
0.01248 -2.1405
0.01264 -2.1612
0.0128 -2.3566
0.01296 -2.6558
0.01312 -2.9593
0.01328 -3.1629
0.01344 -3.1795
0.0136 -3.0133
0.01376 -2.7307
0.01392 -2.4231
0.01408 -2.1696
0.01424 -2.0324
0.0144 -2.0075
0.01456 -2.0615
0.01472 -2.128
0.01488 -2.1363
0.01504 -2.0324
0.0152 -1.8163
0.01536 -1.5461
0.01552 -1.2843
0.01568 -1.0931
0.01584 -1.01
0.016 -1.0224
0.01616 -1.0765
0.01632 -1.118
0.01648 -1.0848
0.01664 -0.947625
0.0168 -0.714875
0.01696 -0.4364063
0.01712 -0.1870313
0.01728 -0.0207812
0.01744 0.0207812
0.0176 -0.0124688
0.01776 -0.0789688
0.01792 -0.1039063
0.01808 -0.03325
0.01824 0.1288437
0.0184 0.3865313
0.01856 0.6733125
0.01872 0.9185313
0.01888 1.0515
0.01904 1.0017
0.0192 0.8063125
0.01936 0.5236875
0.01952 0.2369063
0.01968 0.0290938
0.01984 -0.0581875
0.02 0.0041563
0.02016 0.2244375
0.02032 0.5694063
0.02048 0.96425
0.02064 1.2968
0.0208 1.5129
0.02096 1.5752
0.02112 1.4547
0.02128 1.1596
0.02144 0.7024063
0.0216 0.1745625
0.02176 -0.3325
0.02192 -0.7522813
0.02208 -1.0224
0.02224 -1.1222
0.0224 -1.0889
0.02256 -1.0058
0.02272 -0.9559375
0.02288 -1.0224
0.02304 -1.2469
0.0232 -1.6293
0.02336 -2.1197
0.02352 -2.6392
0.02368 -3.0964
0.02384 -3.3873
0.024 -3.512
0.02416 -3.4954
0.02432 -3.3873
0.02448 -3.2294
0.02464 -3.0341
0.0248 -2.847
0.02496 -2.66
0.02512 -2.4522
0.02528 -2.1987
0.02544 -1.8745
0.0256 -1.542
0.02576 -1.2593
0.02592 -1.0806
0.02608 -1.0474
0.02624 -1.1679
0.0264 -1.4048
0.02656 -1.6958
0.02672 -1.9534
0.02688 -2.1072
0.02704 -2.1072
0.0272 -1.9493
0.02736 -1.6916
0.02752 -1.3923
0.02768 -1.118
0.02784 -0.9393125
0.028 -0.8645
0.02816 -0.8769688
0.02832 -0.947625
0.02848 -1.0391
0.02864 -1.1263
0.0288 -1.1679
0.02896 -1.1596
0.02912 -1.1222
0.02928 -1.0723
0.02944 -1.0391
0.0296 -1.0224
0.02976 -1.0224
0.02992 -1.0391
0.03008 -1.0557
0.03024 -1.0723
0.0304 -1.0806
0.03056 -1.0806
0.03072 -1.0723
0.03088 -1.064
0.03104 -1.0557
0.0312 -1.0474
0.03136 -1.0432
0.03152 -1.0474
0.03168 -1.0598
0.03184 -1.0682
0.032 -1.0806
0.03216 -1.0848
0.03232 -1.0806
0.03248 -1.0682
0.03264 -1.0515
0.0328 -1.0391
0.03296 -1.0349
0.03312 -1.0391
0.03328 -1.0557
0.03344 -1.0889
0.0336 -1.1263
0.03376 -1.1471
0.03392 -1.1388
0.03408 -1.0848
0.03424 -0.9767187
0.0344 -0.814625
0.03456 -0.5985
0.03472 -0.3408125
0.03488 -0.0665
0.03504 0.1620938
0.0352 0.3117188
0.03536 0.3615938
0.03552 0.2867813
0.03568 0.0706562
0.03584 -0.2743125
0.036 -0.7107188
0.03616 -1.1887
0.03632 -1.6459
0.03648 -2.0407
0.03664 -2.3275
0.0368 -2.4813
0.03696 -2.4938
0.03712 -2.3815
0.03728 -2.182
0.03744 -1.9451
0.0376 -1.7498
0.03776 -1.675
0.03792 -1.7664
0.03808 -2.0324
0.03824 -2.4273
0.0384 -2.8553
0.03856 -3.2045
0.03872 -3.3624
0.03888 -3.2585
0.03904 -2.8637
0.0392 -2.2901
0.03936 -1.6916
0.03952 -1.2386
0.03968 -1.0557
0.03984 -1.2012
0.04 -1.6293
0.04016 -2.2111
0.04032 -2.7722
0.04048 -3.1421
0.04064 -3.1878
0.0408 -2.9052
0.04096 -2.3857
0.04112 -1.7581
0.04128 -1.1845
0.04144 -0.781375
0.0416 -0.6068125
0.04176 -0.6400625
0.04192 -0.8063125
0.04208 -1.0141
0.04224 -1.1887
0.0424 -1.2718
0.04256 -1.2593
0.04272 -1.1845
0.04288 -1.0848
0.04304 -1.0183
0.0432 -0.9933438
0.04336 -1.0017
0.04352 -1.0307
0.04368 -1.0557
0.04384 -1.064
0.044 -1.0515
0.04416 -1.0349
0.04432 -1.0307
0.04448 -1.0515
0.04464 -1.1056
0.0448 -1.1638
0.04496 -1.1928
0.04512 -1.1721
0.04528 -1.0889
0.04544 -0.9684063
0.0456 -0.8437188
0.04576 -0.7772188
0.04592 -0.8187813
0.04608 -0.9975
0.04624 -1.2926
0.0464 -1.6459
0.04656 -1.9659
0.04672 -2.1654
0.04688 -2.1612
0.04704 -1.916
0.0472 -1.4796
0.04736 -0.947625
0.04752 -0.4405625
0.04768 -0.0623438
0.04784 0.0914375
0.048 0.0124688
0.04816 -0.2452187
0.04832 -0.6068125
0.04848 -0.9767187
0.04864 -1.2635
0.0488 -1.409
0.04896 -1.4131
0.04912 -1.3051
0.04928 -1.1139
0.04944 -0.8894375
0.0496 -0.6566875
0.04976 -0.4364063
0.04992 -0.2285938
0.05008 -0.0415625
0.05024 0.1122188
0.0504 0.2202813
0.05056 0.2576875
0.05072 0.2078125
0.05088 0.049875
0.05104 -0.1911875
0.0512 -0.4779688
0.05136 -0.748125
0.05152 -0.9559375
0.05168 -1.0557
0.05184 -1.0183
0.052 -0.8561875
0.05216 -0.6068125
0.05232 -0.3241875
0.05248 -0.0581875
0.05264 0.1371563
0.0528 0.2452187
0.05296 0.249375
0.05312 0.1704063
0.05328 0.0374063
0.05344 -0.1205313
0.0536 -0.249375
0.05376 -0.2950937
0.05392 -0.2369063
0.05408 -0.0581875
0.05424 0.2078125
0.0544 0.5195313
0.05456 0.8021563
0.05472 1.0017
0.05488 1.0682
0.05504 0.9684063
0.0552 0.7439688
0.05536 0.4613438
0.05552 0.2036563
0.05568 0.0249375
0.05584 -0.0540313
0.056 -0.0374063
0.05616 0.0207812
0.05632 0.0623438
0.05648 0.0249375
0.05664 -0.133
0.0568 -0.3906875
0.05696 -0.6857813
0.05712 -0.931
0.05728 -1.0557
0.05744 -1.0224
0.0576 -0.8395625
0.05776 -0.5610938
0.05792 -0.266
0.05808 -0.0374063
0.05824 0.0457188
0.0584 -0.03325
0.05856 -0.2618438
0.05872 -0.5943438
0.05888 -0.9684063
0.05904 -1.3258
0.0592 -1.6251
0.05936 -1.8578
0.05952 -2.0158
0.05968 -2.1114
0.05984 -2.1529
0.06 -2.1654
0.06016 -2.1654
0.06032 -2.1529
0.06048 -2.1322
0.06064 -2.1031
0.0608 -2.0864
0.06096 -2.0823
0.06112 -2.0948
0.06128 -2.1197
0.06144 -2.1363
0.0616 -2.1529
0.06176 -2.1571
0.06192 -2.1529
0.06208 -2.1322
0.06224 -2.0948
0.0624 -2.0573
0.06256 -2.0407
0.06272 -2.0532
0.06288 -2.1072
0.06304 -2.1903
0.0632 -2.3275
0.06336 -2.527
0.06352 -2.7972
0.06368 -3.1089
0.06384 -3.3915
0.064 -3.591
0.06416 -3.6658
0.06432 -3.5702
0.06448 -3.2876
0.06464 -2.8221
0.0648 -2.261
0.06496 -1.7207
0.06512 -1.3009
0.06528 -1.0806
0.06544 -1.0848
0.0656 -1.2843
0.06576 -1.5918
0.06592 -1.9036
0.06608 -2.1031
0.06624 -2.0948
0.0664 -1.8537
0.06656 -1.4007
0.06672 -0.8063125
0.06688 -0.1579375
0.06704 0.4364063
0.0672 0.89775
0.06736 1.1762
0.06752 1.2427
0.06768 1.118
0.06784 0.8270938
0.068 0.4779688
0.06816 0.1704063
0.06832 -0.0083125
0.06848 -0.0207812
0.06864 0.116375
0.0688 0.382375
0.06896 0.6899375
0.06912 0.947625
0.06928 1.0598
0.06944 0.9767187
0.0696 0.7315
0.06976 0.4197813
0.06992 0.1454688
0.07008 0.0041563
0.07024 0.0457188
0.0704 0.2535313
0.07056 0.56525
0.07072 0.8603438
0.07088 1.0432
0.07104 1.0391
0.0712 0.8645
0.07136 0.5735625
0.07152 0.2701562
0.07168 0.0374063
0.07184 -0.0665
0.072 -0.0623438
0.07216 0.0041563
0.07232 0.0581875
0.07248 0.0249375
0.07264 -0.1288437
0.0728 -0.3865313
0.07296 -0.6899375
0.07312 -0.9434688
0.07328 -1.064
0.07344 -0.9767187
0.0736 -0.6857813
0.07376 -0.2244375
0.07392 0.3408125
0.07408 0.9226875
0.07424 1.3882
0.0744 1.6625
0.07456 1.7165
0.07472 1.5461
0.07488 1.1804
0.07504 0.648375
0.0752 0.0665
0.07536 -0.4655
0.07552 -0.8561875
0.07568 -1.0474
0.07584 -1.0224
0.076 -0.8229375
0.07616 -0.5195313
0.07632 -0.2202813
0.07648 -0.0207812
0.07664 -0.0249375
0.0768 -0.2535313
0.07696 -0.6940938
0.07712 -1.2843
0.07728 -1.9534
0.07744 -2.6143
0.0776 -3.192
0.07776 -3.6617
0.07792 -4.0066
0.07808 -4.2228
0.07824 -4.2602
0.0784 -4.1687
0.07856 -3.9651
0.07872 -3.6658
0.07888 -3.2918
0.07904 -2.847
0.0792 -2.4314
0.07936 -2.1322
0.07952 -2.0075
0.07968 -2.0781
0.07984 -2.3067
0.08 -2.6309
0.08016 -2.9551
0.08032 -3.1795
0.08048 -3.2169
0.08064 -2.9967
0.0808 -2.581
0.08096 -2.0615
0.08112 -1.5461
0.08128 -1.1388
0.08144 -0.8935938
0.0816 -0.814625
0.08176 -0.8686563
0.08192 -0.9767187
0.08208 -1.0557
0.08224 -1.0474
0.0824 -0.914375
0.08256 -0.6691563
0.08272 -0.36575
0.08288 -0.0665
0.08304 0.149625
0.0832 0.2618438
0.08336 0.266
0.08352 0.182875
0.08368 0.0374063
0.08384 -0.1246875
0.084 -0.249375
0.08416 -0.2909375
0.08432 -0.2285938
0.08448 -0.0540313
0.08464 0.1620938
0.0848 0.4239375
0.08496 0.6899375
0.08512 0.9102188
0.08528 1.0474
0.08544 1.0432
0.0856 0.9102188
0.08576 0.6774688
0.08592 0.382375
0.08608 0.0706562
0.08624 -0.2036563
0.0864 -0.382375
0.08656 -0.4197813
0.08672 -0.3117188
0.08688 -0.0706562
0.08704 0.2202813
0.0872 0.482125
0.08736 0.6068125
0.08752 0.5070625
0.08768 0.133
0.08784 -0.49875
0.088 -1.2926
0.08816 -2.0989
0.08832 -2.7639
0.08848 -3.1463
0.08864 -3.1629
0.0888 -2.8512
0.08896 -2.3192
0.08912 -1.7124
0.08928 -1.1721
0.08944 -0.8104688
0.0896 -0.6733125
0.08976 -0.7231875
0.08992 -0.8769688
0.09008 -1.0307
0.09024 -1.1263
0.0904 -1.1388
0.09056 -1.0931
0.09072 -1.0432
0.09088 -1.0474
0.09104 -1.1471
0.0912 -1.3466
0.09136 -1.6085
0.09152 -1.8745
0.09168 -2.0864
0.09184 -2.1945
0.092 -2.2111
0.09216 -2.1696
0.09232 -2.1238
0.09248 -2.1155
0.09264 -2.1862
0.0928 -2.3441
0.09296 -2.5769
0.09312 -2.847
0.09328 -3.1255
0.09344 -3.3292
0.0936 -3.4455
0.09376 -3.4663
0.09392 -3.3957
0.09408 -3.2377
0.09424 -2.9967
0.0944 -2.7265
0.09456 -2.4771
0.09472 -2.2735
0.09488 -2.1446
0.09504 -2.0698
0.0952 -2.0532
0.09536 -2.074
0.09552 -2.1114
0.09568 -2.128
0.09584 -2.0823
0.096 -1.9534
0.09616 -1.7498
0.09632 -1.4713
0.09648 -1.1471
0.09664 -0.8063125
0.0968 -0.4904375
0.09696 -0.2369063
0.09712 -0.0706562
0.09728 0
0.09744 -0.0374063
0.0976 -0.1122188
0.09776 -0.1704063
0.09792 -0.1579375
0.09808 -0.0457188
0.09824 0.1537812
0.0984 0.4197813
0.09856 0.7065625
0.09872 0.9393125
0.09888 1.0557
0.09904 0.9975
0.0992 0.798
0.09936 0.5278438
0.09952 0.249375
0.09968 0.0374063
0.09984 -0.0955938
0.1 -0.1246875
0.10016 -0.083125
0.10032 -0.016625
0.10048 0.0041563
0.10064 -0.0623438
0.1008 -0.2285938
0.10096 -0.4696563
0.10112 -0.7439688
0.10128 -1.0058
0.10144 -1.2053
0.1016 -1.3175
0.10176 -1.3258
0.10192 -1.2469
0.10208 -1.1014
0.10224 -0.9351563
0.1024 -0.798
0.10256 -0.7439688
0.10272 -0.8104688
0.10288 -0.9975
0.10304 -1.276
0.1032 -1.6002
0.10336 -1.8994
0.10352 -2.0948
0.10368 -2.1405
0.10384 -2.0116
0.104 -1.7539
0.10416 -1.4547
0.10432 -1.2012
0.10448 -1.0682
0.10464 -1.0973
0.1048 -1.2801
0.10496 -1.5669
0.10512 -1.8662
0.10528 -2.0906
0.10544 -2.1612
0.1056 -2.0657
0.10576 -1.8246
0.10592 -1.4921
0.10608 -1.143
0.10624 -0.8561875
0.1064 -0.6899375
0.10656 -0.665
0.10672 -0.7730625
0.10688 -0.9933438
0.10704 -1.2884
0.1072 -1.596
0.10736 -1.862
0.10752 -2.049
0.10768 -2.128
0.10784 -2.0657
0.108 -1.8911
0.10816 -1.6417
0.10832 -1.3674
0.10848 -1.1139
0.10864 -0.931
0.1088 -0.8354063
0.10896 -0.83125
0.10912 -0.9019063
0.10928 -1.0266
0.10944 -1.1887
0.1096 -1.3757
0.10976 -1.5835
0.10992 -1.8163
0.11008 -2.0615
0.11024 -2.2943
0.1104 -2.4647
0.11056 -2.5353
0.11072 -2.4605
0.11088 -2.2153
0.11104 -1.7997
0.1112 -1.2843
0.11136 -0.7605938
0.11152 -0.3200312
0.11168 -0.0374063
0.11184 0.0290938
0.112 -0.0914375
0.11216 -0.3574375
0.11232 -0.681625
0.11248 -0.9933438
0.11264 -1.2178
0.1128 -1.3217
0.11296 -1.3092
0.11312 -1.2178
0.11328 -1.0931
0.11344 -0.9684063
0.1136 -0.8894375
0.11376 -0.8728125
0.11392 -0.9268438
0.11408 -1.0307
0.11424 -1.1554
0.1144 -1.251
0.11456 -1.2926
0.11472 -1.2469
0.11488 -1.1097
0.11504 -0.9060625
0.1152 -0.6566875
0.11536 -0.399
0.11552 -0.1787188
0.11568 -0.0249375
0.11584 0.0374063
0.116 0.0290938
0.11616 -0.0124688
0.11632 -0.0415625
0.11648 -0.016625
0.11664 0.083125
0.1168 0.2743125
0.11696 0.5278438
0.11712 0.7938438
0.11728 1.0183
0.11744 1.1305
0.1176 1.0973
0.11776 0.914375
0.11792 0.581875
0.11808 0.1288437
0.11824 -0.415625
0.1184 -0.9684063
0.11856 -1.4672
0.11872 -1.8578
0.11888 -2.0948
0.11904 -2.1446
0.1192 -2.0324
0.11936 -1.7955
0.11952 -1.4838
0.11968 -1.1471
0.11984 -0.8187813
0.12 -0.548625
0.12016 -0.3325
0.12032 -0.1620938
0.12048 -0.0290938
0.12064 0.0706562
0.1208 0.1371563
0.12096 0.1620938
0.12112 0.133
0.12128 0.03325
0.12144 -0.1288437
0.1216 -0.349125
0.12176 -0.5985
0.12192 -0.83125
0.12208 -1.0224
0.12224 -1.1471
0.1224 -1.2012
0.12256 -1.197
0.12272 -1.1471
0.12288 -1.0806
0.12304 -1.0224
0.1232 -0.9850312
0.12336 -0.9767187
0.12352 -1.0017
0.12368 -1.0474
0.12384 -1.1097
0.124 -1.1679
0.12416 -1.1928
0.12432 -1.1638
0.12448 -1.0889
0.12464 -0.9767187
0.1248 -0.8769688
0.12496 -0.8229375
0.12512 -0.8603438
0.12528 -1.01
0.12544 -1.2552
0.1256 -1.5586
0.12576 -1.8537
0.12592 -2.0657
0.12608 -2.1322
0.12624 -2.0449
0.1264 -1.8246
0.12656 -1.5378
0.12672 -1.2677
0.12688 -1.0848
0.12704 -1.0474
0.1272 -1.1679
0.12736 -1.4131
0.12752 -1.7332
0.12768 -2.0532
0.12784 -2.3067
0.128 -2.4439
0.12816 -2.4605
0.12832 -2.3566
0.12848 -2.1779
0.12864 -1.9493
0.1288 -1.7581
0.12896 -1.6833
0.12912 -1.7747
0.12928 -2.0366
0.12944 -2.4106
0.1296 -2.8179
0.12976 -3.1588
0.12992 -3.325
0.13008 -3.2502
0.13024 -2.8969
0.1304 -2.3566
0.13056 -1.7747
0.13072 -1.3092
0.13088 -1.0765
0.13104 -1.1222
0.1312 -1.3965
0.13136 -1.7706
0.13152 -2.074
0.13168 -2.1529
0.13184 -1.8994
0.132 -1.33
0.13216 -0.56525
0.13232 0.2369063
0.13248 0.9226875
0.13264 1.3549
0.1328 1.5129
0.13296 1.4547
0.13312 1.2801
0.13328 1.0973
0.13344 0.96425
0.1336 0.931
0.13376 0.9850312
0.13392 1.0557
0.13408 1.0723
0.13424 0.9559375
0.1344 0.7356563
0.13456 0.4655
0.13472 0.2036563
0.13488 0.0249375
0.13504 -0.0415625
0.1352 0.0290938
0.13536 0.2535313
0.13552 0.5860313
0.13568 0.9684063
0.13584 1.2926
0.136 1.5046
0.13616 1.5711
0.13632 1.4547
0.13648 1.1596
0.13664 0.7024063
0.1368 0.1579375
0.13696 -0.3574375
0.13712 -0.7772188
0.13728 -1.0266
0.13744 -1.0973
0.1376 -1.0349
0.13776 -0.9351563
0.13792 -0.89775
0.13808 -1.0058
0.13824 -1.2968
0.1384 -1.7456
0.13856 -2.2776
0.13872 -2.7764
0.13888 -3.1338
0.13904 -3.2543
0.1392 -3.1504
0.13936 -2.8803
0.13952 -2.5312
0.13968 -2.1987
0.13984 -1.9451
0.14 -1.8288
0.14016 -1.8495
0.14032 -1.9576
0.14048 -2.0948
0.14064 -2.1862
0.1408 -2.2194
0.14096 -2.2111
0.14112 -2.1737
0.14128 -2.1322
0.14144 -2.0906
0.1416 -2.0698
0.14176 -2.074
0.14192 -2.0948
0.14208 -2.1197
0.14224 -2.1363
0.1424 -2.1405
0.14256 -2.1363
0.14272 -2.1322
0.14288 -2.128
0.14304 -2.1238
0.1432 -2.1238
0.14336 -2.128
0.14352 -2.1322
0.14368 -2.128
0.14384 -2.0864
0.144 -1.9825
0.14416 -1.7997
0.14432 -1.5253
0.14448 -1.1638
0.14464 -0.7564375
0.1448 -0.36575
0.14496 -0.0623438
0.14512 0.0872813
0.14528 0.0415625
0.14544 -0.216125
0.1456 -0.6442188
0.14576 -1.1513
0.14592 -1.65
0.14608 -2.0532
0.14624 -2.261
0.1464 -2.2485
0.14656 -2.0324
0.14672 -1.6542
0.14688 -1.1845
0.14704 -0.7065625
0.1472 -0.29925
0.14736 -0.0249375
0.14752 0.0872813
0.14768 0.03325
0.14784 -0.1745625
0.148 -0.4696563
0.14816 -0.7605938
0.14832 -0.980875
0.14848 -1.064
0.14864 -1.0017
0.1488 -0.814625
0.14896 -0.5444688
0.14912 -0.266
0.14928 -0.0415625
0.14944 0.0706562
0.1496 0.0789688
0.14976 0.03325
0.14992 -0.0124688
0.15008 -0.0083125
0.15024 0.0290938
0.1504 0.1080625
0.15056 0.182875
0.15072 0.1870313
0.15088 0.0540313
0.15104 -0.2576875
0.1512 -0.7190313
0.15136 -1.2302
0.15152 -1.7082
0.15168 -2.0615
0.15184 -2.2444
0.152 -2.2735
0.15216 -2.2111
0.15232 -2.128
0.15248 -2.1155
0.15264 -2.1945
0.1528 -2.3857
0.15296 -2.6517
0.15312 -2.926
0.15328 -3.1463
0.15344 -3.2543
0.1536 -3.2668
0.15376 -3.2211
0.15392 -3.1712
0.15408 -3.1754
0.15424 -3.246
0.1544 -3.4164
0.15456 -3.6658
0.15472 -3.9484
0.15488 -4.202
0.15504 -4.3225
0.1552 -4.2934
0.15536 -4.1105
0.15552 -3.778
0.15568 -3.3208
0.15584 -2.7722
0.156 -2.2153
0.15616 -1.7124
0.15632 -1.3258
0.15648 -1.0931
0.15664 -1.0474
0.1568 -1.1638
0.15696 -1.409
0.15712 -1.7248
0.15728 -2.049
0.15744 -2.3192
0.1576 -2.4813
0.15776 -2.5062
0.15792 -2.4023
0.15808 -2.1903
0.15824 -1.9077
0.1584 -1.6085
0.15856 -1.3383
0.15872 -1.143
0.15888 -1.064
0.15904 -1.1139
0.1592 -1.2718
0.15936 -1.5087
0.15952 -1.783
0.15968 -2.0615
0.15984 -2.2859
0.16 -2.4189
0.16016 -2.4439
0.16032 -2.3607
0.16048 -2.182
0.16064 -1.9243
0.1608 -1.6417
0.16096 -1.384
0.16112 -1.1887
0.16128 -1.0765
0.16144 -1.0474
0.1616 -1.0723
0.16176 -1.1139
0.16192 -1.1305
0.16208 -1.0848
0.16224 -0.9600938
0.1624 -0.7605938
0.16256 -0.5112188
0.16272 -0.2576875
0.16288 -0.0415625
0.16304 0.0872813
0.1632 0.1371563
0.16336 0.1205313
0.16352 0.0665
0.16368 0.0083125
0.16384 -0.03325
0.164 -0.049875
0.16416 -0.03325
0.16432 -0.0124688
0.16448 0
0.16464 -0.0207812
0.1648 -0.0581875
0.16496 -0.083125
0.16512 -0.0748125
0.16528 -0.0207812
0.16544 0.0540313
0.1656 0.133
0.16576 0.182875
0.16592 0.1620938
0.16608 0.0415625
0.16624 -0.1787188
0.1664 -0.4613438
0.16656 -0.748125
0.16672 -0.96425
0.16688 -1.0598
0.16704 -1.01
0.1672 -0.83125
0.16736 -0.56525
0.16752 -0.2784688
0.16768 -0.0415625
0.16784 0.0706562
0.168 0.0290938
0.16816 -0.1745625
0.16832 -0.515375
0.16848 -0.947625
0.16864 -1.3965
0.1688 -1.7955
0.16896 -2.0781
0.16912 -2.207
0.16928 -2.1612
0.16944 -1.9576
0.1696 -1.6667
0.16976 -1.3716
0.16992 -1.1513
0.17008 -1.0598
0.17024 -1.1222
0.1704 -1.3217
0.17056 -1.6043
0.17072 -1.8869
0.17088 -2.0948
0.17104 -2.1529
0.1712 -2.0449
0.17136 -1.8038
0.17152 -1.4796
0.17168 -1.143
0.17184 -0.8561875
0.172 -0.6899375
0.17216 -0.665
0.17232 -0.7730625
0.17248 -0.9933438
0.17264 -1.2884
0.1728 -1.596
0.17296 -1.862
0.17312 -2.049
0.17328 -2.128
0.17344 -2.074
0.1736 -1.9036
0.17376 -1.6583
0.17392 -1.384
0.17408 -1.118
0.17424 -0.9226875
0.1744 -0.8187813
0.17456 -0.8104688
0.17472 -0.881125
0.17488 -1.0183
0.17504 -1.2053
0.1752 -1.4131
0.17536 -1.6293
0.17552 -1.8537
0.17568 -2.074
0.17584 -2.2652
0.176 -2.3982
0.17616 -2.4439
0.17632 -2.3815
0.17648 -2.1903
0.17664 -1.8911
0.1768 -1.5503
0.17696 -1.251
0.17712 -1.064
0.17728 -1.0391
0.17744 -1.1928
0.1776 -1.4672
0.17776 -1.7789
0.17792 -2.0241
0.17808 -2.128
0.17824 -2.049
0.1784 -1.8204
0.17856 -1.517
0.17872 -1.2386
0.17888 -1.0765
0.17904 -1.0765
0.1792 -1.2427
0.17936 -1.5253
0.17952 -1.8371
0.17968 -2.0823
0.17984 -2.1862
0.18 -2.1155
0.18016 -1.8828
0.18032 -1.5378
0.18048 -1.1554
0.18064 -0.814625
0.1808 -0.5943438
0.18096 -0.5361563
0.18112 -0.6691563
0.18128 -0.9684063
0.18144 -1.3674
0.1816 -1.783
0.18176 -2.1072
0.18192 -2.261
0.18208 -2.182
0.18224 -1.8703
0.1824 -1.384
0.18256 -0.8395625
0.18272 -0.3574375
0.18288 -0.0415625
0.18304 0.0207812
0.1832 -0.133
0.18336 -0.4364063
0.18352 -0.7689063
0.18368 -1.0224
0.18384 -1.1097
0.184 -1.0017
0.18416 -0.7356563
0.18432 -0.3948438
0.18448 -0.0665
0.18464 0.149625
0.1848 0.2410625
0.18496 0.2202813
0.18512 0.1371563
0.18528 0.0249375
0.18544 -0.1039063
0.1856 -0.2452187
0.18576 -0.4197813
0.18592 -0.6525313
0.18608 -0.9684063
0.18624 -1.3716
0.1864 -1.783
0.18656 -2.1072
0.18672 -2.2652
0.18688 -2.1862
0.18704 -1.8495
0.1872 -1.3425
0.18736 -0.7772188
0.18752 -0.29925
0.18768 -0.0249375
0.18784 -0.0415625
0.188 -0.3408125
0.18816 -0.8395625
0.18832 -1.4298
0.18848 -1.9992
0.18864 -2.4356
0.1888 -2.7182
0.18896 -2.8928
0.18912 -3.0174
0.18928 -3.1504
0.18944 -3.2959
0.1896 -3.4414
0.18976 -3.5411
0.18992 -3.5037
0.19008 -3.2793
0.19024 -2.8346
0.1904 -2.261
0.19056 -1.6916
0.19072 -1.2593
0.19088 -1.0682
0.19104 -1.1471
0.1912 -1.4381
0.19136 -1.8038
0.19152 -2.0864
0.19168 -2.1529
0.19184 -1.9243
0.192 -1.4547
0.19216 -0.8769688
0.19232 -0.349125
0.19248 -0.03325
0.19264 -0.0374063
0.1928 -0.3532813
0.19296 -0.8852813
0.19312 -1.4921
0.19328 -2.0241
0.19344 -2.3483
0.1936 -2.3898
0.19376 -2.1654
0.19392 -1.7332
0.19408 -1.2012
0.19424 -0.6733125
0.1944 -0.2452187
0.19456 0.0207812
0.19472 0.116375
0.19488 0.0415625
0.19504 -0.1704063
0.1952 -0.448875
0.19536 -0.7231875
0.19552 -0.9351563
0.19568 -1.0515
0.19584 -1.0682
0.196 -1.01
0.19616 -0.9434688
0.19632 -0.9268438
0.19648 -1.0183
0.19664 -1.2261
0.1968 -1.5212
0.19696 -1.8288
0.19712 -2.0657
0.19728 -2.1405
0.19744 -1.9992
0.1976 -1.6542
0.19776 -1.1596
0.19792 -0.615125
0.19808 -0.1080625
0.19824 0.249375
0.1984 0.4280938
0.19856 0.4280938
0.19872 0.2909375
0.19888 0.0623438
0.19904 -0.2078125
0.1992 -0.4779688
0.19936 -0.7107188
0.19952 -0.89775
0.19968 -1.0349
0.19984 -1.1347
0.2 -1.1928
0.20016 -1.2053
0.20032 -1.1679
0.20048 -1.0889
0.20064 -0.9891875
0.2008 -0.9019063
0.20096 -0.8603438
0.20112 -0.8935938
0.20128 -1.0183
0.20144 -1.2386
0.2016 -1.517
0.20176 -1.783
0.20192 -1.995
0.20208 -2.1114
0.20224 -2.1238
0.2024 -2.0698
0.20256 -2.0075
0.20272 -1.9992
0.20288 -2.0864
0.20304 -2.2735
0.2032 -2.5436
0.20336 -2.8387
0.20352 -3.0756
0.20368 -3.1878
0.20384 -3.1172
0.204 -2.9011
0.20416 -2.6101
0.20432 -2.3317
0.20448 -2.1488
0.20464 -2.1031
0.2048 -2.2153
0.20496 -2.4605
0.20512 -2.7847
0.20528 -3.113
0.20544 -3.3458
0.2056 -3.458
0.20576 -3.4622
0.20592 -3.3749
0.20608 -3.2336
0.20624 -3.0341
0.2064 -2.8179
0.20656 -2.6018
0.20672 -2.394
0.20688 -2.1779
0.20704 -1.9493
0.2072 -1.7165
0.20736 -1.4921
0.20752 -1.2884
0.20768 -1.1056
0.20784 -0.947625
0.208 -0.7938438
0.20816 -0.6068125
0.20832 -0.3740625
0.20848 -0.0789688
0.20864 0.2410625
0.2088 0.56525
0.20896 0.8437188
0.20912 1.0266
0.20928 1.0723
0.20944 0.9517813
0.2096 0.714875
0.20976 0.4405625
0.20992 0.1953438
0.21008 0.0249375
0.21024 -0.0540313
0.2104 -0.0374063
0.21056 0.0207812
0.21072 0.0623438
0.21088 0.0249375
0.21104 -0.133
0.2112 -0.3906875
0.21136 -0.6774688
0.21152 -0.9185313
0.21168 -1.0515
0.21184 -1.0307
0.212 -0.8686563
0.21216 -0.6026563
0.21232 -0.3075625
0.21248 -0.049875
0.21264 0.09975
0.2128 0.1454688
0.21296 0.1122188
0.21312 0.0457188
0.21328 0.0041563
0.21344 0
0.2136 0.0290938
0.21376 0.0706562
0.21392 0.083125
0.21408 0.0249375
0.21424 -0.116375
0.2144 -0.3408125
0.21456 -0.5985
0.21472 -0.8437188
0.21488 -1.0307
0.21504 -1.1263
0.2152 -1.143
0.21536 -1.1097
0.21552 -1.064
0.21568 -1.0557
0.21584 -1.1139
0.216 -1.2635
0.21616 -1.4879
0.21632 -1.7664
0.21648 -2.0573
0.21664 -2.2943
0.2168 -2.448
0.21696 -2.4854
0.21712 -2.4023
0.21728 -2.1945
0.21744 -1.8911
0.2176 -1.5586
0.21776 -1.276
0.21792 -1.0931
0.21808 -1.0474
0.21824 -1.1554
0.2184 -1.384
0.21856 -1.6667
0.21872 -1.9327
0.21888 -2.1031
0.21904 -2.1238
0.2192 -1.9908
0.21936 -1.7415
0.21952 -1.4339
0.21968 -1.1305
0.21984 -0.8935938
0.22 -0.76475
0.22016 -0.7522813
0.22032 -0.8437188
0.22048 -1.01
0.22064 -1.2386
0.2208 -1.4796
0.22096 -1.7082
0.22112 -1.916
0.22128 -2.0906
0.22144 -2.2153
0.2216 -2.2818
0.22176 -2.2984
0.22192 -2.2568
0.22208 -2.1571
0.22224 -2.0075
0.2224 -1.8163
0.22256 -1.596
0.22272 -1.3591
0.22288 -1.118
0.22304 -0.914375
0.2232 -0.7730625
0.22336 -0.7231875
0.22352 -0.798
0.22368 -0.9933438
0.22384 -1.2926
0.224 -1.6251
0.22416 -1.9202
0.22432 -2.1114
0.22448 -2.1446
0.22464 -2.0033
0.2248 -1.7373
0.22496 -1.4339
0.22512 -1.1845
0.22528 -1.064
0.22544 -1.1222
0.2256 -1.33
0.22576 -1.6251
0.22592 -1.9077
0.22608 -2.0989
0.22624 -2.1322
0.2264 -1.995
0.22656 -1.7332
0.22672 -1.4173
0.22688 -1.1222
0.22704 -0.9226875
0.2272 -0.8395625
0.22736 -0.8603438
0.22752 -0.9434688
0.22768 -1.0391
0.22784 -1.1097
0.228 -1.1347
0.22816 -1.118
0.22832 -1.0889
0.22848 -1.064
0.22864 -1.064
0.2288 -1.0765
0.22896 -1.0931
0.22912 -1.0973
0.22928 -1.0723
0.22944 -1.0224
0.2296 -0.96425
0.22976 -0.9226875
0.22992 -0.9351563
0.23008 -1.0266
0.23024 -1.197
0.2304 -1.4256
0.23056 -1.6791
0.23072 -1.9202
0.23088 -2.0948
0.23104 -2.1612
0.2312 -2.0906
0.23136 -1.8911
0.23152 -1.5711
0.23168 -1.1679
0.23184 -0.7315
0.232 -0.3366563
0.23216 -0.049875
0.23232 0.0789688
0.23248 0.0374063
0.23264 -0.16625
0.2328 -0.4696563
0.23296 -0.7730625
0.23312 -0.9975
0.23328 -1.0723
0.23344 -0.9725625
0.2336 -0.7315
0.23376 -0.43225
0.23392 -0.1620938
0.23408 -0.0083125
0.23424 -0.0374063
0.2344 -0.2285938
0.23456 -0.515375
0.23472 -0.814625
0.23488 -1.0307
0.23504 -1.1056
0.2352 -1.0141
0.23536 -0.7730625
0.23552 -0.4364063
0.23568 -0.0789688
0.23584 0.1995
0.236 0.36575
0.23616 0.3906875
0.23632 0.282625
0.23648 0.0665
0.23664 -0.2244375
0.2368 -0.532
0.23696 -0.798
0.23712 -0.980875
0.23728 -1.0598
0.23744 -1.0432
0.2376 -0.9684063
0.23776 -0.9060625
0.23792 -0.9060625
0.23808 -1.0141
0.23824 -1.2344
0.2384 -1.5337
0.23856 -1.8329
0.23872 -2.0573
0.23888 -2.1322
0.23904 -2.0283
0.2392 -1.783
0.23936 -1.4755
0.23952 -1.2095
0.23968 -1.0682
0.23984 -1.1014
0.24 -1.2968
0.24016 -1.5918
0.24032 -1.8869
0.24048 -2.0989
0.24064 -2.1238
0.2408 -1.9825
0.24096 -1.7165
0.24112 -1.4007
0.24128 -1.118
0.24144 -0.931
0.2416 -0.8603438
0.24176 -0.8852813
0.24192 -0.96425
0.24208 -1.0474
0.24224 -1.0806
0.2424 -1.0682
0.24256 -1.0349
0.24272 -1.0183
0.24288 -1.0474
0.24304 -1.1263
0.2432 -1.2219
0.24336 -1.2843
0.24352 -1.2635
0.24368 -1.118
0.24384 -0.8437188
0.244 -0.49875
0.24416 -0.1787188
0.24432 0.016625
0.24448 0.0290938
0.24464 -0.1704063
0.2448 -0.5112188
0.24496 -0.8603438
0.24512 -1.0931
0.24528 -1.1014
0.24544 -0.8437188
0.2456 -0.3740625
0.24576 0.1870313
0.24592 0.6940938
0.24608 1.0183
0.24624 1.064
0.2464 0.8603438
0.24656 0.5278438
0.24672 0.1995
0.24688 0.0124688
0.24704 0.0290938
0.2472 0.2452187
0.24736 0.5735625
0.24752 0.8894375
0.24768 1.0557
0.24784 0.96425
0.248 0.6234375
0.24816 0.116375
0.24832 -0.448875
0.24848 -0.9559375
0.24864 -1.3009
0.2488 -1.4464
0.24896 -1.4173
0.24912 -1.276
0.24928 -1.1014
0.24944 -0.9725625
0.2496 -0.9226875
0.24976 -0.9393125
0.24992 -0.9933438
0.25008 -1.0515
0.25024 -1.0765
0.2504 -1.0765
0.25056 -1.0557
0.25072 -1.0432
0.25088 -1.0557
0.25104 -1.0931
0.2512 -1.1471
0.25136 -1.1804
0.25152 -1.1721
0.25168 -1.0931
0.25184 -0.9600938
0.252 -0.8270938
0.25216 -0.7522813
0.25232 -0.7896875
0.25248 -0.9891875
0.25264 -1.3466
0.2528 -1.8246
0.25296 -2.34
0.25312 -2.8055
0.25328 -3.1338
0.25344 -3.2502
0.2536 -3.1588
0.25376 -2.9052
0.25392 -2.5602
0.25408 -2.207
0.25424 -1.9036
0.2544 -1.7207
0.25456 -1.6916
0.25472 -1.8121
0.25488 -2.0532
0.25504 -2.3483
0.2552 -2.6475
0.25536 -2.9135
0.25552 -3.1006
0.25568 -3.1878
0.25584 -3.1297
0.256 -2.9634
0.25616 -2.7265
0.25632 -2.4563
0.25648 -2.1903
0.25664 -1.9243
0.2568 -1.6874
0.25696 -1.4796
0.25712 -1.2968
0.25728 -1.1097
0.25744 -0.9102188
0.2576 -0.6857813
0.25776 -0.4530313
0.25792 -0.2285938
0.25808 -0.0374063
0.25824 0.0872813
0.2584 0.1579375
0.25856 0.1620938
0.25872 0.1122188
0.25888 0.0249375
0.25904 -0.0872813
0.2592 -0.1870313
0.25936 -0.2244375
0.25952 -0.1787188
0.25968 -0.0457188
0.25984 0.1371563
0.26 0.3699063
0.26016 0.6234375
0.26032 0.8561875
0.26048 1.0307
0.26064 1.1097
0.2608 1.118
0.26096 1.0931
0.26112 1.064
0.26128 1.0598
0.26144 1.064
0.2616 1.0931
0.26176 1.1347
0.26192 1.1471
0.26208 1.0931
0.26224 0.914375
0.2624 0.6068125
0.26256 0.182875
0.26272 -0.3366563
0.26288 -0.914375
0.26304 -1.4672
0.2632 -1.9243
0.26336 -2.2236
0.26352 -2.3192
0.26368 -2.1903
0.26384 -1.8537
0.264 -1.3799
0.26416 -0.8645
0.26432 -0.3948438
0.26448 -0.0540313
0.26464 0.0872813
0.2648 0.016625
0.26496 -0.2244375
0.26512 -0.581875
0.26528 -0.9684063
0.26544 -1.2884
0.2656 -1.4796
0.26576 -1.517
0.26592 -1.3965
0.26608 -1.143
0.26624 -0.8021563
0.2664 -0.4530313
0.26656 -0.16625
0.26672 0
0.26688 0.0207812
0.26704 -0.133
0.2672 -0.4571875
0.26736 -0.9060625
0.26752 -1.4422
0.26768 -1.995
0.26784 -2.473
0.268 -2.7847
0.26816 -2.8678
0.26832 -2.6933
0.26848 -2.2652
0.26864 -1.6459
0.2688 -0.9767187
0.26896 -0.3948438
0.26912 -0.0290938
0.26928 0.03325
0.26944 -0.2202813
0.2696 -0.7107188
0.26976 -1.2926
0.26992 -1.8038
0.27008 -2.0989
0.27024 -2.0781
0.2704 -1.7539
0.27056 -1.2219
0.27072 -0.6192813
0.27088 -0.09975
0.27104 0.2078125
0.2712 0.249375
0.27136 0.0249375
0.27152 -0.399
0.27168 -0.9268438
0.27184 -1.4505
0.272 -1.8662
0.27216 -2.128
0.27232 -2.2236
0.27248 -2.1612
0.27264 -1.9784
0.2728 -1.7248
0.27296 -1.4672
0.27312 -1.2427
0.27328 -1.0889
0.27344 -1.0058
0.2736 -0.9933438
0.27376 -1.0266
0.27392 -1.0682
0.27408 -1.0723
0.27424 -1.0058
0.2744 -0.8520313
0.27456 -0.6234375
0.27472 -0.3449688
0.27488 -0.0623438
0.27504 0.16625
0.2752 0.3034063
0.27536 0.3366563
0.27552 0.249375
0.27568 0.0581875
0.27584 -0.216125
0.276 -0.5236875
0.27616 -0.798
0.27632 -0.9891875
0.27648 -1.064
0.27664 -1.0183
0.2768 -0.8561875
0.27696 -0.6068125
0.27712 -0.3241875
0.27728 -0.0581875
0.27744 0.1288437
0.2776 0.2244375
0.27776 0.2285938
0.27792 0.1579375
0.27808 0.03325
0.27824 -0.1288437
0.2784 -0.3200312
0.27856 -0.5236875
0.27872 -0.748125
0.27888 -0.9975
0.27904 -1.251
0.2792 -1.5046
0.27936 -1.7415
0.27952 -1.9451
0.27968 -2.0989
0.27984 -2.1654
0.28 -2.1779
0.28016 -2.1612
0.28032 -2.1363
0.28048 -2.1238
0.28064 -2.1197
0.2808 -2.1322
0.28096 -2.1529
0.28112 -2.1612
0.28128 -2.1405
0.28144 -2.0407
0.2816 -1.8828
0.28176 -1.6667
0.28192 -1.4131
0.28208 -1.1347
0.28224 -0.8395625
0.2824 -0.56525
0.28256 -0.3325
0.28272 -0.149625
0.28288 -0.0207812
0.28304 0.0374063
0.2832 0.049875
0.28336 0.0374063
0.28352 0.016625
0.28368 0
0.28384 -0.0083125
0.284 -0.0083125
0.28416 0
0.28432 0.0083125
0.28448 0.0041563
0.28464 -0.0374063
0.2848 -0.0914375
0.28496 -0.1205313
0.28512 -0.09975
0.28528 -0.0249375
0.28544 0.0665
0.2856 0.16625
0.28576 0.2244375
0.28592 0.1953438
0.28608 0.0540313
0.28624 -0.2119688
0.2864 -0.5444688
0.28656 -0.8561875
0.28672 -1.0557
0.28688 -1.0848
0.28704 -0.9185313
0.2872 -0.6192813
0.28736 -0.2867813
0.28752 -0.0415625
0.28768 0.016625
0.28784 -0.1620938
0.288 -0.5610938
0.28816 -1.0889
0.28832 -1.6209
0.28848 -2.0449
0.28864 -2.2776
0.2888 -2.3233
0.28896 -2.2527
0.28912 -2.1571
0.28928 -2.1197
0.28944 -2.1571
0.2896 -2.261
0.28976 -2.3649
0.28992 -2.3732
0.29008 -2.2028
0.29024 -1.808
0.2904 -1.2635
0.29056 -0.6940938
0.29072 -0.23275
0.29088 -0.0083125
0.29104 -0.0914375
0.2912 -0.4571875
0.29136 -0.9975
0.29152 -1.5711
0.29168 -2.0407
0.29184 -2.2693
0.292 -2.2361
0.29216 -1.9825
0.29232 -1.596
0.29248 -1.1638
0.29264 -0.7772188
0.2928 -0.4779688
0.29296 -0.2701562
0.29312 -0.1288437
0.29328 -0.0207812
0.29344 0.0581875
0.2936 0.133
0.29376 0.1787188
0.29392 0.1537812
0.29408 0.0415625
0.29424 -0.1787188
0.2944 -0.4655
0.29456 -0.748125
0.29472 -0.96425
0.29488 -1.0598
0.29504 -1.0141
0.2952 -0.8354063
0.29536 -0.5777188
0.29552 -0.2950937
0.29568 -0.049875
0.29584 0.1039063
0.296 0.1704063
0.29616 0.1579375
0.29632 0.0914375
0.29648 0.016625
0.29664 -0.0581875
0.2968 -0.09975
0.29696 -0.09975
0.29712 -0.0581875
0.29728 -0.0083125
0.29744 0.0041563
0.2976 -0.0540313
0.29776 -0.2244375
0.29792 -0.5278438
0.29808 -0.9434688
0.29824 -1.4214
0.2984 -1.8786
0.29856 -2.2111
0.29872 -2.3358
0.29888 -2.2028
0.29904 -1.8121
0.2992 -1.2635
0.29936 -0.6857813
0.29952 -0.2285938
0.29968 -0.0083125
0.29984 -0.09975
0.3 -0.4696563
0.30016 -1.0141
0.30032 -1.5877
0.30048 -2.0449
0.30064 -2.2652
0.3008 -2.2236
0.30096 -1.9617
0.30112 -1.5711
0.30128 -1.1554
0.30144 -0.8229375
0.3016 -0.6400625
0.30176 -0.6275938
0.30192 -0.7605938
0.30208 -0.9933438
0.30224 -1.276
0.3024 -1.5544
0.30256 -1.7955
0.30272 -1.9784
0.30288 -2.1031
0.30304 -2.1737
0.3032 -2.1987
0.30336 -2.1903
0.30352 -2.1654
0.30368 -2.1322
0.30384 -2.0989
0.304 -2.0781
0.30416 -2.0823
0.30432 -2.0989
0.30448 -2.1197
0.30464 -2.128
0.3048 -2.1197
0.30496 -2.1114
0.30512 -2.1072
0.30528 -2.1197
0.30544 -2.1322
0.3056 -2.1446
0.30576 -2.1571
0.30592 -2.1529
0.30608 -2.1322
0.30624 -2.074
0.3064 -2.0116
0.30656 -1.9742
0.30672 -1.9992
0.30688 -2.0948
0.30704 -2.2194
0.3072 -2.34
0.30736 -2.4106
0.30752 -2.3774
0.30768 -2.1945
0.30784 -1.8537
0.308 -1.3923
0.30816 -0.8894375
0.30832 -0.4239375
0.30848 -0.0665
0.30864 0.116375
0.3088 0.149625
0.30896 0.0914375
0.30912 0.0124688
0.30928 -0.0083125
0.30944 0.049875
0.3096 0.216125
0.30976 0.4696563
0.30992 0.7522813
0.31008 1.01
0.31024 1.1554
0.3104 1.2012
0.31056 1.1845
0.31072 1.1305
0.31088 1.0765
0.31104 0.9891875
0.3112 0.881125
0.31136 0.7231875
0.31152 0.4779688
0.31168 0.1080625
0.31184 -0.3948438
0.312 -0.96425
0.31216 -1.5004
0.31232 -1.9036
0.31248 -2.1114
0.31264 -2.0864
0.3128 -1.8786
0.31296 -1.5711
0.31312 -1.276
0.31328 -1.0848
0.31344 -1.0598
0.3136 -1.2095
0.31376 -1.4838
0.31392 -1.7997
0.31408 -2.074
0.31424 -2.2278
0.3144 -2.2652
0.31456 -2.2236
0.31472 -2.1571
0.31488 -2.1238
0.31504 -2.1571
0.3152 -2.2984
0.31536 -2.5395
0.31552 -2.8304
0.31568 -3.1213
0.31584 -3.3416
0.316 -3.4622
0.31616 -3.4788
0.31632 -3.3998
0.31648 -3.2419
0.31664 -3.0133
0.3168 -2.7598
0.31696 -2.5187
0.31712 -2.3109
0.31728 -2.1571
0.31744 -2.049
0.3176 -2.0033
0.31776 -2.0075
0.31792 -2.0532
0.31808 -2.1114
0.31824 -2.1654
0.3184 -2.2028
0.31856 -2.2111
0.31872 -2.1862
0.31888 -2.1405
0.31904 -2.0781
0.3192 -2.0199
0.31936 -1.995
0.31952 -2.0241
0.31968 -2.0989
0.31984 -2.2111
0.32 -2.3109
0.32016 -2.3649
0.32032 -2.3275
0.32048 -2.1779
0.32064 -1.9327
0.3208 -1.6293
0.32096 -1.3342
0.32112 -1.1222
0.32128 -1.0557
0.32144 -1.1388
0.3216 -1.3508
0.32176 -1.6293
0.32192 -1.9036
0.32208 -2.0989
0.32224 -2.1405
0.3224 -2.0241
0.32256 -1.7789
0.32272 -1.4588
0.32288 -1.1347
0.32304 -0.8852813
0.3232 -0.7522813
0.32336 -0.7439688
0.32352 -0.8395625
0.32368 -1.0141
0.32384 -1.2095
0.324 -1.3633
0.32416 -1.4256
0.32432 -1.3549
0.32448 -1.1388
0.32464 -0.798
0.3248 -0.3615938
0.32496 0.1205313
0.32512 0.5901875
0.32528 0.9850312
0.32544 1.2053
0.3256 1.2261
0.32576 1.0432
0.32592 0.6691563
0.32608 0.1454688
0.32624 -0.4862813
0.3264 -1.118
0.32656 -1.6417
0.32672 -1.995
0.32688 -2.128
0.32704 -2.049
0.3272 -1.8121
0.32736 -1.5129
0.32752 -1.2386
0.32768 -1.0765
0.32784 -1.0889
0.328 -1.2593
0.32816 -1.5378
0.32832 -1.8371
0.32848 -2.0781
0.32864 -2.2194
0.3288 -2.2485
0.32896 -2.2028
0.32912 -2.1363
0.32928 -2.1197
0.32944 -2.1696
0.3296 -2.3233
0.32976 -2.5644
0.32992 -2.8553
0.33008 -3.1297
0.33024 -3.3084
0.3304 -3.3832
0.33056 -3.3749
0.33072 -3.3084
0.33088 -3.2128
0.33104 -3.1047
0.3312 -3.0258
0.33136 -3.005
0.33152 -3.0507
0.33168 -3.1588
0.33184 -3.271
0.332 -3.3749
0.33216 -3.4289
0.33232 -3.3957
0.33248 -3.246
0.33264 -2.9551
0.3328 -2.5561
0.33296 -2.0948
0.33312 -1.6126
0.33328 -1.1638
0.33344 -0.7855313
0.3336 -0.4904375
0.33376 -0.2743125
0.33392 -0.1246875
0.33408 -0.0207812
0.33424 0.0249375
0.3344 0.0540313
0.33456 0.0665
0.33472 0.0581875
0.33488 0.016625
0.33504 -0.0955938
0.3352 -0.2618438
0.33536 -0.4779688
0.33552 -0.7315
0.33568 -0.9975
0.33584 -1.251
0.336 -1.4256
0.33616 -1.4796
0.33632 -1.384
0.33648 -1.143
0.33664 -0.8063125
0.3368 -0.43225
0.33696 -0.1122188
0.33712 0.0623438
0.33728 0.0374063
0.33744 -0.2202813
0.3376 -0.665
0.33776 -1.197
0.33792 -1.6999
0.33808 -2.0657
0.33824 -2.2028
0.3384 -2.1072
0.33856 -1.8454
0.33872 -1.4879
0.33888 -1.1388
0.33904 -0.8728125
0.3392 -0.748125
0.33936 -0.7522813
0.33952 -0.8561875
0.33968 -1.0183
0.33984 -1.1928
0.34 -1.3716
0.34016 -1.5628
0.34032 -1.7872
0.34048 -2.0532
0.34064 -2.3192
0.3408 -2.5353
0.34096 -2.6309
0.34112 -2.5436
0.34128 -2.2361
0.34144 -1.7124
0.3416 -1.0848
0.34176 -0.49875
0.34192 -0.0955938
0.34208 0.0207812
0.34224 -0.1870313
0.3424 -0.6525313
0.34256 -1.2302
0.34272 -1.7623
0.34288 -2.0906
0.34304 -2.0989
0.3432 -1.8038
0.34336 -1.2801
0.34352 -0.6733125
0.34368 -0.116375
0.34384 0.266
0.344 0.43225
0.34416 0.4031563
0.34432 0.2452187
0.34448 0.0457188
0.34464 -0.1246875
0.3448 -0.2078125
0.34496 -0.1953438
0.34512 -0.1205313
0.34528 -0.0207812
0.34544 0.03325
0.3456 0.049875
0.34576 0.0374063
0.34592 0.0124688
0.34608 0
0.34624 0
0.3464 0.0124688
0.34656 0.0290938
0.34672 0.03325
0.34688 0.0083125
0.34704 -0.0581875
0.3472 -0.133
0.34736 -0.1704063
0.34752 -0.1413125
0.34768 -0.0374063
0.34784 0.1205313
0.348 0.2784688
0.34816 0.3699063
0.34832 0.3200312
0.34848 0.0872813
0.34864 -0.3283438
0.3488 -0.8603438
0.34896 -1.4048
0.34912 -1.8495
0.34928 -2.0989
0.34944 -2.1031
0.3496 -1.9036
0.34976 -1.5918
0.34992 -1.2843
0.35008 -1.0848
0.35024 -1.0598
0.3504 -1.2178
0.35056 -1.5046
0.35072 -1.8246
0.35088 -2.0823
0.35104 -2.182
0.3512 -2.1031
0.35136 -1.862
0.35152 -1.5212
0.35168 -1.1513
0.35184 -0.83125
0.352 -0.6400625
0.35216 -0.6068125
0.35232 -0.7273438
0.35248 -0.9850312
0.35264 -1.3258
0.3528 -1.6791
0.35296 -1.9701
0.35312 -2.1405
0.35328 -2.1488
0.35344 -2.0075
0.3536 -1.7498
0.35376 -1.4588
0.35392 -1.2095
0.35408 -1.0723
0.35424 -1.0889
0.3544 -1.2552
0.35456 -1.5212
0.35472 -1.8163
0.35488 -2.074
0.35504 -2.2319
0.3552 -2.2818
0.35536 -2.2444
0.35552 -2.1737
0.35568 -2.128
0.35584 -2.1405
0.356 -2.2527
0.35616 -2.4688
0.35632 -2.7722
0.35648 -3.1047
0.35664 -3.379
0.3568 -3.5494
0.35696 -3.5952
0.35712 -3.5037
0.35728 -3.271
0.35744 -2.8844
0.3576 -2.4106
0.35776 -1.9243
0.35792 -1.4796
0.35808 -1.1305
0.35824 -0.9060625
0.3584 -0.814625
0.35856 -0.8395625
0.35872 -0.9351563
0.35888 -1.0391
0.35904 -1.1056
0.3592 -1.0682
0.35936 -0.89775
0.35952 -0.5777188
0.35968 -0.1288437
0.35984 0.3865313
0.36 0.9226875
0.36016 1.4173
0.36032 1.8246
0.36048 2.0864
0.36064 2.1238
0.3608 1.9908
0.36096 1.7456
0.36112 1.4464
0.36128 1.1347
0.36144 0.8104688
0.3616 0.5278438
0.36176 0.3200312
0.36192 0.1620938
0.36208 0.0290938
0.36224 -0.1288437
0.3624 -0.3200312
0.36256 -0.5361563
0.36272 -0.7689063
0.36288 -1.0017
0.36304 -1.251
0.3632 -1.4879
0.36336 -1.7041
0.36352 -1.8994
0.36368 -2.0823
0.36384 -2.2444
0.364 -2.4189
0.36416 -2.6143
0.36432 -2.847
0.36448 -3.1172
0.36464 -3.3749
0.3648 -3.6326
0.36496 -3.8736
0.36512 -4.0814
0.36528 -4.2269
0.36544 -4.2685
0.3656 -4.256
0.36576 -4.2269
0.36592 -4.2144
0.36608 -4.2394
0.36624 -4.2602
0.3664 -4.2934
0.36656 -4.335
0.36672 -4.3433
0.36688 -4.2851
0.36704 -4.0814
0.3672 -3.7489
0.36736 -3.3125
0.36752 -2.8013
0.36768 -2.261
0.36784 -1.7456
0.368 -1.3258
0.36816 -1.0515
0.36832 -0.947625
0.36848 -1.0183
0.36864 -1.2344
0.3688 -1.5253
0.36896 -1.8204
0.36912 -2.0407
0.36928 -2.1322
0.36944 -2.0407
0.3696 -1.7498
0.36976 -1.2926
0.36992 -0.7273438
0.37008 -0.1371563
0.37024 0.3532813
0.3704 0.6774688
0.37056 0.7689063
0.37072 0.5860313
0.37088 0.1454688
0.37104 -0.49875
0.3712 -1.197
0.37136 -1.7872
0.37152 -2.1446
0.37168 -2.1737
0.37184 -1.8412
0.372 -1.2136
0.37216 -0.43225
0.37232 0.3366563
0.37248 0.947625
0.37264 1.2718
0.3728 1.2801
0.37296 1.0224
0.37312 0.5901875
0.37328 0.1122188
0.37344 -0.3034063
0.3736 -0.548625
0.37376 -0.581875
0.37392 -0.415625
0.37408 -0.0955938
0.37424 0.2950937
0.3744 0.6691563
0.37456 0.9517813
0.37472 1.1014
0.37488 1.0889
0.37504 0.9060625
0.3752 0.6234375
0.37536 0.3200312
0.37552 0.0914375
0.37568 0
0.37584 0.03325
0.376 0.2078125
0.37616 0.482125
0.37632 0.7772188
0.37648 1.0183
0.37664 1.1056
0.3768 1.0141
0.37696 0.781375
0.37712 0.4530313
0.37728 0.0872813
0.37744 -0.2743125
0.3776 -0.5860313
0.37776 -0.8104688
0.37792 -0.9559375
0.37808 -1.0432
0.37824 -1.1263
0.3784 -1.2386
0.37856 -1.4131
0.37872 -1.675
0.37888 -2.0241
0.37904 -2.4397
0.3792 -2.9011
0.37936 -3.3749
0.37952 -3.8154
0.37968 -4.1812
0.37984 -4.3932
0.38 -4.4763
0.38016 -4.4638
0.38032 -4.3848
0.38048 -4.2809
0.38064 -4.1313
0.3808 -4.0191
0.38096 -3.99
0.38112 -4.0607
0.38128 -4.2061
0.38144 -4.3474
0.3816 -4.4638
0.38176 -4.522
0.38192 -4.4846
0.38208 -4.3183
0.38224 -3.9609
0.3824 -3.4913
0.38256 -2.9925
0.38272 -2.5436
0.38288 -2.1945
0.38304 -1.9493
0.3832 -1.7913
0.38336 -1.6583
0.38352 -1.4672
0.38368 -1.1596
0.38384 -0.7315
0.384 -0.2078125
0.38416 0.3200312
0.38432 0.7605938
0.38448 1.0266
0.38464 1.064
0.3848 0.9559375
0.38496 0.8229375
0.38512 0.798
0.38528 0.980875
0.38544 1.3425
0.3856 1.8703
0.38576 2.4563
0.38592 2.9426
0.38608 3.1795
0.38624 3.005
0.3864 2.473
0.38656 1.7248
0.38672 0.89775
0.38688 0.1537812
0.38704 -0.399
0.3872 -0.6774688
0.38736 -0.6691563
0.38752 -0.4405625
0.38768 -0.0914375
0.38784 0.2369063
0.388 0.4613438
0.38816 0.5278438
0.38832 0.4114688
0.38848 0.1039063
0.38864 -0.3740625
0.3888 -0.980875
0.38896 -1.6625
0.38912 -2.3649
0.38928 -3.0341
0.38944 -3.5868
0.3896 -3.9942
0.38976 -4.2477
0.38992 -4.3433
0.39008 -4.2893
0.39024 -4.0648
0.3904 -3.778
0.39056 -3.512
0.39072 -3.3125
0.39088 -3.2003
0.39104 -3.1546
0.3912 -3.1629
0.39136 -3.2086
0.39152 -3.2377
0.39168 -3.2086
0.39184 -3.0673
0.392 -2.8429
0.39216 -2.5852
0.39232 -2.3441
0.39248 -2.1571
0.39264 -2.0324
0.3928 -1.9867
0.39296 -2.0158
0.39312 -2.074
0.39328 -2.1197
0.39344 -2.0989
0.3936 -1.9784
0.39376 -1.7623
0.39392 -1.4755
0.39408 -1.1471
0.39424 -0.8229375
0.3944 -0.5236875
0.39456 -0.2784688
0.39472 -0.1039063
0.39488 -0.0083125
0.39504 0.0083125
0.3952 0
0.39536 -0.0290938
0.39552 -0.0374063
0.39568 -0.0124688
0.39584 0.0581875
0.396 0.2036563
0.39616 0.4239375
0.39632 0.6940938
0.39648 0.9891875
0.39664 1.2427
0.3968 1.4173
0.39696 1.4713
0.39712 1.3799
0.39728 1.143
0.39744 0.7772188
0.3976 0.382375
0.39776 0.0623438
0.39792 -0.09975
0.39808 -0.0457188
0.39824 0.2285938
0.3984 0.6857813
0.39856 1.2178
0.39872 1.7165
0.39888 2.0698
0.39904 2.1862
0.3992 2.0698
0.39936 1.7955
0.39952 1.4505
0.39968 1.1305
0.39984 0.8686563
0.4 0.7273438
0.40016 0.7273438
0.40032 0.8395625
0.40048 1.0141
0.40064 1.1305
0.4008 1.1721
0.40096 1.1596
0.40112 1.1139
0.40128 1.0682
0.40144 0.9767187
0.4016 0.8894375
0.40176 0.8645
0.40192 0.914375
0.40208 1.0266
0.40224 1.0765
0.4024 1.0432
0.40256 0.9060625
0.40272 0.6192813
0.40288 0.149625
0.40304 -0.5694063
0.4032 -1.5129
0.40336 -2.6143
0.40352 -3.8071
0.40368 -5.0208
0.40384 -6.1388
0.404 -7.1363
0.40416 -8.0133
0.40432 -8.7697
0.40448 -9.4222
0.40464 -9.8794
0.4048 -10.2493
0.40496 -10.6358
0.40512 -11.0723
0.40528 -11.571
0.40544 -11.9201
0.4056 -12.1986
0.40576 -12.4563
0.40592 -12.6599
0.40608 -12.7638
0.40624 -12.4646
0.4064 -11.9118
0.40656 -11.2385
0.40672 -10.5112
0.40688 -9.763
0.40704 -8.7988
0.4072 -7.7473
0.40736 -6.6957
0.40752 -5.6276
0.40768 -4.5345
0.40784 -3.2585
0.408 -1.8953
0.40816 -0.5527813
0.40832 0.7024063
0.40848 1.8537
0.40864 2.9925
0.4088 4.2144
0.40896 5.5902
0.40912 7.1945
0.40928 9.0648
0.40944 11.1803
0.4096 13.4455
0.40976 15.6483
0.40992 17.5394
0.41008 18.9068
0.41024 19.6424
0.4104 19.9043
0.41056 19.817
0.41072 19.5344
0.41088 19.2143
0.41104 19.0813
0.4112 19.3016
0.41136 19.817
0.41152 20.482
0.41168 21.1345
0.41184 21.7497
0.412 22.3191
0.41216 22.7721
0.41232 23.0963
0.41248 23.3415
0.41264 23.7363
0.4128 24.4637
0.41296 25.5942
0.41312 27.1777
0.41328 29.2143
0.41344 31.6956
0.4136 34.5177
0.41376 37.4811
0.41392 40.3738
0.41408 43.0172
0.41424 45.2283
0.4144 47.0945
0.41456 48.732
0.41472 50.2491
0.41488 51.7536
0.41504 53.1293
0.4152 54.5383
0.41536 55.9681
0.41552 57.2814
0.41568 58.3163
0.41584 58.7444
0.416 58.7569
0.41616 58.5117
0.41632 58.0919
0.41648 57.5848
0.41664 56.8824
0.4168 56.2673
0.41696 55.8434
0.41712 55.5732
0.41728 55.3737
0.41744 55.0204
0.4176 54.6214
0.41776 54.2183
0.41792 53.7902
0.41808 53.3205
0.41824 52.7719
0.4184 52.2316
0.41856 51.7619
0.41872 51.3879
0.41888 51.1177
0.41904 50.9681
0.4192 50.9224
0.41936 50.9473
0.41952 51.0055
0.41968 51.0637
0.41984 51.0845
0.42 51.0013
0.42016 50.7977
0.42032 50.486
0.42048 50.1078
0.42064 49.7212
0.4208 49.3596
0.42096 49.0728
0.42112 48.9066
0.42128 48.9108
0.42144 49.1061
0.4216 49.4428
0.42176 49.8958
0.42192 50.4112
0.42208 50.9432
0.42224 51.3962
0.4224 51.683
0.42256 51.7619
0.42272 51.5998
0.42288 51.2008
0.42304 50.5857
0.4232 49.8542
0.42336 49.1144
0.42352 48.466
0.42368 47.9714
0.42384 47.6805
0.424 47.5267
0.42416 47.4145
0.42432 47.2399
0.42448 46.9199
0.42464 46.496
0.4248 45.9806
0.42496 45.4361
0.42512 44.9789
0.42528 44.7129
0.42544 44.8044
0.4256 45.1826
0.42576 45.7229
0.42592 46.2798
0.42608 46.7287
0.42624 47.0903
0.4264 47.2649
0.42656 47.2358
0.42672 47.0654
0.42688 46.8576
0.42704 46.8493
0.4272 47.0529
0.42736 47.4478
0.42752 48.0172
0.42768 48.7403
0.42784 49.6755
0.428 50.7395
0.42816 51.8492
0.42832 52.9506
0.42848 54.0063
0.42864 55.0163
0.4288 55.9681
0.42896 56.8409
0.42912 57.6347
0.42928 58.3496
0.42944 58.9897
0.4296 59.6256
0.42976 60.2698
0.42992 60.914
0.43008 61.5541
0.43024 62.1692
0.4304 62.8176
0.43056 63.4909
0.43072 64.1517
0.43088 64.7627
0.43104 65.2656
0.4312 65.6854
0.43136 65.9846
0.43152 66.1052
0.43168 66.0179
0.43184 65.7477
0.432 65.4277
0.43216 65.1368
0.43232 64.9372
0.43248 64.8874
0.43264 65.0786
0.4328 65.5441
0.43296 66.2215
0.43312 67.0278
0.43328 67.8799
0.43344 68.7444
0.4336 69.6006
0.43376 70.4484
0.43392 71.288
0.43408 72.1359
0.43424 72.9837
0.4344 73.8607
0.43456 74.7502
0.43472 75.6147
0.43488 76.4168
0.43504 77.0943
0.4352 77.7052
0.43536 78.3162
0.43552 78.9521
0.43568 79.6254
0.43584 80.2406
0.436 80.8308
0.43616 81.3586
0.43632 81.7451
0.43648 81.9197
0.43664 81.7534
0.4368 81.3794
0.43696 80.8972
0.43712 80.3819
0.43728 79.9039
0.43744 79.4467
0.4376 79.1059
0.43776 78.9106
0.43792 78.8108
0.43808 78.7485
0.43824 78.6363
0.4384 78.4742
0.43856 78.2622
0.43872 78.0087
0.43888 77.7385
0.43904 77.4808
0.4392 77.3021
0.43936 77.2481
0.43952 77.3437
0.43968 77.5889
0.43984 77.9588
0.44 78.3578
0.44016 78.6861
0.44032 78.8482
0.44048 78.7859
0.44064 78.5116
0.4408 78.0627
0.44096 77.5265
0.44112 77.0195
0.44128 76.662
0.44144 76.554
0.4416 76.6787
0.44176 76.9613
0.44192 77.3021
0.44208 77.6097
0.44224 77.8299
0.4424 77.9172
0.44256 77.884
0.44272 77.7842
0.44288 77.6886
0.44304 77.6554
0.4432 77.7302
0.44336 77.9338
0.44352 78.2456
0.44368 78.6321
0.44384 79.0394
0.444 79.4052
0.44416 79.6753
0.44432 79.8166
0.44448 79.8166
0.44464 79.7044
0.4448 79.5589
0.44496 79.4633
0.44512 79.5049
0.44528 79.7169
0.44544 80.0785
0.4456 80.5149
0.44576 80.8972
0.44592 81.076
0.44608 80.9471
0.44624 80.4733
0.4464 79.7501
0.44656 78.9355
0.44672 78.2082
0.44688 77.7343
0.44704 77.6388
0.4472 77.9089
0.44736 78.4451
0.44752 79.0893
0.44768 79.6753
0.44784 80.0868
0.448 80.2572
0.44816 80.2156
0.44832 80.0411
0.44848 79.8416
0.44864 79.7252
0.4488 79.7626
0.44896 79.9704
0.44912 80.3154
0.44928 80.7476
0.44944 81.1923
0.4496 81.5747
0.44976 81.849
0.44992 81.9779
0.45008 81.9529
0.45024 81.7742
0.4504 81.4999
0.45056 81.209
0.45072 80.9804
0.45088 80.8682
0.45104 80.8557
0.4512 80.9804
0.45136 81.2214
0.45152 81.5332
0.45168 81.8532
0.45184 82.0776
0.452 82.1898
0.45216 82.1982
0.45232 82.1192
0.45248 81.9696
0.45264 81.7327
0.4528 81.4708
0.45296 81.2381
0.45312 81.0469
0.45328 80.8972
0.45344 80.7185
0.4536 80.4733
0.45376 80.1159
0.45392 79.6047
0.45408 78.9272
0.45424 78.1333
0.4544 77.2979
0.45456 76.5249
0.45472 75.9264
0.45488 75.5814
0.45504 75.5731
0.4552 75.81
0.45536 76.155
0.45552 76.4625
0.45568 76.608
0.45584 76.5664
0.456 76.2838
0.45616 75.7934
0.45632 75.1907
0.45648 74.6047
0.45664 74.2722
0.4568 74.2098
0.45696 74.401
0.45712 74.8083
0.45728 75.3819
0.45744 76.1383
0.4576 76.9363
0.45776 77.6554
0.45792 78.2373
0.45808 78.6529
0.45824 78.9604
0.4584 79.1724
0.45856 79.3262
0.45872 79.4924
0.45888 79.721
0.45904 80.0826
0.4592 80.598
0.45936 81.2505
0.45952 81.9987
0.45968 82.7925
0.45984 83.5448
0.46 84.2513
0.46016 84.908
0.46032 85.5107
0.46048 86.0552
0.46064 86.4542
0.4608 86.7659
0.46096 87.0153
0.46112 87.1815
0.46128 87.248
0.46144 87.0942
0.4616 86.8448
0.46176 86.5872
0.46192 86.3669
0.46208 86.2089
0.46224 86.0261
0.4624 85.8557
0.46256 85.6936
0.46272 85.4857
0.46288 85.2031
0.46304 84.7626
0.4632 84.2513
0.46336 83.7526
0.46352 83.3328
0.46368 83.0419
0.46384 82.8424
0.464 82.7177
0.46416 82.5972
0.46432 82.4018
0.46448 82.0444
0.46464 81.4583
0.4648 80.652
0.46496 79.7127
0.46512 78.7443
0.46528 77.8632
0.46544 77.1649
0.4656 76.6828
0.46576 76.4293
0.46592 76.396
0.46608 76.5457
0.46624 76.795
0.4664 77.0569
0.46656 77.2896
0.46672 77.485
0.46688 77.6388
0.46704 77.6678
0.4672 77.5556
0.46736 77.3353
0.46752 77.0444
0.46768 76.6994
0.46784 76.2506
0.468 75.6978
0.46816 75.0619
0.46832 74.3636
0.46848 73.6113
0.46864 72.7801
0.4688 71.8948
0.46896 70.9971
0.46912 70.1409
0.46928 69.347
0.46944 68.6238
0.4696 67.9007
0.46976 67.1151
0.46992 66.2174
0.47008 65.1783
0.47024 64.0935
0.4704 62.9921
0.47056 61.9738
0.47072 61.18
0.47088 60.7062
0.47104 60.6854
0.4712 60.9431
0.47136 61.2964
0.47152 61.5956
0.47168 61.7162
0.47184 61.6871
0.472 61.4502
0.47216 61.0844
0.47232 60.7561
0.47248 60.6314
0.47264 60.8808
0.4728 61.4294
0.47296 62.1692
0.47312 62.9672
0.47328 63.6862
0.47344 64.2265
0.4736 64.4925
0.47376 64.4801
0.47392 64.2556
0.47408 63.9231
0.47424 63.5989
0.4744 63.3703
0.47456 63.3122
0.47472 63.441
0.47488 63.7403
0.47504 64.1143
0.4752 64.4967
0.47536 64.8084
0.47552 64.9788
0.47568 64.9456
0.47584 64.6463
0.476 64.1725
0.47616 63.6363
0.47632 63.1501
0.47648 62.8217
0.47664 62.6638
0.4768 62.7178
0.47696 62.9713
0.47712 63.3454
0.47728 63.7486
0.47744 64.0104
0.4776 64.0728
0.47776 63.9065
0.47792 63.5158
0.47808 62.9423
0.47824 62.1941
0.4784 61.392
0.47856 60.648
0.47872 60.0537
0.47888 59.6505
0.47904 59.4302
0.4792 59.3097
0.47936 59.2058
0.47952 59.0063
0.47968 58.6405
0.47984 58.0628
0.48 57.3188
0.48016 56.5292
0.48032 55.8475
0.48048 55.3945
0.48064 55.2033
0.4808 55.2116
0.48096 55.328
0.48112 55.4153
0.48128 55.3696
0.48144 55.0869
0.4816 54.584
0.48176 53.9938
0.48192 53.4868
0.48208 53.2166
0.48224 53.2125
0.4824 53.4244
0.48256 53.7653
0.48272 54.0894
0.48288 54.2598
0.48304 54.0936
0.4832 53.5408
0.48336 52.6555
0.48352 51.5375
0.48368 50.3114
0.48384 49.1019
0.484 48.0047
0.48416 47.0903
0.48432 46.3796
0.48448 45.8601
0.48464 45.4486
0.4848 45.0205
0.48496 44.476
0.48512 43.7487
0.48528 42.8177
0.48544 41.7454
0.4856 40.6024
0.48576 39.5343
0.48592 38.7279
0.48608 38.3289
0.48624 38.4328
0.4864 38.9358
0.48656 39.6963
0.48672 40.5484
0.48688 41.3298
0.48704 41.8659
0.4872 42.0197
0.48736 41.7994
0.48752 41.2923
0.48768 40.6107
0.48784 39.8626
0.488 39.1103
0.48816 38.4204
0.48832 37.8343
0.48848 37.3439
0.48864 36.9324
0.4888 36.5334
0.48896 36.1303
0.48912 35.698
0.48928 35.2325
0.48944 34.7005
0.4896 34.0937
0.48976 33.4287
0.48992 32.7429
0.49008 32.0779
0.49024 31.442
0.4904 30.8685
0.49056 30.3905
0.49072 30.0372
0.49088 29.8211
0.49104 29.713
0.4912 29.659
0.49136 29.6507
0.49152 29.6923
0.49168 29.7671
0.49184 29.8086
0.492 29.7463
0.49216 29.5634
0.49232 29.26
0.49248 28.8444
0.49264 28.2916
0.4928 27.6266
0.49296 26.9242
0.49312 26.2592
0.49328 25.669
0.49344 25.1287
0.4936 24.5925
0.49376 24.019
0.49392 23.354
0.49408 22.5601
0.49424 21.6167
0.4944 20.5319
0.49456 19.3764
0.49472 18.2459
0.49488 17.2443
0.49504 16.4338
0.4952 15.8104
0.49536 15.3657
0.49552 15.083
0.49568 14.9209
0.49584 14.8004
0.496 14.63
0.49616 14.4014
0.49632 14.1437
0.49648 13.8902
0.49664 13.62
0.4968 13.3208
0.49696 13.0548
0.49712 12.8594
0.49728 12.7722
0.49744 12.714
0.4976 12.6101
0.49776 12.448
0.49792 12.1986
0.49808 11.8204
0.49824 11.2177
0.4984 10.3657
0.49856 9.3017
0.49872 8.0714
0.49888 6.729
0.49904 5.2951
0.4992 3.7947
0.49936 2.2735
0.49952 0.7605938
0.49968 -0.7065625
0.49984 -2.0657
0.5 -3.3292
0.50016 -4.4763
0.50032 -5.453
0.50048 -6.2261
0.50064 -6.7207
0.5008 -7.049
0.50096 -7.2693
0.50112 -7.394
0.50128 -7.4438
0.50144 -7.394
0.5016 -7.3649
0.50176 -7.3857
0.50192 -7.4231
0.50208 -7.4438
0.50224 -7.4106
0.5024 -7.3981
0.50256 -7.4064
0.50272 -7.4189
0.50288 -7.4397
0.50304 -7.4771
0.5032 -7.6475
0.50336 -8.0049
0.50352 -8.5743
0.50368 -9.3516
0.50384 -10.2909
0.504 -11.3133
0.50416 -12.3025
0.50432 -13.1379
0.50448 -13.7239
0.50464 -14.0273
0.5048 -14.1437
0.50496 -14.2144
0.50512 -14.3848
0.50528 -14.7588
0.50544 -15.3615
0.5056 -16.1512
0.50576 -16.9866
0.50592 -17.6724
0.50608 -18.0506
0.50624 -17.9882
0.5064 -17.5851
0.50656 -17.0032
0.50672 -16.4255
0.50688 -16.0182
0.50704 -15.8686
0.5072 -16.014
0.50736 -16.3632
0.50752 -16.7455
0.50768 -16.9949
0.50784 -16.9824
0.508 -16.7538
0.50816 -16.413
0.50832 -16.1013
0.50848 -15.9558
0.50864 -16.0764
0.5088 -16.4546
0.50896 -16.9991
0.50912 -17.5602
0.50928 -18.0049
0.50944 -18.2584
0.5096 -18.3124
0.50976 -18.2376
0.50992 -18.1254
0.51008 -18.0797
0.51024 -18.2085
0.5104 -18.487
0.51056 -18.8112
0.51072 -19.0689
0.51088 -19.1603
0.51104 -19.1104
0.5112 -18.9276
0.51136 -18.6408
0.51152 -18.3415
0.51168 -18.1171
0.51184 -18.1711
0.512 -18.487
0.51216 -18.9733
0.51232 -19.5385
0.51248 -20.0872
0.51264 -20.6483
0.5128 -21.1595
0.51296 -21.5834
0.51312 -21.9325
0.51328 -22.2567
0.51344 -22.7098
0.5136 -23.3332
0.51376 -24.1353
0.51392 -25.1204
0.51408 -26.28
0.51424 -27.6266
0.5144 -29.1145
0.51456 -30.6731
0.51472 -32.2276
0.51488 -33.6989
0.51504 -34.9998
0.5152 -36.1178
0.51536 -37.0405
0.51552 -37.7388
0.51568 -38.2126
0.51584 -38.4495
0.516 -38.6032
0.51616 -38.7737
0.51632 -38.9939
0.51648 -39.2849
0.51664 -39.5924
0.5168 -39.9457
0.51696 -40.2741
0.51712 -40.4777
0.51728 -40.4694
0.51744 -40.2117
0.5176 -39.8584
0.51776 -39.5259
0.51792 -39.3181
0.51808 -39.3223
0.51824 -39.5966
0.5184 -40.1535
0.51856 -40.8975
0.51872 -41.6913
0.51888 -42.4021
0.51904 -42.9507
0.5192 -43.3538
0.51936 -43.6905
0.51952 -44.0563
0.51968 -44.5384
0.51984 -45.1992
0.52 -46.0388
0.52016 -46.9906
0.52032 -47.9382
0.52048 -48.7653
0.52064 -49.3472
0.5208 -49.7004
0.52096 -49.8875
0.52112 -49.9623
0.52128 -49.9997
0.52144 -49.9955
0.5216 -50.0205
0.52176 -50.0703
0.52192 -50.0953
0.52208 -50.0371
0.52224 -49.8293
0.5224 -49.5467
0.52256 -49.2723
0.52272 -49.0562
0.52288 -48.9482
0.52304 -48.9565
0.5232 -49.1061
0.52336 -49.3679
0.52352 -49.6713
0.52368 -49.9498
0.52384 -50.1161
0.524 -50.1826
0.52416 -50.1701
0.52432 -50.1036
0.52448 -50.0246
0.52464 -49.9623
0.5248 -49.9332
0.52496 -49.9415
0.52512 -49.9706
0.52528 -49.9997
0.52544 -50.0454
0.5256 -50.087
0.52576 -50.0953
0.52592 -50.0745
0.52608 -50.0205
0.52624 -50.008
0.5264 -50.0246
0.52656 -50.0371
0.52672 -50.0371
0.52688 -50.0163
0.52704 -50.0454
0.5272 -50.0953
0.52736 -50.1202
0.52752 -50.0953
0.52768 -50.0288
0.52784 -50.008
0.528 -50.062
0.52816 -50.2075
0.52832 -50.486
0.52848 -50.9307
0.52864 -51.6082
0.5288 -52.4519
0.52896 -53.3663
0.52912 -54.2806
0.52928 -55.1285
0.52944 -55.9099
0.5296 -56.6622
0.52976 -57.4394
0.52992 -58.3122
0.53008 -59.3097
0.53024 -60.4028
0.5304 -61.525
0.53056 -62.564
0.53072 -63.362
0.53088 -63.7943
0.53104 -63.7693
0.5312 -63.4077
0.53136 -62.855
0.53152 -62.2689
0.53168 -61.7951
0.53184 -61.525
0.532 -61.5541
0.53216 -61.8367
0.53232 -62.2565
0.53248 -62.6804
0.53264 -62.9797
0.5328 -63.1334
0.53296 -63.1376
0.53312 -63.0129
0.53328 -62.8217
0.53344 -62.643
0.5336 -62.5931
0.53376 -62.7386
0.53392 -63.1002
0.53408 -63.6696
0.53424 -64.4011
0.5344 -65.2074
0.53456 -65.968
0.53472 -66.5831
0.53488 -66.9697
0.53504 -67.1151
0.5352 -67.0736
0.53536 -66.9613
0.53552 -66.8948
0.53568 -66.978
0.53584 -67.2564
0.536 -67.7178
0.53616 -68.2747
0.53632 -68.7776
0.53648 -69.1143
0.53664 -69.1725
0.5368 -69.0021
0.53696 -68.6945
0.53712 -68.3662
0.53728 -68.1292
0.53744 -68.0586
0.5376 -68.1916
0.53776 -68.4784
0.53792 -68.8192
0.53808 -69.106
0.53824 -69.2473
0.5384 -69.2681
0.53856 -69.2099
0.53872 -69.1434
0.53888 -69.1434
0.53904 -69.2431
0.5392 -69.4509
0.53936 -69.7253
0.53952 -69.9913
0.53968 -70.1908
0.53984 -70.2656
0.54 -70.2448
0.54016 -70.1824
0.54032 -70.1492
0.54048 -70.1949
0.54064 -70.3445
0.5408 -70.5856
0.54096 -70.8682
0.54112 -71.1218
0.54128 -71.2714
0.54144 -71.2672
0.5416 -71.1218
0.54176 -70.8682
0.54192 -70.5731
0.54208 -70.2863
0.54224 -70.0661
0.5424 -69.958
0.54256 -69.958
0.54272 -70.0494
0.54288 -70.1866
0.54304 -70.3279
0.5432 -70.4277
0.54336 -70.4568
0.54352 -70.4027
0.54368 -70.2656
0.54384 -70.091
0.544 -69.9331
0.54416 -69.8541
0.54432 -69.9123
0.54448 -70.1367
0.54464 -70.5399
0.5448 -71.0469
0.54496 -71.5789
0.54512 -72.0278
0.54528 -72.3104
0.54544 -72.3811
0.5456 -72.2439
0.54576 -71.9738
0.54592 -71.6454
0.54608 -71.3462
0.54624 -71.1425
0.5464 -71.0636
0.54656 -71.0927
0.54672 -71.1799
0.54688 -71.2672
0.54704 -71.3088
0.5472 -71.3005
0.54736 -71.2631
0.54752 -71.2381
0.54768 -71.2672
0.54784 -71.3711
0.548 -71.554
0.54816 -71.7992
0.54832 -72.0652
0.54848 -72.298
0.54864 -72.4559
0.5488 -72.5307
0.54896 -72.5307
0.54912 -72.4725
0.54928 -72.3769
0.54944 -72.2564
0.5496 -72.1026
0.54976 -71.9114
0.54992 -71.6662
0.55008 -71.367
0.55024 -71.0303
0.5504 -70.6978
0.55056 -70.4193
0.55072 -70.2448
0.55088 -70.2074
0.55104 -70.3238
0.5512 -70.5523
0.55136 -70.8308
0.55152 -71.0843
0.55168 -71.2589
0.55184 -71.3337
0.552 -71.3212
0.55216 -71.2672
0.55232 -71.2257
0.55248 -71.2631
0.55264 -71.4044
0.5528 -71.633
0.55296 -71.9031
0.55312 -72.1567
0.55328 -72.3271
0.55344 -72.3603
0.5536 -72.2439
0.55376 -72.0029
0.55392 -71.687
0.55408 -71.3628
0.55424 -71.0843
0.5544 -70.9098
0.55456 -70.8724
0.55472 -70.9804
0.55488 -71.2132
0.55504 -71.5124
0.5552 -71.8242
0.55536 -72.0985
0.55552 -72.2855
0.55568 -72.352
0.55584 -72.2523
0.556 -72.032
0.55616 -71.7577
0.55632 -71.5
0.55648 -71.3171
0.55664 -71.1882
0.5568 -71.1384
0.55696 -71.1592
0.55712 -71.2215
0.55728 -71.2797
0.55744 -71.234
0.5576 -71.0802
0.55776 -70.8474
0.55792 -70.5731
0.55808 -70.2905
0.55824 -69.9954
0.5584 -69.7294
0.55856 -69.5133
0.55872 -69.3387
0.55888 -69.1933
0.55904 -69.0062
0.5592 -68.7402
0.55936 -68.3662
0.55952 -67.8591
0.55968 -67.2107
0.55984 -66.4418
0.56 -65.569
0.56016 -64.6505
0.56032 -63.7569
0.56048 -62.9506
0.56064 -62.3438
0.5608 -61.9281
0.56096 -61.6871
0.56112 -61.6123
0.56128 -61.6746
0.56144 -61.9073
0.5616 -62.1858
0.56176 -62.4393
0.56192 -62.6347
0.56208 -62.7552
0.56224 -62.8383
0.5624 -62.8508
0.56256 -62.8176
0.56272 -62.7802
0.56288 -62.7718
0.56304 -62.8217
0.5632 -62.8841
0.56336 -62.9173
0.56352 -62.8965
0.56368 -62.8051
0.56384 -62.6763
0.564 -62.5349
0.56416 -62.4393
0.56432 -62.4768
0.56448 -62.6887
0.56464 -63.0753
0.5648 -63.5823
0.56496 -64.1226
0.56512 -64.5881
0.56528 -64.8708
0.56544 -64.8417
0.5656 -64.5424
0.56576 -64.052
0.56592 -63.4701
0.56608 -62.9048
0.56624 -62.3812
0.5664 -61.9822
0.56656 -61.7494
0.56672 -61.6621
0.56688 -61.6912
0.56704 -61.7203
0.5672 -61.7286
0.56736 -61.7245
0.56752 -61.7203
0.56768 -61.712
0.56784 -61.6206
0.568 -61.4626
0.56816 -61.2631
0.56832 -61.0221
0.56848 -60.727
0.56864 -60.3072
0.5688 -59.7627
0.56896 -59.131
0.56912 -58.4203
0.56928 -57.6555
0.56944 -56.8243
0.5696 -55.9473
0.56976 -55.0578
0.56992 -54.1975
0.57008 -53.3912
0.57024 -52.6597
0.5704 -51.9822
0.57056 -51.3422
0.57072 -50.7312
0.57088 -50.1452
0.57104 -49.609
0.5712 -49.0853
0.57136 -48.6032
0.57152 -48.2042
0.57168 -47.9257
0.57184 -47.7969
0.572 -47.7512
0.57216 -47.7553
0.57232 -47.7969
0.57248 -47.8592
0.57264 -47.9299
0.5728 -47.9465
0.57296 -47.9216
0.57312 -47.8925
0.57328 -47.8758
0.57344 -47.8551
0.5736 -47.7719
0.57376 -47.6057
0.57392 -47.3355
0.57408 -46.9365
0.57424 -46.3796
0.5744 -45.6523
0.57456 -44.7836
0.57472 -43.8152
0.57488 -42.8094
0.57504 -41.8451
0.5752 -40.9848
0.57536 -40.2699
0.57552 -39.7462
0.57568 -39.4179
0.57584 -39.2932
0.576 -39.2433
0.57616 -39.1353
0.57632 -38.8859
0.57648 -38.4412
0.57664 -37.8718
0.5768 -37.2151
0.57696 -36.6041
0.57712 -36.2009
0.57728 -36.1303
0.57744 -36.521
0.5776 -37.2234
0.57776 -38.038
0.57792 -38.782
0.57808 -39.289
0.57824 -39.501
0.5784 -39.3846
0.57856 -39.0397
0.57872 -38.6365
0.57888 -38.3414
0.57904 -38.2998
0.5792 -38.4869
0.57936 -38.8111
0.57952 -39.1436
0.57968 -39.3472
0.57984 -39.3181
0.58 -39.0812
0.58016 -38.7363
0.58032 -38.4287
0.58048 -38.2957
0.58064 -38.358
0.5808 -38.649
0.58096 -39.1353
0.58112 -39.7213
0.58128 -40.299
0.58144 -40.6232
0.5816 -40.7188
0.58176 -40.6689
0.58192 -40.5525
0.58208 -40.4486
0.58224 -40.2408
0.5824 -40.0081
0.58256 -39.8086
0.58272 -39.6298
0.58288 -39.4262
0.58304 -38.9648
0.5832 -38.2749
0.58336 -37.4312
0.58352 -36.4628
0.58368 -35.3905
0.58384 -34.1187
0.584 -32.7014
0.58416 -31.201
0.58432 -29.6465
0.58448 -28.0588
0.58464 -26.4504
0.5848 -24.8461
0.58496 -23.2958
0.58512 -21.8453
0.58528 -20.5236
0.58544 -19.4014
0.5856 -18.3582
0.58576 -17.3274
0.58592 -16.2676
0.58608 -15.1703
0.58624 -14.1728
0.5864 -13.1836
0.58656 -12.2194
0.58672 -11.3798
0.58688 -10.7522
0.58704 -10.5486
0.5872 -10.6109
0.58736 -10.8353
0.58752 -11.1762
0.58768 -11.5918
0.58784 -12.157
0.588 -12.6849
0.58816 -13.1088
0.58832 -13.4579
0.58848 -13.7572
0.58864 -14.1022
0.5888 -14.4056
0.58896 -14.6466
0.58912 -14.8087
0.58928 -14.8877
0.58944 -14.8835
0.5896 -14.7672
0.58976 -14.5469
0.58992 -14.2518
0.59008 -13.9151
0.59024 -13.5618
0.5904 -13.2377
0.59056 -12.9758
0.59072 -12.8137
0.59088 -12.7597
0.59104 -12.7929
0.5912 -12.9218
0.59136 -13.1421
0.59152 -13.433
0.59168 -13.753
0.59184 -13.9858
0.592 -14.1229
0.59216 -14.1603
0.59232 -14.0814
0.59248 -13.8943
0.59264 -13.5286
0.5928 -13.0589
0.59296 -12.5727
0.59312 -12.1279
0.59328 -11.7747
0.59344 -11.492
0.5936 -11.2801
0.59376 -11.1013
0.59392 -10.9226
0.59408 -10.7023
0.59424 -10.3657
0.5944 -9.9334
0.59456 -9.4638
0.59472 -9.0066
0.59488 -8.6034
0.59504 -8.1795
0.5952 -7.7057
0.59536 -7.1404
0.59552 -6.4339
0.59568 -5.5611
0.59584 -4.4347
0.596 -3.2045
0.59616 -2.0948
0.59632 -1.3217
0.59648 -1.0474
0.59664 -1.1305
0.5968 -1.5004
0.59696 -2.0615
0.59712 -2.6517
0.59728 -3.113
0.59744 -3.0507
0.5976 -2.5062
0.59776 -1.7124
0.59792 -0.8686563
0.59808 -0.149625
0.59824 0.615125
0.5984 1.4256
0.59856 2.3067
0.59872 3.404
0.59888 4.8753
0.59904 7.0074
0.5992 9.7963
0.59936 13.0756
0.59952 16.6624
0.59968 20.3573
0.59984 23.9525
0.6 27.3855
0.60016 30.6399
0.60032 33.6947
0.60048 36.5542
0.60064 38.9898
0.6008 41.1843
0.60096 43.2001
0.60112 44.9956
0.60128 46.4918
0.60144 47.269
0.6016 47.6473
0.60176 47.8468
0.60192 47.9257
0.60208 47.9008
0.60224 47.3854
0.6024 46.7329
0.60256 46.1136
0.60272 45.5026
0.60288 44.8584
0.60304 43.7986
0.6032 42.6265
0.60336 41.5251
0.60352 40.5193
0.60368 39.5924
0.60384 38.4786
0.604 37.2982
0.60416 36.0929
0.60432 34.8003
0.60448 33.3664
0.60464 31.6706
0.6048 29.7671
0.60496 27.7596
0.60512 25.748
0.60528 23.8527
0.60544 22.1653
0.6056 20.7023
0.60576 19.5219
0.60592 18.674
0.60608 18.1628
0.60624 17.9051
0.6064 17.7139
0.60656 17.5228
0.60672 17.3149
0.60688 17.0822
0.60704 16.7206
0.6072 16.1928
0.60736 15.6275
0.60752 15.1662
0.60768 14.9168
0.60784 14.7547
0.608 14.6092
0.60816 14.4679
0.60832 14.2767
0.60848 13.9484
0.60864 13.1878
0.6088 11.8786
0.60896 10.0457
0.60912 7.7223
0.60928 4.9875
0.60944 1.8786
0.6096 -1.4588
0.60976 -4.8296
0.60992 -8.059
0.61008 -11.0141
0.61024 -13.5369
0.6104 -15.7605
0.61056 -17.8262
0.61072 -19.8336
0.61088 -21.8369
0.61104 -23.595
0.6112 -25.2866
0.61136 -26.9325
0.61152 -28.408
0.61168 -29.5634
0.61184 -29.9333
0.612 -29.8336
0.61216 -29.5218
0.61232 -29.1436
0.61248 -28.7987
0.61264 -28.1794
0.6128 -27.6224
0.61296 -27.2608
0.61312 -26.9948
0.61328 -26.6956
0.61344 -25.9184
0.6136 -24.8419
0.61376 -23.5826
0.61392 -22.1694
0.61408 -20.6192
0.61424 -18.8029
0.6144 -16.8785
0.61456 -14.9376
0.61472 -13.0091
0.61488 -11.1097
0.61504 -9.2227
0.6152 -7.369
0.61536 -5.5943
0.61552 -3.9443
0.61568 -2.4647
0.61584 -1.2593
0.616 -0.216125
0.61616 0.6899375
0.61632 1.4464
0.61648 2.0158
0.61664 2.3317
0.6168 2.5228
0.61696 2.5935
0.61712 2.5021
0.61728 2.2236
0.61744 1.7789
0.6176 1.3716
0.61776 1.0848
0.61792 0.9600938
0.61808 1.0183
0.61824 1.3051
0.6184 1.8537
0.61856 2.5769
0.61872 3.3541
0.61888 4.0856
0.61904 4.7672
0.6192 5.4613
0.61936 6.2178
0.61952 7.1072
0.61968 8.2003
0.61984 9.5469
0.62 11.1471
0.62016 12.9343
0.62032 14.7879
0.62048 16.5918
0.62064 18.142
0.6208 19.4388
0.62096 20.5277
0.62112 21.4296
0.62128 22.1778
0.62144 22.61
0.6216 22.8885
0.62176 23.1088
0.62192 23.2833
0.62208 23.3914
0.62224 23.2043
0.6224 22.9176
0.62256 22.664
0.62272 22.477
0.62288 22.3606
0.62304 22.0822
0.6232 21.7497
0.62336 21.3922
0.62352 20.96
0.62368 20.3864
0.62384 19.4637
0.624 18.3
0.62416 16.9991
0.62432 15.6067
0.62448 14.1853
0.62464 12.7264
0.6248 11.305
0.62496 9.9833
0.62512 8.778
0.62528 7.6974
0.62544 6.7539
0.6256 5.8645
0.62576 5.0041
0.62592 4.1687
0.62608 3.379
0.62624 2.714
0.6264 2.1155
0.62656 1.6043
0.62672 1.2427
0.62688 1.0723
0.62704 1.1679
0.6272 1.384
0.62736 1.6417
0.62752 1.8911
0.62768 2.0906
0.62784 2.2153
0.628 2.1612
0.62816 1.9285
0.62832 1.5794
0.62848 1.1679
0.62864 0.7564375
0.6288 0.3325
0.62896 -0.09975
0.62912 -0.532
0.62928 -0.9559375
0.62944 -1.3716
0.6296 -1.7997
0.62976 -2.2402
0.62992 -2.6766
0.63008 -3.0923
0.63024 -3.4622
0.6304 -3.8321
0.63056 -4.2352
0.63072 -4.6841
0.63088 -5.187
0.63104 -5.6733
0.6312 -6.1554
0.63136 -6.6209
0.63152 -7.0407
0.63168 -7.3773
0.63184 -7.5145
0.632 -7.5311
0.63216 -7.4979
0.63232 -7.4563
0.63248 -7.4438
0.63264 -7.369
0.6328 -7.315
0.63296 -7.3233
0.63312 -7.3773
0.63328 -7.4355
0.63344 -7.3483
0.6336 -7.1197
0.63376 -6.7498
0.63392 -6.2136
0.63408 -5.5153
0.63424 -4.6259
0.6344 -3.6533
0.63456 -2.7099
0.63472 -1.8745
0.63488 -1.2012
0.63504 -0.69825
0.6352 -0.3200312
0.63536 0.0207812
0.63552 0.4073125
0.63568 0.914375
0.63584 1.5212
0.636 2.2402
0.63616 2.9967
0.63632 3.6783
0.63648 4.1687
0.63664 4.3308
0.6368 4.2269
0.63696 3.9526
0.63712 3.5993
0.63728 3.2627
0.63744 2.9551
0.6376 2.793
0.63776 2.8013
0.63792 2.9385
0.63808 3.138
0.63824 3.2668
0.6384 3.3375
0.63856 3.3499
0.63872 3.3125
0.63888 3.2211
0.63904 3.0507
0.6392 2.847
0.63936 2.6309
0.63952 2.4106
0.63968 2.182
0.63984 1.9451
0.64 1.7082
0.64016 1.4921
0.64032 1.2884
0.64048 1.1056
0.64064 0.9393125
0.6408 0.76475
0.64096 0.56525
0.64112 0.3325
0.64128 0.0665
0.64144 -0.1911875
0.6416 -0.448875
0.64176 -0.69825
0.64192 -0.9019063
0.64208 -1.0432
0.64224 -1.0474
0.6424 -0.9393125
0.64256 -0.7315
0.64272 -0.4405625
0.64288 -0.0914375
0.64304 0.3034063
0.6432 0.6691563
0.64336 0.9434688
0.64352 1.0889
0.64368 1.0848
0.64384 0.947625
0.644 0.714875
0.64416 0.4364063
0.64432 0.1870313
0.64448 0.0207812
0.64464 0.0124688
0.6448 0.1537812
0.64496 0.399
0.64512 0.6940938
0.64528 0.9933438
0.64544 1.2677
0.6456 1.5046
0.64576 1.7041
0.64592 1.8953
0.64608 2.0781
0.64624 2.261
0.6464 2.4189
0.64656 2.4938
0.64672 2.4314
0.64688 2.207
0.64704 1.8578
0.6472 1.4672
0.64736 1.1305
0.64752 0.947625
0.64768 1.0058
0.64784 1.3799
0.648 2.0075
0.64816 2.7681
0.64832 3.5203
0.64848 4.1355
0.64864 4.5178
0.6488 4.6633
0.64896 4.6134
0.64912 4.4555
0.64928 4.2893
0.64944 4.1937
0.6496 4.2435
0.64976 4.4513
0.64992 4.788
0.65008 5.2036
0.65024 5.6317
0.6504 6.0557
0.65056 6.4796
0.65072 6.9035
0.65088 7.3358
0.65104 7.6766
0.6512 7.9551
0.65136 8.192
0.65152 8.3748
0.65168 8.4954
0.65184 8.3832
0.652 8.1421
0.65216 7.8761
0.65232 7.6433
0.65248 7.4771
0.65264 7.236
0.6528 6.9908
0.65296 6.783
0.65312 6.6084
0.65328 6.4297
0.65344 6.064
0.6536 5.5361
0.65376 4.9127
0.65392 4.1978
0.65408 3.404
0.65424 2.4148
0.6544 1.2718
0.65456 0.0083125
0.65472 -1.3549
0.65488 -2.8138
0.65504 -4.3516
0.6552 -5.931
0.65536 -7.4979
0.65552 -8.9858
0.65568 -10.3283
0.65584 -11.4214
0.656 -12.3524
0.65616 -13.1919
0.65632 -13.9733
0.65648 -14.7131
0.65664 -15.2908
0.6568 -15.8145
0.65696 -16.305
0.65712 -16.7164
0.65728 -16.9824
0.65744 -16.9243
0.6576 -16.7123
0.65776 -16.4587
0.65792 -16.2094
0.65808 -16.0016
0.65824 -15.7231
0.6584 -15.4945
0.65856 -15.3324
0.65872 -15.1745
0.65888 -14.9625
0.65904 -14.5801
0.6592 -14.1146
0.65936 -13.6408
0.65952 -13.2044
0.65968 -12.8428
0.65984 -12.5394
0.66 -12.3108
0.66016 -12.1404
0.66032 -11.9742
0.66048 -11.7663
0.66064 -11.4546
0.6608 -11.0432
0.66096 -10.5652
0.66112 -10.0831
0.66128 -9.6633
0.66144 -9.3349
0.6616 -9.1313
0.66176 -9.0814
0.66192 -9.2019
0.66208 -9.4846
0.66224 -9.8171
0.6624 -10.1288
0.66256 -10.3906
0.66272 -10.5693
0.66288 -10.64
0.66304 -10.4821
0.6632 -10.1205
0.66336 -9.6508
0.66352 -9.1354
0.66368 -8.6325
0.66384 -8.0798
0.664 -7.502
0.66416 -6.8994
0.66432 -6.251
0.66448 -5.5153
0.66464 -4.6508
0.6648 -3.6533
0.66496 -2.5602
0.66512 -1.4173
0.66528 -0.2784688
0.66544 0.7896875
0.6656 1.8288
0.66576 2.8678
0.66592 3.9318
0.66608 5.0374
0.66624 6.0432
0.6664 7.0033
0.66656 7.9301
0.66672 8.7655
0.66688 9.4388
0.66704 9.6882
0.6672 9.6508
0.66736 9.4305
0.66752 9.0773
0.66768 8.6325
0.66784 7.9301
0.668 7.1737
0.66816 6.4921
0.66832 5.9143
0.66848 5.4281
0.66864 4.842
0.6688 4.2269
0.66896 3.6201
0.66912 2.9925
0.66928 2.3109
0.66944 1.4381
0.6696 0.43225
0.66976 -0.648375
0.66992 -1.7747
0.67008 -2.9094
0.67024 -4.0565
0.6704 -5.1953
0.67056 -6.2884
0.67072 -7.3233
0.67088 -8.2834
0.67104 -9.1188
0.6712 -9.9334
0.67136 -10.7813
0.67152 -11.6624
0.67168 -12.5477
0.67184 -13.3
0.672 -13.9608
0.67216 -14.4928
0.67232 -14.8295
0.67248 -14.9126
0.67264 -14.6633
0.6728 -14.285
0.67296 -13.9442
0.67312 -13.7447
0.67328 -13.7821
0.67344 -14.0066
0.6736 -14.4596
0.67376 -15.0373
0.67392 -15.5735
0.67408 -15.9184
0.67424 -15.9018
0.6744 -15.5943
0.67456 -15.0872
0.67472 -14.497
0.67488 -13.9484
0.67504 -13.5244
0.6752 -13.3208
0.67536 -13.3291
0.67552 -13.4995
0.67568 -13.7613
0.67584 -14.019
0.676 -14.2476
0.67616 -14.443
0.67632 -14.63
0.67648 -14.8378
0.67664 -15.0415
0.6768 -15.2534
0.67696 -15.4779
0.67712 -15.7065
0.67728 -15.9101
0.67744 -16.014
0.6776 -16.0348
0.67776 -16.0182
0.67792 -15.9891
0.67808 -15.9642
0.67824 -15.8686
0.6784 -15.7189
0.67856 -15.5319
0.67872 -15.2908
0.67888 -14.9833
0.67904 -14.5261
0.6792 -13.9525
0.67936 -13.3042
0.67952 -12.6059
0.67968 -11.8827
0.67984 -11.1429
0.68 -10.4197
0.68016 -9.7547
0.68032 -9.1562
0.68048 -8.6325
0.68064 -8.2003
0.6808 -7.8137
0.68096 -7.4189
0.68112 -6.9908
0.68128 -6.5087
0.68144 -6.064
0.6816 -5.6068
0.68176 -5.1413
0.68192 -4.7049
0.68208 -4.3308
0.68224 -4.177
0.6824 -4.1479
0.68256 -4.1729
0.68272 -4.2144
0.68288 -4.2477
0.68304 -4.3765
0.6832 -4.5262
0.68336 -4.6924
0.68352 -4.9085
0.68368 -5.2203
0.68384 -5.7398
0.684 -6.3882
0.68416 -7.0739
0.68432 -7.7473
0.68448 -8.3665
0.68464 -8.9609
0.6848 -9.5344
0.68496 -10.1205
0.68512 -10.7647
0.68528 -11.5003
0.68544 -12.3108
0.6856 -13.1421
0.68576 -13.9068
0.68592 -14.5053
0.68608 -14.8544
0.68624 -14.8544
0.6864 -14.6217
0.68656 -14.2892
0.68672 -13.9858
0.68688 -13.8362
0.68704 -13.8528
0.6872 -14.1188
0.68736 -14.6009
0.68752 -15.2119
0.68768 -15.8187
0.68784 -16.2343
0.688 -16.4463
0.68816 -16.4754
0.68832 -16.3299
0.68848 -16.0473
0.68864 -15.5943
0.6888 -15.0955
0.68896 -14.6258
0.68912 -14.2227
0.68928 -13.8985
0.68944 -13.62
0.6896 -13.3956
0.68976 -13.2044
0.68992 -13.0215
0.69008 -12.822
0.69024 -12.5727
0.6904 -12.2983
0.69056 -12.0365
0.69072 -11.837
0.69088 -11.7206
0.69104 -11.6333
0.6912 -11.5378
0.69136 -11.3964
0.69152 -11.1429
0.69168 -10.7564
0.69184 -10.1994
0.692 -9.5552
0.69216 -8.9609
0.69232 -8.566
0.69248 -8.4788
0.69264 -8.6907
0.6928 -9.1271
0.69296 -9.6841
0.69312 -10.2202
0.69328 -10.586
0.69344 -10.5818
0.6936 -10.1828
0.69376 -9.4887
0.69392 -8.6117
0.69408 -7.6766
0.69424 -6.7207
0.6944 -5.8021
0.69456 -4.9626
0.69472 -4.1687
0.69488 -3.3873
0.69504 -2.581
0.6952 -1.7207
0.69536 -0.8270938
0.69552 0.0457188
0.69568 0.8686563
0.69584 1.5711
0.696 2.2028
0.69616 2.8138
0.69632 3.4289
0.69648 4.0856
0.69664 4.7173
0.6968 5.3865
0.69696 6.0806
0.69712 6.7456
0.69728 7.3233
0.69744 7.635
0.6976 7.768
0.69776 7.7805
0.69792 7.6891
0.69808 7.5062
0.69824 7.1238
0.6984 6.6791
0.69856 6.2344
0.69872 5.8104
0.69888 5.4114
0.69904 4.9792
0.6992 4.6051
0.69936 4.3474
0.69952 4.2228
0.69968 4.2352
0.69984 4.3017
0.7 4.4056
0.70016 4.4846
0.70032 4.4763
0.70048 4.3225
0.70064 3.9443
0.7008 3.379
0.70096 2.6933
0.70112 1.9493
0.70128 1.2302
0.70144 0.5860313
0.7016 0.03325
0.70176 -0.4114688
0.70192 -0.7605938
0.70208 -1.0141
0.70224 -1.1845
0.7024 -1.3009
0.70256 -1.3508
0.70272 -1.2968
0.70288 -1.1263
0.70304 -0.8728125
0.7032 -0.5985
0.70336 -0.3325
0.70352 -0.1205313
0.70368 -0.0041563
0.70384 -0.0872813
0.704 -0.3699063
0.70416 -0.8104688
0.70432 -1.3633
0.70448 -1.9701
0.70464 -2.6184
0.7048 -3.2834
0.70496 -3.936
0.70512 -4.5636
0.70528 -5.1704
0.70544 -5.7523
0.7056 -6.3466
0.70576 -6.9742
0.70592 -7.635
0.70608 -8.3333
0.70624 -8.9983
0.7064 -9.6591
0.70656 -10.3241
0.70672 -10.9642
0.70688 -11.5627
0.70704 -11.9783
0.7072 -12.2776
0.70736 -12.4978
0.70752 -12.6558
0.70768 -12.7514
0.70784 -12.6516
0.708 -12.4563
0.70816 -12.2443
0.70832 -12.0157
0.70848 -11.7663
0.70864 -11.3715
0.7088 -10.9226
0.70896 -10.4862
0.70912 -10.0706
0.70928 -9.6716
0.70944 -9.1812
0.7096 -8.6284
0.70976 -8.0257
0.70992 -7.3441
0.71008 -6.5835
0.71024 -5.7107
0.7104 -4.7797
0.71056 -3.8612
0.71072 -3.0133
0.71088 -2.2859
0.71104 -1.7165
0.7112 -1.2552
0.71136 -0.8603438
0.71152 -0.4862813
0.71168 -0.09975
0.71184 0.2784688
0.712 0.7107188
0.71216 1.1887
0.71232 1.6542
0.71248 2.0449
0.71264 2.2693
0.7128 2.3732
0.71296 2.3815
0.71312 2.3026
0.71328 2.1654
0.71344 1.9742
0.7136 1.8329
0.71376 1.7997
0.71392 1.8828
0.71408 2.0657
0.71424 2.3026
0.7144 2.5852
0.71456 2.8595
0.71472 3.0715
0.71488 3.1795
0.71504 3.1546
0.7152 3.059
0.71536 2.9759
0.71552 2.98
0.71568 3.1255
0.71584 3.4164
0.716 3.8445
0.71616 4.3433
0.71632 4.8379
0.71648 5.241
0.71664 5.4281
0.7168 5.4488
0.71696 5.3782
0.71712 5.3075
0.71728 5.3034
0.71744 5.3075
0.7176 5.3533
0.71776 5.4156
0.71792 5.4405
0.71808 5.3616
0.71824 5.0457
0.7184 4.5677
0.71856 4.0482
0.71872 3.5827
0.71888 3.2502
0.71904 3.0008
0.7192 2.8387
0.71936 2.7223
0.71952 2.5561
0.71968 2.2402
0.71984 1.6459
0.72 0.7730625
0.72016 -0.3200312
0.72032 -1.5669
0.72048 -2.8678
0.72064 -4.1105
0.7208 -5.2203
0.72096 -6.1471
0.72112 -6.8578
0.72128 -7.3524
0.72144 -7.5935
0.7216 -7.7223
0.72176 -7.8595
0.72192 -8.0798
0.72208 -8.4123
0.72224 -8.7364
0.7224 -9.0856
0.72256 -9.4139
0.72272 -9.6176
0.72288 -9.6134
0.72304 -9.2684
0.7232 -8.7323
0.72336 -8.167
0.72352 -7.7098
0.72368 -7.4646
0.72384 -7.4064
0.724 -7.581
0.72416 -7.9135
0.72432 -8.2668
0.72448 -8.4912
0.72464 -8.3873
0.7248 -7.9592
0.72496 -7.2818
0.72512 -6.4422
0.72528 -5.5403
0.72544 -4.6592
0.7256 -3.8695
0.72576 -3.2086
0.72592 -2.6683
0.72608 -2.2236
0.72624 -1.8828
0.7264 -1.6002
0.72656 -1.3633
0.72672 -1.1845
0.72688 -1.0765
0.72704 -1.064
0.7272 -1.0889
0.72736 -1.1139
0.72752 -1.118
0.72768 -1.0806
0.72784 -0.9975
0.728 -0.8395625
0.72816 -0.6068125
0.72832 -0.3366563
0.72848 -0.0623438
0.72864 0.1537812
0.7288 0.3075625
0.72896 0.36575
0.72912 0.2909375
0.72928 0.0706562
0.72944 -0.266
0.7296 -0.6400625
0.72976 -0.9517813
0.72992 -1.1222
0.73008 -1.1014
0.73024 -0.8561875
0.7304 -0.4364063
0.73056 0.0706562
0.73072 0.5694063
0.73088 0.980875
0.73104 1.2427
0.7312 1.3882
0.73136 1.5087
0.73152 1.6916
0.73168 2.0199
0.73184 2.4647
0.732 3.0091
0.73216 3.5619
0.73232 4.0108
0.73248 4.2394
0.73264 4.1355
0.7328 3.7531
0.73296 3.2211
0.73312 2.6642
0.73328 2.2111
0.73344 1.8953
0.7336 1.7747
0.73376 1.8288
0.73392 1.9742
0.73408 2.1072
0.73424 2.0781
0.7344 1.8495
0.73456 1.4381
0.73472 0.8603438
0.73488 0.1745625
0.73504 -0.5444688
0.7352 -1.2095
0.73536 -1.7248
0.73552 -2.0366
0.73568 -2.1322
0.73584 -2.0324
0.736 -1.8454
0.73616 -1.7124
0.73632 -1.7415
0.73648 -2.0158
0.73664 -2.5187
0.7368 -3.2211
0.73696 -4.0066
0.73712 -4.7257
0.73728 -5.2369
0.73744 -5.3907
0.7376 -5.2535
0.73776 -4.9418
0.73792 -4.5843
0.73808 -4.3017
0.73824 -4.123
0.7384 -4.0939
0.73856 -4.1853
0.73872 -4.2809
0.73888 -4.2809
0.73904 -4.0607
0.7392 -3.6575
0.73936 -3.1504
0.73952 -2.6351
0.73968 -2.207
0.73984 -1.9327
0.74 -1.8329
0.74016 -1.8786
0.74032 -1.9992
0.74048 -2.1072
0.74064 -2.128
0.7408 -2.0241
0.74096 -1.8038
0.74112 -1.4963
0.74128 -1.1471
0.74144 -0.83125
0.7416 -0.5527813
0.74176 -0.3241875
0.74192 -0.1454688
0.74208 -0.0207812
0.74224 0.03325
0.7424 0.0623438
0.74256 0.0623438
0.74272 0.0457188
0.74288 0.0083125
0.74304 -0.0665
0.7432 -0.1371563
0.74336 -0.1620938
0.74352 -0.1288437
0.74368 -0.0290938
0.74384 0.0914375
0.744 0.216125
0.74416 0.282625
0.74432 0.2410625
0.74448 0.0623438
0.74464 -0.2410625
0.7448 -0.6026563
0.74496 -0.9185313
0.74512 -1.1014
0.74528 -1.0973
0.74544 -0.8935938
0.7456 -0.5610938
0.74576 -0.2119688
0.74592 0.016625
0.74608 0.03325
0.74624 -0.2119688
0.7464 -0.6774688
0.74656 -1.2469
0.74672 -1.7623
0.74688 -2.0864
0.74704 -2.128
0.7472 -1.8703
0.74736 -1.3799
0.74752 -0.7605938
0.74768 -0.1371563
0.74784 0.3366563
0.748 0.5943438
0.74816 0.615125
0.74832 0.4280938
0.74848 0.0955938
0.74864 -0.3200312
0.7488 -0.7231875
0.74896 -1.0141
0.74912 -1.143
0.74928 -1.0973
0.74944 -0.9393125
0.7496 -0.7439688
0.74976 -0.6192813
0.74992 -0.665
0.75008 -0.9517813
0.75024 -1.4963
0.7504 -2.2278
0.75056 -3.0133
0.75072 -3.7115
0.75088 -4.1812
0.75104 -4.3391
0.7512 -4.2103
0.75136 -3.8903
0.75152 -3.5203
0.75168 -3.2336
0.75184 -3.113
0.752 -3.2003
0.75216 -3.4705
0.75232 -3.8279
0.75248 -4.177
0.75264 -4.389
0.7528 -4.4555
0.75296 -4.4181
0.75312 -4.335
0.75328 -4.2643
0.75344 -4.1978
0.7536 -4.1812
0.75376 -4.2144
0.75392 -4.256
0.75408 -4.2602
0.75424 -4.1479
0.7544 -3.9484
0.75456 -3.7032
0.75472 -3.4538
0.75488 -3.2377
0.75504 -3.0507
0.7552 -2.8844
0.75536 -2.7182
0.75552 -2.5021
0.75568 -2.2111
0.75584 -1.8204
0.756 -1.3591
0.75616 -0.881125
0.75632 -0.4364063
0.75648 -0.0748125
0.75664 0.1704063
0.7568 0.349125
0.75696 0.5070625
0.75712 0.7065625
0.75728 0.980875
0.75744 1.3134
0.7576 1.6791
0.75776 2.0075
0.75792 2.1945
0.75808 2.1696
0.75824 1.8786
0.7584 1.4048
0.75856 0.8686563
0.75872 0.3782188
0.75888 0.0457188
0.75904 -0.0581875
0.7592 0.0581875
0.75936 0.3449688
0.75952 0.6940938
0.75968 1.0017
0.75984 1.1804
0.76 1.2261
0.76016 1.1762
0.76032 1.0973
0.76048 1.0598
0.76064 1.1097
0.7608 1.2677
0.76096 1.517
0.76112 1.8038
0.76128 2.0698
0.76144 2.2402
0.7616 2.3067
0.76176 2.2859
0.76192 2.2194
0.76208 2.1446
0.76224 2.0657
0.7624 2.0158
0.76256 2.0116
0.76272 2.0449
0.76288 2.1072
0.76304 2.1322
0.7632 2.1363
0.76336 2.1322
0.76352 2.1322
0.76368 2.128
0.76384 2.0657
0.764 1.941
0.76416 1.7581
0.76432 1.5004
0.76448 1.1596
0.76464 0.7190313
0.7648 0.23275
0.76496 -0.2410625
0.76512 -0.665
0.76528 -0.9933438
0.76544 -1.2261
0.7656 -1.3965
0.76576 -1.5544
0.76592 -1.7539
0.76608 -2.0366
0.76624 -2.4148
0.7664 -2.8803
0.76656 -3.3749
0.76672 -3.8321
0.76688 -4.1895
0.76704 -4.3682
0.7672 -4.4098
0.76736 -4.3682
0.76752 -4.3017
0.76768 -4.256
0.76784 -4.2352
0.768 -4.2602
0.76816 -4.3183
0.76832 -4.3433
0.76848 -4.2851
0.76864 -4.0981
0.7688 -3.8321
0.76896 -3.5536
0.76912 -3.325
0.76928 -3.2003
0.76944 -3.2086
0.7696 -3.3541
0.76976 -3.6118
0.76992 -3.911
0.77008 -4.1937
0.77024 -4.3848
0.7704 -4.468
0.77056 -4.4597
0.77072 -4.3807
0.77088 -4.2809
0.77104 -4.1812
0.7712 -4.1147
0.77136 -4.1064
0.77152 -4.1479
0.77168 -4.2311
0.77184 -4.3017
0.772 -4.3516
0.77216 -4.3682
0.77232 -4.3474
0.77248 -4.2768
0.77264 -4.1188
0.7728 -3.8986
0.77296 -3.6575
0.77312 -3.4206
0.77328 -3.2294
0.77344 -3.0715
0.7736 -2.9759
0.77376 -2.9593
0.77392 -3.0258
0.77408 -3.1504
0.77424 -3.271
0.7744 -3.3499
0.77456 -3.3873
0.77472 -3.3583
0.77488 -3.2377
0.77504 -2.9634
0.7752 -2.5228
0.77536 -1.9119
0.77552 -1.1388
0.77568 -0.2369063
0.77584 0.714875
0.776 1.6334
0.77616 2.4065
0.77632 2.9426
0.77648 3.1795
0.77664 3.0715
0.7768 2.7514
0.77696 2.3898
0.77712 2.128
0.77728 2.0906
0.77744 2.2901
0.7776 2.7182
0.77776 3.2793
0.77792 3.8154
0.77808 4.1978
0.77824 4.256
0.7784 4.0648
0.77856 3.7406
0.77872 3.4164
0.77888 3.2128
0.77904 3.1297
0.7792 3.1878
0.77936 3.3167
0.77952 3.3832
0.77968 3.2585
0.77984 2.8346
0.78 2.1737
0.78016 1.3965
0.78032 0.6608438
0.78048 0.09975
0.78064 -0.1953438
0.7808 -0.2535313
0.78096 -0.149625
0.78112 -0.016625
0.78128 0.016625
0.78144 -0.1288437
0.7816 -0.482125
0.78176 -0.9850312
0.78192 -1.5378
0.78208 -2.0241
0.78224 -2.3358
0.7824 -2.4688
0.78256 -2.448
0.78272 -2.3275
0.78288 -2.1654
0.78304 -2.0158
0.7832 -1.941
0.78336 -1.9493
0.78352 -2.0158
0.78368 -2.1031
0.78384 -2.1612
0.784 -2.1903
0.78416 -2.1945
0.78432 -2.1737
0.78448 -2.1363
0.78464 -2.0989
0.7848 -2.074
0.78496 -2.0657
0.78512 -2.0781
0.78528 -2.1114
0.78544 -2.1737
0.7856 -2.2901
0.78576 -2.4813
0.78592 -2.7514
0.78608 -3.0964
0.78624 -3.4746
0.7864 -3.8362
0.78656 -4.123
0.78672 -4.2809
0.78688 -4.2809
0.78704 -4.0856
0.7872 -3.7822
0.78736 -3.4663
0.78752 -3.2377
0.78768 -3.1754
0.78784 -3.2751
0.788 -3.5203
0.78816 -3.8279
0.78832 -4.0981
0.78848 -4.2435
0.78864 -4.1812
0.7888 -3.9526
0.78896 -3.6492
0.78912 -3.3707
0.78928 -3.2086
0.78944 -3.1712
0.7896 -3.2502
0.78976 -3.3666
0.78992 -3.404
0.79008 -3.2627
0.79024 -2.8678
0.7904 -2.2527
0.79056 -1.5087
0.79072 -0.7564375
0.79088 -0.1246875
0.79104 0.2701562
0.7912 0.43225
0.79136 0.3948438
0.79152 0.2369063
0.79168 0.0415625
0.79184 -0.1288437
0.792 -0.2244375
0.79216 -0.216125
0.79232 -0.1371563
0.79248 -0.0249375
0.79264 0.0457188
0.7928 0.083125
0.79296 0.0872813
0.79312 0.0581875
0.79328 0.0124688
0.79344 -0.0581875
0.7936 -0.116375
0.79376 -0.133
0.79392 -0.09975
0.79408 -0.0207812
0.79424 0.049875
0.7944 0.116375
0.79456 0.1537812
0.79472 0.133
0.79488 0.03325
0.79504 -0.149625
0.7952 -0.3948438
0.79536 -0.6566875
0.79552 -0.8852813
0.79568 -1.0391
0.79584 -1.0973
0.796 -1.0806
0.79616 -1.0307
0.79632 -1.0017
0.79648 -1.0391
0.79664 -1.1679
0.7968 -1.3799
0.79696 -1.6376
0.79712 -1.8911
0.79728 -2.0906
0.79744 -2.1903
0.7976 -2.2111
0.79776 -2.1779
0.79792 -2.1363
0.79808 -2.1238
0.79824 -2.1405
0.7984 -2.1945
0.79856 -2.2444
0.79872 -2.2485
0.79888 -2.1654
0.79904 -1.9617
0.7992 -1.6874
0.79936 -1.4048
0.79952 -1.1762
0.79968 -1.0682
0.79984 -1.1056
0.8 -1.2801
0.80016 -1.542
0.80032 -1.8288
0.80048 -2.0781
0.80064 -2.2236
0.8008 -2.2693
0.80096 -2.2402
0.80112 -2.182
0.80128 -2.1322
0.80144 -2.1114
0.8016 -2.1238
0.80176 -2.1571
0.80192 -2.1737
0.80208 -2.1446
0.80224 -2.0366
0.8024 -1.8495
0.80256 -1.6126
0.80272 -1.3549
0.80288 -1.1139
0.80304 -0.9393125
0.8032 -0.8437188
0.80336 -0.8354063
0.80352 -0.9060625
0.80368 -1.0266
0.80384 -1.1638
0.804 -1.276
0.80416 -1.3217
0.80432 -1.2718
0.80448 -1.1139
0.80464 -0.8645
0.8048 -0.56525
0.80496 -0.2784688
0.80512 -0.0706562
0.80528 0.0041563
0.80544 -0.0748125
0.8056 -0.282625
0.80576 -0.5527813
0.80592 -0.8229375
0.80608 -1.0266
0.80624 -1.1347
0.8064 -1.1513
0.80656 -1.1056
0.80672 -1.0598
0.80688 -1.0515
0.80704 -1.1097
0.8072 -1.2053
0.80736 -1.2843
0.80752 -1.276
0.80768 -1.1263
0.80784 -0.8104688
0.808 -0.3574375
0.80816 0.149625
0.80832 0.63175
0.80848 0.9975
0.80864 1.1471
0.8088 1.0806
0.80896 0.8354063
0.80912 0.4862813
0.80928 0.0955938
0.80944 -0.2784688
0.8096 -0.5860313
0.80976 -0.8104688
0.80992 -0.9559375
0.81008 -1.0474
0.81024 -1.0973
0.8104 -1.1222
0.81056 -1.1263
0.81072 -1.1097
0.81088 -1.0723
0.81104 -1.0307
0.8112 -0.9891875
0.81136 -0.9684063
0.81152 -0.9767187
0.81168 -1.0391
0.81184 -1.1638
0.812 -1.3466
0.81216 -1.5711
0.81232 -1.8204
0.81248 -2.0698
0.81264 -2.2568
0.8128 -2.3732
0.81296 -2.4065
0.81312 -2.3441
0.81328 -2.182
0.81344 -1.9368
0.8136 -1.6625
0.81376 -1.4007
0.81392 -1.1928
0.81408 -1.0723
0.81424 -1.0765
0.8144 -1.2012
0.81456 -1.4256
0.81472 -1.7248
0.81488 -2.049
0.81504 -2.3317
0.8152 -2.5228
0.81536 -2.5769
0.81552 -2.4688
0.81568 -2.2111
0.81584 -1.8371
0.816 -1.4422
0.81616 -1.1222
0.81632 -0.9559375
0.81648 -1.01
0.81664 -1.2968
0.8168 -1.7539
0.81696 -2.2818
0.81712 -2.7764
0.81728 -3.1297
0.81744 -3.2668
0.8176 -3.1837
0.81776 -2.9218
0.81792 -2.5644
0.81808 -2.207
0.81824 -1.9119
0.8184 -1.7456
0.81856 -1.7373
0.81872 -1.8578
0.81888 -2.0698
0.81904 -2.2943
0.8192 -2.4688
0.81936 -2.527
0.81952 -2.4397
0.81968 -2.2028
0.81984 -1.8537
0.82 -1.4672
0.82016 -1.1471
0.82032 -0.980875
0.82048 -1.0183
0.82064 -1.2552
0.8208 -1.6168
0.82096 -1.9701
0.82112 -2.1903
0.82128 -2.1737
0.82144 -1.8745
0.8216 -1.3633
0.82176 -0.7689063
0.82192 -0.2701562
0.82208 -0.0124688
0.82224 -0.0872813
0.8224 -0.4613438
0.82256 -1.0183
0.82272 -1.6002
0.82288 -2.049
0.82304 -2.2402
0.8232 -2.1571
0.82336 -1.8662
0.82352 -1.4838
0.82368 -1.1305
0.82384 -0.9019063
0.824 -0.8229375
0.82416 -0.8728125
0.82432 -0.9767187
0.82448 -1.0557
0.82464 -1.0266
0.8248 -0.8769688
0.82496 -0.6275938
0.82512 -0.3366563
0.82528 -0.0581875
0.82544 0.133
0.8256 0.2244375
0.82576 0.2202813
0.82592 0.1413125
0.82608 0.0290938
0.82624 -0.083125
0.8264 -0.1620938
0.82656 -0.182875
0.82672 -0.1413125
0.82688 -0.03325
0.82704 0.1246875
0.8272 0.3325
0.82736 0.56525
0.82752 0.8021563
0.82768 1.0141
0.82784 1.1554
0.828 1.2261
0.82816 1.2344
0.82832 1.1887
0.82848 1.0931
0.82864 0.9268438
0.8288 0.7231875
0.82896 0.49875
0.82912 0.2701562
0.82928 0.049875
0.82944 -0.1454688
0.8296 -0.2909375
0.82976 -0.3366563
0.82992 -0.2618438
0.83008 -0.0665
0.83024 0.216125
0.8304 0.5403125
0.83056 0.83125
0.83072 1.0307
0.83088 1.0806
0.83104 0.9185313
0.8312 0.5735625
0.83136 0.1039063
0.83152 -0.4239375
0.83168 -0.9393125
0.83184 -1.3757
0.832 -1.7041
0.83216 -1.9202
0.83232 -2.049
0.83248 -2.1155
0.83264 -2.128
0.8328 -2.1238
0.83296 -2.1238
0.83312 -2.128
0.83328 -2.128
0.83344 -2.1155
0.8336 -2.1031
0.83376 -2.0948
0.83392 -2.1031
0.83408 -2.1197
0.83424 -2.1446
0.8344 -2.1737
0.83456 -2.1987
0.83472 -2.1945
0.83488 -2.1488
0.83504 -2.0241
0.8352 -1.8495
0.83536 -1.6293
0.83552 -1.3799
0.83568 -1.1222
0.83584 -0.9019063
0.836 -0.7522813
0.83616 -0.7024063
0.83632 -0.7772188
0.83648 -0.9891875
0.83664 -1.3342
0.8368 -1.7706
0.83696 -2.2444
0.83712 -2.7057
0.83728 -3.1047
0.83744 -3.379
0.8376 -3.512
0.83776 -3.5203
0.83792 -3.4206
0.83808 -3.2419
0.83824 -3.005
0.8384 -2.7514
0.83856 -2.5145
0.83872 -2.3109
0.83888 -2.1571
0.83904 -2.0407
0.8392 -1.9742
0.83936 -1.9659
0.83952 -2.0116
0.83968 -2.0989
0.83984 -2.1987
0.84 -2.2859
0.84016 -2.3275
0.84032 -2.2943
0.84048 -2.1696
0.84064 -1.9617
0.8408 -1.6999
0.84096 -1.4381
0.84112 -1.2219
0.84128 -1.0848
0.84144 -1.0391
0.8416 -1.0682
0.84176 -1.1139
0.84192 -1.1347
0.84208 -1.0889
0.84224 -0.9517813
0.8424 -0.7356563
0.84256 -0.4655
0.84272 -0.2119688
0.84288 -0.0249375
0.84304 0.016625
0.8432 -0.0748125
0.84336 -0.2950937
0.84352 -0.6068125
0.84368 -0.9684063
0.84384 -1.3342
0.844 -1.6542
0.84416 -1.8953
0.84432 -2.0532
0.84448 -2.1238
0.84464 -2.0989
0.8448 -1.9867
0.84496 -1.783
0.84512 -1.5004
0.84528 -1.1554
0.84544 -0.781375
0.8456 -0.4280938
0.84576 -0.1537812
0.84592 0
0.84608 0.016625
0.84624 -0.1122188
0.8464 -0.3532813
0.84656 -0.6275938
0.84672 -0.8728125
0.84688 -1.0349
0.84704 -1.0931
0.8472 -1.0682
0.84736 -1.0058
0.84752 -0.9767187
0.84768 -1.0307
0.84784 -1.1845
0.848 -1.4256
0.84816 -1.7041
0.84832 -1.9534
0.84848 -2.1072
0.84864 -2.1114
0.8488 -1.9659
0.84896 -1.7124
0.84912 -1.409
0.84928 -1.1222
0.84944 -0.9060625
0.8496 -0.798
0.84976 -0.798
0.84992 -0.8852813
0.85008 -1.0266
0.85024 -1.1554
0.8504 -1.2469
0.85056 -1.276
0.85072 -1.2302
0.85088 -1.1056
0.85104 -0.9102188
0.8512 -0.6733125
0.85136 -0.4280938
0.85152 -0.2078125
0.85168 -0.03325
0.85184 0.0623438
0.852 0.0955938
0.85216 0.083125
0.85232 0.0457188
0.85248 0.0041563
0.85264 -0.0290938
0.8528 -0.0457188
0.85296 -0.0374063
0.85312 -0.016625
0.85328 0
0.85344 -0.0124688
0.8536 -0.0415625
0.85376 -0.0623438
0.85392 -0.0540313
0.85408 -0.0124688
0.85424 0.0374063
0.8544 0.0955938
0.85456 0.1371563
0.85472 0.1205313
0.85488 0.03325
0.85504 -0.1413125
0.8552 -0.382375
0.85536 -0.6400625
0.85552 -0.8686563
0.85568 -1.0349
0.85584 -1.118
0.856 -1.1263
0.85616 -1.0931
0.85632 -1.0515
0.85648 -1.0515
0.85664 -1.1263
0.8568 -1.2926
0.85696 -1.5253
0.85712 -1.7955
0.85728 -2.0657
0.85744 -2.2735
0.8576 -2.3982
0.85776 -2.4231
0.85792 -2.3483
0.85808 -2.1779
0.85824 -1.9368
0.8584 -1.6667
0.85856 -1.4131
0.85872 -1.2053
0.85888 -1.0765
0.85904 -1.0557
0.8592 -1.1513
0.85936 -1.3591
0.85952 -1.6625
0.85968 -2.0283
0.85984 -2.4023
0.86 -2.7348
0.86016 -2.9967
0.86032 -3.1546
0.86048 -3.2003
0.86064 -3.1089
0.8608 -2.9177
0.86096 -2.6766
0.86112 -2.4148
0.86128 -2.1779
0.86144 -2.0033
0.8616 -1.916
0.86176 -1.9119
0.86192 -1.9825
0.86208 -2.0948
0.86224 -2.207
0.8624 -2.2818
0.86256 -2.3109
0.86272 -2.2693
0.86288 -2.1612
0.86304 -1.9784
0.8632 -1.7456
0.86336 -1.4963
0.86352 -1.2677
0.86368 -1.0931
0.86384 -1.0141
0.864 -1.0598
0.86416 -1.2427
0.86432 -1.5669
0.86448 -2.0033
0.86464 -2.4938
0.8648 -2.9468
0.86496 -3.271
0.86512 -3.3957
0.86528 -3.2627
0.86544 -2.8595
0.8656 -2.2402
0.86576 -1.5087
0.86592 -0.7730625
0.86608 -0.133
0.86624 0.3117188
0.8664 0.5278438
0.86656 0.5278438
0.86672 0.3574375
0.86688 0.0789688
0.86704 -0.2576875
0.8672 -0.5735625
0.86736 -0.8187813
0.86752 -0.980875
0.86768 -1.0557
0.86784 -1.064
0.868 -1.0391
0.86816 -1.01
0.86832 -1.01
0.86848 -1.0474
0.86864 -1.1139
0.8688 -1.1887
0.86896 -1.2302
0.86912 -1.2095
0.86928 -1.1014
0.86944 -0.9102188
0.8696 -0.6566875
0.86976 -0.382375
0.86992 -0.149625
0.87008 -0.0124688
0.87024 -0.016625
0.8704 -0.1579375
0.87056 -0.4073125
0.87072 -0.7065625
0.87088 -0.9975
0.87104 -1.2136
0.8712 -1.3217
0.87136 -1.3342
0.87152 -1.251
0.87168 -1.1056
0.87184 -0.9185313
0.872 -0.7107188
0.87216 -0.4945938
0.87232 -0.2701562
0.87248 -0.049875
0.87264 0.1371563
0.8728 0.2701562
0.87296 0.315875
0.87312 0.2535313
0.87328 0.0623438
0.87344 -0.23275
0.8736 -0.5735625
0.87376 -0.8728125
0.87392 -1.0598
0.87408 -1.0848
0.87424 -0.9226875
0.8744 -0.6400625
0.87456 -0.3283438
0.87472 -0.083125
0.87488 0.0041563
0.87504 -0.083125
0.8752 -0.3241875
0.87536 -0.63175
0.87552 -0.9019063
0.87568 -1.0515
0.87584 -1.0183
0.876 -0.814625
0.87616 -0.515375
0.87632 -0.216125
0.87648 -0.0207812
0.87664 0
0.8768 -0.1537812
0.87696 -0.43225
0.87712 -0.7522813
0.87728 -1.0141
0.87744 -1.1305
0.8776 -1.0682
0.87776 -0.8437188
0.87792 -0.49875
0.87808 -0.09975
0.87824 0.2618438
0.8784 0.5527813
0.87856 0.781375
0.87872 0.9393125
0.87888 1.0432
0.87904 1.064
0.8792 0.9891875
0.87936 0.814625
0.87952 0.5195313
0.87968 0.1122188
0.87984 -0.3699063
0.88 -0.8395625
0.88016 -1.1762
0.88032 -1.2884
0.88048 -1.1388
0.88064 -0.7564375
0.8808 -0.266
0.88096 0.149625
0.88112 0.3200312
0.88128 0.116375
0.88144 -0.515375
0.8816 -1.463
0.88176 -2.5353
0.88192 -3.5037
0.88208 -4.1563
0.88224 -4.335
0.8824 -4.069
0.88256 -3.5037
0.88272 -2.8304
0.88288 -2.2402
0.88304 -1.8537
0.8832 -1.7332
0.88336 -1.8246
0.88352 -1.9992
0.88368 -2.1155
0.88384 -2.074
0.884 -1.8662
0.88416 -1.5586
0.88432 -1.2635
0.88448 -1.0806
0.88464 -1.0848
0.8848 -1.2718
0.88496 -1.5752
0.88512 -1.8828
0.88528 -2.0989
0.88544 -2.128
0.8856 -1.9784
0.88576 -1.7041
0.88592 -1.3882
0.88608 -1.118
0.88624 -0.9517813
0.8864 -0.89775
0.88656 -0.931
0.88672 -0.9975
0.88688 -1.0557
0.88704 -1.0723
0.8872 -1.0474
0.88736 -1.0058
0.88752 -0.9891875
0.88768 -1.0349
0.88784 -1.1721
0.888 -1.384
0.88816 -1.6417
0.88832 -1.8953
0.88848 -2.0906
0.88864 -2.1654
0.8888 -2.0948
0.88896 -1.8869
0.88912 -1.5669
0.88928 -1.1679
0.88944 -0.748125
0.8896 -0.3699063
0.88976 -0.0914375
0.88992 0.0457188
0.89008 0.0249375
0.89024 -0.1371563
0.8904 -0.399
0.89056 -0.6857813
0.89072 -0.9226875
0.89088 -1.0515
0.89104 -1.0307
0.8912 -0.8686563
0.89136 -0.6109688
0.89152 -0.315875
0.89168 -0.0540313
0.89184 0.116375
0.892 0.182875
0.89216 0.16625
0.89232 0.0955938
0.89248 0.016625
0.89264 -0.0457188
0.8928 -0.0706562
0.89296 -0.0540313
0.89312 -0.0249375
0.89328 0
0.89344 -0.016625
0.8936 -0.0540313
0.89376 -0.0872813
0.89392 -0.0789688
0.89408 -0.0207812
0.89424 0.0665
0.8944 0.16625
0.89456 0.2244375
0.89472 0.1953438
0.89488 0.0540313
0.89504 -0.2119688
0.8952 -0.5444688
0.89536 -0.8561875
0.89552 -1.0557
0.89568 -1.0848
0.89584 -0.9268438
0.896 -0.63175
0.89616 -0.3075625
0.89632 -0.0540313
0.89648 0.0124688
0.89664 -0.1371563
0.8968 -0.5029063
0.89696 -1.01
0.89712 -1.5544
0.89728 -2.0283
0.89744 -2.3358
0.8976 -2.4563
0.89776 -2.4231
0.89792 -2.3026
0.89808 -2.1571
0.89824 -2.0366
0.8984 -1.9867
0.89856 -2.0075
0.89872 -2.0698
0.89888 -2.1197
0.89904 -2.1072
0.8992 -2.0033
0.89936 -1.808
0.89952 -1.517
0.89968 -1.1596
0.89984 -0.7772188
0.9 -0.4280938
0.90016 -0.1579375
0.90032 -0.0041563
0.90048 0.0124688
0.90064 -0.0914375
0.9008 -0.3034063
0.90096 -0.5610938
0.90112 -0.8187813
0.90128 -1.0224
0.90144 -1.1513
0.9016 -1.1845
0.90176 -1.1554
0.90192 -1.0973
0.90208 -1.0598
0.90224 -1.0973
0.9024 -1.2219
0.90256 -1.4381
0.90272 -1.7248
0.90288 -2.0449
0.90304 -2.34
0.9032 -2.5478
0.90336 -2.6143
0.90352 -2.5062
0.90368 -2.2236
0.90384 -1.7789
0.904 -1.251
0.90416 -0.7273438
0.90432 -0.29925
0.90448 -0.03325
0.90464 0.0207812
0.9048 -0.1080625
0.90496 -0.3740625
0.90512 -0.69825
0.90528 -0.9975
0.90544 -1.2053
0.9056 -1.2884
0.90576 -1.2677
0.90592 -1.1845
0.90608 -1.0848
0.90624 -1.01
0.9064 -0.9850312
0.90656 -1.0017
0.90672 -1.0307
0.90688 -1.0598
0.90704 -1.0682
0.9072 -1.0515
0.90736 -1.0307
0.90752 -1.0224
0.90768 -1.0474
0.90784 -1.118
0.908 -1.1928
0.90816 -1.2427
0.90832 -1.2219
0.90848 -1.1056
0.90864 -0.8894375
0.9088 -0.615125
0.90896 -0.3366563
0.90912 -0.116375
0.90928 -0.0041563
0.90944 -0.049875
0.9096 -0.2285938
0.90976 -0.4904375
0.90992 -0.7730625
0.91008 -1.0141
0.91024 -1.1762
0.9104 -1.2386
0.91056 -1.2178
0.91072 -1.1513
0.91088 -1.0765
0.91104 -1.0349
0.9112 -1.0266
0.91136 -1.0432
0.91152 -1.064
0.91168 -1.0682
0.91184 -1.0515
0.912 -1.0183
0.91216 -0.9891875
0.91232 -0.9850312
0.91248 -1.0391
0.91264 -1.1638
0.9128 -1.3591
0.91296 -1.596
0.91312 -1.8495
0.91328 -2.0781
0.91344 -2.2319
0.9136 -2.3109
0.91376 -2.315
0.91392 -2.2568
0.91408 -2.1529
0.91424 -2.0407
0.9144 -1.9576
0.91456 -1.9285
0.91472 -1.9701
0.91488 -2.0864
0.91504 -2.2652
0.9152 -2.4938
0.91536 -2.7431
0.91552 -2.9759
0.91568 -3.1588
0.91584 -3.2294
0.916 -3.1754
0.91616 -2.9842
0.91632 -2.6642
0.91648 -2.2402
0.91664 -1.7623
0.9168 -1.3217
0.91696 -1.0058
0.91712 -0.881125
0.91728 -0.9975
0.91744 -1.3383
0.9176 -1.8288
0.91776 -2.3691
0.91792 -2.8387
0.91808 -3.1463
0.91824 -3.2253
0.9184 -3.0923
0.91856 -2.8096
0.91872 -2.4771
0.91888 -2.1862
0.91904 -1.9867
0.9192 -1.916
0.91936 -1.9493
0.91952 -2.0407
0.91968 -2.1155
0.91984 -2.1238
0.92 -2.0199
0.92016 -1.8038
0.92032 -1.4921
0.92048 -1.1471
0.92064 -0.8354063
0.9208 -0.6192813
0.92096 -0.5569375
0.92112 -0.6774688
0.92128 -0.9684063
0.92144 -1.3799
0.9216 -1.808
0.92176 -2.1529
0.92192 -2.3026
0.92208 -2.1945
0.92224 -1.8329
0.9224 -1.2884
0.92256 -0.7065625
0.92272 -0.2369063
0.92288 -0.0083125
0.92304 -0.09975
0.9232 -0.4696563
0.92336 -1.0141
0.92352 -1.5877
0.92368 -2.0449
0.92384 -2.261
0.924 -2.207
0.92416 -1.9368
0.92432 -1.5461
0.92448 -1.1513
0.92464 -0.847875
0.9248 -0.7024063
0.92496 -0.714875
0.92512 -0.8395625
0.92528 -1.0183
0.92544 -1.1845
0.9256 -1.2884
0.92576 -1.3092
0.92592 -1.2427
0.92608 -1.1056
0.92624 -0.9185313
0.9264 -0.7065625
0.92656 -0.4779688
0.92672 -0.2535313
0.92688 -0.0457188
0.92704 0.1122188
0.9272 0.2078125
0.92736 0.23275
0.92752 0.1745625
0.92768 0.0415625
0.92784 -0.1704063
0.928 -0.4239375
0.92816 -0.6733125
0.92832 -0.881125
0.92848 -1.0349
0.92864 -1.1305
0.9288 -1.1679
0.92896 -1.1554
0.92912 -1.118
0.92928 -1.0723
0.92944 -1.0474
0.9296 -1.0474
0.92976 -1.0598
0.92992 -1.0723
0.93008 -1.0682
0.93024 -1.0557
0.9304 -1.0349
0.93056 -1.0183
0.93072 -1.0141
0.93088 -1.0474
0.93104 -1.1347
0.9312 -1.2926
0.93136 -1.5087
0.93152 -1.7664
0.93168 -2.0532
0.93184 -2.3317
0.932 -2.5893
0.93216 -2.8221
0.93232 -3.0133
0.93248 -3.1588
0.93264 -3.2419
0.9328 -3.2751
0.93296 -3.271
0.93312 -3.2419
0.93328 -3.2003
0.93344 -3.138
0.9336 -3.0839
0.93376 -3.0673
0.93392 -3.0964
0.93408 -3.1671
0.93424 -3.2502
0.9344 -3.325
0.93456 -3.3666
0.93472 -3.3416
0.93488 -3.2336
0.93504 -3.0091
0.9352 -2.7265
0.93536 -2.4522
0.93552 -2.2444
0.93568 -2.1363
0.93584 -2.1322
0.936 -2.1945
0.93616 -2.2693
0.93632 -2.2776
0.93648 -2.1737
0.93664 -1.9451
0.9368 -1.6376
0.93696 -1.3258
0.93712 -1.1014
0.93728 -1.0474
0.93744 -1.1887
0.9376 -1.4713
0.93776 -1.7997
0.93792 -2.0573
0.93808 -2.1405
0.93824 -1.9825
0.9384 -1.6002
0.93856 -1.0723
0.93872 -0.5278438
0.93888 -0.0789688
0.93904 0.1454688
0.9392 0.1288437
0.93936 -0.0955938
0.93952 -0.482125
0.93968 -0.9434688
0.93984 -1.3882
0.94 -1.7415
0.94016 -1.9825
0.94032 -2.1072
0.94048 -2.1322
0.94064 -2.0615
0.9408 -1.9119
0.94096 -1.7041
0.94112 -1.4422
0.94128 -1.143
0.94144 -0.8187813
0.9416 -0.5070625
0.94176 -0.2535313
0.94192 -0.0789688
0.94208 -0.0041563
0.94224 -0.0207812
0.9424 -0.083125
0.94256 -0.1413125
0.94272 -0.1371563
0.94288 -0.0374063
0.94304 0.149625
0.9432 0.4073125
0.94336 0.681625
0.94352 0.914375
0.94368 1.0474
0.94384 1.0224
0.944 0.8520313
0.94416 0.5985
0.94432 0.3075625
0.94448 0.049875
0.94464 -0.1371563
0.9448 -0.2285938
0.94496 -0.2202813
0.94512 -0.1371563
0.94528 -0.0249375
0.94544 0.0623438
0.9456 0.1122188
0.94576 0.1246875
0.94592 0.0914375
0.94608 0.0207812
0.94624 -0.0872813
0.9464 -0.2369063
0.94656 -0.43225
0.94672 -0.681625
0.94688 -0.980875
0.94704 -1.3092
0.9472 -1.6334
0.94736 -1.9077
0.94752 -2.0864
0.94768 -2.1363
0.94784 -2.0283
0.948 -1.8121
0.94816 -1.5503
0.94832 -1.2926
0.94848 -1.0973
0.94864 -0.9767187
0.9488 -0.9434688
0.94896 -0.9725625
0.94912 -1.0224
0.94928 -1.0598
0.94944 -1.0515
0.9496 -1.0141
0.94976 -0.9725625
0.94992 -0.9684063
0.95008 -1.0307
0.95024 -1.1721
0.9504 -1.384
0.95056 -1.6417
0.95072 -1.8953
0.95088 -2.0906
0.95104 -2.1696
0.9512 -2.1114
0.95136 -1.9077
0.95152 -1.5835
0.95168 -1.1721
0.95184 -0.7315
0.952 -0.3366563
0.95216 -0.049875
0.95232 0.0789688
0.95248 0.0374063
0.95264 -0.1745625
0.9528 -0.482125
0.95296 -0.7938438
0.95312 -1.0141
0.95328 -1.0765
0.95344 -0.9559375
0.9536 -0.7024063
0.95376 -0.3906875
0.95392 -0.1288437
0.95408 0
0.95424 -0.0665
0.9544 -0.2950937
0.95456 -0.6026563
0.95472 -0.8852813
0.95488 -1.0474
0.95504 -1.0307
0.9552 -0.8354063
0.95536 -0.5361563
0.95552 -0.23275
0.95568 -0.0249375
0.95584 0
0.956 -0.1413125
0.95616 -0.415625
0.95632 -0.7315
0.95648 -1.0058
0.95664 -1.1762
0.9568 -1.2302
0.95696 -1.1928
0.95712 -1.1222
0.95728 -1.0682
0.95744 -1.0598
0.9576 -1.0931
0.95776 -1.143
0.95792 -1.1554
0.95808 -1.0931
0.95824 -0.9226875
0.9584 -0.6733125
0.95856 -0.399
0.95872 -0.1579375
0.95888 -0.0124688
0.95904 -0.016625
0.9592 -0.1579375
0.95936 -0.4073125
0.95952 -0.7065625
0.95968 -0.9975
0.95984 -1.2178
0.96 -1.3383
0.96016 -1.3508
0.96032 -1.2677
0.96048 -1.1097
0.96064 -0.9102188
0.9608 -0.6940938
0.96096 -0.4696563
0.96112 -0.249375
0.96128 -0.0457188
0.96144 0.1080625
0.9616 0.2036563
0.96176 0.23275
0.96192 0.182875
0.96208 0.0457188
0.96224 -0.1704063
0.9624 -0.4280938
0.96256 -0.681625
0.96272 -0.89775
0.96288 -1.0391
0.96304 -1.0889
0.9632 -1.0765
0.96336 -1.0349
0.96352 -1.01
0.96368 -1.0432
0.96384 -1.1471
0.964 -1.33
0.96416 -1.5711
0.96432 -1.8288
0.96448 -2.074
0.96464 -2.2361
0.9648 -2.315
0.96496 -2.315
0.96512 -2.2568
0.96528 -2.1571
0.96544 -1.9992
0.9656 -1.8121
0.96576 -1.6085
0.96592 -1.3799
0.96608 -1.1305
0.96624 -0.847875
0.9664 -0.5694063
0.96656 -0.3241875
0.96672 -0.133
0.96688 -0.016625
0.96704 0.0207812
0.9672 0.0041563
0.96736 -0.0290938
0.96752 -0.049875
0.96768 -0.016625
0.96784 0.0789688
0.968 0.2535313
0.96816 0.4862813
0.96832 0.748125
0.96848 1.0058
0.96864 1.1887
0.9688 1.2968
0.96896 1.3175
0.96912 1.251
0.96928 1.1097
0.96944 0.8852813
0.9696 0.63175
0.96976 0.3906875
0.96992 0.182875
0.97008 0.0290938
0.97024 -0.0665
0.9704 -0.1080625
0.97056 -0.0955938
0.97072 -0.0581875
0.97088 -0.0083125
0.97104 0.03325
0.9712 0.0581875
0.97136 0.0623438
0.97152 0.0415625
0.97168 0.0083125
0.97184 -0.0249375
0.972 -0.0540313
0.97216 -0.0623438
0.97232 -0.0457188
0.97248 -0.0083125
0.97264 0.03325
0.9728 0.0706562
0.97296 0.0872813
0.97312 0.0665
0.97328 0.016625
0.97344 -0.049875
0.9736 -0.116375
0.97376 -0.1537812
0.97392 -0.133
0.97408 -0.0374063
0.97424 0.133
0.9744 0.3615938
0.97456 0.615125
0.97472 0.8561875
0.97488 1.0307
0.97504 1.0889
0.9752 1.0141
0.97536 0.8063125
0.97552 0.4904375
0.97568 0.09975
0.97584 -0.3075625
0.976 -0.6774688
0.97616 -0.9517813
0.97632 -1.0931
0.97648 -1.0848
0.97664 -0.9434688
0.9768 -0.7065625
0.97696 -0.43225
0.97712 -0.1870313
0.97728 -0.0207812
0.97744 0.0041563
0.9776 -0.1039063
0.97776 -0.3366563
0.97792 -0.648375
0.97808 -0.980875
0.97824 -1.2801
0.9784 -1.4755
0.97856 -1.5295
0.97872 -1.4173
0.97888 -1.1471
0.97904 -0.7730625
0.9792 -0.3782188
0.97936 -0.0623438
0.97952 0.0914375
0.97968 0.0457188
0.97984 -0.2244375
0.98 -0.6733125
0.98016 -1.1928
0.98032 -1.6874
0.98048 -2.0573
0.98064 -2.2485
0.9808 -2.2693
0.98096 -2.1945
0.98112 -2.1114
0.98128 -2.1072
0.98144 -2.2153
0.9816 -2.4397
0.98176 -2.7307
0.98192 -3.005
0.98208 -3.1712
0.98224 -3.1546
0.9824 -2.9717
0.98256 -2.6933
0.98272 -2.3982
0.98288 -2.1654
0.98304 -2.0324
0.9832 -2.0116
0.98336 -2.0657
0.98352 -2.1322
0.98368 -2.1405
0.98384 -2.0324
0.984 -1.8288
0.98416 -1.5669
0.98432 -1.3051
0.98448 -1.1014
0.98464 -0.96425
0.9848 -0.914375
0.98496 -0.9393125
0.98512 -0.9975
0.98528 -1.0557
0.98544 -1.0557
0.9856 -0.96425
0.98576 -0.7605938
0.98592 -0.4571875
0.98608 -0.0914375
0.98624 0.2576875
0.9864 0.5236875
0.98656 0.6234375
0.98672 0.4904375
0.98688 0.1205313
0.98704 -0.4197813
0.9872 -0.9933438
0.98736 -1.4173
0.98752 -1.5212
0.98768 -1.2095
0.98784 -0.4655
0.988 0.56525
0.98816 1.6708
0.98832 2.6018
0.98848 3.1338
0.98864 3.113
0.9888 2.6018
0.98896 1.7789
0.98912 0.881125
0.98928 0.1371563
0.98944 -0.282625
0.9896 -0.315875
0.98976 -0.0249375
0.98992 0.4530313
0.99008 0.9517813
0.99024 1.3134
0.9904 1.4755
0.99056 1.4422
0.99072 1.2884
0.99088 1.1056
0.99104 0.9517813
0.9912 0.8852813
0.99136 0.9102188
0.99152 0.9850312
0.99168 1.0515
0.99184 1.0391
0.992 0.9268438
0.99216 0.7190313
0.99232 0.4280938
0.99248 0.0872813
0.99264 -0.2535313
0.9928 -0.5777188
0.99296 -0.8354063
0.99312 -1.01
0.99328 -1.0682
0.99344 -0.9933438
0.9936 -0.8187813
0.99376 -0.5735625
0.99392 -0.3034063
0.99408 -0.0540313
0.99424 0.133
0.9944 0.2369063
0.99456 0.2452187
0.99472 0.1745625
0.99488 0.0374063
0.99504 -0.1413125
0.9952 -0.3574375
0.99536 -0.5860313
0.99552 -0.8104688
0.99568 -1.0141
0.99584 -1.1721
0.996 -1.2677
0.99616 -1.2968
0.99632 -1.2469
0.99648 -1.1097
0.99664 -0.8935938
0.9968 -0.6400625
0.99696 -0.382375
0.99712 -0.1620938
0.99728 -0.016625
0.99744 0.0124688
0.9976 -0.0748125
0.99776 -0.2784688
0.99792 -0.5860313
0.99808 -0.9600938
0.99824 -1.3508
0.9984 -1.7041
0.99856 -1.9742
0.99872 -2.128
0.99888 -2.1446
0.99904 -1.9992
0.9992 -1.7581
0.99936 -1.4838
0.99952 -1.2469
0.99968 -1.0848
0.99984 -1.0141
1 -1.0266
1.00016 -1.0765
1.00032 -1.1097
1.00048 -1.0848
1.00064 -0.9517813
1.0008 -0.7315
1.00096 -0.4655
1.00112 -0.216125
1.00128 -0.0290938
1.00144 0.0623438
1.0016 0.0706562
1.00176 0.0249375
1.00192 -0.016625
1.00208 -0.0124688
1.00224 0.0789688
1.0024 0.2618438
1.00256 0.515375
1.00272 0.781375
1.00288 1.0141
1.00304 1.1513
1.0032 1.2012
1.00336 1.1845
1.00352 1.1305
1.00368 1.0723
1.00384 1.0224
1.004 0.9975
1.00416 1.0058
1.00432 1.0349
1.00448 1.0557
1.00464 1.0515
1.0048 1.0349
1.00496 1.0183
1.00512 1.0183
1.00528 1.0515
1.00544 1.0806
1.0056 1.1139
1.00576 1.1388
1.00592 1.1347
1.00608 1.0848
1.00624 0.9434688
1.0064 0.7356563
1.00656 0.4945938
1.00672 0.2535313
1.00688 0.0415625
1.00704 -0.1039063
1.0072 -0.182875
1.00736 -0.1870313
1.00752 -0.1288437
1.00768 -0.0249375
1.00784 0.0706562
1.008 0.1454688
1.00816 0.1745625
1.00832 0.1413125
1.00848 0.0374063
1.00864 -0.1579375
1.0088 -0.4655
1.00896 -0.881125
1.00912 -1.3965
1.00928 -1.9742
1.00944 -2.5395
1.0096 -3.0216
1.00976 -3.3416
1.00992 -3.4372
1.01008 -3.271
1.01024 -2.847
1.0104 -2.2818
1.01056 -1.7124
1.01072 -1.2718
1.01088 -1.0682
1.01104 -1.143
1.0112 -1.4796
1.01136 -1.9992
1.01152 -2.581
1.01168 -3.0881
1.01184 -3.3832
1.012 -3.4164
1.01216 -3.2045
1.01232 -2.7972
1.01248 -2.2652
1.01264 -1.7041
1.0128 -1.2386
1.01296 -0.9434688
1.01312 -0.8645
1.01328 -1.0017
1.01344 -1.2884
1.0136 -1.6459
1.01376 -1.9659
1.01392 -2.1612
1.01408 -2.1612
1.01424 -1.9285
1.0144 -1.5046
1.01456 -0.9767187
1.01472 -0.4613438
1.01488 -0.0665
1.01504 0.1205313
1.0152 0.083125
1.01536 -0.1579375
1.01552 -0.5361563
1.01568 -0.9600938
1.01584 -1.33
1.016 -1.5586
1.01616 -1.6043
1.01632 -1.463
1.01648 -1.1596
1.01664 -0.7273438
1.0168 -0.2410625
1.01696 0.2452187
1.01712 0.6691563
1.01728 0.9975
1.01744 1.1928
1.0176 1.2635
1.01776 1.2469
1.01792 1.1721
1.01808 1.0848
1.01824 0.9975
1.0184 0.9517813
1.01856 0.9559375
1.01872 0.9975
1.01888 1.0515
1.01904 1.0806
1.0192 1.0931
1.01936 1.0848
1.01952 1.0723
1.01968 1.064
1.01984 1.0432
1.02 1.0307
1.02016 1.0307
1.02032 1.0432
1.02048 1.0598
1.02064 1.0474
1.0208 1.0307
1.02096 1.0183
1.02112 1.0266
1.02128 1.0515
1.02144 1.0806
1.0216 1.1097
1.02176 1.1263
1.02192 1.1222
1.02208 1.0806
1.02224 0.980875
1.0224 0.8270938
1.02256 0.6192813
1.02272 0.3615938
1.02288 0.0748125
1.02304 -0.2285938
1.0232 -0.5236875
1.02336 -0.7730625
1.02352 -0.9517813
1.02368 -1.0515
1.02384 -1.0682
1.024 -1.0432
1.02416 -1.01
1.02432 -1.0017
1.02448 -1.0432
1.02464 -1.1263
1.0248 -1.2219
1.02496 -1.2843
1.02512 -1.2635
1.02528 -1.118
1.02544 -0.8561875
1.0256 -0.5361563
1.02576 -0.2285938
1.02592 -0.016625
1.02608 0.0207812
1.02624 -0.1454688
1.0264 -0.5112188
1.02656 -1.0058
1.02672 -1.542
1.02688 -2.0241
1.02704 -2.3607
1.0272 -2.5228
1.02736 -2.5145
1.02752 -2.3815
1.02768 -2.1779
1.02784 -1.9576
1.028 -1.7955
1.02816 -1.7498
1.02832 -1.8371
1.02848 -2.0532
1.02864 -2.3566
1.0288 -2.6849
1.02896 -2.9759
1.02912 -3.1629
1.02928 -3.2045
1.02944 -3.059
1.0296 -2.7888
1.02976 -2.4896
1.02992 -2.2485
1.03008 -2.1322
1.03024 -2.128
1.0304 -2.2153
1.03056 -2.3192
1.03072 -2.34
1.03088 -2.1945
1.03104 -1.8329
1.0312 -1.3175
1.03136 -0.7564375
1.03152 -0.2867813
1.03168 -0.0249375
1.03184 -0.0290938
1.032 -0.2535313
1.03216 -0.5943438
1.03232 -0.9102188
1.03248 -1.0598
1.03264 -0.9600938
1.0328 -0.6109688
1.03296 -0.0914375
1.03312 0.4738125
1.03328 0.96425
1.03344 1.2552
1.0336 1.3508
1.03376 1.2968
1.03392 1.1804
1.03408 1.0765
1.03424 1.0224
1.0344 1.0432
1.03456 1.1056
1.03472 1.143
1.03488 1.0931
1.03504 0.9060625
1.0352 0.6275938
1.03536 0.3325
1.03552 0.0955938
1.03568 0
1.03584 0.0623438
1.036 0.2784688
1.03616 0.5694063
1.03632 0.8520313
1.03648 1.0391
1.03664 1.0598
1.0368 0.9226875
1.03696 0.665
1.03712 0.3574375
1.03728 0.0623438
1.03744 -0.1537812
1.0376 -0.2618438
1.03776 -0.2618438
1.03792 -0.1704063
1.03808 -0.03325
1.03824 0.09975
1.0384 0.1911875
1.03856 0.2078125
1.03872 0.1537812
1.03888 0.03325
1.03904 -0.1080625
1.0392 -0.2410625
1.03936 -0.29925
1.03952 -0.2452187
1.03968 -0.0623438
1.03984 0.216125
1.04 0.548625
1.04016 0.8520313
1.04032 1.0557
1.04048 1.0848
1.04064 0.89775
1.0408 0.515375
1.04096 0.016625
1.04112 -0.5070625
1.04128 -0.96425
1.04144 -1.2843
1.0416 -1.4256
1.04176 -1.409
1.04192 -1.2843
1.04208 -1.1056
1.04224 -0.9434688
1.0424 -0.8437188
1.04256 -0.8270938
1.04272 -0.8935938
1.04288 -1.0224
1.04304 -1.1928
1.0432 -1.3882
1.04336 -1.6085
1.04352 -1.8412
1.04368 -2.0698
1.04384 -2.261
1.044 -2.3898
1.04416 -2.4314
1.04432 -2.3691
1.04448 -2.1862
1.04464 -1.8869
1.0448 -1.5503
1.04496 -1.2552
1.04512 -1.0723
1.04528 -1.0432
1.04544 -1.1762
1.0456 -1.4339
1.04576 -1.729
1.04592 -1.9867
1.04608 -2.1155
1.04624 -2.0781
1.0464 -1.8869
1.04656 -1.6085
1.04672 -1.3217
1.04688 -1.1014
1.04704 -0.9850312
1.0472 -0.980875
1.04736 -1.0391
1.04752 -1.0931
1.04768 -1.0806
1.04784 -0.9684063
1.048 -0.7522813
1.04816 -0.4779688
1.04832 -0.2119688
1.04848 -0.0249375
1.04864 0.0083125
1.0488 -0.1039063
1.04896 -0.3449688
1.04912 -0.6608438
1.04928 -0.9850312
1.04944 -1.2552
1.0496 -1.409
1.04976 -1.4381
1.04992 -1.3383
1.05008 -1.1263
1.05024 -0.8520313
1.0504 -0.5610938
1.05056 -0.2950937
1.05072 -0.09975
1.05088 -0.0041563
1.05104 -0.0290938
1.0512 -0.1704063
1.05136 -0.4031563
1.05152 -0.6899375
1.05168 -0.9891875
1.05184 -1.251
1.052 -1.4214
1.05216 -1.463
1.05232 -1.3674
1.05248 -1.1347
1.05264 -0.8104688
1.0528 -0.4613438
1.05296 -0.16625
1.05312 0.0041563
1.05328 0.0207812
1.05344 -0.1371563
1.0536 -0.4197813
1.05376 -0.7273438
1.05392 -0.9684063
1.05408 -1.064
1.05424 -0.9975
1.0544 -0.7896875
1.05456 -0.4945938
1.05472 -0.2119688
1.05488 -0.0207812
1.05504 0
1.0552 -0.133
1.05536 -0.3948438
1.05552 -0.7107188
1.05568 -1.0017
1.05584 -1.2012
1.056 -1.2843
1.05616 -1.2635
1.05632 -1.1721
1.05648 -1.0806
1.05664 -1.0474
1.0568 -1.118
1.05696 -1.3134
1.05712 -1.6209
1.05728 -2.0199
1.05744 -2.4314
1.0576 -2.8096
1.05776 -3.1006
1.05792 -3.2502
1.05808 -3.2253
1.05824 -2.9967
1.0584 -2.6101
1.05856 -2.1322
1.05872 -1.6251
1.05888 -1.1638
1.05904 -0.798
1.0592 -0.5943438
1.05936 -0.5694063
1.05952 -0.7065625
1.05968 -0.980875
1.05984 -1.3217
1.06 -1.6583
1.06016 -1.9368
1.06032 -2.1072
1.06048 -2.1405
1.06064 -2.0199
1.0608 -1.7955
1.06096 -1.5253
1.06112 -1.276
1.06128 -1.0931
1.06144 -0.9975
1.0616 -0.9891875
1.06176 -1.0266
1.06192 -1.0723
1.06208 -1.0723
1.06224 -1.0058
1.0624 -0.847875
1.06256 -0.6109688
1.06272 -0.3325
1.06288 -0.0581875
1.06304 0.1537812
1.0632 0.2867813
1.06336 0.3034063
1.06352 0.2202813
1.06368 0.049875
1.06384 -0.1704063
1.064 -0.3615938
1.06416 -0.4405625
1.06432 -0.3574375
1.06448 -0.0914375
1.06464 0.2950937
1.0648 0.7231875
1.06496 1.0765
1.06512 1.2427
1.06528 1.1347
1.06544 0.7231875
1.0656 0.1288437
1.06576 -0.4779688
1.06592 -0.931
1.06608 -1.0806
1.06624 -0.8686563
1.0664 -0.36575
1.06656 0.2535313
1.06672 0.7938438
1.06688 1.0557
1.06704 0.9102188
1.0672 0.382375
1.06736 -0.399
1.06752 -1.251
1.06768 -1.9784
1.06784 -2.4439
1.068 -2.606
1.06816 -2.5312
1.06832 -2.34
1.06848 -2.1571
1.06864 -2.0698
1.0688 -2.0989
1.06896 -2.1903
1.06912 -2.2485
1.06928 -2.1737
1.06944 -1.9036
1.0696 -1.4672
1.06976 -0.947625
1.06992 -0.4447188
1.07008 -0.0623438
1.07024 0.0872813
1.0704 0.016625
1.07056 -0.2369063
1.07072 -0.5943438
1.07088 -0.9725625
1.07104 -1.3092
1.0712 -1.5711
1.07136 -1.7664
1.07152 -1.9243
1.07168 -2.0823
1.07184 -2.2735
1.072 -2.4938
1.07216 -2.7348
1.07232 -2.9634
1.07248 -3.1546
1.07264 -3.2627
1.0728 -3.3001
1.07296 -3.2876
1.07312 -3.2419
1.07328 -3.2003
1.07344 -3.1671
1.0736 -3.1629
1.07376 -3.1754
1.07392 -3.192
1.07408 -3.192
1.07424 -3.1421
1.0744 -3.0839
1.07456 -3.0507
1.07472 -3.0715
1.07488 -3.1588
1.07504 -3.2668
1.0752 -3.3749
1.07536 -3.4414
1.07552 -3.4081
1.07568 -3.2502
1.07584 -2.9509
1.076 -2.5893
1.07616 -2.2652
1.07632 -2.0781
1.07648 -2.0906
1.07664 -2.2984
1.0768 -2.6392
1.07696 -2.9967
1.07712 -3.2336
1.07728 -3.2336
1.07744 -2.926
1.0776 -2.34
1.07776 -1.5877
1.07792 -0.8063125
1.07808 -0.133
1.07824 0.29925
1.0784 0.4779688
1.07856 0.4405625
1.07872 0.266
1.07888 0.049875
1.07904 -0.1371563
1.0792 -0.23275
1.07936 -0.2202813
1.07952 -0.133
1.07968 -0.0249375
1.07984 0.0540313
1.08 0.0955938
1.08016 0.09975
1.08032 0.0623438
1.08048 0.0124688
1.08064 -0.0374063
1.0808 -0.0706562
1.08096 -0.0706562
1.08112 -0.049875
1.08128 -0.0083125
1.08144 0.0124688
1.0816 0.03325
1.08176 0.0457188
1.08192 0.0374063
1.08208 0.0083125
1.08224 -0.0249375
1.0824 -0.0665
1.08256 -0.0872813
1.08272 -0.0789688
1.08288 -0.0207812
1.08304 0.083125
1.0832 0.249375
1.08336 0.4655
1.08352 0.7231875
1.08368 0.9975
1.08384 1.2136
1.084 1.3508
1.08416 1.3882
1.08432 1.3092
1.08448 1.1222
1.08464 0.83125
1.0848 0.49875
1.08496 0.2078125
1.08512 0.0207812
1.08528 -0.0124688
1.08544 0.0872813
1.0856 0.315875
1.08576 0.6026563
1.08592 0.8728125
1.08608 1.0432
1.08624 1.0183
1.0864 0.7772188
1.08656 0.349125
1.08672 -0.2285938
1.08688 -0.8894375
1.08704 -1.5669
1.0872 -2.1737
1.08736 -2.66
1.08752 -3.0008
1.08768 -3.1712
1.08784 -3.1588
1.088 -3.005
1.08816 -2.7639
1.08832 -2.4771
1.08848 -2.1903
1.08864 -1.9368
1.0888 -1.7789
1.08896 -1.7456
1.08912 -1.8454
1.08928 -2.0573
1.08944 -2.3317
1.0896 -2.6309
1.08976 -2.9011
1.08992 -3.1006
1.09008 -3.1878
1.09024 -3.1297
1.0904 -2.9593
1.09056 -2.7182
1.09072 -2.448
1.09088 -2.1862
1.09104 -1.9701
1.0912 -1.8495
1.09136 -1.8329
1.09152 -1.916
1.09168 -2.0781
1.09184 -2.2735
1.092 -2.4314
1.09216 -2.4938
1.09232 -2.4231
1.09248 -2.2028
1.09264 -1.8662
1.0928 -1.4879
1.09296 -1.1638
1.09312 -0.9850312
1.09328 -1.0183
1.09344 -1.2843
1.0936 -1.729
1.09376 -2.261
1.09392 -2.7639
1.09408 -3.1297
1.09424 -3.2751
1.0944 -3.1878
1.09456 -2.9177
1.09472 -2.5602
1.09488 -2.2028
1.09504 -1.9243
1.0952 -1.7789
1.09536 -1.7789
1.09552 -1.8953
1.09568 -2.0781
1.09584 -2.2527
1.096 -2.3691
1.09616 -2.4023
1.09632 -2.3358
1.09648 -2.1779
1.09664 -1.9451
1.0968 -1.6833
1.09696 -1.4298
1.09712 -1.2219
1.09728 -1.0848
1.09744 -1.0183
1.0976 -1.0141
1.09776 -1.0474
1.09792 -1.0806
1.09808 -1.0723
1.09824 -1.0141
1.0984 -0.8769688
1.09856 -0.6566875
1.09872 -0.3740625
1.09888 -0.0706562
1.09904 0.1745625
1.0992 0.3366563
1.09936 0.382375
1.09952 0.2909375
1.09968 0.0706562
1.09984 -0.2576875
1.1 -0.615125
1.10016 -0.914375
1.10032 -1.0848
1.10048 -1.0889
1.10064 -0.9393125
1.1008 -0.6733125
1.10096 -0.3615938
1.10112 -0.1080625
1.10128 0
1.10144 -0.09975
1.1016 -0.415625
1.10176 -0.8935938
1.10192 -1.4505
1.10208 -2.0033
1.10224 -2.4273
1.1024 -2.6642
1.10256 -2.6974
1.10272 -2.5353
1.10288 -2.2194
1.10304 -1.8163
1.1032 -1.4256
1.10336 -1.1263
1.10352 -0.980875
1.10368 -1.0224
1.10384 -1.2552
1.104 -1.65
1.10416 -2.1446
1.10432 -2.6558
1.10448 -3.1006
1.10464 -3.3624
1.1048 -3.404
1.10496 -3.2169
1.10512 -2.8304
1.10528 -2.2818
1.10544 -1.6376
1.1056 -1.0141
1.10576 -0.4945938
1.10592 -0.1413125
1.10608 0
1.10624 -0.1039063
1.1064 -0.4197813
1.10656 -0.8894375
1.10672 -1.4422
1.10688 -1.9992
1.10704 -2.4688
1.1072 -2.7722
1.10736 -2.8429
1.10752 -2.6683
1.10768 -2.2568
1.10784 -1.6708
1.108 -1.0307
1.10816 -0.4655
1.10832 -0.0914375
1.10848 0.016625
1.10864 -0.1745625
1.1088 -0.5985
1.10896 -1.1347
1.10912 -1.6583
1.10928 -2.0532
1.10944 -2.2527
1.1096 -2.2652
1.10976 -2.1696
1.10992 -2.0823
1.11008 -2.0989
1.11024 -2.2444
1.1104 -2.5187
1.11056 -2.8429
1.11072 -3.1047
1.11088 -3.2003
1.11104 -3.0424
};
\addplot [semithick, color1]
table {%
0 0
0.00016 0.0403125
0.00032 -0.1249687
0.00048 -0.4313438
0.00064 -0.7659375
0.0008 -1.032
0.00096 -1.1288
0.00112 -1.036
0.00128 -0.7780312
0.00144 -0.4071562
0.0016 0
0.00176 0.35475
0.00192 0.6369375
0.00208 0.8344688
0.00224 0.9594375
0.0024 1.032
0.00256 1.0401
0.00272 1.036
0.00288 1.032
0.00304 1.028
0.0032 1.032
0.00336 0.9916875
0.00352 0.9675
0.00368 0.9715313
0.00384 0.9957188
0.004 1.032
0.00416 0.9957188
0.00432 0.8949375
0.00448 0.7054688
0.00464 0.4071562
0.0048 0
0.00496 -0.5039063
0.00512 -1.028
0.00528 -1.5037
0.00544 -1.8665
0.0056 -2.064
0.00576 -2.064
0.00592 -1.9027
0.00608 -1.6286
0.00624 -1.3102
0.0064 -1.032
0.00656 -0.8586562
0.00672 -0.8667188
0.00688 -1.0844
0.00704 -1.4956
0.0072 -2.064
0.00736 -2.6929
0.00752 -3.3016
0.00768 -3.7934
0.00784 -4.0877
0.008 -4.128
0.00816 -3.87
0.00832 -3.4346
0.00848 -2.9227
0.00864 -2.4349
0.0088 -2.064
0.00896 -1.8423
0.00912 -1.7899
0.00928 -1.8584
0.00944 -1.9713
0.0096 -2.064
0.00976 -2.0962
0.00992 -2.068
0.01008 -2.0197
0.01024 -2.0035
0.0104 -2.064
0.01056 -2.2051
0.01072 -2.4349
0.01088 -2.6969
0.01104 -2.9388
0.0112 -3.096
0.01136 -3.096
0.01152 -2.9549
0.01168 -2.705
0.01184 -2.3865
0.012 -2.064
0.01216 -1.802
0.01232 -1.6528
0.01248 -1.6448
0.01264 -1.7858
0.0128 -2.064
0.01296 -2.4026
0.01312 -2.7412
0.01328 -3.0113
0.01344 -3.1444
0.0136 -3.096
0.01376 -2.8259
0.01392 -2.4107
0.01408 -1.9148
0.01424 -1.4311
0.0144 -1.032
0.01456 -0.7619063
0.01472 -0.6570938
0.01488 -0.693375
0.01504 -0.8385
0.0152 -1.032
0.01536 -1.2053
0.01552 -1.3061
0.01568 -1.3102
0.01584 -1.2134
0.016 -1.032
0.01616 -0.790125
0.01632 -0.532125
0.01648 -0.2942813
0.01664 -0.1088437
0.0168 0
0.01696 0.03225
0.01712 0.0201563
0.01728 -0.0080625
0.01744 -0.0282188
0.0176 0
0.01776 0.0765938
0.01792 0.241875
0.01808 0.4756875
0.01824 0.7498125
0.0184 1.032
0.01856 1.2336
0.01872 1.3464
0.01888 1.3545
0.01904 1.2416
0.0192 1.032
0.01936 0.7135313
0.01952 0.3950625
0.01968 0.1370625
0.01984 -0.0040313
0.02 0
0.02016 0.1249687
0.02032 0.3587813
0.02048 0.6329063
0.02064 0.8788125
0.0208 1.032
0.02096 1.0159
0.02112 0.8626875
0.02128 0.6046875
0.02144 0.3023437
0.0216 0
0.02176 -0.2620312
0.02192 -0.4716563
0.02208 -0.6490313
0.02224 -0.8264062
0.0224 -1.032
0.02256 -1.2618
0.02272 -1.5198
0.02288 -1.7657
0.02304 -1.9552
0.0232 -2.064
0.02336 -2.0559
0.02352 -2.0035
0.02368 -1.9592
0.02384 -1.9713
0.024 -2.064
0.02416 -2.2051
0.02432 -2.3623
0.02448 -2.4389
0.02464 -2.3543
0.0248 -2.064
0.02496 -1.5601
0.02512 -0.9715313
0.02528 -0.4313438
0.02544 -0.0765938
0.0256 0
0.02576 -0.2378437
0.02592 -0.7095
0.02608 -1.2739
0.02624 -1.7697
0.0264 -2.064
0.02656 -2.068
0.02672 -1.8382
0.02688 -1.4996
0.02704 -1.1852
0.0272 -1.032
0.02736 -1.0965
0.02752 -1.3424
0.02768 -1.6649
0.02784 -1.9431
0.028 -2.064
0.02816 -1.9592
0.02832 -1.6891
0.02848 -1.3585
0.02864 -1.1046
0.0288 -1.032
0.02896 -1.1771
0.02912 -1.4795
0.02928 -1.8181
0.02944 -2.0479
0.0296 -2.064
0.02976 -1.7939
0.02992 -1.3222
0.03008 -0.7699688
0.03024 -0.2862187
0.0304 0
0.03056 0.0120938
0.03072 -0.1935
0.03088 -0.5240625
0.03104 -0.8425313
0.0312 -1.032
0.03136 -1.0199
0.03152 -0.822375
0.03168 -0.516
0.03184 -0.2055938
0.032 0
0.03216 0.016125
0.03232 -0.1370625
0.03248 -0.41925
0.03264 -0.74175
0.0328 -1.032
0.03296 -1.2174
0.03312 -1.2819
0.03328 -1.2416
0.03344 -1.1408
0.0336 -1.032
0.03376 -0.9634687
0.03392 -0.9392812
0.03408 -0.9594375
0.03424 -0.9957188
0.0344 -1.032
0.03456 -1.0481
0.03472 -1.0522
0.03488 -1.0441
0.03504 -1.032
0.0352 -1.032
0.03536 -1.0441
0.03552 -1.0683
0.03568 -1.0804
0.03584 -1.0683
0.036 -1.032
0.03616 -0.9755625
0.03632 -0.9231563
0.03648 -0.903
0.03664 -0.9312187
0.0368 -1.032
0.03696 -1.1973
0.03712 -1.4109
0.03728 -1.6488
0.03744 -1.8745
0.0376 -2.064
0.03776 -2.1648
0.03792 -2.193
0.03808 -2.1728
0.03824 -2.1204
0.0384 -2.064
0.03856 -2.0035
0.03872 -1.9793
0.03888 -1.9834
0.03904 -2.0156
0.0392 -2.064
0.03936 -2.0922
0.03952 -2.1164
0.03968 -2.1245
0.03984 -2.1083
0.04 -2.064
0.04016 -1.9511
0.04032 -1.7939
0.04048 -1.5843
0.04064 -1.3263
0.0408 -1.032
0.04096 -0.7215937
0.04112 -0.4313438
0.04128 -0.1975312
0.04144 -0.048375
0.0416 0
0.04176 -0.0725625
0.04192 -0.2378437
0.04208 -0.4756875
0.04224 -0.7538437
0.0424 -1.032
0.04256 -1.2537
0.04272 -1.3827
0.04288 -1.3868
0.04304 -1.2658
0.0432 -1.032
0.04336 -0.7135313
0.04352 -0.3910313
0.04368 -0.1249687
0.04384 0.0120938
0.044 0
0.04416 -0.177375
0.04432 -0.4474687
0.04448 -0.7296563
0.04464 -0.9473438
0.0448 -1.032
0.04496 -0.9554063
0.04512 -0.7498125
0.04528 -0.4716563
0.04544 -0.2015625
0.0456 0
0.04576 0.0927188
0.04592 0.0846563
0.04608 0.0282188
0.04624 -0.016125
0.0464 0
0.04656 0.112875
0.04672 0.3225
0.04688 0.5885625
0.04704 0.8465625
0.0472 1.032
0.04736 1.0683
0.04752 0.9634687
0.04768 0.7296563
0.04784 0.3910313
0.048 0
0.04816 -0.387
0.04832 -0.7135313
0.04848 -0.9392812
0.04864 -1.0481
0.0488 -1.032
0.04896 -0.8909063
0.04912 -0.67725
0.04928 -0.4273125
0.04944 -0.1894688
0.0496 0
0.04976 0.112875
0.04992 0.1491563
0.05008 0.1249687
0.05024 0.0645
0.0504 0
0.05056 -0.0564375
0.05072 -0.080625
0.05088 -0.0765938
0.05104 -0.0443437
0.0512 0
0.05136 0.0443437
0.05152 0.0725625
0.05168 0.0725625
0.05184 0.0443437
0.052 0
0.05216 -0.0443437
0.05232 -0.0765938
0.05248 -0.080625
0.05264 -0.0564375
0.0528 0
0.05296 0.0645
0.05312 0.1249687
0.05328 0.1491563
0.05344 0.112875
0.0536 0
0.05376 -0.1975312
0.05392 -0.4434375
0.05408 -0.693375
0.05424 -0.9070312
0.0544 -1.032
0.05456 -1.036
0.05472 -0.919125
0.05488 -0.6893438
0.05504 -0.370875
0.0552 0
0.05536 0.370875
0.05552 0.6893438
0.05568 0.919125
0.05584 1.036
0.056 1.032
0.05616 0.8989688
0.05632 0.6893438
0.05648 0.4474687
0.05664 0.209625
0.0568 0
0.05696 -0.1814063
0.05712 -0.3466875
0.05728 -0.5240625
0.05744 -0.7457812
0.0576 -1.032
0.05776 -1.3545
0.05792 -1.681
0.05808 -1.9511
0.05824 -2.0922
0.0584 -2.064
0.05856 -1.8503
0.05872 -1.4754
0.05888 -0.9957188
0.05904 -0.4756875
0.0592 0
0.05936 0.3426562
0.05952 0.5119687
0.05968 0.4958438
0.05984 0.3104063
0.06 0
0.06016 -0.3749063
0.06032 -0.7296563
0.06048 -0.9836251
0.06064 -1.0884
0.0608 -1.032
0.06096 -0.8667188
0.06112 -0.6812813
0.06128 -0.5885625
0.06144 -0.6812813
0.0616 -1.032
0.06176 -1.6246
0.06192 -2.3704
0.06208 -3.1323
0.06224 -3.7612
0.0624 -4.128
0.06256 -4.1199
0.06272 -3.8015
0.06288 -3.2613
0.06304 -2.6364
0.0632 -2.064
0.06336 -1.6407
0.06352 -1.4513
0.06368 -1.4996
0.06384 -1.7254
0.064 -2.064
0.06416 -2.3865
0.06432 -2.6727
0.06448 -2.8944
0.06464 -3.0355
0.0648 -3.096
0.06496 -3.0315
0.06512 -2.8904
0.06528 -2.6727
0.06544 -2.3905
0.0656 -2.064
0.06576 -1.7012
0.06592 -1.3827
0.06608 -1.1449
0.06624 -1.028
0.0664 -1.032
0.06656 -1.1408
0.06672 -1.2698
0.06688 -1.3384
0.06704 -1.2698
0.0672 -1.032
0.06736 -0.6248438
0.06752 -0.1209375
0.06768 0.387
0.06784 0.7981875
0.068 1.032
0.06816 1.0239
0.06832 0.8385
0.06848 0.5442188
0.06864 0.2378437
0.0688 0
0.06896 -0.145125
0.06912 -0.16125
0.06928 -0.09675
0.06944 -0.016125
0.0696 0
0.06976 -0.1007813
0.06992 -0.2983125
0.07008 -0.5603437
0.07024 -0.8264062
0.0704 -1.032
0.07056 -1.1126
0.07072 -1.036
0.07088 -0.80625
0.07104 -0.4434375
0.0712 0
0.07136 0.4474687
0.07152 0.8344688
0.07168 1.0884
0.07184 1.161
0.072 1.032
0.07216 0.7014375
0.07232 0.2459063
0.07248 -0.2459063
0.07264 -0.6974062
0.0728 -1.032
0.07296 -1.2134
0.07312 -1.2537
0.07328 -1.1973
0.07344 -1.1046
0.0736 -1.032
0.07376 -1.0159
0.07392 -1.0562
0.07408 -1.1046
0.07424 -1.1126
0.0744 -1.032
0.07456 -0.87075
0.07472 -0.6409687
0.07488 -0.3789375
0.07504 -0.145125
0.0752 0
0.07536 0
0.07552 -0.1410938
0.07568 -0.3910313
0.07584 -0.7014375
0.076 -1.032
0.07616 -1.3142
0.07632 -1.5561
0.07648 -1.7496
0.07664 -1.9108
0.0768 -2.064
0.07696 -2.2253
0.07712 -2.4067
0.07728 -2.6203
0.07744 -2.8501
0.0776 -3.096
0.07776 -3.3177
0.07792 -3.5394
0.07808 -3.7571
0.07824 -3.9587
0.0784 -4.128
0.07856 -4.1643
0.07872 -4.0957
0.07888 -3.8982
0.07904 -3.5636
0.0792 -3.096
0.07936 -2.4953
0.07952 -1.9027
0.07968 -1.4109
0.07984 -1.1005
0.08 -1.032
0.08016 -1.1973
0.08032 -1.5762
0.08048 -2.0842
0.08064 -2.6284
0.0808 -3.096
0.08096 -3.354
0.08112 -3.3782
0.08128 -3.1524
0.08144 -2.6969
0.0816 -2.064
0.08176 -1.3222
0.08192 -0.5724375
0.08208 0.1048125
0.08224 0.6530625
0.0824 1.032
0.08256 1.1933
0.08272 1.2174
0.08288 1.161
0.08304 1.0844
0.0832 1.032
0.08336 0.9836251
0.08352 0.9876563
0.08368 1.028
0.08384 1.0522
0.084 1.032
0.08416 0.903
0.08432 0.7014375
0.08448 0.4595625
0.08464 0.209625
0.0848 0
0.08496 -0.1652812
0.08512 -0.2378437
0.08528 -0.2136563
0.08544 -0.1249687
0.0856 0
0.08576 0.1088437
0.08592 0.1854375
0.08608 0.2015625
0.08624 0.1370625
0.0864 0
0.08656 -0.2136563
0.08672 -0.467625
0.08688 -0.7135313
0.08704 -0.9110625
0.0872 -1.032
0.08736 -1.0643
0.08752 -1.0481
0.08768 -1.0118
0.08784 -0.99975
0.088 -1.032
0.08816 -1.0884
0.08832 -1.1691
0.08848 -1.2134
0.08864 -1.1771
0.0888 -1.032
0.08896 -0.7780312
0.08912 -0.4756875
0.08928 -0.1975312
0.08944 -0.016125
0.0896 0
0.08976 -0.2015625
0.08992 -0.5845313
0.09008 -1.0804
0.09024 -1.6004
0.0904 -2.064
0.09056 -2.3784
0.09072 -2.5276
0.09088 -2.4994
0.09104 -2.3301
0.0912 -2.064
0.09136 -1.7536
0.09152 -1.4633
0.09168 -1.2336
0.09184 -1.0844
0.092 -1.032
0.09216 -1.0965
0.09232 -1.2537
0.09248 -1.4835
0.09264 -1.7617
0.0928 -2.064
0.09296 -2.3381
0.09312 -2.584
0.09328 -2.7937
0.09344 -2.963
0.0936 -3.096
0.09376 -3.1524
0.09392 -3.1766
0.09408 -3.1726
0.09424 -3.1403
0.0944 -3.096
0.09456 -3.0033
0.09472 -2.9307
0.09488 -2.9106
0.09504 -2.963
0.0952 -3.096
0.09536 -3.2371
0.09552 -3.3661
0.09568 -3.4185
0.09584 -3.3379
0.096 -3.096
0.09616 -2.6566
0.09632 -2.1406
0.09648 -1.6367
0.09664 -1.2457
0.0968 -1.032
0.09696 -0.9916875
0.09712 -1.0763
0.09728 -1.1771
0.09744 -1.1892
0.0976 -1.032
0.09776 -0.6812813
0.09792 -0.1854375
0.09808 0.3426562
0.09824 0.7860938
0.0984 1.032
0.09856 0.9916875
0.09872 0.757875
0.09888 0.435375
0.09904 0.145125
0.0992 0
0.09936 0.0120938
0.09952 0.2136563
0.09968 0.5200313
0.09984 0.8264062
0.1 1.032
0.10016 1.0239
0.10032 0.8505938
0.10048 0.5724375
0.10064 0.2620312
0.1008 0
0.10096 -0.1693125
0.10112 -0.2176875
0.10128 -0.1652812
0.10144 -0.0725625
0.1016 0
0.10176 -0.0120938
0.10192 -0.129
0.10208 -0.3507188
0.10224 -0.6651562
0.1024 -1.032
0.10256 -1.3868
0.10272 -1.7012
0.10288 -1.935
0.10304 -2.06
0.1032 -2.064
0.10336 -1.935
0.10352 -1.7294
0.10368 -1.4795
0.10384 -1.2336
0.104 -1.032
0.10416 -0.8989688
0.10432 -0.854625
0.10448 -0.8828437
0.10464 -0.951375
0.1048 -1.032
0.10496 -1.0884
0.10512 -1.1167
0.10528 -1.1126
0.10544 -1.0804
0.1056 -1.032
0.10576 -0.9836251
0.10592 -0.9554063
0.10608 -0.951375
0.10624 -0.9755625
0.1064 -1.032
0.10656 -1.1247
0.10672 -1.2658
0.10688 -1.4714
0.10704 -1.7375
0.1072 -2.064
0.10736 -2.3986
0.10752 -2.713
0.10768 -2.9549
0.10784 -3.092
0.108 -3.096
0.10816 -2.9509
0.10832 -2.7251
0.10848 -2.4631
0.10864 -2.2293
0.1088 -2.064
0.10896 -1.9713
0.10912 -1.9632
0.10928 -2.0076
0.10944 -2.0559
0.1096 -2.064
0.10976 -1.9753
0.10992 -1.81
0.11008 -1.5722
0.11024 -1.2981
0.1104 -1.032
0.11056 -0.8183438
0.11072 -0.6974062
0.11088 -0.6974062
0.11104 -0.8102813
0.1112 -1.032
0.11136 -1.3182
0.11152 -1.6165
0.11168 -1.8665
0.11184 -2.0237
0.112 -2.064
0.11216 -1.9632
0.11232 -1.7576
0.11248 -1.4956
0.11264 -1.2376
0.1128 -1.032
0.11296 -0.9110625
0.11312 -0.8747813
0.11328 -0.9070312
0.11344 -0.9675
0.1136 -1.032
0.11376 -1.0723
0.11392 -1.0763
0.11408 -1.0602
0.11424 -1.036
0.1144 -1.032
0.11456 -1.0401
0.11472 -1.0602
0.11488 -1.0844
0.11504 -1.0804
0.1152 -1.032
0.11536 -0.9110625
0.11552 -0.725625
0.11568 -0.4958438
0.11584 -0.241875
0.116 0
0.11616 0.1814063
0.11632 0.29025
0.11648 0.2983125
0.11664 0.1975312
0.1168 0
0.11696 -0.2821875
0.11712 -0.5805
0.11728 -0.8344688
0.11744 -0.9957188
0.1176 -1.032
0.11776 -0.919125
0.11792 -0.7054688
0.11808 -0.4394063
0.11824 -0.1894688
0.1184 0
0.11856 0.0725625
0.11872 0.0685312
0.11888 0.0201563
0.11904 -0.0080625
0.1192 0
0.11936 0.0403125
0.11952 0.112875
0.11968 0.16125
0.11984 0.1330313
0.12 0
0.12016 -0.258
0.12032 -0.5684062
0.12048 -0.854625
0.12064 -1.028
0.1208 -1.032
0.12096 -0.8626875
0.12112 -0.5845313
0.12128 -0.2781563
0.12144 -0.0524062
0.1216 0
0.12176 -0.1814063
0.12192 -0.564375
0.12208 -1.0683
0.12224 -1.6004
0.1224 -2.064
0.12256 -2.3543
0.12272 -2.4631
0.12288 -2.4187
0.12304 -2.2615
0.1232 -2.064
0.12336 -1.8665
0.12352 -1.7536
0.12368 -1.7455
0.12384 -1.8584
0.124 -2.064
0.12416 -2.3139
0.12432 -2.5881
0.12448 -2.834
0.12464 -3.0154
0.1248 -3.096
0.12496 -3.0396
0.12512 -2.8783
0.12528 -2.6405
0.12544 -2.3583
0.1256 -2.064
0.12576 -1.7818
0.12592 -1.544
0.12608 -1.3464
0.12624 -1.1771
0.1264 -1.032
0.12656 -0.919125
0.12672 -0.8465625
0.12688 -0.8264062
0.12704 -0.886875
0.1272 -1.032
0.12736 -1.2537
0.12752 -1.5077
0.12768 -1.7536
0.12784 -1.9471
0.128 -2.064
0.12816 -2.0842
0.12832 -2.0438
0.12848 -1.9955
0.12864 -1.9874
0.1288 -2.064
0.12896 -2.2212
0.12912 -2.451
0.12928 -2.705
0.12944 -2.9388
0.1296 -3.096
0.12976 -3.1
0.12992 -2.9751
0.13008 -2.7372
0.13024 -2.4147
0.1304 -2.064
0.13056 -1.6931
0.13072 -1.3868
0.13088 -1.1691
0.13104 -1.0522
0.1312 -1.032
0.13136 -1.0602
0.13152 -1.1126
0.13168 -1.1489
0.13184 -1.1288
0.132 -1.032
0.13216 -0.8505938
0.13232 -0.6167812
0.13248 -0.3668437
0.13264 -0.1491563
0.1328 0
0.13296 0.0443437
0.13312 0.0282188
0.13328 -0.0120938
0.13344 -0.0362812
0.1336 0
0.13376 0.0765938
0.13392 0.2459063
0.13408 0.4877812
0.13424 0.7619063
0.1344 1.032
0.13456 1.1933
0.13472 1.2739
0.13488 1.2658
0.13504 1.1812
0.1352 1.032
0.13536 0.8022187
0.13552 0.564375
0.13568 0.3426562
0.13584 0.1531875
0.136 0
0.13616 -0.1531875
0.13632 -0.306375
0.13648 -0.4877812
0.13664 -0.7215937
0.1368 -1.032
0.13696 -1.423
0.13712 -1.8665
0.13728 -2.322
0.13744 -2.7493
0.1376 -3.096
0.13776 -3.2976
0.13792 -3.3822
0.13808 -3.3621
0.13824 -3.2572
0.1384 -3.096
0.13856 -2.8703
0.13872 -2.6485
0.13888 -2.4389
0.13904 -2.2414
0.1392 -2.064
0.13936 -1.8907
0.13952 -1.7818
0.13968 -1.7697
0.13984 -1.8624
0.14 -2.064
0.14016 -2.322
0.14032 -2.6122
0.14048 -2.8743
0.14064 -3.0476
0.1408 -3.096
0.14096 -2.967
0.14112 -2.7332
0.14128 -2.4631
0.14144 -2.2212
0.1416 -2.064
0.14176 -1.9955
0.14192 -2.0156
0.14208 -2.068
0.14224 -2.1003
0.1424 -2.064
0.14256 -1.939
0.14272 -1.7294
0.14288 -1.4714
0.14304 -1.2255
0.1432 -1.032
0.14336 -0.9150938
0.14352 -0.8828437
0.14368 -0.919125
0.14384 -0.9836251
0.144 -1.032
0.14416 -1.0118
0.14432 -0.8989688
0.14448 -0.67725
0.14464 -0.3628125
0.1448 0
0.14496 0.3305625
0.14512 0.5603437
0.14528 0.61275
0.14544 0.4273125
0.1456 0
0.14576 -0.61275
0.14592 -1.2658
0.14608 -1.8141
0.14624 -2.1083
0.1464 -2.064
0.14656 -1.6367
0.14672 -0.9473438
0.14688 -0.1572187
0.14704 0.5522813
0.1472 1.032
0.14736 1.1731
0.14752 1.0239
0.14768 0.6812813
0.14784 0.29025
0.148 0
0.14816 -0.1088437
0.14832 0
0.14848 0.2862187
0.14864 0.6651562
0.1488 1.032
0.14896 1.2698
0.14912 1.3666
0.14928 1.3263
0.14944 1.1933
0.1496 1.032
0.14976 0.854625
0.14992 0.7498125
0.15008 0.7457812
0.15024 0.8465625
0.1504 1.032
0.15056 1.1933
0.15072 1.3222
0.15088 1.3666
0.15104 1.2779
0.1512 1.032
0.15136 0.61275
0.15152 0.1088437
0.15168 -0.3789375
0.15184 -0.7820625
0.152 -1.032
0.15216 -1.1086
0.15232 -1.0602
0.15248 -0.9675
0.15264 -0.9312187
0.1528 -1.032
0.15296 -1.286
0.15312 -1.6972
0.15328 -2.197
0.15344 -2.6929
0.1536 -3.096
0.15376 -3.3137
0.15392 -3.3782
0.15408 -3.3218
0.15424 -3.2089
0.1544 -3.096
0.15456 -2.9831
0.15472 -2.9468
0.15488 -2.9791
0.15504 -3.0436
0.1552 -3.096
0.15536 -3.0557
0.15552 -2.9388
0.15568 -2.7292
0.15584 -2.4308
0.156 -2.064
0.15616 -1.6528
0.15632 -1.29
0.15648 -1.036
0.15664 -0.9433125
0.1568 -1.032
0.15696 -1.2739
0.15712 -1.5923
0.15728 -1.8866
0.15744 -2.064
0.1576 -2.064
0.15776 -1.8745
0.15792 -1.5641
0.15808 -1.2497
0.15824 -1.0441
0.1584 -1.032
0.15856 -1.2376
0.15872 -1.5722
0.15888 -1.9068
0.15904 -2.1043
0.1592 -2.064
0.15936 -1.7496
0.15952 -1.2336
0.15968 -0.6651562
0.15984 -0.2055938
0.16 0
0.16016 -0.1209375
0.16032 -0.5200313
0.16048 -1.0763
0.16064 -1.6407
0.1608 -2.064
0.16096 -2.2091
0.16112 -2.0962
0.16128 -1.7899
0.16144 -1.3948
0.1616 -1.032
0.16176 -0.7820625
0.16192 -0.6893438
0.16208 -0.7377188
0.16224 -0.87075
0.1624 -1.032
0.16256 -1.1449
0.16272 -1.1933
0.16288 -1.1771
0.16304 -1.1207
0.1632 -1.032
0.16336 -0.919125
0.16352 -0.7699688
0.16368 -0.5764688
0.16384 -0.3184688
0.164 0
0.16416 0.3426562
0.16432 0.6691875
0.16448 0.9231563
0.16464 1.0522
0.1648 1.032
0.16496 0.8344688
0.16512 0.5563125
0.16528 0.274125
0.16544 0.0685312
0.1656 0
0.16576 0.0765938
0.16592 0.29025
0.16608 0.5764688
0.16624 0.8465625
0.1664 1.032
0.16656 1.0401
0.16672 0.903
0.16688 0.6490313
0.16704 0.3265312
0.1672 0
0.16736 -0.3144375
0.16752 -0.5684062
0.16768 -0.7619063
0.16784 -0.9110625
0.168 -1.032
0.16816 -1.1288
0.16832 -1.1973
0.16848 -1.2174
0.16864 -1.165
0.1688 -1.032
0.16896 -0.8143125
0.16912 -0.5603437
0.16928 -0.3104063
0.16944 -0.1088437
0.1696 0
0.16976 -0.0120938
0.16992 -0.145125
0.17008 -0.3829687
0.17024 -0.6893438
0.1704 -1.032
0.17056 -1.3464
0.17072 -1.6206
0.17088 -1.8342
0.17104 -1.9834
0.1712 -2.064
0.17136 -2.06
0.17152 -2.0358
0.17168 -2.0116
0.17184 -2.0197
0.172 -2.064
0.17216 -2.1285
0.17232 -2.197
0.17248 -2.2293
0.17264 -2.193
0.1728 -2.064
0.17296 -1.8423
0.17312 -1.5762
0.17328 -1.3222
0.17344 -1.1288
0.1736 -1.032
0.17376 -1.0239
0.17392 -1.0844
0.17408 -1.1449
0.17424 -1.1449
0.1744 -1.032
0.17456 -0.8022187
0.17472 -0.5039063
0.17488 -0.2136563
0.17504 -0.0201563
0.1752 0
0.17536 -0.2015625
0.17552 -0.5845313
0.17568 -1.0804
0.17584 -1.6044
0.176 -2.064
0.17616 -2.3744
0.17632 -2.5074
0.17648 -2.4752
0.17664 -2.3099
0.1768 -2.064
0.17696 -1.7738
0.17712 -1.4996
0.17728 -1.2779
0.17744 -1.1207
0.1776 -1.032
0.17776 -0.9957188
0.17792 -0.99975
0.17808 -1.0159
0.17824 -1.032
0.1784 -1.032
0.17856 -1.0038
0.17872 -0.9675
0.17888 -0.951375
0.17904 -0.9675
0.1792 -1.032
0.17936 -1.1167
0.17952 -1.2053
0.17968 -1.2457
0.17984 -1.1933
0.18 -1.032
0.18016 -0.7659375
0.18032 -0.4555313
0.18048 -0.177375
0.18064 -0.0040313
0.1808 0
0.18096 -0.177375
0.18112 -0.4595625
0.18128 -0.7619063
0.18144 -0.9755625
0.1816 -1.032
0.18176 -0.8989688
0.18192 -0.628875
0.18208 -0.3225
0.18224 -0.0846563
0.1824 0
0.18256 -0.1048125
0.18272 -0.3587813
0.18288 -0.6691875
0.18304 -0.9231563
0.1832 -1.032
0.18336 -0.9433125
0.18352 -0.7054688
0.18368 -0.3990938
0.18384 -0.129
0.184 0
0.18416 -0.0604688
0.18432 -0.2821875
0.18448 -0.5845313
0.18464 -0.8667188
0.1848 -1.032
0.18496 -1.028
0.18512 -0.8626875
0.18528 -0.5845313
0.18544 -0.274125
0.1856 0
0.18576 0.1531875
0.18592 0.1975312
0.18608 0.1531875
0.18624 0.0725625
0.1864 0
0.18656 -0.0564375
0.18672 -0.0645
0.18688 -0.0403125
0.18704 -0.0080625
0.1872 0
0.18736 -0.0685312
0.18752 -0.2136563
0.18768 -0.4313438
0.18784 -0.7095
0.188 -1.032
0.18816 -1.3464
0.18832 -1.6327
0.18848 -1.8544
0.18864 -1.9995
0.1888 -2.064
0.18896 -2.0318
0.18912 -1.9713
0.18928 -1.9269
0.18944 -1.9471
0.1896 -2.064
0.18976 -2.2333
0.18992 -2.4712
0.19008 -2.7332
0.19024 -2.9549
0.1904 -3.096
0.19056 -3.0758
0.19072 -2.9347
0.19088 -2.6929
0.19104 -2.3905
0.1912 -2.064
0.19136 -1.7375
0.19152 -1.4674
0.19168 -1.2698
0.19184 -1.1328
0.192 -1.032
0.19216 -0.93525
0.19232 -0.80625
0.19248 -0.6167812
0.19264 -0.3466875
0.1928 0
0.19296 0.387
0.19312 0.757875
0.19328 1.032
0.19344 1.1368
0.1936 1.032
0.19376 0.6893438
0.19392 0.2176875
0.19408 -0.2862187
0.19424 -0.7296563
0.1944 -1.032
0.19456 -1.161
0.19472 -1.1449
0.19488 -1.0602
0.19504 -0.99975
0.1952 -1.032
0.19536 -1.161
0.19552 -1.3303
0.19568 -1.4311
0.19584 -1.3585
0.196 -1.032
0.19616 -0.4434375
0.19632 0.306375
0.19648 1.0763
0.19664 1.7093
0.1968 2.064
0.19696 2.0277
0.19712 1.677
0.19728 1.1247
0.19744 0.516
0.1976 0
0.19776 -0.3225
0.19792 -0.4273125
0.19808 -0.35475
0.19824 -0.1814063
0.1984 0
0.19856 0.1088437
0.19872 0.1370625
0.19888 0.1007813
0.19904 0.0403125
0.1992 0
0.19936 -0.0201563
0.19952 -0.0040313
0.19968 0.0241875
0.19984 0.0403125
0.2 0
0.20016 -0.1410938
0.20032 -0.35475
0.20048 -0.6006563
0.20064 -0.8385
0.2008 -1.032
0.20096 -1.1489
0.20112 -1.1933
0.20128 -1.1771
0.20144 -1.1167
0.2016 -1.032
0.20176 -0.9554063
0.20192 -0.9150938
0.20208 -0.9110625
0.20224 -0.9473438
0.2024 -1.032
0.20256 -1.165
0.20272 -1.3424
0.20288 -1.5561
0.20304 -1.802
0.2032 -2.064
0.20336 -2.3139
0.20352 -2.5518
0.20368 -2.7735
0.20384 -2.9549
0.204 -3.096
0.20416 -3.1484
0.20432 -3.1645
0.20448 -3.1484
0.20464 -3.1242
0.2048 -3.096
0.20496 -3.0355
0.20512 -3.0033
0.20528 -2.9993
0.20544 -3.0315
0.2056 -3.096
0.20576 -3.1323
0.20592 -3.1686
0.20608 -3.1847
0.20624 -3.1645
0.2064 -3.096
0.20656 -2.9347
0.20672 -2.7332
0.20688 -2.5074
0.20704 -2.2817
0.2072 -2.064
0.20736 -1.8463
0.20752 -1.6448
0.20768 -1.4472
0.20784 -1.2457
0.208 -1.032
0.20816 -0.8102813
0.20832 -0.5845313
0.20848 -0.3587813
0.20864 -0.1572187
0.2088 0
0.20896 0.0846563
0.20912 0.1209375
0.20928 0.1088437
0.20944 0.0645
0.2096 0
0.20976 -0.0886875
0.20992 -0.145125
0.21008 -0.1531875
0.21024 -0.1007813
0.2104 0
0.21056 0.1048125
0.21072 0.1975312
0.21088 0.2338125
0.21104 0.1733438
0.2112 0
0.21136 -0.2942813
0.21152 -0.6248438
0.21168 -0.8989688
0.21184 -1.0522
0.212 -1.032
0.21216 -0.8425313
0.21232 -0.54825
0.21248 -0.2459063
0.21264 -0.0362812
0.2128 0
0.21296 -0.145125
0.21312 -0.4232812
0.21328 -0.7336875
0.21344 -0.9634687
0.2136 -1.032
0.21376 -0.8949375
0.21392 -0.6167812
0.21408 -0.2942813
0.21424 -0.0524062
0.2144 0
0.21456 -0.2015625
0.21472 -0.6208125
0.21488 -1.1529
0.21504 -1.673
0.2152 -2.064
0.21536 -2.2172
0.21552 -2.1366
0.21568 -1.8584
0.21584 -1.4633
0.216 -1.032
0.21616 -0.6490313
0.21632 -0.3587813
0.21648 -0.1652812
0.21664 -0.048375
0.2168 0
0.21696 -0.0241875
0.21712 -0.1249687
0.21728 -0.3144375
0.21744 -0.6167812
0.2176 -1.032
0.21776 -1.5359
0.21792 -2.064
0.21808 -2.5437
0.21824 -2.9025
0.2184 -3.096
0.21856 -3.0597
0.21872 -2.8743
0.21888 -2.5961
0.21904 -2.3018
0.2192 -2.064
0.21936 -1.9108
0.21952 -1.8665
0.21968 -1.9027
0.21984 -1.9834
0.22 -2.064
0.22016 -2.1043
0.22032 -2.1124
0.22048 -2.0962
0.22064 -2.0721
0.2208 -2.064
0.22096 -2.0559
0.22112 -2.0761
0.22128 -2.1003
0.22144 -2.1043
0.2216 -2.064
0.22176 -1.935
0.22192 -1.7415
0.22208 -1.5077
0.22224 -1.2618
0.2224 -1.032
0.22256 -0.8667188
0.22272 -0.774
0.22288 -0.7699688
0.22304 -0.8586562
0.2232 -1.032
0.22336 -1.2658
0.22352 -1.5238
0.22368 -1.7617
0.22384 -1.9552
0.224 -2.064
0.22416 -2.0479
0.22432 -1.9189
0.22448 -1.6891
0.22464 -1.3787
0.2248 -1.032
0.22496 -0.6893438
0.22512 -0.387
0.22528 -0.1572187
0.22544 -0.0241875
0.2256 0
0.22576 -0.09675
0.22592 -0.2781563
0.22608 -0.516
0.22624 -0.7780312
0.2264 -1.032
0.22656 -1.2215
0.22672 -1.3303
0.22688 -1.3303
0.22704 -1.2295
0.2272 -1.032
0.22736 -0.7699688
0.22752 -0.4918125
0.22768 -0.2459063
0.22784 -0.0725625
0.228 0
0.22816 -0.0725625
0.22832 -0.2459063
0.22848 -0.4918125
0.22864 -0.7699688
0.2288 -1.032
0.22896 -1.2295
0.22912 -1.3303
0.22928 -1.3303
0.22944 -1.2215
0.2296 -1.032
0.22976 -0.7860938
0.22992 -0.532125
0.23008 -0.2942813
0.23024 -0.1088437
0.2304 0
0.23056 0.0040313
0.23072 -0.0927188
0.23088 -0.3104063
0.23104 -0.6329063
0.2312 -1.032
0.23136 -1.4472
0.23152 -1.8181
0.23168 -2.068
0.23184 -2.1567
0.232 -2.064
0.23216 -1.81
0.23232 -1.4875
0.23248 -1.1973
0.23264 -1.028
0.2328 -1.032
0.23296 -1.2013
0.23312 -1.4916
0.23328 -1.802
0.23344 -2.0237
0.2336 -2.064
0.23376 -1.8624
0.23392 -1.4674
0.23408 -0.9554063
0.23424 -0.4273125
0.2344 0
0.23456 0.2217188
0.23472 0.2015625
0.23488 -0.0524062
0.23504 -0.4918125
0.2352 -1.032
0.23536 -1.5722
0.23552 -2.0076
0.23568 -2.2575
0.23584 -2.2777
0.236 -2.064
0.23616 -1.6568
0.23632 -1.1449
0.23648 -0.6369375
0.23664 -0.2297812
0.2368 0
0.23696 0
0.23712 -0.1854375
0.23728 -0.4797188
0.23744 -0.790125
0.2376 -1.032
0.23776 -1.161
0.23792 -1.1691
0.23808 -1.1046
0.23824 -1.036
0.2384 -1.032
0.23856 -1.1126
0.23872 -1.3021
0.23888 -1.5601
0.23904 -1.8342
0.2392 -2.064
0.23936 -2.1849
0.23952 -2.2132
0.23968 -2.1769
0.23984 -2.1124
0.24 -2.064
0.24016 -2.0277
0.24032 -2.0318
0.24048 -2.0559
0.24064 -2.0761
0.2408 -2.064
0.24096 -1.9592
0.24112 -1.7939
0.24128 -1.5762
0.24144 -1.3142
0.2416 -1.032
0.24176 -0.7457812
0.24192 -0.4877812
0.24208 -0.2700938
0.24224 -0.1088437
0.2424 0
0.24256 0.0403125
0.24272 0.048375
0.24288 0.03225
0.24304 0.0120938
0.2432 0
0.24336 -0.0080625
0.24352 -0.0040313
0.24368 0.0040313
0.24384 0.0080625
0.244 0
0.24416 -0.0443437
0.24432 -0.0846563
0.24448 -0.09675
0.24464 -0.0725625
0.2448 0
0.24496 0.080625
0.24512 0.16125
0.24528 0.1975312
0.24544 0.1491563
0.2456 0
0.24576 -0.258
0.24592 -0.5563125
0.24608 -0.8264062
0.24624 -0.99975
0.2464 -1.032
0.24656 -0.8989688
0.24672 -0.6570938
0.24688 -0.370875
0.24704 -0.129
0.2472 0
0.24736 -0.0201563
0.24752 -0.1854375
0.24768 -0.4555313
0.24784 -0.757875
0.248 -1.032
0.24816 -1.2094
0.24832 -1.2779
0.24848 -1.2497
0.24864 -1.1529
0.2488 -1.032
0.24896 -0.9231563
0.24912 -0.8667188
0.24928 -0.87075
0.24944 -0.93525
0.2496 -1.032
0.24976 -1.1328
0.24992 -1.2134
0.25008 -1.2336
0.25024 -1.1771
0.2504 -1.032
0.25056 -0.8264062
0.25072 -0.5805
0.25088 -0.3305625
0.25104 -0.1249687
0.2512 0
0.25136 0
0.25152 -0.1249687
0.25168 -0.3587813
0.25184 -0.6732187
0.252 -1.032
0.25216 -1.3868
0.25232 -1.6931
0.25248 -1.9148
0.25264 -2.0358
0.2528 -2.064
0.25296 -2.0116
0.25312 -1.9431
0.25328 -1.9027
0.25344 -1.935
0.2536 -2.064
0.25376 -2.2736
0.25392 -2.5397
0.25408 -2.8058
0.25424 -3.0073
0.2544 -3.096
0.25456 -3.0073
0.25472 -2.8017
0.25488 -2.5316
0.25504 -2.2656
0.2552 -2.064
0.25536 -1.9189
0.25552 -1.8745
0.25568 -1.9108
0.25584 -1.9874
0.256 -2.064
0.25616 -2.068
0.25632 -2.0438
0.25648 -2.0116
0.25664 -2.0116
0.2568 -2.064
0.25696 -2.1325
0.25712 -2.2172
0.25728 -2.2615
0.25744 -2.2253
0.2576 -2.064
0.25776 -1.7455
0.25792 -1.3222
0.25808 -0.8465625
0.25824 -0.3910313
0.2584 0
0.25856 0.2821875
0.25872 0.4918125
0.25888 0.6570938
0.25904 0.822375
0.2592 1.032
0.25936 1.2577
0.25952 1.5198
0.25968 1.7738
0.25984 1.9713
0.26 2.064
0.26016 1.9673
0.26032 1.7657
0.26048 1.5037
0.26064 1.2416
0.2608 1.032
0.26096 0.8667188
0.26112 0.80625
0.26128 0.8385
0.26144 0.9271875
0.2616 1.032
0.26176 1.0844
0.26192 1.1126
0.26208 1.1086
0.26224 1.0804
0.2624 1.032
0.26256 0.93525
0.26272 0.8022187
0.26288 0.61275
0.26304 0.3466875
0.2632 0
0.26336 -0.41925
0.26352 -0.8788125
0.26368 -1.3384
0.26384 -1.7455
0.264 -2.064
0.26416 -2.2293
0.26432 -2.2897
0.26448 -2.2656
0.26464 -2.1809
0.2648 -2.064
0.26496 -1.8907
0.26512 -1.7093
0.26528 -1.5117
0.26544 -1.286
0.2656 -1.032
0.26576 -0.7296563
0.26592 -0.4434375
0.26608 -0.2055938
0.26624 -0.0443437
0.2664 0
0.26656 -0.080625
0.26672 -0.2700938
0.26688 -0.5240625
0.26704 -0.7941563
0.2672 -1.032
0.26736 -1.1933
0.26752 -1.2698
0.26768 -1.2577
0.26784 -1.1691
0.268 -1.032
0.26816 -0.8505938
0.26832 -0.645
0.26848 -0.4273125
0.26864 -0.2055938
0.2688 0
0.26896 0.1410938
0.26912 0.2297812
0.26928 0.2459063
0.26944 0.1733438
0.2696 0
0.26976 -0.2821875
0.26992 -0.6570938
0.27008 -1.1005
0.27024 -1.5843
0.2704 -2.064
0.27056 -2.4429
0.27072 -2.6768
0.27088 -2.705
0.27104 -2.4994
0.2712 -2.064
0.27136 -1.4553
0.27152 -0.8183438
0.27168 -0.2862187
0.27184 0.0080625
0.272 0
0.27216 -0.3144375
0.27232 -0.8304375
0.27248 -1.3988
0.27264 -1.8544
0.2728 -2.064
0.27296 -1.939
0.27312 -1.548
0.27328 -0.9957188
0.27344 -0.4313438
0.2736 0
0.27376 0.1854375
0.27392 0.1048125
0.27408 -0.1854375
0.27424 -0.6046875
0.2744 -1.032
0.27456 -1.3706
0.27472 -1.544
0.27488 -1.5278
0.27504 -1.3384
0.2752 -1.032
0.27536 -0.67725
0.27552 -0.3466875
0.27568 -0.1007813
0.27584 0.016125
0.276 0
0.27616 -0.1652812
0.27632 -0.4111875
0.27648 -0.6812813
0.27664 -0.903
0.2768 -1.032
0.27696 -1.0562
0.27712 -1.0078
0.27728 -0.9433125
0.27744 -0.9312187
0.2776 -1.032
0.27776 -1.2457
0.27792 -1.5359
0.27808 -1.8262
0.27824 -2.0237
0.2784 -2.064
0.27856 -1.9068
0.27872 -1.6246
0.27888 -1.3182
0.27904 -1.0925
0.2792 -1.032
0.27936 -1.1529
0.27952 -1.415
0.27968 -1.7254
0.27984 -1.9713
0.28 -2.064
0.28016 -1.9431
0.28032 -1.681
0.28048 -1.3706
0.28064 -1.1207
0.2808 -1.032
0.28096 -1.1247
0.28112 -1.3706
0.28128 -1.6851
0.28144 -1.9511
0.2816 -2.064
0.28176 -1.9269
0.28192 -1.5762
0.28208 -1.0723
0.28224 -0.5119687
0.2824 0
0.28256 0.35475
0.28272 0.5200313
0.28288 0.4958438
0.28304 0.3023437
0.2832 0
0.28336 -0.3587813
0.28352 -0.6853125
0.28368 -0.9271875
0.28384 -1.0441
0.284 -1.032
0.28416 -0.8989688
0.28432 -0.6812813
0.28448 -0.4232812
0.28464 -0.1814063
0.2848 0
0.28496 0.080625
0.28512 0.0846563
0.28528 0.048375
0.28544 0.0080625
0.2856 0
0.28576 0.0080625
0.28592 0.0524062
0.28608 0.0927188
0.28624 0.0846563
0.2864 0
0.28656 -0.1894688
0.28672 -0.4434375
0.28688 -0.7054688
0.28704 -0.9150938
0.2872 -1.032
0.28736 -1.0481
0.28752 -0.9957188
0.28768 -0.93525
0.28784 -0.9312187
0.288 -1.032
0.28816 -1.2497
0.28832 -1.5399
0.28848 -1.8221
0.28864 -2.0156
0.2888 -2.064
0.28896 -1.9189
0.28912 -1.6609
0.28928 -1.3666
0.28944 -1.1288
0.2896 -1.032
0.28976 -1.1046
0.28992 -1.3222
0.29008 -1.6165
0.29024 -1.8866
0.2904 -2.064
0.29056 -2.0801
0.29072 -2.0116
0.29088 -1.9269
0.29104 -1.9229
0.2912 -2.064
0.29136 -2.3301
0.29152 -2.6768
0.29168 -2.9912
0.29184 -3.1605
0.292 -3.096
0.29216 -2.7412
0.29232 -2.2212
0.29248 -1.6649
0.29264 -1.2295
0.2928 -1.032
0.29296 -1.1046
0.29312 -1.3908
0.29328 -1.7455
0.29344 -2.0116
0.2936 -2.064
0.29376 -1.8262
0.29392 -1.3666
0.29408 -0.8102813
0.29424 -0.3104063
0.2944 0
0.29456 0.0282188
0.29472 -0.1693125
0.29488 -0.5039063
0.29504 -0.8304375
0.2952 -1.032
0.29536 -1.028
0.29552 -0.8264062
0.29568 -0.5119687
0.29584 -0.2015625
0.296 0
0.29616 0.0040313
0.29632 -0.1572187
0.29648 -0.4394063
0.29664 -0.757875
0.2968 -1.032
0.29696 -1.1892
0.29712 -1.2255
0.29728 -1.1771
0.29744 -1.0965
0.2976 -1.032
0.29776 -1.0199
0.29792 -1.0481
0.29808 -1.0884
0.29824 -1.0925
0.2984 -1.032
0.29856 -0.8828437
0.29872 -0.6691875
0.29888 -0.4232812
0.29904 -0.1854375
0.2992 0
0.29936 0.0846563
0.29952 0.0927188
0.29968 0.0604688
0.29984 0.0201563
0.3 0
0.30016 0
0.30032 0.0201563
0.30048 0.048375
0.30064 0.048375
0.3008 0
0.30096 -0.1491563
0.30112 -0.35475
0.30128 -0.5885625
0.30144 -0.822375
0.3016 -1.032
0.30176 -1.2134
0.30192 -1.3747
0.30208 -1.5561
0.30224 -1.7778
0.3024 -2.064
0.30256 -2.3825
0.30272 -2.705
0.30288 -2.967
0.30304 -3.1081
0.3032 -3.096
0.30336 -2.9025
0.30352 -2.6243
0.30368 -2.3381
0.30384 -2.1285
0.304 -2.064
0.30416 -2.1406
0.30432 -2.3623
0.30448 -2.6526
0.30464 -2.9227
0.3048 -3.096
0.30496 -3.0718
0.30512 -2.8985
0.30528 -2.6284
0.30544 -2.326
0.3056 -2.064
0.30576 -1.8544
0.30592 -1.7617
0.30608 -1.7899
0.30624 -1.9027
0.3064 -2.064
0.30656 -2.2011
0.30672 -2.2897
0.30688 -2.2978
0.30704 -2.2212
0.3072 -2.064
0.30736 -1.8342
0.30752 -1.5843
0.30768 -1.3464
0.30784 -1.161
0.308 -1.032
0.30816 -0.93525
0.30832 -0.8304375
0.30848 -0.6691875
0.30864 -0.3990938
0.3088 0
0.30896 0.48375
0.30912 1.0199
0.30928 1.5198
0.30944 1.8947
0.3096 2.064
0.30976 1.9148
0.30992 1.548
0.31008 1.0401
0.31024 0.4877812
0.3104 0
0.31056 -0.370875
0.31072 -0.5401875
0.31088 -0.5119687
0.31104 -0.3104063
0.3112 0
0.31136 0.3265312
0.31152 0.6329063
0.31168 0.8747813
0.31184 1.0159
0.312 1.032
0.31216 0.854625
0.31232 0.5401875
0.31248 0.1007813
0.31264 -0.435375
0.3128 -1.032
0.31296 -1.6407
0.31312 -2.193
0.31328 -2.6485
0.31344 -2.9549
0.3136 -3.096
0.31376 -3.0315
0.31392 -2.8461
0.31408 -2.5881
0.31424 -2.3099
0.3144 -2.064
0.31456 -1.8705
0.31472 -1.7818
0.31488 -1.7939
0.31504 -1.8947
0.3152 -2.064
0.31536 -2.2575
0.31552 -2.4752
0.31568 -2.6969
0.31584 -2.9065
0.316 -3.096
0.31616 -3.1968
0.31632 -3.2572
0.31648 -3.2613
0.31664 -3.2089
0.3168 -3.096
0.31696 -2.9065
0.31712 -2.6888
0.31728 -2.4591
0.31744 -2.2454
0.3176 -2.064
0.31776 -1.939
0.31792 -1.8826
0.31808 -1.8866
0.31824 -1.9552
0.3184 -2.064
0.31856 -2.1809
0.31872 -2.2696
0.31888 -2.2938
0.31904 -2.2293
0.3192 -2.064
0.31936 -1.802
0.31952 -1.5077
0.31968 -1.2416
0.31984 -1.0683
0.32 -1.032
0.32016 -1.1489
0.32032 -1.3706
0.32048 -1.6367
0.32064 -1.8866
0.3208 -2.064
0.32096 -2.1366
0.32112 -2.1245
0.32128 -2.0721
0.32144 -2.0398
0.3216 -2.064
0.32176 -2.1245
0.32192 -2.2132
0.32208 -2.2736
0.32224 -2.2373
0.3224 -2.064
0.32256 -1.7093
0.32272 -1.2416
0.32288 -0.74175
0.32304 -0.2983125
0.3232 0
0.32336 0.0765938
0.32352 -0.0403125
0.32368 -0.3144375
0.32384 -0.6691875
0.324 -1.032
0.32416 -1.3102
0.32432 -1.4633
0.32448 -1.4593
0.32464 -1.3102
0.3248 -1.032
0.32496 -0.661125
0.32512 -0.2378437
0.32528 0.2055938
0.32544 0.6409687
0.3256 1.032
0.32576 1.286
0.32592 1.4271
0.32608 1.4351
0.32624 1.2981
0.3264 1.032
0.32656 0.61275
0.32672 0.1491563
0.32688 -0.3104063
0.32704 -0.7175625
0.3272 -1.032
0.32736 -1.2255
0.32752 -1.2981
0.32768 -1.2698
0.32784 -1.165
0.328 -1.032
0.32816 -0.9070312
0.32832 -0.8304375
0.32848 -0.822375
0.32864 -0.8909063
0.3288 -1.032
0.32896 -1.2295
0.32912 -1.4553
0.32928 -1.6851
0.32944 -1.8947
0.3296 -2.064
0.32976 -2.1527
0.32992 -2.189
0.33008 -2.1769
0.33024 -2.1325
0.3304 -2.064
0.33056 -1.9673
0.33072 -1.9068
0.33088 -1.9027
0.33104 -1.9592
0.3312 -2.064
0.33136 -2.1648
0.33152 -2.2494
0.33168 -2.2777
0.33184 -2.2212
0.332 -2.064
0.33216 -1.81
0.33232 -1.5198
0.33248 -1.2577
0.33264 -1.0804
0.3328 -1.032
0.33296 -1.1288
0.33312 -1.3464
0.33328 -1.6206
0.33344 -1.8826
0.3336 -2.064
0.33376 -2.0922
0.33392 -1.9713
0.33408 -1.7213
0.33424 -1.3908
0.3344 -1.032
0.33456 -0.7135313
0.33472 -0.4515
0.33488 -0.2459063
0.33504 -0.09675
0.3352 0
0.33536 0.0080625
0.33552 -0.0645
0.33568 -0.258
0.33584 -0.5845313
0.336 -1.032
0.33616 -1.5561
0.33632 -2.0237
0.33648 -2.322
0.33664 -2.3502
0.3368 -2.064
0.33696 -1.5117
0.33712 -0.8465625
0.33728 -0.2620312
0.33744 0.0604688
0.3376 0
0.33776 -0.48375
0.33792 -1.2457
0.33808 -2.0801
0.33824 -2.7574
0.3384 -3.096
0.33856 -2.9952
0.33872 -2.5518
0.33888 -1.935
0.33904 -1.3626
0.3392 -1.032
0.33936 -1.0522
0.33952 -1.415
0.33968 -1.9995
0.33984 -2.6163
0.34 -3.096
0.34016 -3.2774
0.34032 -3.1726
0.34048 -2.8501
0.34064 -2.4389
0.3408 -2.064
0.34096 -1.802
0.34112 -1.7093
0.34128 -1.7657
0.34144 -1.9068
0.3416 -2.064
0.34176 -2.1487
0.34192 -2.1728
0.34208 -2.1446
0.34224 -2.0962
0.3424 -2.064
0.34256 -2.0479
0.34272 -2.0559
0.34288 -2.0761
0.34304 -2.0882
0.3432 -2.064
0.34336 -1.9592
0.34352 -1.7899
0.34368 -1.5682
0.34384 -1.3102
0.344 -1.032
0.34416 -0.774
0.34432 -0.5361562
0.34448 -0.3225
0.34464 -0.1410938
0.3448 0
0.34496 0.080625
0.34512 0.1169063
0.34528 0.112875
0.34544 0.0685312
0.3456 0
0.34576 -0.09675
0.34592 -0.1693125
0.34608 -0.1814063
0.34624 -0.1249687
0.3464 0
0.34656 0.1330313
0.34672 0.2539688
0.34688 0.306375
0.34704 0.2297812
0.3472 0
0.34736 -0.3950625
0.34752 -0.8788125
0.34768 -1.3747
0.34784 -1.7939
0.348 -2.064
0.34816 -2.1325
0.34832 -2.0116
0.34848 -1.7455
0.34864 -1.3908
0.3488 -1.032
0.34896 -0.7457812
0.34912 -0.5925937
0.34928 -0.596625
0.34944 -0.7538437
0.3496 -1.032
0.34976 -1.3585
0.34992 -1.681
0.35008 -1.935
0.35024 -2.068
0.3504 -2.064
0.35056 -1.9229
0.35072 -1.6931
0.35088 -1.4311
0.35104 -1.1892
0.3512 -1.032
0.35136 -1.0038
0.35152 -1.1247
0.35168 -1.3706
0.35184 -1.7052
0.352 -2.064
0.35216 -2.3663
0.35232 -2.5518
0.35248 -2.5679
0.35264 -2.3986
0.3528 -2.064
0.35296 -1.6206
0.35312 -1.1892
0.35328 -0.8909063
0.35344 -0.8183438
0.3536 -1.032
0.35376 -1.5278
0.35392 -2.2172
0.35408 -2.971
0.35424 -3.6483
0.3544 -4.128
0.35456 -4.2812
0.35472 -4.1643
0.35488 -3.8579
0.35504 -3.4669
0.3552 -3.096
0.35536 -2.7735
0.35552 -2.5558
0.35568 -2.4067
0.35584 -2.2575
0.356 -2.064
0.35616 -1.7657
0.35632 -1.4392
0.35648 -1.1529
0.35664 -0.99975
0.3568 -1.032
0.35696 -1.2376
0.35712 -1.552
0.35728 -1.8705
0.35744 -2.068
0.3576 -2.064
0.35776 -1.7899
0.35792 -1.3303
0.35808 -0.7981875
0.35824 -0.3184688
0.3584 0
0.35856 0.09675
0.35872 0.0403125
0.35888 -0.0645
0.35904 -0.112875
0.3592 0
0.35936 0.2821875
0.35952 0.7336875
0.35968 1.2577
0.35984 1.7375
0.36 2.064
0.36016 2.1043
0.36032 1.935
0.36048 1.6327
0.36064 1.3021
0.3608 1.032
0.36096 0.8344688
0.36112 0.7780312
0.36128 0.8344688
0.36144 0.9433125
0.3616 1.032
0.36176 0.9836251
0.36192 0.8344688
0.36208 0.596625
0.36224 0.306375
0.3624 0
0.36256 -0.306375
0.36272 -0.5684062
0.36288 -0.7699688
0.36304 -0.919125
0.3632 -1.032
0.36336 -1.1408
0.36352 -1.2779
0.36368 -1.4674
0.36384 -1.7254
0.364 -2.064
0.36416 -2.4389
0.36432 -2.8703
0.36448 -3.3218
0.36464 -3.7571
0.3648 -4.128
0.36496 -4.3457
0.36512 -4.4505
0.36528 -4.4424
0.36544 -4.3255
0.3656 -4.128
0.36576 -3.8176
0.36592 -3.5233
0.36608 -3.2855
0.36624 -3.1403
0.3664 -3.096
0.36656 -3.092
0.36672 -3.1363
0.36688 -3.1847
0.36704 -3.1807
0.3672 -3.096
0.36736 -2.8743
0.36752 -2.6002
0.36768 -2.3381
0.36784 -2.1446
0.368 -2.064
0.36816 -2.0761
0.36832 -2.1608
0.36848 -2.2333
0.36864 -2.2212
0.3688 -2.064
0.36896 -1.7334
0.36912 -1.2779
0.36928 -0.774
0.36944 -0.3184688
0.3696 0
0.36976 0.1249687
0.36992 0.1169063
0.37008 0.0403125
0.37024 -0.0201563
0.3704 0
0.37056 0.129
0.37072 0.370875
0.37088 0.6570938
0.37104 0.903
0.3712 1.032
0.37136 0.9675
0.37152 0.774
0.37168 0.499875
0.37184 0.2176875
0.372 0
0.37216 -0.129
0.37232 -0.1531875
0.37248 -0.1048125
0.37264 -0.0362812
0.3728 0
0.37296 -0.0282188
0.37312 -0.0886875
0.37328 -0.1330313
0.37344 -0.1169063
0.3736 0
0.37376 0.22575
0.37392 0.5119687
0.37408 0.7860938
0.37424 0.9795937
0.3744 1.032
0.37456 0.8949375
0.37472 0.6409687
0.37488 0.35475
0.37504 0.1169063
0.3752 0
0.37536 -0.0040313
0.37552 0.0645
0.37568 0.1410938
0.37584 0.1370625
0.376 0
0.37616 -0.2983125
0.37632 -0.6651562
0.37648 -0.9876563
0.37664 -1.1368
0.3768 -1.032
0.37696 -0.6812813
0.37712 -0.1572187
0.37728 0.403125
0.37744 0.8425313
0.3776 1.032
0.37776 0.8304375
0.37792 0.306375
0.37808 -0.4394063
0.37824 -1.2779
0.3784 -2.064
0.37856 -2.6687
0.37872 -3.0476
0.37888 -3.2129
0.37904 -3.2008
0.3792 -3.096
0.37936 -2.9307
0.37952 -2.8178
0.37968 -2.7977
0.37984 -2.8944
0.38 -3.096
0.38016 -3.3177
0.38032 -3.5717
0.38048 -3.8136
0.38064 -4.0071
0.3808 -4.128
0.38096 -4.1078
0.38112 -4.0474
0.38128 -4.003
0.38144 -4.0192
0.3816 -4.128
0.38176 -4.2248
0.38192 -4.3497
0.38208 -4.4223
0.38224 -4.3699
0.3824 -4.128
0.38256 -3.612
0.38272 -2.9428
0.38288 -2.2212
0.38304 -1.552
0.3832 -1.032
0.38336 -0.6853125
0.38352 -0.4918125
0.38368 -0.3749063
0.38384 -0.2338125
0.384 0
0.38416 0.3466875
0.38432 0.80625
0.38448 1.3021
0.38464 1.7496
0.3848 2.064
0.38496 2.1487
0.38512 2.1124
0.38528 2.0277
0.38544 1.9874
0.3856 2.064
0.38576 2.189
0.38592 2.4308
0.38608 2.7251
0.38624 2.9751
0.3864 3.096
0.38656 2.9307
0.38672 2.584
0.38688 2.1043
0.38704 1.5601
0.3872 1.032
0.38736 0.5442188
0.38752 0.1975312
0.38768 0.0080625
0.38784 -0.0362812
0.388 0
0.38816 0.0241875
0.38832 -0.0120938
0.38848 -0.1814063
0.38864 -0.516
0.3888 -1.032
0.38896 -1.677
0.38912 -2.4026
0.38928 -3.1162
0.38944 -3.7168
0.3896 -4.128
0.38976 -4.2328
0.38992 -4.1159
0.39008 -3.8337
0.39024 -3.4669
0.3904 -3.096
0.39056 -2.7493
0.39072 -2.5639
0.39088 -2.576
0.39104 -2.7654
0.3912 -3.096
0.39136 -3.4346
0.39152 -3.7652
0.39168 -4.0272
0.39184 -4.1562
0.392 -4.128
0.39216 -3.866
0.39232 -3.4749
0.39248 -2.9993
0.39264 -2.5115
0.3928 -2.064
0.39296 -1.6689
0.39312 -1.3787
0.39328 -1.1973
0.39344 -1.0925
0.3936 -1.032
0.39376 -0.9715313
0.39392 -0.8505938
0.39408 -0.6490313
0.39424 -0.3587813
0.3944 0
0.39456 0.3628125
0.39472 0.7014375
0.39488 0.9554063
0.39504 1.0723
0.3952 1.032
0.39536 0.8143125
0.39552 0.5240625
0.39568 0.241875
0.39584 0.048375
0.396 0
0.39616 0.0927188
0.39632 0.3265312
0.39648 0.6248438
0.39664 0.886875
0.3968 1.032
0.39696 0.99975
0.39712 0.8264062
0.39728 0.5563125
0.39744 0.258
0.3976 0
0.39776 -0.16125
0.39792 -0.209625
0.39808 -0.1693125
0.39824 -0.0846563
0.3984 0
0.39856 0.048375
0.39872 0.0524062
0.39888 0.0282188
0.39904 0
0.3992 0
0.39936 0.0524062
0.39952 0.1814063
0.39968 0.3910313
0.39984 0.6812813
0.4 1.032
0.40016 1.3222
0.40032 1.5883
0.40048 1.806
0.40064 1.9713
0.4008 2.064
0.40096 1.9673
0.40112 1.7778
0.40128 1.5359
0.40144 1.2779
0.4016 1.032
0.40176 0.7175625
0.40192 0.4474687
0.40208 0.2378437
0.40224 0.09675
0.4024 0
0.40256 -0.1854375
0.40272 -0.4474687
0.40288 -0.822375
0.40304 -1.3545
0.4032 -2.064
0.40336 -2.9589
0.40352 -3.9829
0.40368 -5.0794
0.40384 -6.1839
0.404 -7.224
0.40416 -8.0585
0.40432 -8.768
0.40448 -9.3606
0.40464 -9.8725
0.4048 -10.32
0.40496 -10.5216
0.40512 -10.707
0.40528 -10.9086
0.40544 -11.1303
0.4056 -11.352
0.40576 -11.2512
0.40592 -11.0859
0.40608 -10.8723
0.40624 -10.6143
0.4064 -10.32
0.40656 -9.679
0.40672 -9.0018
0.40688 -8.3447
0.40704 -7.7521
0.4072 -7.224
0.40736 -6.4823
0.40752 -5.7284
0.40768 -4.9383
0.40784 -4.0756
0.408 -3.096
0.40816 -1.7858
0.40832 -0.2700938
0.40848 1.423
0.40864 3.2532
0.4088 5.16
0.40896 7.1474
0.40912 9.1429
0.40928 11.0698
0.40944 12.8557
0.4096 14.448
0.40976 15.8267
0.40992 17.0603
0.41008 18.1245
0.41024 18.9791
0.4104 19.608
0.41056 20.0756
0.41072 20.4586
0.41088 20.7085
0.41104 20.7771
0.4112 20.64
0.41136 20.4667
0.41152 20.3578
0.41168 20.3336
0.41184 20.4183
0.412 20.64
0.41216 21.2044
0.41232 22.0751
0.41248 23.1837
0.41264 24.4415
0.4128 25.8
0.41296 27.3601
0.41312 29.154
0.41328 31.2019
0.41344 33.5279
0.4136 36.12
0.41376 38.9096
0.41392 41.8162
0.41408 44.6743
0.41424 47.2946
0.4144 49.536
0.41456 51.1646
0.41472 52.4748
0.41488 53.5955
0.41504 54.6517
0.4152 55.728
0.41536 56.5867
0.41552 57.4776
0.41568 58.2637
0.41584 58.7635
0.416 58.824
0.41616 58.1266
0.41632 57.1712
0.41648 56.1593
0.41664 55.2846
0.4168 54.696
0.41696 54.2727
0.41712 54.2405
0.41728 54.4461
0.41744 54.6678
0.4176 54.696
0.41776 54.2848
0.41792 53.6438
0.41808 52.886
0.41824 52.1563
0.4184 51.6
0.41856 51.2775
0.41872 51.213
0.41888 51.3299
0.41904 51.4952
0.4192 51.6
0.41936 51.5557
0.41952 51.346
0.41968 51.0437
0.41984 50.7494
0.42 50.568
0.42016 50.5761
0.42032 50.7333
0.42048 50.9993
0.42064 51.3097
0.4208 51.6
0.42096 51.7733
0.42112 51.8298
0.42128 51.7935
0.42144 51.7008
0.4216 51.6
0.42176 51.467
0.42192 51.3138
0.42208 51.1283
0.42224 50.8824
0.4224 50.568
0.42256 50.1568
0.42272 49.6973
0.42288 49.2417
0.42304 48.8305
0.4232 48.504
0.42336 48.2662
0.42352 48.0888
0.42368 47.9275
0.42384 47.734
0.424 47.472
0.42416 47.1455
0.42432 46.7383
0.42448 46.2828
0.42464 45.8192
0.4248 45.408
0.42496 45.2145
0.42512 45.146
0.42528 45.1782
0.42544 45.275
0.4256 45.408
0.42576 45.7063
0.42592 45.9925
0.42608 46.2223
0.42624 46.3755
0.4264 46.44
0.42656 46.573
0.42672 46.6416
0.42688 46.6335
0.42704 46.5529
0.4272 46.44
0.42736 46.5085
0.42752 46.7061
0.42768 47.0769
0.42784 47.6655
0.428 48.504
0.42816 49.6973
0.42832 51.1082
0.42848 52.6481
0.42864 54.2163
0.4288 55.728
0.42896 57.1107
0.42912 58.3322
0.42928 59.3722
0.42944 60.2188
0.4296 60.888
0.42976 61.3879
0.42992 61.8353
0.43008 62.2465
0.43024 62.6214
0.4304 62.952
0.43056 63.214
0.43072 63.4841
0.43088 63.722
0.43104 63.8993
0.4312 63.984
0.43136 63.98
0.43152 63.9759
0.43168 63.9719
0.43184 63.9679
0.432 63.984
0.43216 64.0888
0.43232 64.3549
0.43248 64.7822
0.43264 65.3587
0.4328 66.048
0.43296 66.8301
0.43312 67.6887
0.43328 68.5635
0.43344 69.402
0.4336 70.176
0.43376 70.8653
0.43392 71.5023
0.43408 72.0989
0.43424 72.6794
0.4344 73.272
0.43456 73.8646
0.43472 74.566
0.43488 75.4045
0.43504 76.368
0.4352 77.4
0.43536 78.303
0.43552 79.073
0.43568 79.5809
0.43584 79.726
0.436 79.464
0.43616 78.7223
0.43632 77.8434
0.43648 77.0372
0.43664 76.501
0.4368 76.368
0.43696 76.5736
0.43712 77.1098
0.43728 77.7548
0.43744 78.2627
0.4376 78.432
0.43776 78.1055
0.43792 77.4443
0.43808 76.6179
0.43824 75.848
0.4384 75.336
0.43856 75.1586
0.43872 75.332
0.43888 75.7149
0.43904 76.1181
0.4392 76.368
0.43936 76.3438
0.43952 76.0253
0.43968 75.4852
0.43984 74.8563
0.44 74.304
0.44016 74.0057
0.44032 73.9815
0.44048 74.2395
0.44064 74.7192
0.4408 75.336
0.44096 75.9891
0.44112 76.5051
0.44128 76.7792
0.44144 76.7308
0.4416 76.368
0.44176 75.7835
0.44192 75.1102
0.44208 74.5378
0.44224 74.2314
0.4424 74.304
0.44256 74.8039
0.44272 75.5658
0.44288 76.3922
0.44304 77.0573
0.4432 77.4
0.44336 77.3678
0.44352 77.0533
0.44368 76.6381
0.44384 76.3478
0.444 76.368
0.44416 76.759
0.44432 77.4645
0.44448 78.299
0.44464 79.0327
0.4448 79.464
0.44496 79.4479
0.44512 79.0649
0.44528 78.4602
0.44544 77.8354
0.4456 77.4
0.44576 77.2871
0.44592 77.5371
0.44608 78.0772
0.44624 78.7747
0.4464 79.464
0.44656 79.9598
0.44672 80.2178
0.44688 80.1977
0.44704 79.9276
0.4472 79.464
0.44736 78.8754
0.44752 78.303
0.44768 77.8273
0.44784 77.5129
0.448 77.4
0.44816 77.4887
0.44832 77.7225
0.44848 78.0127
0.44864 78.2707
0.4488 78.432
0.44896 78.4804
0.44912 78.436
0.44928 78.3594
0.44944 78.3393
0.4496 78.432
0.44976 78.6416
0.44992 78.9399
0.45008 79.2342
0.45024 79.4317
0.4504 79.464
0.45056 79.2826
0.45072 78.9722
0.45088 78.6457
0.45104 78.432
0.4512 78.432
0.45136 78.6578
0.45152 79.0851
0.45168 79.6172
0.45184 80.1251
0.452 80.496
0.45216 80.5807
0.45232 80.4436
0.45248 80.1533
0.45264 79.7946
0.4528 79.464
0.45296 79.1697
0.45312 78.9641
0.45328 78.8109
0.45344 78.6497
0.4536 78.432
0.45376 78.1014
0.45392 77.6822
0.45408 77.2226
0.45424 76.7631
0.4544 76.368
0.45456 76.0818
0.45472 75.8762
0.45488 75.7109
0.45504 75.5416
0.4552 75.336
0.45536 75.1465
0.45552 74.9208
0.45568 74.6829
0.45584 74.4612
0.456 74.304
0.45616 74.3524
0.45632 74.5056
0.45648 74.7434
0.45664 75.0296
0.4568 75.336
0.45696 75.7109
0.45712 76.0253
0.45728 76.243
0.45744 76.3519
0.4576 76.368
0.45776 76.4123
0.45792 76.4849
0.45808 76.6381
0.45824 76.9324
0.4584 77.4
0.45856 78.0853
0.45872 78.9117
0.45888 79.8187
0.45904 80.7137
0.4592 81.528
0.45936 82.169
0.45952 82.6729
0.45968 83.0518
0.45984 83.3421
0.46 83.592
0.46016 83.7452
0.46032 83.9266
0.46048 84.1483
0.46064 84.3902
0.4608 84.624
0.46096 84.7087
0.46112 84.757
0.46128 84.757
0.46144 84.7127
0.4616 84.624
0.46176 84.3902
0.46192 84.1564
0.46208 83.9387
0.46224 83.7492
0.4624 83.592
0.46256 83.3703
0.46272 83.1768
0.46288 82.9873
0.46304 82.7898
0.4632 82.56
0.46336 82.2053
0.46352 81.7941
0.46368 81.3587
0.46384 80.9193
0.464 80.496
0.46416 80.0687
0.46432 79.6494
0.46448 79.2383
0.46464 78.8311
0.4648 78.432
0.46496 78.0289
0.46512 77.6741
0.46528 77.4202
0.46544 77.3234
0.4656 77.4
0.46576 77.5492
0.46592 77.7064
0.46608 77.7789
0.46624 77.6943
0.4664 77.4
0.46656 76.8195
0.46672 76.0778
0.46688 75.3158
0.46704 74.687
0.4672 74.304
0.46736 74.1307
0.46752 74.1508
0.46768 74.2677
0.46784 74.3564
0.468 74.304
0.46816 73.9412
0.46832 73.2801
0.46848 72.365
0.46864 71.2927
0.4688 70.176
0.46896 69.1037
0.46912 68.1402
0.46928 67.3138
0.46944 66.6245
0.4696 66.048
0.46976 65.5804
0.46992 65.0765
0.47008 64.4839
0.47024 63.7744
0.4704 62.952
0.47056 62.1538
0.47072 61.3476
0.47088 60.626
0.47104 60.1019
0.4712 59.856
0.47136 60.0092
0.47152 60.3841
0.47168 60.892
0.47184 61.4322
0.472 61.92
0.47216 62.3594
0.47232 62.6255
0.47248 62.7666
0.47264 62.8472
0.4728 62.952
0.47296 63.1576
0.47312 63.4035
0.47328 63.6534
0.47344 63.8631
0.4736 63.984
0.47376 63.9719
0.47392 63.8228
0.47408 63.5728
0.47424 63.2624
0.4744 62.952
0.47456 62.6618
0.47472 62.4199
0.47488 62.2264
0.47504 62.0651
0.4752 61.92
0.47536 61.7587
0.47552 61.6499
0.47568 61.6217
0.47584 61.7104
0.476 61.92
0.47616 62.1699
0.47632 62.4562
0.47648 62.7142
0.47664 62.8956
0.4768 62.952
0.47696 62.7988
0.47712 62.5408
0.47728 62.2546
0.47744 62.0288
0.4776 61.92
0.47776 61.8878
0.47792 61.9402
0.47808 62.0127
0.47824 62.0288
0.4784 61.92
0.47856 61.5773
0.47872 61.0412
0.47888 60.3599
0.47904 59.598
0.4792 58.824
0.47936 58.0621
0.47952 57.3687
0.47968 56.756
0.47984 56.2158
0.48 55.728
0.48016 55.2523
0.48032 54.7887
0.48048 54.3493
0.48064 53.9623
0.4808 53.664
0.48096 53.4503
0.48112 53.3294
0.48128 53.3133
0.48144 53.4262
0.4816 53.664
0.48176 53.926
0.48192 54.184
0.48208 54.4138
0.48224 54.5952
0.4824 54.696
0.48256 54.6073
0.48272 54.3412
0.48288 53.9059
0.48304 53.3213
0.4832 52.632
0.48336 51.8056
0.48352 50.9348
0.48368 50.0681
0.48384 49.2498
0.484 48.504
0.48416 47.7784
0.48432 47.0407
0.48448 46.2425
0.48464 45.3556
0.4848 44.376
0.48496 43.3359
0.48512 42.2918
0.48528 41.3606
0.48544 40.6511
0.4856 40.248
0.48576 40.1795
0.48592 40.3407
0.48608 40.6471
0.48624 40.9897
0.4864 41.28
0.48656 41.4009
0.48672 41.2961
0.48688 41.0099
0.48704 40.6269
0.4872 40.248
0.48736 39.9215
0.48752 39.6554
0.48768 39.4579
0.48784 39.3208
0.488 39.216
0.48816 39.0588
0.48832 38.7887
0.48848 38.3815
0.48864 37.8373
0.4888 37.152
0.48896 36.3095
0.48912 35.3299
0.48928 34.2495
0.48944 33.1127
0.4896 31.992
0.48976 30.956
0.48992 30.0731
0.49008 29.4039
0.49024 29.0089
0.4904 28.896
0.49056 29.025
0.49072 29.2669
0.49088 29.545
0.49104 29.7829
0.4912 29.928
0.49136 29.8796
0.49152 29.666
0.49168 29.3636
0.49184 29.0774
0.492 28.896
0.49216 28.8033
0.49232 28.7952
0.49248 28.8436
0.49264 28.9
0.4928 28.896
0.49296 28.6783
0.49312 28.2389
0.49328 27.5899
0.49344 26.7594
0.4936 25.8
0.49376 24.6592
0.49392 23.4296
0.49408 22.1517
0.49424 20.8617
0.4944 19.608
0.49456 18.4268
0.49472 17.3707
0.49488 16.4999
0.49504 15.8589
0.4952 15.48
0.49536 15.3228
0.49552 15.2946
0.49568 15.347
0.49584 15.4276
0.496 15.48
0.49616 15.3671
0.49632 15.0648
0.49648 14.5972
0.49664 14.0207
0.4968 13.416
0.49696 12.775
0.49712 12.1784
0.49728 11.7027
0.49744 11.4246
0.4976 11.352
0.49776 11.3722
0.49792 11.3762
0.49808 11.2593
0.49824 10.9287
0.4984 10.32
0.49856 9.2961
0.49872 7.9456
0.49888 6.3774
0.49904 4.7166
0.4992 3.096
0.49936 1.5923
0.49952 0.2297812
0.49968 -0.9876563
0.49984 -2.0922
0.5 -3.096
0.50016 -3.9547
0.50032 -4.7488
0.50048 -5.4301
0.50064 -5.9259
0.5008 -6.192
0.50096 -6.188
0.50112 -6.1073
0.50128 -6.0307
0.50144 -6.0428
0.5016 -6.192
0.50176 -6.4742
0.50192 -6.9338
0.50208 -7.4659
0.50224 -7.9416
0.5024 -8.256
0.50256 -8.3245
0.50272 -8.2883
0.50288 -8.2117
0.50304 -8.1754
0.5032 -8.256
0.50336 -8.5221
0.50352 -9.03
0.50368 -9.7234
0.50384 -10.5296
0.504 -11.352
0.50416 -12.0897
0.50432 -12.7629
0.50448 -13.3676
0.50464 -13.9239
0.5048 -14.448
0.50496 -14.9156
0.50512 -15.3792
0.50528 -15.8186
0.50544 -16.2016
0.5056 -16.512
0.50576 -16.6894
0.50592 -16.8506
0.50608 -17.0361
0.50624 -17.2658
0.5064 -17.544
0.50656 -17.7778
0.50672 -17.9834
0.50688 -18.06
0.50704 -17.931
0.5072 -17.544
0.50736 -16.8748
0.50752 -16.1008
0.50768 -15.355
0.50784 -14.7665
0.508 -14.448
0.50816 -14.452
0.50832 -14.7705
0.50848 -15.3026
0.50864 -15.9234
0.5088 -16.512
0.50896 -16.9756
0.50912 -17.282
0.50928 -17.4392
0.50944 -17.5037
0.5096 -17.544
0.50976 -17.6569
0.50992 -17.8383
0.51008 -18.0802
0.51024 -18.3422
0.5104 -18.576
0.51056 -18.8179
0.51072 -18.967
0.51088 -18.9872
0.51104 -18.8501
0.5112 -18.576
0.51136 -18.3422
0.51152 -18.0963
0.51168 -17.8625
0.51184 -17.6649
0.512 -17.544
0.51216 -17.6891
0.51232 -18.0036
0.51248 -18.4591
0.51264 -19.0033
0.5128 -19.608
0.51296 -20.3699
0.51312 -21.1802
0.51328 -22.0147
0.51344 -22.8612
0.5136 -23.736
0.51376 -24.7478
0.51392 -25.8726
0.51408 -27.1142
0.51424 -28.4727
0.5144 -29.928
0.51456 -31.4317
0.51472 -32.9756
0.51488 -34.4914
0.51504 -35.9104
0.5152 -37.152
0.51536 -38.0832
0.51552 -38.7726
0.51568 -39.1958
0.51584 -39.3369
0.516 -39.216
0.51616 -38.8451
0.51632 -38.4783
0.51648 -38.2122
0.51664 -38.1074
0.5168 -38.184
0.51696 -38.3815
0.51712 -38.7242
0.51728 -39.0628
0.51744 -39.2603
0.5176 -39.216
0.51776 -38.9177
0.51792 -38.5347
0.51808 -38.1921
0.51824 -38.0348
0.5184 -38.184
0.51856 -38.7363
0.51872 -39.6635
0.51888 -40.8527
0.51904 -42.1306
0.5192 -43.344
0.51936 -44.3599
0.51952 -45.1379
0.51968 -45.7023
0.51984 -46.1094
0.52 -46.44
0.52016 -46.7423
0.52032 -47.1092
0.52048 -47.5486
0.52064 -48.0283
0.5208 -48.504
0.52096 -48.8708
0.52112 -49.1812
0.52128 -49.407
0.52144 -49.5279
0.5216 -49.536
0.52176 -49.3868
0.52192 -49.1853
0.52208 -48.9595
0.52224 -48.7257
0.5224 -48.504
0.52256 -48.2662
0.52272 -48.1412
0.52288 -48.1452
0.52304 -48.2742
0.5232 -48.504
0.52336 -48.7701
0.52352 -49.0563
0.52368 -49.3103
0.52384 -49.4755
0.524 -49.536
0.52416 -49.4796
0.52432 -49.3868
0.52448 -49.3264
0.52464 -49.3627
0.5248 -49.536
0.52496 -49.8464
0.52512 -50.2133
0.52528 -50.5237
0.52544 -50.6688
0.5256 -50.568
0.52576 -50.2536
0.52592 -49.7698
0.52608 -49.2377
0.52624 -48.7781
0.5264 -48.504
0.52656 -48.5483
0.52672 -48.7983
0.52688 -49.1369
0.52704 -49.4191
0.5272 -49.536
0.52736 -49.5521
0.52752 -49.4473
0.52768 -49.3304
0.52784 -49.3183
0.528 -49.536
0.52816 -50.1125
0.52832 -50.9429
0.52848 -51.9023
0.52864 -52.8537
0.5288 -53.664
0.52896 -54.2848
0.52912 -54.7162
0.52928 -55.0266
0.52944 -55.3289
0.5296 -55.728
0.52976 -56.3045
0.52992 -57.0744
0.53008 -57.9855
0.53024 -58.9449
0.5304 -59.856
0.53056 -60.5736
0.53072 -61.1379
0.53088 -61.5411
0.53104 -61.7991
0.5312 -61.92
0.53136 -61.8797
0.53152 -61.7749
0.53168 -61.5854
0.53184 -61.2871
0.532 -60.888
0.53216 -60.4043
0.53232 -59.9971
0.53248 -59.7351
0.53264 -59.6786
0.5328 -59.856
0.53296 -60.2591
0.53312 -60.8033
0.53328 -61.3435
0.53344 -61.7426
0.5336 -61.92
0.53376 -61.8676
0.53392 -61.6983
0.53408 -61.5572
0.53424 -61.5935
0.5344 -61.92
0.53456 -62.569
0.53472 -63.468
0.53488 -64.4637
0.53504 -65.3708
0.5352 -66.048
0.53536 -66.3826
0.53552 -66.4269
0.53568 -66.2858
0.53584 -66.1125
0.536 -66.048
0.53616 -66.1649
0.53632 -66.5197
0.53648 -67.0437
0.53664 -67.6162
0.5368 -68.112
0.53696 -68.366
0.53712 -68.3821
0.53728 -68.1483
0.53744 -67.6927
0.5376 -67.08
0.53776 -66.3947
0.53792 -65.7779
0.53808 -65.3103
0.53824 -65.0482
0.5384 -65.016
0.53856 -65.2256
0.53872 -65.6207
0.53888 -66.1206
0.53904 -66.6285
0.5392 -67.08
0.53936 -67.4227
0.53952 -67.6605
0.53968 -67.8177
0.53984 -67.9507
0.54 -68.112
0.54016 -68.3015
0.54032 -68.5393
0.54048 -68.7893
0.54064 -69.0029
0.5408 -69.144
0.54096 -69.1843
0.54112 -69.1601
0.54128 -69.1117
0.54144 -69.0916
0.5416 -69.144
0.54176 -69.269
0.54192 -69.4786
0.54208 -69.7366
0.54224 -69.9906
0.5424 -70.176
0.54256 -70.1962
0.54272 -70.0913
0.54288 -69.8656
0.54304 -69.535
0.5432 -69.144
0.54336 -68.7248
0.54352 -68.374
0.54368 -68.1362
0.54384 -68.0475
0.544 -68.112
0.54416 -68.3176
0.54432 -68.5998
0.54448 -68.8779
0.54464 -69.0755
0.5448 -69.144
0.54496 -69.0553
0.54512 -68.8457
0.54528 -68.5716
0.54544 -68.3055
0.5456 -68.112
0.54576 -68.0394
0.54592 -68.0556
0.54608 -68.112
0.54624 -68.1443
0.5464 -68.112
0.54656 -68.0273
0.54672 -67.9104
0.54688 -67.8379
0.54704 -67.8862
0.5472 -68.112
0.54736 -68.5232
0.54752 -69.0432
0.54768 -69.5632
0.54784 -69.9704
0.548 -70.176
0.54816 -70.1196
0.54832 -69.8777
0.54848 -69.5552
0.54864 -69.277
0.5488 -69.144
0.54896 -69.2085
0.54912 -69.4383
0.54928 -69.7487
0.54944 -70.0268
0.5496 -70.176
0.54976 -70.1196
0.54992 -69.9059
0.55008 -69.6116
0.55024 -69.3294
0.5504 -69.144
0.55056 -69.0795
0.55072 -69.1279
0.55088 -69.2085
0.55104 -69.2367
0.5512 -69.144
0.55136 -68.9021
0.55152 -68.5837
0.55168 -68.2733
0.55184 -68.0878
0.552 -68.112
0.55216 -68.3861
0.55232 -68.8376
0.55248 -69.3697
0.55264 -69.8495
0.5528 -70.176
0.55296 -70.2566
0.55312 -70.1196
0.55328 -69.8213
0.55344 -69.4625
0.5536 -69.144
0.55376 -68.9263
0.55392 -68.8497
0.55408 -68.89
0.55424 -69.0029
0.5544 -69.144
0.55456 -69.2367
0.55472 -69.273
0.55488 -69.2569
0.55504 -69.2045
0.5552 -69.144
0.55536 -69.0634
0.55552 -69.019
0.55568 -69.015
0.55584 -69.0593
0.556 -69.144
0.55616 -69.1964
0.55632 -69.2367
0.55648 -69.2488
0.55664 -69.2206
0.5568 -69.144
0.55696 -68.9586
0.55712 -68.7328
0.55728 -68.495
0.55744 -68.2773
0.5576 -68.112
0.55776 -67.9266
0.55792 -67.7532
0.55808 -67.5678
0.55824 -67.3501
0.5584 -67.08
0.55856 -66.7132
0.55872 -66.2858
0.55888 -65.8343
0.55904 -65.399
0.5592 -65.016
0.55936 -64.6975
0.55952 -64.4516
0.55968 -64.2662
0.55984 -64.1211
0.56 -63.984
0.56016 -63.8066
0.56032 -63.5285
0.56048 -63.1173
0.56064 -62.569
0.5608 -61.92
0.56096 -61.2669
0.56112 -60.6502
0.56128 -60.1583
0.56144 -59.8762
0.5616 -59.856
0.56176 -60.1462
0.56192 -60.6018
0.56208 -61.1218
0.56224 -61.5935
0.5624 -61.92
0.56256 -62.0611
0.56272 -61.9684
0.56288 -61.6862
0.56304 -61.2911
0.5632 -60.888
0.56336 -60.5937
0.56352 -60.4405
0.56368 -60.4446
0.56384 -60.6058
0.564 -60.888
0.56416 -61.2347
0.56432 -61.5572
0.56448 -61.8031
0.56464 -61.9281
0.5648 -61.92
0.56496 -61.7587
0.56512 -61.5088
0.56528 -61.2347
0.56544 -61.0089
0.5656 -60.888
0.56576 -60.8396
0.56592 -60.9404
0.56608 -61.1783
0.56624 -61.5249
0.5664 -61.92
0.56656 -62.1901
0.56672 -62.3433
0.56688 -62.3513
0.56704 -62.2022
0.5672 -61.92
0.56736 -61.4121
0.56752 -60.8678
0.56768 -60.3801
0.56784 -60.0293
0.568 -59.856
0.56816 -59.727
0.56832 -59.6544
0.56848 -59.5416
0.56864 -59.2916
0.5688 -58.824
0.56896 -58.0339
0.56912 -57.014
0.56928 -55.8651
0.56944 -54.7121
0.5696 -53.664
0.56976 -52.7852
0.56992 -52.0797
0.57008 -51.5194
0.57024 -51.0356
0.5704 -50.568
0.57056 -50.0722
0.57072 -49.4755
0.57088 -48.8023
0.57104 -48.1089
0.5712 -47.472
0.57136 -46.9681
0.57152 -46.5851
0.57168 -46.3553
0.57184 -46.307
0.572 -46.44
0.57216 -46.702
0.57232 -46.9842
0.57248 -47.2261
0.57264 -47.3954
0.5728 -47.472
0.57296 -47.4115
0.57312 -47.218
0.57328 -46.9479
0.57344 -46.6738
0.5736 -46.44
0.57376 -46.2183
0.57392 -45.9603
0.57408 -45.6055
0.57424 -45.0936
0.5744 -44.376
0.57456 -43.4448
0.57472 -42.3604
0.57488 -41.2195
0.57504 -40.1351
0.5752 -39.216
0.57536 -38.5508
0.57552 -38.0792
0.57568 -37.7365
0.57584 -37.4503
0.576 -37.152
0.57616 -36.8819
0.57632 -36.5755
0.57648 -36.2933
0.57664 -36.116
0.5768 -36.12
0.57696 -36.4062
0.57712 -36.8335
0.57728 -37.3253
0.57744 -37.801
0.5776 -38.184
0.57776 -38.4662
0.57792 -38.5629
0.57808 -38.5065
0.57824 -38.3533
0.5784 -38.184
0.57856 -38.0913
0.57872 -38.0469
0.57888 -38.0631
0.57904 -38.1114
0.5792 -38.184
0.57936 -38.2324
0.57952 -38.2525
0.57968 -38.2445
0.57984 -38.2163
0.58 -38.184
0.58016 -38.0913
0.58032 -38.0187
0.58048 -37.9986
0.58064 -38.051
0.5808 -38.184
0.58096 -38.2646
0.58112 -38.4017
0.58128 -38.6113
0.58144 -38.8895
0.5816 -39.216
0.58176 -39.349
0.58192 -39.4176
0.58208 -39.4176
0.58224 -39.3531
0.5824 -39.216
0.58256 -38.7726
0.58272 -38.2042
0.58288 -37.5511
0.58304 -36.8497
0.5832 -36.12
0.58336 -35.1727
0.58352 -34.1648
0.58368 -33.1248
0.58384 -32.0565
0.584 -30.96
0.58416 -29.7224
0.58432 -28.384
0.58448 -26.9288
0.58464 -25.3727
0.5848 -23.736
0.58496 -22.0993
0.58512 -20.4787
0.58528 -18.959
0.58544 -17.6166
0.5856 -16.512
0.58576 -15.7622
0.58592 -15.1535
0.58608 -14.5931
0.58624 -14.0247
0.5864 -13.416
0.58656 -12.9202
0.58672 -12.3679
0.58688 -11.8438
0.58704 -11.4689
0.5872 -11.352
0.58736 -11.6987
0.58752 -12.2631
0.58768 -12.9685
0.58784 -13.7264
0.588 -14.448
0.58816 -15.1172
0.58832 -15.5405
0.58848 -15.7138
0.58864 -15.6695
0.5888 -15.48
0.58896 -15.2422
0.58912 -14.9761
0.58928 -14.7262
0.58944 -14.5448
0.5896 -14.448
0.58976 -14.4077
0.58992 -14.4158
0.59008 -14.448
0.59024 -14.4682
0.5904 -14.448
0.59056 -14.3271
0.59072 -14.1416
0.59088 -13.9038
0.59104 -13.6498
0.5912 -13.416
0.59136 -13.158
0.59152 -13.0048
0.59168 -12.9887
0.59184 -13.1338
0.592 -13.416
0.59216 -13.6861
0.59232 -13.8998
0.59248 -13.9642
0.59264 -13.8111
0.5928 -13.416
0.59296 -12.7267
0.59312 -11.9204
0.59328 -11.1545
0.59344 -10.5861
0.5936 -10.32
0.59376 -10.3079
0.59392 -10.4692
0.59408 -10.6385
0.59424 -10.6385
0.5944 -10.32
0.59456 -9.5541
0.59472 -8.4656
0.59488 -7.2321
0.59504 -6.067
0.5952 -5.16
0.59536 -4.5271
0.59552 -4.2046
0.59568 -4.1119
0.59584 -4.132
0.596 -4.128
0.59616 -3.7853
0.59632 -3.2572
0.59648 -2.6848
0.59664 -2.2414
0.5968 -2.064
0.59696 -1.8826
0.59712 -1.8584
0.59728 -1.931
0.59744 -2.0358
0.5976 -2.064
0.59776 -1.5641
0.59792 -0.790125
0.59808 0.1410938
0.59824 1.1126
0.5984 2.064
0.59856 3.3137
0.59872 4.769
0.59888 6.5306
0.59904 8.6954
0.5992 11.352
0.59936 14.6415
0.59952 18.3664
0.59968 22.313
0.59984 26.2434
0.6 29.928
0.60016 33.0321
0.60032 35.7532
0.60048 38.1518
0.60064 40.3165
0.6008 42.312
0.60096 43.7955
0.60112 45.2185
0.60128 46.5448
0.60144 47.6736
0.6016 48.504
0.60176 48.4597
0.60192 48.2138
0.60208 47.9114
0.60224 47.6453
0.6024 47.472
0.60256 46.8673
0.60272 46.3674
0.60288 45.8797
0.60304 45.2588
0.6032 44.376
0.60336 42.7595
0.60352 40.9777
0.60368 39.1838
0.60384 37.5309
0.604 36.12
0.60416 34.7655
0.60432 33.6085
0.60448 32.512
0.60464 31.3309
0.6048 29.928
0.60496 28.1784
0.60512 26.183
0.60528 24.123
0.60544 22.2122
0.6056 20.64
0.60576 19.479
0.60592 18.6607
0.60608 18.1205
0.60624 17.7818
0.6064 17.544
0.60656 17.2094
0.60672 16.7257
0.60688 16.1855
0.60704 15.7259
0.6072 15.48
0.60736 15.3026
0.60752 15.226
0.60768 15.2542
0.60784 15.3631
0.608 15.48
0.60816 15.218
0.60832 14.581
0.60848 13.545
0.60864 12.1139
0.6088 10.32
0.60896 8.0182
0.60912 5.3938
0.60928 2.576
0.60944 -0.2821875
0.6096 -3.096
0.60976 -5.7969
0.60992 -8.4576
0.61008 -11.1101
0.61024 -13.7909
0.6104 -16.512
0.61056 -19.0033
0.61072 -21.4906
0.61088 -23.8811
0.61104 -26.0499
0.6112 -27.864
0.61136 -28.8154
0.61152 -29.412
0.61168 -29.7426
0.61184 -29.8917
0.612 -29.928
0.61216 -29.3959
0.61232 -28.9242
0.61248 -28.5453
0.61264 -28.2187
0.6128 -27.864
0.61296 -26.9449
0.61312 -25.9613
0.61328 -24.9333
0.61344 -23.8489
0.6136 -22.704
0.61376 -21.1842
0.61392 -19.6685
0.61408 -18.2011
0.61424 -16.8063
0.6144 -15.48
0.61456 -14.061
0.61472 -12.646
0.61488 -11.2149
0.61504 -9.7556
0.6152 -8.256
0.61536 -6.7282
0.61552 -5.156
0.61568 -3.6201
0.61584 -2.2172
0.616 -1.032
0.61616 -0.1572187
0.61632 0.5684062
0.61648 1.1691
0.61664 1.6649
0.6168 2.064
0.61696 2.3543
0.61712 2.6727
0.61728 2.9509
0.61744 3.1121
0.6176 3.096
0.61776 2.9347
0.61792 2.7977
0.61808 2.7453
0.61824 2.8299
0.6184 3.096
0.61856 3.612
0.61872 4.386
0.61888 5.3132
0.61904 6.2888
0.6192 7.224
0.61936 8.1028
0.61952 8.9978
0.61968 9.9693
0.61984 11.0779
0.62 12.384
0.62016 13.8554
0.62032 15.5203
0.62048 17.29
0.62064 19.0396
0.6208 20.64
0.62096 21.7889
0.62112 22.6476
0.62128 23.2281
0.62144 23.5748
0.6216 23.736
0.62176 23.4861
0.62192 23.2079
0.62208 22.962
0.62224 22.7927
0.6224 22.704
0.62256 22.3855
0.62272 22.1396
0.62288 21.9622
0.62304 21.8212
0.6232 21.672
0.62336 21.2124
0.62352 20.6763
0.62368 20.0635
0.62384 19.3661
0.624 18.576
0.62416 17.4997
0.62432 16.3266
0.62448 15.0728
0.62464 13.7546
0.6248 12.384
0.62496 10.9368
0.62512 9.4533
0.62528 7.9658
0.62544 6.5185
0.6256 5.16
0.62576 4.0111
0.62592 2.9912
0.62608 2.1325
0.62624 1.4714
0.6264 1.032
0.62656 0.8909063
0.62672 0.854625
0.62688 0.8828437
0.62704 0.9473438
0.6272 1.032
0.62736 1.1449
0.62752 1.165
0.62768 1.1167
0.62784 1.0562
0.628 1.032
0.62816 1.0884
0.62832 1.1207
0.62848 1.1288
0.62864 1.0965
0.6288 1.032
0.62896 0.9271875
0.62912 0.7498125
0.62928 0.5200313
0.62944 0.2660625
0.6296 0
0.62976 -0.3104063
0.62992 -0.7296563
0.63008 -1.3102
0.63024 -2.0922
0.6304 -3.096
0.63056 -4.2328
0.63072 -5.4382
0.63088 -6.5911
0.63104 -7.5667
0.6312 -8.256
0.63136 -8.4656
0.63152 -8.3648
0.63168 -8.0464
0.63184 -7.6271
0.632 -7.224
0.63216 -6.7765
0.63232 -6.4702
0.63248 -6.3008
0.63264 -6.2283
0.6328 -6.192
0.63296 -6.0106
0.63312 -5.805
0.63328 -5.5833
0.63344 -5.3656
0.6336 -5.16
0.63376 -4.8496
0.63392 -4.5271
0.63408 -4.1482
0.63424 -3.6805
0.6344 -3.096
0.63456 -2.3462
0.63472 -1.4996
0.63488 -0.6167812
0.63504 0.241875
0.6352 1.032
0.63536 1.6367
0.63552 2.1325
0.63568 2.5236
0.63584 2.834
0.636 3.096
0.63616 3.2411
0.63632 3.4145
0.63648 3.6241
0.63664 3.87
0.6368 4.128
0.63696 4.2691
0.63712 4.3739
0.63728 4.4062
0.63744 4.3296
0.6376 4.128
0.63776 3.7208
0.63792 3.2613
0.63808 2.8017
0.63824 2.3905
0.6384 2.064
0.63856 1.7697
0.63872 1.5601
0.63888 1.3988
0.63904 1.2336
0.6392 1.032
0.63936 0.7619063
0.63952 0.4716563
0.63968 0.209625
0.63984 0.0362812
0.64 0
0.64016 0.1209375
0.64032 0.3507188
0.64048 0.6248438
0.64064 0.8747813
0.6408 1.032
0.64096 1.0643
0.64112 0.9433125
0.64128 0.693375
0.64144 0.3587813
0.6416 0
0.64176 -0.306375
0.64192 -0.5724375
0.64208 -0.7780312
0.64224 -0.9271875
0.6424 -1.032
0.64256 -1.0441
0.64272 -1.0401
0.64288 -1.036
0.64304 -1.036
0.6432 -1.032
0.64336 -0.9634687
0.64352 -0.8465625
0.64368 -0.6530625
0.64384 -0.3668437
0.644 0
0.64416 0.4273125
0.64432 0.8264062
0.64448 1.0965
0.64464 1.1771
0.6448 1.032
0.64496 0.7135313
0.64512 0.3225
0.64528 -0.0080625
0.64544 -0.1491563
0.6456 0
0.64576 0.4918125
0.64592 1.2094
0.64608 1.9914
0.64624 2.6687
0.6464 3.096
0.64656 3.1887
0.64672 3.0073
0.64688 2.6647
0.64704 2.3059
0.6472 2.064
0.64736 2.0358
0.64752 2.2212
0.64768 2.5397
0.64784 2.8662
0.648 3.096
0.64816 3.2008
0.64832 3.1847
0.64848 3.1041
0.64864 3.0436
0.6488 3.096
0.64896 3.3298
0.64912 3.7249
0.64928 4.2248
0.64944 4.7327
0.6496 5.16
0.64976 5.4019
0.64992 5.4865
0.65008 5.4382
0.65024 5.3051
0.6504 5.16
0.65056 5.0189
0.65072 5.0149
0.65088 5.1963
0.65104 5.5913
0.6512 6.192
0.65136 6.8128
0.65152 7.4981
0.65168 8.1834
0.65184 8.8002
0.652 9.288
0.65216 9.3646
0.65232 9.2316
0.65248 8.9615
0.65264 8.6188
0.6528 8.256
0.65296 7.6715
0.65312 7.0789
0.65328 6.4782
0.65344 5.8453
0.6536 5.16
0.65376 4.2288
0.65392 3.2008
0.65408 2.1285
0.65424 1.0522
0.6544 0
0.65456 -1.1167
0.65472 -2.2656
0.65488 -3.4749
0.65504 -4.777
0.6552 -6.192
0.65536 -7.6715
0.65552 -9.2195
0.65568 -10.7594
0.65584 -12.1865
0.656 -13.416
0.65616 -14.2343
0.65632 -14.8229
0.65648 -15.2018
0.65664 -15.4074
0.6568 -15.48
0.65696 -15.3067
0.65712 -15.1938
0.65728 -15.1857
0.65744 -15.2905
0.6576 -15.48
0.65776 -15.5405
0.65792 -15.6372
0.65808 -15.6977
0.65824 -15.6614
0.6584 -15.48
0.65856 -15.0204
0.65872 -14.5125
0.65888 -14.0328
0.65904 -13.6498
0.6592 -13.416
0.65936 -13.2668
0.65952 -13.2628
0.65968 -13.3475
0.65984 -13.4241
0.66 -13.416
0.66016 -13.2185
0.66032 -12.8597
0.66048 -12.384
0.66064 -11.8559
0.6608 -11.352
0.66096 -10.8965
0.66112 -10.5538
0.66128 -10.3482
0.66144 -10.2757
0.6616 -10.32
0.66176 -10.3523
0.66192 -10.3764
0.66208 -10.3805
0.66224 -10.3603
0.6624 -10.32
0.66256 -10.1386
0.66272 -9.9048
0.66288 -9.6548
0.66304 -9.4412
0.6632 -9.288
0.66336 -9.0864
0.66352 -8.897
0.66368 -8.7115
0.66384 -8.51
0.664 -8.256
0.66416 -7.7884
0.66432 -7.1353
0.66448 -6.3008
0.66464 -5.293
0.6648 -4.128
0.66496 -2.8138
0.66512 -1.3868
0.66528 0.1048125
0.66544 1.6206
0.6656 3.096
0.66576 4.3336
0.66592 5.4019
0.66608 6.2525
0.66624 6.8612
0.6664 7.224
0.66656 7.1555
0.66672 7.0063
0.66688 6.9055
0.66704 6.962
0.6672 7.224
0.66736 7.4014
0.66752 7.6997
0.66768 8.0222
0.66784 8.2439
0.668 8.256
0.66816 7.7037
0.66832 6.9055
0.66848 5.9743
0.66864 5.0149
0.6688 4.128
0.66896 3.1726
0.66912 2.3462
0.66928 1.6004
0.66944 0.8505938
0.6696 0
0.66976 -1.1126
0.66992 -2.4389
0.67008 -3.9506
0.67024 -5.5752
0.6704 -7.224
0.67056 -8.7438
0.67072 -10.1588
0.67088 -11.4326
0.67104 -12.5251
0.6712 -13.416
0.67136 -13.9642
0.67152 -14.3472
0.67168 -14.5528
0.67184 -14.581
0.672 -14.448
0.67216 -14.0691
0.67232 -13.7264
0.67248 -13.4845
0.67264 -13.3837
0.6728 -13.416
0.67296 -13.4603
0.67312 -13.5853
0.67328 -13.6901
0.67344 -13.6538
0.6736 -13.416
0.67376 -12.9202
0.67392 -12.3477
0.67408 -11.8196
0.67424 -11.4528
0.6744 -11.352
0.67456 -11.5616
0.67472 -12.0817
0.67488 -12.8234
0.67504 -13.6538
0.6752 -14.448
0.67536 -15.0607
0.67552 -15.4518
0.67568 -15.6171
0.67584 -15.605
0.676 -15.48
0.67616 -15.2946
0.67632 -15.1575
0.67648 -15.1293
0.67664 -15.2381
0.6768 -15.48
0.67696 -15.742
0.67712 -16.0323
0.67728 -16.2903
0.67744 -16.4677
0.6776 -16.512
0.67776 -16.2903
0.67792 -15.9154
0.67808 -15.4357
0.67824 -14.9237
0.6784 -14.448
0.67856 -13.9602
0.67872 -13.6055
0.67888 -13.4079
0.67904 -13.3596
0.6792 -13.416
0.67936 -13.4523
0.67952 -13.4281
0.67968 -13.2709
0.67984 -12.9282
0.68 -12.384
0.68016 -11.6302
0.68032 -10.7513
0.68048 -9.8362
0.68064 -8.9776
0.6808 -8.256
0.68096 -7.744
0.68112 -7.357
0.68128 -7.0184
0.68144 -6.6475
0.6816 -6.192
0.68176 -5.7405
0.68192 -5.2245
0.68208 -4.7166
0.68224 -4.3215
0.6824 -4.128
0.68256 -4.3255
0.68272 -4.7246
0.68288 -5.2406
0.68304 -5.7566
0.6832 -6.192
0.68336 -6.6072
0.68352 -6.8652
0.68368 -7.0103
0.68384 -7.1031
0.684 -7.224
0.68416 -7.5183
0.68432 -7.9013
0.68448 -8.3527
0.68464 -8.8284
0.6848 -9.288
0.68496 -9.7435
0.68512 -10.1628
0.68528 -10.5498
0.68544 -10.9368
0.6856 -11.352
0.68576 -11.8075
0.68592 -12.3558
0.68608 -12.9927
0.68624 -13.7022
0.6864 -14.448
0.68656 -15.093
0.68672 -15.6695
0.68688 -16.125
0.68704 -16.4152
0.6872 -16.512
0.68736 -16.3266
0.68752 -16.0484
0.68768 -15.7662
0.68784 -15.5606
0.688 -15.48
0.68816 -15.4236
0.68832 -15.476
0.68848 -15.5566
0.68864 -15.5808
0.6888 -15.48
0.68896 -15.1333
0.68912 -14.6697
0.68928 -14.1698
0.68944 -13.7264
0.6896 -13.416
0.68976 -13.2265
0.68992 -13.2023
0.69008 -13.2789
0.69024 -13.3797
0.6904 -13.416
0.69056 -13.2588
0.69072 -12.9363
0.69088 -12.4687
0.69104 -11.9204
0.6912 -11.352
0.69136 -10.7715
0.69152 -10.2515
0.69168 -9.8201
0.69184 -9.4936
0.692 -9.288
0.69216 -9.1227
0.69232 -9.03
0.69248 -9.0179
0.69264 -9.1066
0.6928 -9.288
0.69296 -9.4251
0.69312 -9.5218
0.69328 -9.546
0.69344 -9.4734
0.6936 -9.288
0.69376 -8.8607
0.69392 -8.2883
0.69408 -7.615
0.69424 -6.9015
0.6944 -6.192
0.69456 -5.4301
0.69472 -4.644
0.69488 -3.8297
0.69504 -2.9751
0.6952 -2.064
0.69536 -1.0925
0.69552 -0.0524062
0.69568 1.0199
0.69584 2.0842
0.696 3.096
0.69616 3.9184
0.69632 4.64
0.69648 5.2568
0.69664 5.7687
0.6968 6.192
0.69696 6.4057
0.69712 6.6032
0.69728 6.8048
0.69744 7.0144
0.6976 7.224
0.69776 7.2643
0.69792 7.3087
0.69808 7.3369
0.69824 7.3167
0.6984 7.224
0.69856 6.9015
0.69872 6.5226
0.69888 6.0953
0.69904 5.6357
0.6992 5.16
0.69936 4.5997
0.69952 4.0998
0.69968 3.6725
0.69984 3.3419
0.7 3.096
0.70016 2.8703
0.70032 2.6808
0.70048 2.4953
0.70064 2.2897
0.7008 2.064
0.70096 1.7858
0.70112 1.4996
0.70128 1.2497
0.70144 1.0804
0.7016 1.032
0.70176 1.0522
0.70192 1.1086
0.70208 1.1529
0.70224 1.1408
0.7024 1.032
0.70256 0.7538437
0.70272 0.3426562
0.70288 -0.1410938
0.70304 -0.6208125
0.7032 -1.032
0.70336 -1.3827
0.70352 -1.6448
0.70368 -1.8262
0.70384 -1.9511
0.704 -2.064
0.70416 -2.2454
0.70432 -2.5357
0.70448 -2.9549
0.70464 -3.4911
0.7048 -4.128
0.70496 -4.8254
0.70512 -5.543
0.70528 -6.2122
0.70544 -6.7765
0.7056 -7.224
0.70576 -7.5183
0.70592 -7.7964
0.70608 -8.1431
0.70624 -8.6269
0.7064 -9.288
0.70656 -9.9854
0.70672 -10.7594
0.70688 -11.4971
0.70704 -12.0736
0.7072 -12.384
0.70736 -12.2107
0.70752 -11.7874
0.70768 -11.2391
0.70784 -10.707
0.708 -10.32
0.70816 -10.0015
0.70832 -9.9169
0.70848 -10.0056
0.70864 -10.1789
0.7088 -10.32
0.70896 -10.187
0.70912 -9.8806
0.70928 -9.4291
0.70944 -8.8728
0.7096 -8.256
0.70976 -7.5142
0.70992 -6.7403
0.71008 -5.9259
0.71024 -5.0552
0.7104 -4.128
0.71056 -3.1444
0.71072 -2.1688
0.71088 -1.2739
0.71104 -0.532125
0.7112 0
0.71136 0.2862187
0.71152 0.4595625
0.71168 0.596625
0.71184 0.7659375
0.712 1.032
0.71216 1.3626
0.71232 1.8181
0.71248 2.322
0.71264 2.7775
0.7128 3.096
0.71296 3.1484
0.71312 3.0355
0.71328 2.7775
0.71344 2.4349
0.7136 2.064
0.71376 1.7052
0.71392 1.5077
0.71408 1.5037
0.71424 1.7012
0.7144 2.064
0.71456 2.5236
0.71472 3.0355
0.71488 3.5153
0.71504 3.8982
0.7152 4.128
0.71536 4.1764
0.71552 4.128
0.71568 4.0554
0.71584 4.0393
0.716 4.128
0.71616 4.2772
0.71632 4.519
0.71648 4.7932
0.71664 5.027
0.7168 5.16
0.71696 5.0512
0.71712 4.8093
0.71728 4.515
0.71744 4.257
0.7176 4.128
0.71776 4.0635
0.71792 4.1723
0.71808 4.4344
0.71824 4.7891
0.7184 5.16
0.71856 5.3051
0.71872 5.285
0.71888 5.0834
0.71904 4.6964
0.7192 4.128
0.71936 3.2976
0.71952 2.318
0.71968 1.2416
0.71984 0.1048125
0.72 -1.032
0.72016 -2.1285
0.72032 -3.1283
0.72048 -3.9788
0.72064 -4.6561
0.7208 -5.16
0.72096 -5.4744
0.72112 -5.7607
0.72128 -6.1073
0.72144 -6.5871
0.7216 -7.224
0.72176 -7.8972
0.72192 -8.6551
0.72208 -9.3888
0.72224 -9.9773
0.7224 -10.32
0.72256 -10.2031
0.72272 -9.8443
0.72288 -9.3323
0.72304 -8.768
0.7232 -8.256
0.72336 -7.744
0.72352 -7.4175
0.72368 -7.2643
0.72384 -7.224
0.724 -7.224
0.72416 -7.1071
0.72432 -6.9458
0.72448 -6.7282
0.72464 -6.4742
0.7248 -6.192
0.72496 -5.8413
0.72512 -5.4664
0.72528 -5.0633
0.72544 -4.6158
0.7256 -4.128
0.72576 -3.5999
0.72592 -3.0839
0.72608 -2.6243
0.72624 -2.2736
0.7264 -2.064
0.72656 -2.0197
0.72672 -2.0519
0.72688 -2.1043
0.72704 -2.1204
0.7272 -2.064
0.72736 -1.9229
0.72752 -1.6972
0.72768 -1.4311
0.72784 -1.1892
0.728 -1.032
0.72816 -0.9755625
0.72832 -0.9876563
0.72848 -1.028
0.72864 -1.0562
0.7288 -1.032
0.72896 -0.919125
0.72912 -0.7175625
0.72928 -0.4595625
0.72944 -0.2015625
0.7296 0
0.72976 0.1088437
0.72992 0.1370625
0.73008 0.1048125
0.73024 0.0443437
0.7304 0
0.73056 0.0040313
0.73072 0.1007813
0.73088 0.3104063
0.73104 0.6248438
0.7312 1.032
0.73136 1.4714
0.73152 1.939
0.73168 2.3865
0.73184 2.7816
0.732 3.096
0.73216 3.2411
0.73232 3.2895
0.73248 3.2653
0.73264 3.1928
0.7328 3.096
0.73296 2.8985
0.73312 2.6929
0.73328 2.4873
0.73344 2.2777
0.7336 2.064
0.73376 1.7738
0.73392 1.4956
0.73408 1.2658
0.73424 1.1046
0.7344 1.032
0.73456 0.9916875
0.73472 1.0118
0.73488 1.0562
0.73504 1.0763
0.7352 1.032
0.73536 0.8505938
0.73552 0.5401875
0.73568 0.112875
0.73584 -0.41925
0.736 -1.032
0.73616 -1.6689
0.73632 -2.322
0.73648 -2.967
0.73664 -3.5757
0.7368 -4.128
0.73696 -4.519
0.73712 -4.8173
0.73728 -5.027
0.73744 -5.1398
0.7376 -5.16
0.73776 -5.0068
0.73792 -4.8012
0.73808 -4.5714
0.73824 -4.3417
0.7384 -4.128
0.73856 -3.8579
0.73872 -3.6241
0.73888 -3.4266
0.73904 -3.2532
0.7392 -3.096
0.73936 -2.8904
0.73952 -2.6888
0.73968 -2.4833
0.73984 -2.2777
0.74 -2.064
0.74016 -1.8141
0.74032 -1.5762
0.74048 -1.3626
0.74064 -1.1812
0.7408 -1.032
0.74096 -0.903
0.74112 -0.7619063
0.74128 -0.5724375
0.74144 -0.3144375
0.7416 0
0.74176 0.3225
0.74192 0.645
0.74208 0.9070312
0.74224 1.0441
0.7424 1.032
0.74256 0.8264062
0.74272 0.5401875
0.74288 0.258
0.74304 0.0564375
0.7432 0
0.74336 0.0765938
0.74352 0.2983125
0.74368 0.5885625
0.74384 0.8586562
0.744 1.032
0.74416 1.0239
0.74432 0.8667188
0.74448 0.6006563
0.74464 0.29025
0.7448 0
0.74496 -0.2176875
0.74512 -0.3225
0.74528 -0.2983125
0.74544 -0.177375
0.7456 0
0.74576 0.1693125
0.74592 0.2821875
0.74608 0.3023437
0.74624 0.2055938
0.7464 0
0.74656 -0.2942813
0.74672 -0.61275
0.74688 -0.8788125
0.74704 -1.032
0.7472 -1.032
0.74736 -0.8909063
0.74752 -0.6490313
0.74768 -0.370875
0.74784 -0.1330313
0.748 0
0.74816 -0.0362812
0.74832 -0.2176875
0.74848 -0.4918125
0.74864 -0.7860938
0.7488 -1.032
0.74896 -1.1852
0.74912 -1.2336
0.74928 -1.1933
0.74944 -1.1046
0.7496 -1.032
0.74976 -1.036
0.74992 -1.1408
0.75008 -1.3585
0.75024 -1.673
0.7504 -2.064
0.75056 -2.4873
0.75072 -2.9347
0.75088 -3.3701
0.75104 -3.7773
0.7512 -4.128
0.75136 -4.3497
0.75152 -4.4666
0.75168 -4.4666
0.75184 -4.3497
0.752 -4.128
0.75216 -3.8015
0.75232 -3.483
0.75248 -3.229
0.75264 -3.092
0.7528 -3.096
0.75296 -3.1968
0.75312 -3.4104
0.75328 -3.6765
0.75344 -3.9345
0.7536 -4.128
0.75376 -4.1482
0.75392 -4.0313
0.75408 -3.7934
0.75424 -3.4669
0.7544 -3.096
0.75456 -2.6929
0.75472 -2.3583
0.75488 -2.1325
0.75504 -2.0358
0.7552 -2.064
0.75536 -2.1608
0.75552 -2.2696
0.75568 -2.326
0.75584 -2.2656
0.756 -2.064
0.75616 -1.6931
0.75632 -1.2215
0.75648 -0.7296563
0.75664 -0.2983125
0.7568 0
0.75696 0.1249687
0.75712 0.1169063
0.75728 0.048375
0.75744 -0.0120938
0.7576 0
0.75776 0.1209375
0.75792 0.35475
0.75808 0.6329063
0.75824 0.8788125
0.7584 1.032
0.75856 1.0562
0.75872 1.0038
0.75888 0.9392812
0.75904 0.9312187
0.7592 1.032
0.75936 1.2376
0.75952 1.5278
0.75968 1.8181
0.75984 2.0197
0.76 2.064
0.76016 1.9027
0.76032 1.6246
0.76048 1.3182
0.76064 1.0884
0.7608 1.032
0.76096 1.1731
0.76112 1.5238
0.76128 2.0237
0.76144 2.58
0.7616 3.096
0.76176 3.4145
0.76192 3.5515
0.76208 3.5193
0.76224 3.35
0.7624 3.096
0.76256 2.7453
0.76272 2.4268
0.76288 2.193
0.76304 2.0721
0.7632 2.064
0.76336 2.0761
0.76352 2.1245
0.76368 2.1648
0.76384 2.1567
0.764 2.064
0.76416 1.806
0.76432 1.4351
0.76448 0.9836251
0.76464 0.4918125
0.7648 0
0.76496 -0.4877812
0.76512 -0.9312187
0.76528 -1.3303
0.76544 -1.7012
0.7656 -2.064
0.76576 -2.4147
0.76592 -2.8058
0.76608 -3.229
0.76624 -3.6805
0.7664 -4.128
0.76656 -4.4908
0.76672 -4.8053
0.76688 -5.035
0.76704 -5.156
0.7672 -5.16
0.76736 -5.0028
0.76752 -4.7891
0.76768 -4.5472
0.76784 -4.3134
0.768 -4.128
0.76816 -3.9748
0.76832 -3.9224
0.76848 -3.9506
0.76864 -4.0313
0.7688 -4.128
0.76896 -4.1764
0.76912 -4.2006
0.76928 -4.1965
0.76944 -4.1683
0.7696 -4.128
0.76976 -4.0635
0.76992 -4.0192
0.77008 -4.0071
0.77024 -4.0393
0.7704 -4.128
0.77056 -4.261
0.77072 -4.4465
0.77088 -4.6682
0.77104 -4.9141
0.7712 -5.16
0.77136 -5.3414
0.77152 -5.4503
0.77168 -5.4623
0.77184 -5.3696
0.772 -5.16
0.77216 -4.7891
0.77232 -4.3497
0.77248 -3.8861
0.77264 -3.4508
0.7728 -3.096
0.77296 -2.8219
0.77312 -2.6848
0.77328 -2.6929
0.77344 -2.838
0.7736 -3.096
0.77376 -3.358
0.77392 -3.6241
0.77408 -3.8579
0.77424 -4.0353
0.7744 -4.128
0.77456 -4.0232
0.77472 -3.7571
0.77488 -3.3298
0.77504 -2.7533
0.7752 -2.064
0.77536 -1.2981
0.77552 -0.5442188
0.77568 0.129
0.77584 0.6651562
0.776 1.032
0.77616 1.2215
0.77632 1.3424
0.77648 1.4714
0.77664 1.6891
0.7768 2.064
0.77696 2.5598
0.77712 3.1968
0.77728 3.9103
0.77744 4.5997
0.7776 5.16
0.77776 5.4019
0.77792 5.3938
0.77808 5.1439
0.77824 4.7004
0.7784 4.128
0.77856 3.4185
0.77872 2.7896
0.77888 2.326
0.77904 2.0801
0.7792 2.064
0.77936 2.1849
0.77952 2.4349
0.77968 2.7332
0.77984 2.9831
0.78 3.096
0.78016 2.963
0.78032 2.6405
0.78048 2.1688
0.78064 1.6085
0.7808 1.032
0.78096 0.48375
0.78112 0.0120938
0.78128 -0.3829687
0.78144 -0.725625
0.7816 -1.032
0.78176 -1.2981
0.78192 -1.5641
0.78208 -1.7979
0.78224 -1.9713
0.7824 -2.064
0.78256 -2.0438
0.78272 -1.9995
0.78288 -1.9632
0.78304 -1.9793
0.7832 -2.064
0.78336 -2.1769
0.78352 -2.2978
0.78368 -2.3583
0.78384 -2.2897
0.784 -2.064
0.78416 -1.6972
0.78432 -1.3061
0.78448 -0.99975
0.78464 -0.8828437
0.7848 -1.032
0.78496 -1.4674
0.78512 -2.1245
0.78528 -2.8823
0.78544 -3.5959
0.7856 -4.128
0.78576 -4.3618
0.78592 -4.3094
0.78608 -4.0151
0.78624 -3.5717
0.7864 -3.096
0.78656 -2.6848
0.78672 -2.451
0.78688 -2.447
0.78704 -2.6687
0.7872 -3.096
0.78736 -3.608
0.78752 -4.1482
0.78768 -4.6319
0.78784 -4.9826
0.788 -5.16
0.78816 -5.0874
0.78832 -4.8738
0.78848 -4.5876
0.78864 -4.3175
0.7888 -4.128
0.78896 -4.0111
0.78912 -4.0071
0.78928 -4.0716
0.78944 -4.1361
0.7896 -4.128
0.78976 -3.9345
0.78992 -3.5918
0.79008 -3.1283
0.79024 -2.6002
0.7904 -2.064
0.79056 -1.552
0.79072 -1.0965
0.79088 -0.693375
0.79104 -0.3265312
0.7912 0
0.79136 0.2539688
0.79152 0.41925
0.79168 0.4515
0.79184 0.3104063
0.792 0
0.79216 -0.467625
0.79232 -0.9231563
0.79248 -1.2376
0.79264 -1.294
0.7928 -1.032
0.79296 -0.4797188
0.79312 0.2660625
0.79328 1.0562
0.79344 1.7012
0.7936 2.064
0.79376 1.9834
0.79392 1.6004
0.79408 1.036
0.79424 0.4515
0.7944 0
0.79456 -0.258
0.79472 -0.2942813
0.79488 -0.1854375
0.79504 -0.0524062
0.7952 0
0.79536 -0.1048125
0.79552 -0.3466875
0.79568 -0.6409687
0.79584 -0.8949375
0.796 -1.032
0.79616 -1.0118
0.79632 -0.8989688
0.79648 -0.7941563
0.79664 -0.8102813
0.7968 -1.032
0.79696 -1.4633
0.79712 -2.0237
0.79728 -2.58
0.79744 -2.9751
0.7976 -3.096
0.79776 -2.8582
0.79792 -2.3865
0.79808 -1.8221
0.79824 -1.3263
0.7984 -1.032
0.79856 -0.9876563
0.79872 -1.1892
0.79888 -1.5238
0.79904 -1.8584
0.7992 -2.064
0.79936 -2.0519
0.79952 -1.8544
0.79968 -1.548
0.79984 -1.2376
0.8 -1.032
0.80016 -0.99975
0.80032 -1.1489
0.80048 -1.4392
0.80064 -1.7738
0.8008 -2.064
0.80096 -2.2011
0.80112 -2.1527
0.80128 -1.9148
0.80144 -1.5198
0.8016 -1.032
0.80176 -0.5401875
0.80192 -0.1330313
0.80208 0.1209375
0.80224 0.1693125
0.8024 0
0.80256 -0.3789375
0.80272 -0.87075
0.80288 -1.3827
0.80304 -1.806
0.8032 -2.064
0.80336 -2.0922
0.80352 -1.931
0.80368 -1.6407
0.80384 -1.3142
0.804 -1.032
0.80416 -0.8465625
0.80432 -0.790125
0.80448 -0.8304375
0.80464 -0.9312187
0.8048 -1.032
0.80496 -1.1005
0.80512 -1.1167
0.80528 -1.0925
0.80544 -1.0562
0.8056 -1.032
0.80576 -1.0239
0.80592 -1.0441
0.80608 -1.0683
0.80624 -1.0723
0.8064 -1.032
0.80656 -0.9110625
0.80672 -0.725625
0.80688 -0.4958438
0.80704 -0.241875
0.8072 0
0.80736 0.1733438
0.80752 0.274125
0.80768 0.2781563
0.80784 0.1854375
0.808 0
0.80816 -0.2700938
0.80832 -0.5603437
0.80848 -0.8143125
0.80864 -0.9795937
0.8088 -1.032
0.80896 -0.9594375
0.80912 -0.774
0.80928 -0.5240625
0.80944 -0.2459063
0.8096 0
0.80976 0.1572187
0.80992 0.22575
0.81008 0.209625
0.81024 0.129
0.8104 0
0.81056 -0.1854375
0.81072 -0.387
0.81088 -0.596625
0.81104 -0.8102813
0.8112 -1.032
0.81136 -1.2457
0.81152 -1.4633
0.81168 -1.681
0.81184 -1.8826
0.812 -2.064
0.81216 -2.1849
0.81232 -2.2535
0.81248 -2.2575
0.81264 -2.189
0.8128 -2.064
0.81296 -1.8987
0.81312 -1.7697
0.81328 -1.7294
0.81344 -1.8221
0.8136 -2.064
0.81376 -2.3825
0.81392 -2.7332
0.81408 -3.0275
0.81424 -3.1686
0.8144 -3.096
0.81456 -2.7816
0.81472 -2.318
0.81488 -1.7979
0.81504 -1.3343
0.8152 -1.032
0.81536 -0.9392812
0.81552 -1.0683
0.81568 -1.3585
0.81584 -1.7254
0.816 -2.064
0.81616 -2.2656
0.81632 -2.2696
0.81648 -2.0438
0.81664 -1.6085
0.8168 -1.032
0.81696 -0.4152188
0.81712 0.09675
0.81728 0.3910313
0.81744 0.3668437
0.8176 0
0.81776 -0.6691875
0.81792 -1.4875
0.81808 -2.2696
0.81824 -2.8461
0.8184 -3.096
0.81856 -2.9509
0.81872 -2.5115
0.81888 -1.9229
0.81904 -1.3787
0.8192 -1.032
0.81936 -0.9594375
0.81952 -1.1529
0.81968 -1.4996
0.81984 -1.8544
0.82 -2.064
0.82016 -2.0076
0.82032 -1.6851
0.82048 -1.165
0.82064 -0.5603437
0.8208 0
0.82096 0.4071562
0.82112 0.596625
0.82128 0.5603437
0.82144 0.3426562
0.8216 0
0.82176 -0.3789375
0.82192 -0.7215937
0.82208 -0.9634687
0.82224 -1.0683
0.8224 -1.032
0.82256 -0.87075
0.82272 -0.6248438
0.82288 -0.3587813
0.82304 -0.1370625
0.8232 0
0.82336 0.048375
0.82352 0.0241875
0.82368 -0.0241875
0.82384 -0.048375
0.824 0
0.82416 0.145125
0.82432 0.3749063
0.82448 0.645
0.82464 0.8828437
0.8248 1.032
0.82496 1.028
0.82512 0.8828437
0.82528 0.6248438
0.82544 0.3104063
0.8256 0
0.82576 -0.2459063
0.82592 -0.3829687
0.82608 -0.387
0.82624 -0.2539688
0.8264 0
0.82656 0.3507188
0.82672 0.7659375
0.82688 1.2174
0.82704 1.6609
0.8272 2.064
0.82736 2.3341
0.82752 2.4792
0.82768 2.4792
0.82784 2.3381
0.828 2.064
0.82816 1.6367
0.82832 1.165
0.82848 0.7014375
0.82864 0.2983125
0.8288 0
0.82896 -0.1814063
0.82912 -0.2459063
0.82928 -0.2136563
0.82944 -0.1209375
0.8296 0
0.82976 0.0886875
0.82992 0.1410938
0.83008 0.1410938
0.83024 0.0927188
0.8304 0
0.83056 -0.1491563
0.83072 -0.3305625
0.83088 -0.5442188
0.83104 -0.7780312
0.8312 -1.032
0.83136 -1.2739
0.83152 -1.5198
0.83168 -1.7415
0.83184 -1.9269
0.832 -2.064
0.83216 -2.1124
0.83232 -2.1285
0.83248 -2.1124
0.83264 -2.0882
0.8328 -2.064
0.83296 -2.0318
0.83312 -2.0197
0.83328 -2.0277
0.83344 -2.0438
0.8336 -2.064
0.83376 -2.0721
0.83392 -2.0801
0.83408 -2.0842
0.83424 -2.0761
0.8344 -2.064
0.83456 -2.0237
0.83472 -1.9955
0.83488 -1.9874
0.83504 -2.0116
0.8352 -2.064
0.83536 -2.1285
0.83552 -2.1849
0.83568 -2.2051
0.83584 -2.1648
0.836 -2.064
0.83616 -1.9108
0.83632 -1.7778
0.83648 -1.7254
0.83664 -1.8141
0.8368 -2.064
0.83696 -2.4187
0.83712 -2.8058
0.83728 -3.1081
0.83744 -3.229
0.8376 -3.096
0.83776 -2.6929
0.83792 -2.1406
0.83808 -1.5843
0.83824 -1.1731
0.8384 -1.032
0.83856 -1.1973
0.83872 -1.6246
0.83888 -2.189
0.83904 -2.7292
0.8392 -3.096
0.83936 -3.1484
0.83952 -2.8904
0.83968 -2.3784
0.83984 -1.7173
0.84 -1.032
0.84016 -0.4434375
0.84032 -0.0403125
0.84048 0.1531875
0.84064 0.1531875
0.8408 0
0.84096 -0.2700938
0.84112 -0.5603437
0.84128 -0.8143125
0.84144 -0.9795937
0.8416 -1.032
0.84176 -0.9795937
0.84192 -0.8143125
0.84208 -0.5684062
0.84224 -0.2781563
0.8424 0
0.84256 0.2055938
0.84272 0.3184688
0.84288 0.3184688
0.84304 0.2055938
0.8432 0
0.84336 -0.2942813
0.84352 -0.5885625
0.84368 -0.8264062
0.84384 -0.9795937
0.844 -1.032
0.84416 -0.9957188
0.84432 -0.9231563
0.84448 -0.87075
0.84464 -0.8949375
0.8448 -1.032
0.84496 -1.2658
0.84512 -1.5601
0.84528 -1.8423
0.84544 -2.0277
0.8456 -2.064
0.84576 -1.9269
0.84592 -1.673
0.84608 -1.3787
0.84624 -1.1368
0.8464 -1.032
0.84656 -1.0844
0.84672 -1.286
0.84688 -1.5762
0.84704 -1.8665
0.8472 -2.064
0.84736 -2.0882
0.84752 -1.9552
0.84768 -1.6931
0.84784 -1.3626
0.848 -1.032
0.84816 -0.7619063
0.84832 -0.6208125
0.84848 -0.6208125
0.84864 -0.7699688
0.8488 -1.032
0.84896 -1.3464
0.84912 -1.6568
0.84928 -1.9068
0.84944 -2.0519
0.8496 -2.064
0.84976 -1.9269
0.84992 -1.7012
0.85008 -1.4351
0.85024 -1.1933
0.8504 -1.032
0.85056 -0.9473438
0.85072 -0.9473438
0.85088 -0.9957188
0.85104 -1.0401
0.8512 -1.032
0.85136 -0.9433125
0.85152 -0.7699688
0.85168 -0.5280938
0.85184 -0.2539688
0.852 0
0.85216 0.1854375
0.85232 0.2781563
0.85248 0.2660625
0.85264 0.1652812
0.8528 0
0.85296 -0.1975312
0.85312 -0.3466875
0.85328 -0.3829687
0.85344 -0.2700938
0.8536 0
0.85376 0.3507188
0.85392 0.725625
0.85408 1.0199
0.85424 1.1408
0.8544 1.032
0.85456 0.6530625
0.85472 0.1330313
0.85488 -0.403125
0.85504 -0.8264062
0.8552 -1.032
0.85536 -0.9876563
0.85552 -0.74175
0.85568 -0.403125
0.85584 -0.1169063
0.856 0
0.85616 -0.1370625
0.85632 -0.5119687
0.85648 -1.036
0.85664 -1.5883
0.8568 -2.064
0.85696 -2.3341
0.85712 -2.4147
0.85728 -2.3502
0.85744 -2.2051
0.8576 -2.064
0.85776 -1.9632
0.85792 -1.9471
0.85808 -1.9834
0.85824 -2.0358
0.8584 -2.064
0.85856 -2.0197
0.85872 -1.9632
0.85888 -1.9269
0.85904 -1.9552
0.8592 -2.064
0.85936 -2.2253
0.85952 -2.3825
0.85968 -2.4591
0.85984 -2.3663
0.86 -2.064
0.86016 -1.5561
0.86032 -0.9675
0.86048 -0.4273125
0.86064 -0.0725625
0.8608 0
0.86096 -0.2338125
0.86112 -0.6974062
0.86128 -1.2537
0.86144 -1.7496
0.8616 -2.064
0.86176 -2.1285
0.86192 -1.9552
0.86208 -1.6327
0.86224 -1.286
0.8624 -1.032
0.86256 -0.9675
0.86272 -1.1046
0.86288 -1.3908
0.86304 -1.7455
0.8632 -2.064
0.86336 -2.2293
0.86352 -2.2011
0.86368 -1.9632
0.86384 -1.552
0.864 -1.032
0.86416 -0.4918125
0.86432 -0.0443437
0.86448 0.2176875
0.86464 0.2459063
0.8648 0
0.86496 -0.4918125
0.86512 -1.157
0.86528 -1.8907
0.86544 -2.5719
0.8656 -3.096
0.86576 -3.3459
0.86592 -3.3218
0.86608 -3.0557
0.86624 -2.6122
0.8664 -2.064
0.86656 -1.4835
0.86672 -0.9594375
0.86688 -0.532125
0.86704 -0.2136563
0.8672 0
0.86736 0.09675
0.86752 0.1249687
0.86768 0.1048125
0.86784 0.0564375
0.868 0
0.86816 -0.0564375
0.86832 -0.0886875
0.86848 -0.0886875
0.86864 -0.0564375
0.8688 0
0.86896 0.0524062
0.86912 0.09675
0.86928 0.1088437
0.86944 0.080625
0.8696 0
0.86976 -0.1572187
0.86992 -0.35475
0.87008 -0.5764688
0.87024 -0.8102813
0.8704 -1.032
0.87056 -1.2134
0.87072 -1.3222
0.87088 -1.3303
0.87104 -1.2295
0.8712 -1.032
0.87136 -0.7538437
0.87152 -0.4635938
0.87168 -0.209625
0.87184 -0.0443437
0.872 0
0.87216 -0.080625
0.87232 -0.274125
0.87248 -0.5280938
0.87264 -0.7981875
0.8728 -1.032
0.87296 -1.165
0.87312 -1.2094
0.87328 -1.1812
0.87344 -1.1086
0.8736 -1.032
0.87376 -0.9554063
0.87392 -0.9231563
0.87408 -0.9312187
0.87424 -0.9715313
0.8744 -1.032
0.87456 -1.0763
0.87472 -1.1046
0.87488 -1.1086
0.87504 -1.0844
0.8752 -1.032
0.87536 -0.919125
0.87552 -0.7619063
0.87568 -0.5522813
0.87584 -0.2942813
0.876 0
0.87616 0.306375
0.87632 0.596625
0.87648 0.8304375
0.87664 0.9795937
0.8768 1.032
0.87696 0.9473438
0.87712 0.774
0.87728 0.532125
0.87744 0.2620312
0.8776 0
0.87776 -0.2176875
0.87792 -0.3466875
0.87808 -0.3507188
0.87824 -0.2297812
0.8784 0
0.87856 0.2781563
0.87872 0.5724375
0.87888 0.8264062
0.87904 0.9876563
0.8792 1.032
0.87936 0.8989688
0.87952 0.6732187
0.87968 0.4071562
0.87984 0.1652812
0.88 0
0.88016 -0.0846563
0.88032 -0.0846563
0.88048 -0.0403125
0.88064 0.0040313
0.8808 0
0.88096 -0.0927188
0.88112 -0.274125
0.88128 -0.516
0.88144 -0.7820625
0.8816 -1.032
0.88176 -1.2457
0.88192 -1.4351
0.88208 -1.6165
0.88224 -1.8181
0.8824 -2.064
0.88256 -2.3301
0.88272 -2.6122
0.88288 -2.8622
0.88304 -3.0355
0.8832 -3.096
0.88336 -2.9872
0.88352 -2.7856
0.88368 -2.5276
0.88384 -2.2736
0.884 -2.064
0.88416 -1.9189
0.88432 -1.8745
0.88448 -1.9027
0.88464 -1.9793
0.8848 -2.064
0.88496 -2.1124
0.88512 -2.1285
0.88528 -2.1204
0.88544 -2.0922
0.8856 -2.064
0.88576 -2.0318
0.88592 -2.0197
0.88608 -2.0277
0.88624 -2.0438
0.8864 -2.064
0.88656 -2.068
0.88672 -2.064
0.88688 -2.06
0.88704 -2.0559
0.8872 -2.064
0.88736 -2.064
0.88752 -2.068
0.88768 -2.068
0.88784 -2.064
0.888 -2.064
0.88816 -2.0438
0.88832 -2.0277
0.88848 -2.0197
0.88864 -2.0318
0.8888 -2.064
0.88896 -2.0801
0.88912 -2.1003
0.88928 -2.1083
0.88944 -2.1003
0.8896 -2.064
0.88976 -1.9552
0.88992 -1.7939
0.89008 -1.5762
0.89024 -1.3182
0.8904 -1.032
0.89056 -0.7457812
0.89072 -0.4797188
0.89088 -0.258
0.89104 -0.09675
0.8912 0
0.89136 0.0241875
0.89152 0.0120938
0.89168 -0.0120938
0.89184 -0.0241875
0.892 0
0.89216 0.0564375
0.89232 0.1935
0.89248 0.4111875
0.89264 0.7014375
0.8928 1.032
0.89296 1.3021
0.89312 1.4875
0.89328 1.5238
0.89344 1.3747
0.8936 1.032
0.89376 0.499875
0.89392 -0.0765938
0.89408 -0.5885625
0.89424 -0.9271875
0.8944 -1.032
0.89456 -0.903
0.89472 -0.6167812
0.89488 -0.2862187
0.89504 -0.0443437
0.8952 0
0.89536 -0.2055938
0.89552 -0.6248438
0.89568 -1.157
0.89584 -1.677
0.896 -2.064
0.89616 -2.2011
0.89632 -2.1083
0.89648 -1.8181
0.89664 -1.4271
0.8968 -1.032
0.89696 -0.7215937
0.89712 -0.5684062
0.89728 -0.5885625
0.89744 -0.757875
0.8976 -1.032
0.89776 -1.3343
0.89792 -1.6246
0.89808 -1.8624
0.89824 -2.0156
0.8984 -2.064
0.89856 -1.9834
0.89872 -1.81
0.89888 -1.5682
0.89904 -1.294
0.8992 -1.032
0.89936 -0.8143125
0.89952 -0.693375
0.89968 -0.6893438
0.89984 -0.8102813
0.9 -1.032
0.90016 -1.3061
0.90032 -1.5883
0.90048 -1.8342
0.90064 -1.9995
0.9008 -2.064
0.90096 -2.0035
0.90112 -1.8382
0.90128 -1.5964
0.90144 -1.3142
0.9016 -1.032
0.90176 -0.8102813
0.90192 -0.6893438
0.90208 -0.6893438
0.90224 -0.8143125
0.9024 -1.032
0.90256 -1.29
0.90272 -1.4956
0.90288 -1.552
0.90304 -1.4069
0.9032 -1.032
0.90336 -0.4797188
0.90352 0.1330313
0.90368 0.67725
0.90384 1.0078
0.904 1.032
0.90416 0.6812813
0.90432 0.0604688
0.90448 -0.6974062
0.90464 -1.4553
0.9048 -2.064
0.90496 -2.4187
0.90512 -2.5195
0.90528 -2.4268
0.90544 -2.2414
0.9056 -2.064
0.90576 -1.9471
0.90592 -1.931
0.90608 -1.9874
0.90624 -2.0519
0.9064 -2.064
0.90656 -1.9632
0.90672 -1.7697
0.90688 -1.5158
0.90704 -1.2497
0.9072 -1.032
0.90736 -0.8828437
0.90752 -0.8304375
0.90768 -0.8586562
0.90784 -0.93525
0.908 -1.032
0.90816 -1.0965
0.90832 -1.1288
0.90848 -1.1247
0.90864 -1.0884
0.9088 -1.032
0.90896 -0.9836251
0.90912 -0.9473438
0.90928 -0.9433125
0.90944 -0.9715313
0.9096 -1.032
0.90976 -1.1005
0.90992 -1.161
0.91008 -1.1812
0.91024 -1.1408
0.9104 -1.032
0.91056 -0.8667188
0.91072 -0.6530625
0.91088 -0.4111875
0.91104 -0.177375
0.9112 0
0.91136 0.0645
0.91152 0
0.91168 -0.209625
0.91184 -0.564375
0.912 -1.032
0.91216 -1.5278
0.91232 -1.9632
0.91248 -2.2373
0.91264 -2.2777
0.9128 -2.064
0.91296 -1.6246
0.91312 -1.0763
0.91328 -0.5361562
0.91344 -0.145125
0.9136 0
0.91376 -0.1572187
0.91392 -0.5603437
0.91408 -1.1005
0.91424 -1.6448
0.9144 -2.064
0.91456 -2.2575
0.91472 -2.2011
0.91488 -1.9229
0.91504 -1.4956
0.9152 -1.032
0.91536 -0.6329063
0.91552 -0.4071562
0.91568 -0.3990938
0.91584 -0.61275
0.916 -1.032
0.91616 -1.5682
0.91632 -2.1325
0.91648 -2.6243
0.91664 -2.9589
0.9168 -3.096
0.91696 -3.0073
0.91712 -2.7695
0.91728 -2.4712
0.91744 -2.2051
0.9176 -2.064
0.91776 -2.0761
0.91792 -2.2535
0.91808 -2.5397
0.91824 -2.8501
0.9184 -3.096
0.91856 -3.1645
0.91872 -3.0678
0.91888 -2.8138
0.91904 -2.4591
0.9192 -2.064
0.91936 -1.673
0.91952 -1.3626
0.91968 -1.1489
0.91984 -1.0401
0.92 -1.032
0.92016 -1.1046
0.92032 -1.2577
0.92048 -1.4795
0.92064 -1.7576
0.9208 -2.064
0.92096 -2.3381
0.92112 -2.5155
0.92128 -2.5518
0.92144 -2.4026
0.9216 -2.064
0.92176 -1.5722
0.92192 -1.0199
0.92208 -0.516
0.92224 -0.1531875
0.9224 0
0.92256 -0.09675
0.92272 -0.35475
0.92288 -0.6651562
0.92304 -0.919125
0.9232 -1.032
0.92336 -0.9594375
0.92352 -0.7296563
0.92368 -0.4273125
0.92384 -0.1491563
0.924 0
0.92416 -0.0282188
0.92432 -0.2136563
0.92448 -0.499875
0.92464 -0.7981875
0.9248 -1.032
0.92496 -1.1489
0.92512 -1.1529
0.92528 -1.0884
0.92544 -1.028
0.9256 -1.032
0.92576 -1.1288
0.92592 -1.3222
0.92608 -1.5843
0.92624 -1.8503
0.9264 -2.064
0.92656 -2.1366
0.92672 -2.068
0.92688 -1.8423
0.92704 -1.4795
0.9272 -1.032
0.92736 -0.564375
0.92752 -0.1693125
0.92768 0.080625
0.92784 0.1410938
0.928 0
0.92816 -0.3144375
0.92832 -0.6893438
0.92848 -1.0038
0.92864 -1.1408
0.9288 -1.032
0.92896 -0.6853125
0.92912 -0.1733438
0.92928 0.3749063
0.92944 0.8143125
0.9296 1.032
0.92976 0.9110625
0.92992 0.5280938
0.93008 -0.0080625
0.93024 -0.5684062
0.9304 -1.032
0.93056 -1.3222
0.93072 -1.4029
0.93088 -1.3222
0.93104 -1.165
0.9312 -1.032
0.93136 -1.0038
0.93152 -1.1126
0.93168 -1.3545
0.93184 -1.6891
0.932 -2.064
0.93216 -2.4026
0.93232 -2.6888
0.93248 -2.8985
0.93264 -3.0275
0.9328 -3.096
0.93296 -3.1
0.93312 -3.0839
0.93328 -3.0718
0.93344 -3.0718
0.9336 -3.096
0.93376 -3.092
0.93392 -3.1081
0.93408 -3.1283
0.93424 -3.1323
0.9344 -3.096
0.93456 -2.967
0.93472 -2.7816
0.93488 -2.5518
0.93504 -2.3018
0.9352 -2.064
0.93536 -1.8544
0.93552 -1.7334
0.93568 -1.7254
0.93584 -1.8382
0.936 -2.064
0.93616 -2.3623
0.93632 -2.6768
0.93648 -2.9428
0.93664 -3.096
0.9368 -3.096
0.93696 -2.8864
0.93712 -2.5195
0.93728 -2.0438
0.93744 -1.5238
0.9376 -1.032
0.93776 -0.6369375
0.93792 -0.3466875
0.93808 -0.1572187
0.93824 -0.0443437
0.9384 0
0.93856 -0.0241875
0.93872 -0.112875
0.93888 -0.3023437
0.93904 -0.6087188
0.9392 -1.032
0.93936 -1.5037
0.93952 -1.939
0.93968 -2.2253
0.93984 -2.2817
0.94 -2.064
0.94016 -1.5802
0.94032 -0.9715313
0.94048 -0.403125
0.94064 -0.0403125
0.9408 0
0.94096 -0.306375
0.94112 -0.8465625
0.94128 -1.4392
0.94144 -1.9027
0.9416 -2.064
0.94176 -1.8221
0.94192 -1.2457
0.94208 -0.4595625
0.94224 0.35475
0.9424 1.032
0.94256 1.4271
0.94272 1.5359
0.94288 1.4311
0.94304 1.2255
0.9432 1.032
0.94336 0.8909063
0.94352 0.8667188
0.94368 0.9312187
0.94384 1.0078
0.944 1.032
0.94416 0.919125
0.94432 0.7175625
0.94448 0.4595625
0.94464 0.2015625
0.9448 0
0.94496 -0.129
0.94512 -0.1652812
0.94528 -0.1249687
0.94544 -0.0564375
0.9456 0
0.94576 -0.0282188
0.94592 -0.1491563
0.94608 -0.3628125
0.94624 -0.6651562
0.9464 -1.032
0.94656 -1.3948
0.94672 -1.7254
0.94688 -1.9673
0.94704 -2.0842
0.9472 -2.064
0.94736 -1.8826
0.94752 -1.6246
0.94768 -1.3545
0.94784 -1.1408
0.948 -1.032
0.94816 -1.0441
0.94832 -1.2013
0.94848 -1.4593
0.94864 -1.7657
0.9488 -2.064
0.94896 -2.2656
0.94912 -2.3663
0.94928 -2.3583
0.94944 -2.2494
0.9496 -2.064
0.94976 -1.806
0.94992 -1.548
0.95008 -1.3182
0.95024 -1.1408
0.9504 -1.032
0.95056 -0.9715313
0.95072 -0.9594375
0.95088 -0.9795937
0.95104 -1.0078
0.9512 -1.032
0.95136 -1.032
0.95152 -1.0199
0.95168 -1.0038
0.95184 -1.0038
0.952 -1.032
0.95216 -1.0763
0.95232 -1.1288
0.95248 -1.157
0.95264 -1.1328
0.9528 -1.032
0.95296 -0.8505938
0.95312 -0.6167812
0.95328 -0.3668437
0.95344 -0.1491563
0.9536 0
0.95376 0.048375
0.95392 0.0362812
0.95408 0
0.95424 -0.016125
0.9544 0
0.95456 0.0564375
0.95472 0.129
0.95488 0.177375
0.95504 0.1410938
0.9552 0
0.95536 -0.258
0.95552 -0.5684062
0.95568 -0.8465625
0.95584 -1.0199
0.956 -1.032
0.95616 -0.8788125
0.95632 -0.6167812
0.95648 -0.3184688
0.95664 -0.0886875
0.9568 0
0.95696 -0.0927188
0.95712 -0.3265312
0.95728 -0.6248438
0.95744 -0.886875
0.9576 -1.032
0.95776 -0.9916875
0.95792 -0.8022187
0.95808 -0.516
0.95824 -0.2176875
0.9584 0
0.95856 0.0725625
0.95872 -0.0201563
0.95888 -0.2660625
0.95904 -0.6248438
0.9592 -1.032
0.95936 -1.4109
0.95952 -1.7254
0.95968 -1.9471
0.95984 -2.0559
0.96 -2.064
0.96016 -1.9552
0.96032 -1.7778
0.96048 -1.548
0.96064 -1.294
0.9608 -1.032
0.96096 -0.7699688
0.96112 -0.5361562
0.96128 -0.3305625
0.96144 -0.1491563
0.9616 0
0.96176 0.1048125
0.96192 0.1652812
0.96208 0.1733438
0.96224 0.1209375
0.9624 0
0.96256 -0.1814063
0.96272 -0.403125
0.96288 -0.6369375
0.96304 -0.854625
0.9632 -1.032
0.96336 -1.1368
0.96352 -1.1731
0.96368 -1.157
0.96384 -1.1005
0.964 -1.032
0.96416 -0.9675
0.96432 -0.9312187
0.96448 -0.9271875
0.96464 -0.9634687
0.9648 -1.032
0.96496 -1.1005
0.96512 -1.161
0.96528 -1.1771
0.96544 -1.1368
0.9656 -1.032
0.96576 -0.8344688
0.96592 -0.6006563
0.96608 -0.3587813
0.96624 -0.1491563
0.9664 0
0.96656 0.0765938
0.96672 0.0927188
0.96688 0.0645
0.96704 0.0241875
0.9672 0
0.96736 -0.0040313
0.96752 0.0080625
0.96768 0.0282188
0.96784 0.03225
0.968 0
0.96816 -0.0765938
0.96832 -0.1531875
0.96848 -0.1894688
0.96864 -0.145125
0.9688 0
0.96896 0.2338125
0.96912 0.516
0.96928 0.790125
0.96944 0.9755625
0.9696 1.032
0.96976 0.9150938
0.96992 0.6893438
0.97008 0.4152188
0.97024 0.1652812
0.9704 0
0.97056 -0.0645
0.97072 -0.0443437
0.97088 0.0120938
0.97104 0.0443437
0.9712 0
0.97136 -0.145125
0.97152 -0.3789375
0.97168 -0.6490313
0.97184 -0.886875
0.972 -1.032
0.97216 -1.032
0.97232 -0.8909063
0.97248 -0.6409687
0.97264 -0.3225
0.9728 0
0.97296 0.258
0.97312 0.4071562
0.97328 0.4071562
0.97344 0.2660625
0.9736 0
0.97376 -0.3305625
0.97392 -0.661125
0.97408 -0.9231563
0.97424 -1.0562
0.9744 -1.032
0.97456 -0.8465625
0.97472 -0.5764688
0.97488 -0.29025
0.97504 -0.0765938
0.9752 0
0.97536 -0.0846563
0.97552 -0.306375
0.97568 -0.5925937
0.97584 -0.8626875
0.976 -1.032
0.97616 -1.0481
0.97632 -0.903
0.97648 -0.6369375
0.97664 -0.3144375
0.9768 0
0.97696 0.22575
0.97712 0.338625
0.97728 0.3225
0.97744 0.2015625
0.9776 0
0.97776 -0.2459063
0.97792 -0.499875
0.97808 -0.7215937
0.97824 -0.903
0.9784 -1.032
0.97856 -1.0925
0.97872 -1.1046
0.97888 -1.0844
0.97904 -1.0522
0.9792 -1.032
0.97936 -1.0239
0.97952 -1.036
0.97968 -1.0481
0.97984 -1.0522
0.98 -1.032
0.98016 -0.9876563
0.98032 -0.9433125
0.98048 -0.9231563
0.98064 -0.9473438
0.9808 -1.032
0.98096 -1.1408
0.98112 -1.2457
0.98128 -1.294
0.98144 -1.2295
0.9816 -1.032
0.98176 -0.7175625
0.98192 -0.3668437
0.98208 -0.0725625
0.98224 0.0685312
0.9824 0
0.98256 -0.3023437
0.98272 -0.7659375
0.98288 -1.294
0.98304 -1.7576
0.9832 -2.064
0.98336 -2.1164
0.98352 -1.9552
0.98368 -1.6568
0.98384 -1.3182
0.984 -1.032
0.98416 -0.8425313
0.98432 -0.790125
0.98448 -0.8385
0.98464 -0.9392812
0.9848 -1.032
0.98496 -1.0643
0.98512 -1.0481
0.98528 -1.0118
0.98544 -0.9957188
0.9856 -1.032
0.98576 -1.1005
0.98592 -1.1852
0.98608 -1.2376
0.98624 -1.2013
0.9864 -1.032
0.98656 -0.7054688
0.98672 -0.2700938
0.98688 0.2055938
0.98704 0.661125
0.9872 1.032
0.98736 1.2457
0.98752 1.3222
0.98768 1.286
0.98784 1.1691
0.988 1.032
0.98816 0.8909063
0.98832 0.8143125
0.98848 0.8143125
0.98864 0.8949375
0.9888 1.032
0.98896 1.1489
0.98912 1.2416
0.98928 1.2658
0.98944 1.1973
0.9896 1.032
0.98976 0.7619063
0.98992 0.4716563
0.99008 0.209625
0.99024 0.0403125
0.9904 0
0.99056 0.0886875
0.99072 0.2983125
0.99088 0.5684062
0.99104 0.8385
0.9912 1.032
0.99136 1.0844
0.99152 0.9876563
0.99168 0.7498125
0.99184 0.403125
0.992 0
0.99216 -0.387
0.99232 -0.7175625
0.99248 -0.9433125
0.99264 -1.0522
0.9928 -1.032
0.99296 -0.8788125
0.99312 -0.6570938
0.99328 -0.4111875
0.99344 -0.1814063
0.9936 0
0.99376 0.1048125
0.99392 0.1330313
0.99408 0.1048125
0.99424 0.0524062
0.9944 0
0.99456 -0.0362812
0.99472 -0.048375
0.99488 -0.0362812
0.99504 -0.0120938
0.9952 0
0.99536 -0.0040313
0.99552 -0.0282188
0.99568 -0.0443437
0.99584 -0.0403125
0.996 0
0.99616 0.0524062
0.99632 0.1048125
0.99648 0.1330313
0.99664 0.1048125
0.9968 0
0.99696 -0.1935
0.99712 -0.435375
0.99728 -0.6812813
0.99744 -0.8909063
0.9976 -1.032
0.99776 -1.0844
0.99792 -1.0723
0.99808 -1.032
0.99824 -1.0038
0.9984 -1.032
0.99856 -1.1207
0.99872 -1.294
0.99888 -1.5278
0.99904 -1.7979
0.9992 -2.064
0.99936 -2.2333
0.99952 -2.326
0.99968 -2.322
0.99984 -2.2293
1 -2.064
1.00016 -1.81
1.00032 -1.548
1.00048 -1.3142
1.00064 -1.1368
1.0008 -1.032
1.00096 -0.9755625
1.00112 -0.9755625
1.00128 -1.0078
1.00144 -1.036
1.0016 -1.032
1.00176 -0.9433125
1.00192 -0.7860938
1.00208 -0.5603437
1.00224 -0.29025
1.0024 0
1.00256 0.2862187
1.00272 0.5442188
1.00288 0.7619063
1.00304 0.9231563
1.0032 1.032
1.00336 1.0522
1.00352 1.0481
1.00368 1.032
1.00384 1.0239
1.004 1.032
1.00416 1.0199
1.00432 1.028
1.00448 1.0401
1.00464 1.0441
1.0048 1.032
1.00496 0.9675
1.00512 0.9110625
1.00528 0.886875
1.00544 0.9271875
1.0056 1.032
1.00576 1.1449
1.00592 1.2618
1.00608 1.3142
1.00624 1.2497
1.0064 1.032
1.00656 0.6369375
1.00672 0.1491563
1.00688 -0.3426562
1.00704 -0.757875
1.0072 -1.032
1.00736 -1.1328
1.00752 -1.1086
1.00768 -1.028
1.00784 -0.9795937
1.008 -1.032
1.00816 -1.1812
1.00832 -1.4311
1.00848 -1.7133
1.00864 -1.9471
1.0088 -2.064
1.00896 -1.9914
1.00912 -1.7858
1.00928 -1.5037
1.00944 -1.2255
1.0096 -1.032
1.00976 -0.9715313
1.00992 -1.0844
1.01008 -1.3424
1.01024 -1.6931
1.0104 -2.064
1.01056 -2.3462
1.01072 -2.5115
1.01088 -2.5236
1.01104 -2.3704
1.0112 -2.064
1.01136 -1.6246
1.01152 -1.1328
1.01168 -0.6570938
1.01184 -0.2620312
1.012 0
1.01216 0.0846563
1.01232 0
1.01248 -0.2459063
1.01264 -0.6046875
1.0128 -1.032
1.01296 -1.4472
1.01312 -1.7899
1.01328 -2.0197
1.01344 -2.1083
1.0136 -2.064
1.01376 -1.8705
1.01392 -1.6165
1.01408 -1.3585
1.01424 -1.1489
1.0144 -1.032
1.01456 -0.9957188
1.01472 -1.0239
1.01488 -1.0683
1.01504 -1.0844
1.0152 -1.032
1.01536 -0.8828437
1.01552 -0.661125
1.01568 -0.4071562
1.01584 -0.1693125
1.016 0
1.01616 0.0886875
1.01632 0.09675
1.01648 0.0524062
1.01664 0.0040313
1.0168 0
1.01696 0.080625
1.01712 0.2539688
1.01728 0.4958438
1.01744 0.7699688
1.0176 1.032
1.01776 1.2013
1.01792 1.294
1.01808 1.29
1.01824 1.1973
1.0184 1.032
1.01856 0.790125
1.01872 0.5401875
1.01888 0.3023437
1.01904 0.112875
1.0192 0
1.01936 -0.0201563
1.01952 0.0725625
1.01968 0.29025
1.01984 0.6208125
1.02 1.032
1.02016 1.4472
1.02032 1.8141
1.02048 2.0721
1.02064 2.1608
1.0208 2.064
1.02096 1.7415
1.02112 1.2981
1.02128 0.80625
1.02144 0.3466875
1.0216 0
1.02176 -0.209625
1.02192 -0.274125
1.02208 -0.22575
1.02224 -0.1169063
1.0224 0
1.02256 0.080625
1.02272 0.1088437
1.02288 0.0886875
1.02304 0.0403125
1.0232 0
1.02336 -0.0282188
1.02352 -0.0282188
1.02368 -0.0080625
1.02384 0.0040313
1.024 0
1.02416 -0.0685312
1.02432 -0.209625
1.02448 -0.4273125
1.02464 -0.7095
1.0248 -1.032
1.02496 -1.3464
1.02512 -1.6407
1.02528 -1.8786
1.02544 -2.0277
1.0256 -2.064
1.02576 -1.9632
1.02592 -1.7697
1.02608 -1.5238
1.02624 -1.2618
1.0264 -1.032
1.02656 -0.854625
1.02672 -0.774
1.02688 -0.7820625
1.02704 -0.8747813
1.0272 -1.032
1.02736 -1.2215
1.02752 -1.4392
1.02768 -1.6609
1.02784 -1.8745
1.028 -2.064
1.02816 -2.1849
1.02832 -2.2575
1.02848 -2.2615
1.02864 -2.197
1.0288 -2.064
1.02896 -1.8463
1.02912 -1.6044
1.02928 -1.3666
1.02944 -1.1691
1.0296 -1.032
1.02976 -0.9675
1.02992 -0.9594375
1.03008 -0.9836251
1.03024 -1.0159
1.0304 -1.032
1.03056 -1.0159
1.03072 -0.9876563
1.03088 -0.9634687
1.03104 -0.9755625
1.0312 -1.032
1.03136 -1.1167
1.03152 -1.2053
1.03168 -1.2457
1.03184 -1.1933
1.032 -1.032
1.03216 -0.7659375
1.03232 -0.4555313
1.03248 -0.177375
1.03264 -0.0040313
1.0328 0
1.03296 -0.1693125
1.03312 -0.4474687
1.03328 -0.74175
1.03344 -0.9594375
1.0336 -1.032
1.03376 -0.9150938
1.03392 -0.6651562
1.03408 -0.3628125
1.03424 -0.112875
1.0344 0
1.03456 -0.0524062
1.03472 -0.2539688
1.03488 -0.5442188
1.03504 -0.8344688
1.0352 -1.032
1.03536 -1.0602
1.03552 -0.9231563
1.03568 -0.6530625
1.03584 -0.3184688
1.036 0
1.03616 0.2338125
1.03632 0.3466875
1.03648 0.3265312
1.03664 0.1975312
1.0368 0
1.03696 -0.2055938
1.03712 -0.3507188
1.03728 -0.387
1.03744 -0.274125
1.0376 0
1.03776 0.3950625
1.03792 0.8667188
1.03808 1.3505
1.03824 1.7738
1.0384 2.064
1.03856 2.1366
1.03872 2.0318
1.03888 1.7778
1.03904 1.423
1.0392 1.032
1.03936 0.6369375
1.03952 0.3225
1.03968 0.112875
1.03984 0.0120938
1.04 0
1.04016 0.03225
1.04032 0.0846563
1.04048 0.112875
1.04064 0.0886875
1.0408 0
1.04096 -0.1693125
1.04112 -0.387
1.04128 -0.628875
1.04144 -0.854625
1.0416 -1.032
1.04176 -1.1489
1.04192 -1.1933
1.04208 -1.1771
1.04224 -1.1167
1.0424 -1.032
1.04256 -0.9433125
1.04272 -0.8949375
1.04288 -0.8989688
1.04304 -0.9473438
1.0432 -1.032
1.04336 -1.1167
1.04352 -1.1812
1.04368 -1.2013
1.04384 -1.1489
1.044 -1.032
1.04416 -0.8747813
1.04432 -0.74175
1.04448 -0.693375
1.04464 -0.7780312
1.0448 -1.032
1.04496 -1.419
1.04512 -1.8987
1.04528 -2.3946
1.04544 -2.8178
1.0456 -3.096
1.04576 -3.1524
1.04592 -3.0234
1.04608 -2.7533
1.04624 -2.4067
1.0464 -2.064
1.04656 -1.7858
1.04672 -1.6407
1.04688 -1.6568
1.04704 -1.81
1.0472 -2.064
1.04736 -2.322
1.04752 -2.5115
1.04768 -2.5639
1.04784 -2.4187
1.048 -2.064
1.04816 -1.5319
1.04832 -0.9433125
1.04848 -0.41925
1.04864 -0.0765938
1.0488 0
1.04896 -0.2176875
1.04912 -0.6570938
1.04928 -1.2013
1.04944 -1.7093
1.0496 -2.064
1.04976 -2.1688
1.04992 -2.0358
1.05008 -1.7294
1.05024 -1.3585
1.0504 -1.032
1.05056 -0.8102813
1.05072 -0.7336875
1.05088 -0.7780312
1.05104 -0.8989688
1.0512 -1.032
1.05136 -1.1328
1.05152 -1.1691
1.05168 -1.1489
1.05184 -1.0925
1.052 -1.032
1.05216 -0.9836251
1.05232 -0.9675
1.05248 -0.9836251
1.05264 -1.0118
1.0528 -1.032
1.05296 -1.0118
1.05312 -0.919125
1.05328 -0.725625
1.05344 -0.4111875
1.0536 0
1.05376 0.4313438
1.05392 0.8304375
1.05408 1.1086
1.05424 1.1852
1.0544 1.032
1.05456 0.6208125
1.05472 0.0886875
1.05488 -0.4434375
1.05504 -0.8465625
1.0552 -1.032
1.05536 -0.9715313
1.05552 -0.7175625
1.05568 -0.387
1.05584 -0.1048125
1.056 0
1.05616 -0.1531875
1.05632 -0.5280938
1.05648 -1.0481
1.05664 -1.5964
1.0568 -2.064
1.05696 -2.3502
1.05712 -2.447
1.05728 -2.3865
1.05744 -2.2333
1.0576 -2.064
1.05776 -1.9269
1.05792 -1.8745
1.05808 -1.9027
1.05824 -1.9753
1.0584 -2.064
1.05856 -2.0922
1.05872 -2.0962
1.05888 -2.0801
1.05904 -2.064
1.0592 -2.064
1.05936 -2.0559
1.05952 -2.064
1.05968 -2.0761
1.05984 -2.0761
1.06 -2.064
1.06016 -1.9955
1.06032 -1.939
1.06048 -1.9148
1.06064 -1.9552
1.0608 -2.064
1.06096 -2.193
1.06112 -2.3099
1.06128 -2.3623
1.06144 -2.2897
1.0616 -2.064
1.06176 -1.677
1.06192 -1.1973
1.06208 -0.7054688
1.06224 -0.2821875
1.0624 0
1.06256 0.1088437
1.06272 0.0927188
1.06288 0.0201563
1.06304 -0.03225
1.0632 0
1.06336 0.129
1.06352 0.3587813
1.06368 0.6369375
1.06384 0.8828437
1.064 1.032
1.06416 0.9957188
1.06432 0.8264062
1.06448 0.5603437
1.06464 0.2660625
1.0648 0
1.06496 -0.2055938
1.06512 -0.2983125
1.06528 -0.274125
1.06544 -0.1572187
1.0656 0
1.06576 0.1410938
1.06592 0.241875
1.06608 0.258
1.06624 0.177375
1.0664 0
1.06656 -0.2660625
1.06672 -0.5603437
1.06688 -0.8143125
1.06704 -0.9795937
1.0672 -1.032
1.06736 -0.9795937
1.06752 -0.886875
1.06768 -0.822375
1.06784 -0.854625
1.068 -1.032
1.06816 -1.3182
1.06832 -1.6689
1.06848 -1.9793
1.06864 -2.1366
1.0688 -2.064
1.06896 -1.7375
1.06912 -1.2416
1.06928 -0.6974062
1.06944 -0.241875
1.0696 0
1.06976 -0.0725625
1.06992 -0.4232812
1.07008 -0.9554063
1.07024 -1.544
1.0704 -2.064
1.07056 -2.4067
1.07072 -2.5437
1.07088 -2.4953
1.07104 -2.3139
1.0712 -2.064
1.07136 -1.802
1.07152 -1.5722
1.07168 -1.3706
1.07184 -1.1892
1.072 -1.032
1.07216 -0.9392812
1.07232 -0.9473438
1.07248 -1.1126
1.07264 -1.4835
1.0728 -2.064
1.07296 -2.8058
1.07312 -3.6039
1.07328 -4.3417
1.07344 -4.8899
1.0736 -5.16
1.07376 -5.0471
1.07392 -4.6722
1.07408 -4.1361
1.07424 -3.5717
1.0744 -3.096
1.07456 -2.7614
1.07472 -2.6364
1.07488 -2.6929
1.07504 -2.8703
1.0752 -3.096
1.07536 -3.2572
1.07552 -3.3419
1.07568 -3.3338
1.07584 -3.2452
1.076 -3.096
1.07616 -2.8823
1.07632 -2.6647
1.07648 -2.455
1.07664 -2.2575
1.0768 -2.064
1.07696 -1.8544
1.07712 -1.6448
1.07728 -1.4392
1.07744 -1.2336
1.0776 -1.032
1.07776 -0.8425313
1.07792 -0.645
1.07808 -0.4273125
1.07824 -0.2055938
1.0784 0
1.07856 0.1531875
1.07872 0.2459063
1.07888 0.2620312
1.07904 0.1814063
1.0792 0
1.07936 -0.2700938
1.07952 -0.5603437
1.07968 -0.8143125
1.07984 -0.9795937
1.08 -1.032
1.08016 -0.951375
1.08032 -0.757875
1.08048 -0.4958438
1.08064 -0.22575
1.0808 0
1.08096 0.1370625
1.08112 0.1894688
1.08128 0.16125
1.08144 0.0886875
1.0816 0
1.08176 -0.0765938
1.08192 -0.1169063
1.08208 -0.112875
1.08224 -0.0685312
1.0824 0
1.08256 0.0685312
1.08272 0.112875
1.08288 0.1169063
1.08304 0.0765938
1.0832 0
1.08336 -0.1048125
1.08352 -0.1935
1.08368 -0.22575
1.08384 -0.1652812
1.084 0
1.08416 0.2297812
1.08432 0.5039063
1.08448 0.7659375
1.08464 0.9594375
1.0848 1.032
1.08496 0.9271875
1.08512 0.7135313
1.08528 0.4394063
1.08544 0.1854375
1.0856 0
1.08576 -0.09675
1.08592 -0.1048125
1.08608 -0.0524062
1.08624 0
1.0864 0
1.08656 -0.0927188
1.08672 -0.274125
1.08688 -0.516
1.08704 -0.7820625
1.0872 -1.032
1.08736 -1.2457
1.08752 -1.4351
1.08768 -1.6165
1.08784 -1.8181
1.088 -2.064
1.08816 -2.3381
1.08832 -2.6284
1.08848 -2.8823
1.08864 -3.0517
1.0888 -3.096
1.08896 -2.9791
1.08912 -2.7654
1.08928 -2.5074
1.08944 -2.2575
1.0896 -2.064
1.08976 -1.9431
1.08992 -1.9148
1.09008 -1.9511
1.09024 -2.0116
1.0904 -2.064
1.09056 -2.064
1.09072 -2.0438
1.09088 -2.0156
1.09104 -2.0116
1.0912 -2.064
1.09136 -2.1849
1.09152 -2.3784
1.09168 -2.6163
1.09184 -2.8703
1.092 -3.096
1.09216 -3.2411
1.09232 -3.3137
1.09248 -3.3056
1.09264 -3.225
1.0928 -3.096
1.09296 -2.9146
1.09312 -2.709
1.09328 -2.4913
1.09344 -2.2696
1.0936 -2.064
1.09376 -1.8866
1.09392 -1.7778
1.09408 -1.7617
1.09424 -1.8584
1.0944 -2.064
1.09456 -2.3341
1.09472 -2.6284
1.09488 -2.8904
1.09504 -3.0557
1.0952 -3.096
1.09536 -2.967
1.09552 -2.7332
1.09568 -2.4631
1.09584 -2.2212
1.096 -2.064
1.09616 -2.0035
1.09632 -2.0277
1.09648 -2.0842
1.09664 -2.1124
1.0968 -2.064
1.09696 -1.9108
1.09712 -1.677
1.09728 -1.415
1.09744 -1.1812
1.0976 -1.032
1.09776 -0.9876563
1.09792 -1.0239
1.09808 -1.0804
1.09824 -1.1005
1.0984 -1.032
1.09856 -0.8505938
1.09872 -0.5925937
1.09888 -0.3184688
1.09904 -0.09675
1.0992 0
1.09936 -0.0846563
1.09952 -0.2942813
1.09968 -0.564375
1.09984 -0.8344688
1.1 -1.032
1.10016 -1.1288
1.10032 -1.1328
1.10048 -1.0804
1.10064 -1.032
1.1008 -1.032
1.10096 -1.1086
1.10112 -1.2094
1.10128 -1.2698
1.10144 -1.2255
1.1016 -1.032
1.10176 -0.725625
1.10192 -0.3668437
1.10208 -0.0604688
1.10224 0.080625
1.1024 0
1.10256 -0.338625
1.10272 -0.8425313
1.10288 -1.3868
1.10304 -1.8302
1.1032 -2.064
1.10336 -2.0398
1.10352 -1.81
1.10368 -1.4754
1.10384 -1.1771
1.104 -1.032
1.10416 -1.1247
1.10432 -1.4553
1.10448 -1.9632
1.10464 -2.5478
1.1048 -3.096
1.10496 -3.4709
1.10512 -3.6443
1.10528 -3.612
1.10544 -3.4104
1.1056 -3.096
1.10576 -2.709
1.10592 -2.3704
1.10608 -2.1285
1.10624 -2.0237
1.1064 -2.064
1.10656 -2.2172
1.10672 -2.4591
1.10688 -2.7292
1.10704 -2.9589
1.1072 -3.096
1.10736 -3.0758
1.10752 -2.9267
1.10768 -2.6768
1.10784 -2.3744
1.108 -2.064
1.10816 -1.7657
1.10832 -1.5238
1.10848 -1.3303
1.10864 -1.1691
1.1088 -1.032
1.10896 -0.9231563
1.10912 -0.8425313
1.10928 -0.8183438
1.10944 -0.8788125
1.1096 -1.032
1.10976 -1.2779
1.10992 -1.5561
1.11008 -1.8141
1.11024 -1.9995
1.1104 -2.064
1.11056 -1.9673
1.11072 -1.7617
1.11088 -1.4916
1.11104 -1.2295
};
\end{axis}

\end{tikzpicture}
    % This file was created by matplotlib2tikz v0.6.17.
\begin{tikzpicture}

\definecolor{color0}{rgb}{0.12156862745098,0.466666666666667,0.705882352941177}
\definecolor{color1}{rgb}{1,0.498039215686275,0.0549019607843137}

\begin{axis}[
xlabel={$t$ in \si{\micro \second}},
ylabel={$U_{out}$ in \si{\milli \volt}},
xmin=-0.0555, xmax=1.1655,
ymin=-873.008455, ymax=867.887955,
width=\figurewidth,
height=\figureheight,
tick align=outside,
tick pos=left,
x grid style={white!69.01960784313725!black},
y grid style={white!69.01960784313725!black},
legend entries={{$U_{out,0}$},{$U_{out,1}$}},
legend cell align={left},
legend style={draw=white!80.0!black}
]
\addlegendimage{no markers, color0}
\addlegendimage{no markers, color1}
\addplot [semithick, color0]
table {%
0 19.0069
0.002 13.4269
0.004 8.1956
0.006 11.6831
0.008 1.9762
0.01 18.6581
0.012 11.7412
0.014 -0.93
0.016 -24.3544
0.018 -3.1969
0.02 1.2788
0.022 15.6938
0.024 17.8444
0.026 15.8681
0.028 7.3237
0.03 -0.11625
0.032 12.2644
0.034 1.3369
0.036 -9.2419
0.038 -1.4531
0.04 -3.6619
0.042 -4.5338
0.044 6.3938
0.046 -13.3106
0.048 -6.1031
0.05 -17.205
0.052 -10.5206
0.054 -16.1006
0.056 5.115
0.058 -2.1506
0.06 -3.0225
0.062 -2.9644
0.064 13.8338
0.066 15.4613
0.068 11.5087
0.07 15.5194
0.072 -21.4481
0.074 -15.6938
0.076 -0.465
0.078 -16.2169
0.08 1.7437
0.082 -2.4994
0.084 -12.4387
0.086 -3.0225
0.088 0.58125
0.09 0.34875
0.092 -7.2075
0.094 -13.3106
0.096 -0.81375
0.098 2.2087
0.1 16.7981
0.102 -0.755625
0.104 7.2656
0.106 16.1006
0.108 16.4494
0.11 14.88
0.112 14.9963
0.114 16.275
0.116 -6.7425
0.118 -4.4175
0.12 -0.81375
0.122 13.0781
0.124 0.058125
0.126 -15.9262
0.128 -2.2669
0.13 -4.3012
0.132 -1.6856
0.134 1.5113
0.136 3.1388
0.138 1.5113
0.14 8.9513
0.142 -0.058125
0.144 -12.0319
0.146 2.4994
0.148 11.7412
0.15 2.0925
0.152 2.7319
0.154 0.174375
0.156 0.406875
0.158 1.4531
0.16 23.4244
0.162 0.11625
0.164 -10.8694
0.166 14.1244
0.168 -4.2431
0.17 14.4731
0.172 -9.0094
0.174 16.8562
0.176 12.3806
0.178 -1.9181
0.18 -5.7544
0.182 -23.8313
0.184 11.5669
0.186 0.58125
0.188 -1.1044
0.19 16.4494
0.192 4.8244
0.194 26.8537
0.196 14.5894
0.198 0
0.2 -12.7294
0.202 16.9144
0.204 12.0319
0.206 -0.871875
0.208 -9.3
0.21 -25.5169
0.212 -7.3819
0.214 -15.6356
0.216 -6.8006
0.218 -1.395
0.22 1.1625
0.222 14.3569
0.224 -1.3369
0.226 0.465
0.228 3.1969
0.23 -0.290625
0.232 -6.4519
0.234 -1.6275
0.236 -9.0675
0.238 -2.3831
0.24 -36.27
0.242 -14.9381
0.244 14.7638
0.246 1.6856
0.248 0.11625
0.25 -19.065
0.252 2.0925
0.254 0.639375
0.256 -6.045
0.258 -10.0556
0.26 -4.4756
0.262 0.988125
0.264 -5.4056
0.266 2.9063
0.268 7.905
0.27 -1.6275
0.272 3.0806
0.274 -1.86
0.276 2.9644
0.278 -13.6013
0.28 1.2788
0.282 -11.7412
0.284 -9.6487
0.286 16.9725
0.288 -4.8825
0.29 13.02
0.292 8.6025
0.294 -4.3012
0.296 -3.9525
0.298 -15.5775
0.3 -2.9644
0.302 1.8019
0.304 7.905
0.306 -2.4994
0.308 13.5431
0.31 14.8219
0.312 0.11625
0.314 16.5656
0.316 3.0225
0.318 10.1719
0.32 -23.4244
0.322 4.5338
0.324 1.6856
0.326 1.2788
0.328 5.9288
0.33 8.9513
0.332 8.8931
0.334 0.755625
0.336 1.0462
0.338 1.5113
0.34 19.1812
0.342 -5.2894
0.344 -14.0663
0.346 -12.4387
0.348 12.8456
0.35 -11.9737
0.352 2.4413
0.354 14.3569
0.356 1.4531
0.358 11.4506
0.36 -6.5681
0.362 4.8244
0.364 0
0.366 -0.34875
0.368 -19.8788
0.37 3.9525
0.372 -14.6475
0.374 -14.6475
0.376 24.8194
0.378 -2.0925
0.38 10.7531
0.382 14.415
0.384 4.7663
0.386 -10.1137
0.388 26.6213
0.39 4.4175
0.392 34.9912
0.394 0.93
0.396 -18.8906
0.398 -12.8456
0.4 5.115
0.402 -40.5713
0.404 -2.0344
0.406 -2.0925
0.408 11.8575
0.41 -2.2669
0.412 -7.4981
0.414 15.5775
0.416 -2.9644
0.418 9.6487
0.42 -5.2894
0.422 -15.0544
0.424 -2.79
0.426 -16.3331
0.428 13.2525
0.43 -1.7437
0.432 -6.1031
0.434 -0.755625
0.436 -2.6738
0.438 11.625
0.44 -14.2406
0.442 -3.9525
0.444 -20.9831
0.446 -5.2313
0.448 -24.5287
0.45 -16.1006
0.452 3.0225
0.454 -16.8562
0.456 -19.7044
0.458 -13.8338
0.46 -1.4531
0.462 -0.58125
0.464 8.9513
0.466 -15.6938
0.468 2.9063
0.47 0.871875
0.472 20.8088
0.474 -0.81375
0.476 -13.3687
0.478 27.9581
0.48 25.8075
0.482 -11.4506
0.484 16.2169
0.486 1.9181
0.488 -4.1269
0.49 6.045
0.492 -16.4494
0.494 0.523125
0.496 1.395
0.498 -3.9525
0.5 -3.72
0.502 15.1125
0.504 -8.835
0.506 -11.2762
0.508 -5.9869
0.51 17.5538
0.512 52.08
0.514 98.6381
0.516 137.2913
0.518 189.6037
0.52 225.99
0.522 286.2656
0.524 351.54
0.526 394.1456
0.528 438.9019
0.53 457.5019
0.532 505.2225
0.534 533.5294
0.536 566.0794
0.538 626.7037
0.54 623.6231
0.542 663.9038
0.544 682.62
0.546 711.3919
0.548 719.5875
0.55 728.4806
0.552 777.0731
0.554 744.93
0.556 770.5631
0.558 779.1075
0.56 788.7563
0.562 776.1431
0.564 788.0588
0.566 774.6319
0.568 760.9144
0.57 744.8719
0.572 733.0144
0.574 686.8631
0.576 671.1694
0.578 663.8456
0.58 611.8819
0.582 585.0281
0.584 550.4438
0.586 549.2812
0.588 494.1787
0.59 428.9044
0.592 388.0425
0.594 360.3169
0.596 320.5012
0.598 240.7538
0.6 214.2487
0.602 149.4394
0.604 98.8125
0.606 45.105
0.608 25.2263
0.61 -15.7519
0.612 -54.6375
0.614 -112.1231
0.616 -162.8662
0.618 -248.7169
0.62 -254.6456
0.622 -306.3188
0.624 -348.4594
0.626 -387.6938
0.628 -440.7037
0.63 -452.6194
0.632 -483.0188
0.634 -543.8175
0.636 -590.6662
0.638 -595.9556
0.64 -621.5888
0.642 -680.9925
0.644 -711.1012
0.646 -740.0475
0.648 -755.625
0.65 -737.49
0.652 -775.6781
0.654 -768.5869
0.656 -784.5712
0.658 -785.2106
0.66 -774.9806
0.662 -760.9144
0.664 -762.8325
0.666 -773.4694
0.668 -780.5025
0.67 -729.585
0.672 -730.8056
0.674 -714.2981
0.676 -663.2063
0.678 -644.49
0.68 -625.7737
0.682 -599.6757
0.684 -562.4756
0.686 -533.2969
0.688 -481.4494
0.69 -419.4881
0.692 -385.3688
0.694 -343.5769
0.696 -312.3637
0.698 -255.6919
0.7 -187.1044
0.702 -122.0044
0.704 -45.3375
0.706 -14.0663
0.708 -0.93
0.71 -13.3687
0.712 -2.9644
0.714 2.2669
0.716 26.2725
0.718 -33.3637
0.72 24.4125
0.722 2.3831
0.724 19.1231
0.726 -14.7056
0.728 -12.3225
0.73 -9.9394
0.732 -8.1375
0.734 -10.0556
0.736 7.0331
0.738 -5.8706
0.74 -17.7281
0.742 -22.9594
0.744 -16.3912
0.746 -7.0331
0.748 3.1969
0.75 16.1588
0.752 26.7375
0.754 17.3212
0.756 6.5681
0.758 -13.1362
0.76 23.3081
0.762 13.1362
0.764 5.5219
0.766 14.5313
0.768 -21.9131
0.77 16.74
0.772 7.6725
0.774 17.1469
0.776 9.765
0.778 30.69
0.78 -6.4519
0.782 2.9063
0.784 -14.7056
0.786 2.1506
0.788 -3.3131
0.79 0
0.792 -16.0425
0.794 -14.9381
0.796 2.9644
0.798 15.345
0.8 26.8537
0.802 -14.9381
0.804 1.6275
0.806 -1.9762
0.808 -22.2619
0.81 -26.505
0.812 0
0.814 -0.639375
0.816 12.4387
0.818 -1.9181
0.82 4.7081
0.822 15.7519
0.824 21.1575
0.826 -1.1044
0.828 -6.1613
0.83 -11.625
0.832 1.6856
0.834 1.9762
0.836 -0.290625
0.838 -11.7994
0.84 -0.11625
0.842 -3.4875
0.844 -16.9725
0.846 -28.6556
0.848 -4.0106
0.85 -15.9262
0.852 13.5431
0.854 -15.9844
0.856 1.395
0.858 18.0769
0.86 19.7044
0.862 -16.1588
0.864 -9.3
0.866 -3.3131
0.868 6.8587
0.87 0.2325
0.872 -14.2406
0.874 2.4413
0.876 8.835
0.878 -32.3175
0.88 -6.9169
0.882 1.8019
0.884 14.7056
0.886 1.3369
0.888 10.4044
0.89 -0.6975
0.892 9.1256
0.894 -9.3581
0.896 2.4994
0.898 -26.2725
0.9 -4.8244
0.902 -15.345
0.904 -6.8587
0.906 13.02
0.908 -21.2738
0.91 -15.5194
0.912 -23.25
0.914 -14.8219
0.916 -5.6381
0.918 15.7519
0.92 4.185
0.922 1.2788
0.924 7.2075
0.926 17.5538
0.928 22.2038
0.93 1.6275
0.932 0.34875
0.934 -10.1137
0.936 2.0344
0.938 -16.2169
0.94 -6.1031
0.942 14.2406
0.944 -10.2881
0.946 11.9737
0.948 -1.3369
0.95 -30.7481
0.952 -4.4756
0.954 -10.9856
0.956 13.5431
0.958 23.8894
0.96 9.4162
0.962 1.2788
0.964 1.3369
0.966 -15.4031
0.968 10.9275
0.97 9.7069
0.972 -6.9169
0.974 0.81375
0.976 -0.465
0.978 -0.406875
0.98 -18.4838
0.982 -1.1044
0.984 -5.2313
0.986 1.4531
0.988 -13.6013
0.99 -15.5194
0.992 -9.9394
0.994 11.5087
0.996 0.2325
0.998 15.4031
1 -25.11
1.002 1.6275
1.004 -1.7437
1.006 -15.2288
1.008 -8.1956
1.01 -17.4375
1.012 -2.2087
1.014 -5.0569
1.016 -13.95
1.018 -0.58125
1.02 0
1.022 -0.34875
1.024 5.6381
1.026 -13.8919
1.028 10.23
1.03 -3.1388
1.032 21.855
1.034 -20.8088
1.036 14.5894
1.038 -14.9963
1.04 -1.4531
1.042 -12.6131
1.044 1.7437
1.046 0.871875
1.048 11.8575
1.05 0.58125
1.052 -4.4175
1.054 -1.9762
1.056 0.93
1.058 2.6156
1.06 -10.4044
1.062 -2.0925
1.064 12.555
1.066 1.2788
1.068 10.4625
1.07 16.5075
1.072 -13.7175
1.074 0.174375
1.076 0.755625
1.078 0.6975
1.08 28.0163
1.082 14.88
1.084 11.16
1.086 2.5575
1.088 5.4056
1.09 -1.7437
1.092 8.0213
1.094 16.6238
1.096 -3.4875
1.098 19.8206
1.1 29.5856
1.102 -1.6275
1.104 -3.3713
1.106 -14.7056
1.108 5.7544
1.11 12.0319
};
\addplot [semithick, color1]
table {%
0 0
0.002 11.4375
0.004 0
0.006 7.6631
0.008 29.28
0.01 17.8425
0.012 0
0.014 15.0403
0.016 14.64
0.018 2.0016
0.02 14.64
0.022 9.0356
0.024 14.64
0.026 0
0.028 -14.64
0.03 -12.7528
0.032 0
0.034 -11.5519
0.036 -14.64
0.038 -12.2953
0.04 -14.64
0.042 5.3184
0.044 0
0.046 -19.9012
0.048 14.64
0.05 16.1269
0.052 0
0.054 -5.5472
0.056 14.64
0.058 -7.7775
0.06 0
0.062 -6.8053
0.064 -14.64
0.066 26.2491
0.068 0
0.07 -9.9506
0.072 -14.64
0.074 -11.0944
0.076 -14.64
0.078 9.0928
0.08 -14.64
0.082 -1.2009
0.084 0
0.086 -28.6509
0.088 -14.64
0.09 -2.2875
0.092 -14.64
0.094 -9.3216
0.096 0
0.098 -4.575
0.1 0
0.102 -7.0341
0.104 -14.64
0.106 -4.6322
0.108 29.28
0.11 24.3047
0.112 14.64
0.114 5.3756
0.116 0
0.118 3.0881
0.12 0
0.122 -7.6059
0.124 -14.64
0.126 6.7481
0.128 -14.64
0.13 -17.0419
0.132 -14.64
0.134 1.7156
0.136 0
0.138 1.7156
0.14 14.64
0.142 22.7034
0.144 0
0.146 2.5734
0.148 0
0.15 16.6416
0.152 -14.64
0.154 6.9197
0.156 14.64
0.158 -10.1794
0.16 -14.64
0.162 -11.6091
0.164 14.64
0.166 10.5225
0.168 -29.28
0.17 5.49
0.172 29.28
0.174 0
0.176 0
0.178 0.571875
0.18 -14.64
0.182 1.1437
0.184 0
0.186 16.0697
0.188 14.64
0.19 9.9506
0.192 0
0.194 -1.5441
0.196 -29.28
0.198 -0.915
0.2 14.64
0.202 19.2722
0.204 0
0.206 -7.32
0.208 14.64
0.21 -14.8116
0.212 0
0.214 -24.8194
0.216 -29.28
0.218 -4.6322
0.22 0
0.222 1.5441
0.224 -14.64
0.226 -35.9138
0.228 -14.64
0.23 -22.1316
0.232 0
0.234 1.7728
0.236 0
0.238 9.3787
0.24 0
0.242 0.915
0.244 14.64
0.246 -4.8037
0.248 -14.64
0.25 5.5472
0.252 14.64
0.254 -9.7791
0.256 14.64
0.258 -12.6384
0.26 -14.64
0.262 -13.4962
0.264 -14.64
0.266 16.6416
0.268 14.64
0.27 -14.5256
0.272 14.64
0.274 4.0031
0.276 0
0.278 9.3787
0.28 0
0.282 -6.6909
0.284 0
0.286 16.9275
0.288 0
0.29 -4.8037
0.292 14.64
0.294 -18.9862
0.296 0
0.298 1.4297
0.3 -14.64
0.302 -7.6059
0.304 0
0.306 -7.8919
0.308 -14.64
0.31 2.5734
0.312 0
0.314 11.8378
0.316 -14.64
0.318 16.9275
0.32 14.64
0.322 4.0031
0.324 -14.64
0.326 -17.2134
0.328 14.64
0.33 -4.3463
0.332 14.64
0.334 -0.8578125
0.336 0
0.338 15.4978
0.34 14.64
0.342 14.0109
0.344 14.64
0.346 -6.1762
0.348 14.64
0.35 22.2459
0.352 0
0.354 -7.0341
0.356 14.64
0.358 0.8578125
0.36 14.64
0.362 20.7591
0.364 0
0.366 -0.0571875
0.368 0
0.37 -8.9784
0.372 0
0.374 -4.5178
0.376 29.28
0.378 2.0588
0.38 -29.28
0.382 -33.9694
0.384 0
0.386 8.5209
0.388 0
0.39 -4.9753
0.392 14.64
0.394 -23.3325
0.396 -14.64
0.398 -1.3153
0.4 -14.64
0.402 0.4003125
0.404 14.64
0.406 -14.9259
0.408 -14.64
0.41 12.9816
0.412 14.64
0.414 12.1237
0.416 -14.64
0.418 8.9784
0.42 0
0.422 5.8331
0.424 14.64
0.426 -11.0372
0.428 -14.64
0.43 2.9166
0.432 14.64
0.434 5.0325
0.436 14.64
0.438 -7.6059
0.44 0
0.442 6.9769
0.444 0
0.446 20.7591
0.448 14.64
0.45 -1.2009
0.452 0
0.454 -2.0016
0.456 0
0.458 2.0588
0.46 0
0.462 -1.4297
0.464 -14.64
0.466 -8.5209
0.468 -14.64
0.47 -0.2859375
0.472 -14.64
0.474 -27.5072
0.476 -14.64
0.478 18.3572
0.48 0
0.482 9.0928
0.484 0
0.486 -5.8903
0.488 -14.64
0.49 18.8719
0.492 -14.64
0.494 2.1731
0.496 -29.28
0.498 -7.6631
0.5 0
0.502 14.9831
0.504 0
0.506 -6.7481
0.508 0
0.51 7.2628
0.512 43.92
0.514 100.0209
0.516 131.76
0.518 211.4794
0.52 219.6
0.522 290.3409
0.524 322.08
0.526 368.8022
0.528 409.92
0.53 464.2481
0.532 512.4
0.534 540.8221
0.536 541.68
0.538 587.8303
0.54 614.88
0.542 649.7071
0.544 688.08
0.546 700.89
0.548 702.72
0.55 719.3615
0.552 732
0.554 746.2397
0.556 746.64
0.558 764.9972
0.56 746.64
0.562 750.0141
0.564 746.64
0.566 742.1793
0.568 732
0.57 718.275
0.572 688.08
0.574 667.7212
0.576 644.16
0.578 630.435
0.58 629.52
0.582 560.9522
0.584 527.04
0.586 499.8187
0.588 468.48
0.59 435.4828
0.592 409.92
0.594 334.0894
0.596 292.8
0.598 246.764
0.6 175.68
0.602 154.5206
0.604 102.48
0.606 54.0994
0.608 29.28
0.61 -17.4994
0.612 -73.2
0.614 -116.4909
0.616 -161.04
0.618 -201.7003
0.62 -248.88
0.622 -308.6981
0.624 -366
0.626 -373.5487
0.628 -439.2
0.63 -466.7644
0.632 -497.76
0.634 -529.7278
0.636 -556.32
0.638 -587.0869
0.64 -629.52
0.642 -657.9993
0.644 -702.72
0.646 -700.0322
0.648 -732
0.65 -751.7868
0.652 -732
0.654 -765.7978
0.656 -746.64
0.658 -780.0947
0.66 -775.92
0.662 -793.8768
0.664 -732
0.666 -757.0481
0.668 -732
0.67 -685.6781
0.672 -688.08
0.674 -666.12
0.676 -644.16
0.678 -627.8616
0.68 -614.88
0.682 -565.3556
0.684 -512.4
0.686 -489.8109
0.688 -468.48
0.69 -419.3559
0.692 -380.64
0.694 -334.3181
0.696 -336.72
0.698 -253.2263
0.7 -190.32
0.702 -129.015
0.704 -43.92
0.706 13.8966
0.708 29.28
0.71 4.5178
0.712 29.28
0.714 -22.9894
0.716 14.64
0.718 -10.7512
0.72 14.64
0.722 -1.8872
0.724 0
0.726 -18.0712
0.728 0
0.73 -1.7156
0.732 14.64
0.734 0.4003125
0.736 29.28
0.738 13.4391
0.74 0
0.742 6.8625
0.744 0
0.746 5.8331
0.748 14.64
0.75 5.6616
0.752 0
0.754 -16.3556
0.756 -14.64
0.758 -2.6306
0.76 0
0.762 -12.9244
0.764 -29.28
0.766 -12.3525
0.768 -14.64
0.77 -10.4653
0.772 0
0.774 -0.0571875
0.776 -29.28
0.778 -7.6631
0.78 14.64
0.782 0.7434375
0.784 0
0.786 -9.3787
0.788 -14.64
0.79 8.6353
0.792 0
0.794 16.6987
0.796 14.64
0.798 -8.6925
0.8 0
0.802 -18.3572
0.804 0
0.806 -16.3556
0.808 -14.64
0.81 -10.8656
0.812 -14.64
0.814 -1.7728
0.816 0
0.818 -8.6925
0.82 14.64
0.822 1.6012
0.824 -14.64
0.826 -6.8053
0.828 -14.64
0.83 -3.3169
0.832 0
0.834 32.3681
0.836 0
0.838 0.7434375
0.84 -14.64
0.842 1.1437
0.844 0
0.846 -1.1437
0.848 0
0.85 -1.4869
0.852 14.64
0.854 -5.3184
0.856 -14.64
0.858 15.3834
0.86 -14.64
0.862 1.0866
0.864 -14.64
0.866 0.8578125
0.868 14.64
0.87 14.0109
0.872 -14.64
0.874 -21.3881
0.876 -14.64
0.878 -4.0603
0.88 14.64
0.882 2.1731
0.884 14.64
0.886 -18.9291
0.888 0
0.89 -9.8934
0.892 0
0.894 11.3803
0.896 0
0.898 -4.9181
0.9 14.64
0.902 -16.5272
0.904 14.64
0.906 0.68625
0.908 0
0.91 -9.0356
0.912 0
0.914 -0.571875
0.916 0
0.918 -9.0356
0.92 0
0.922 6.405
0.924 0
0.926 -14.4112
0.928 0
0.93 -6.6909
0.932 0
0.934 4.5178
0.936 14.64
0.938 1.0294
0.94 14.64
0.942 14.4112
0.944 0
0.946 6.7481
0.948 0
0.95 22.3603
0.952 14.64
0.954 14.7544
0.956 14.64
0.958 12.2953
0.96 14.64
0.962 1.8872
0.964 14.64
0.966 -17.6709
0.968 -14.64
0.97 14.2397
0.972 14.64
0.974 -16.4128
0.976 0
0.978 -9.5503
0.98 0
0.982 -7.7203
0.984 0
0.986 -24.2475
0.988 -14.64
0.99 -1.2009
0.992 0
0.994 7.6059
0.996 -14.64
0.998 -12.81
1 -29.28
1.002 -11.1516
1.004 0
1.006 4.4034
1.008 -14.64
1.01 -2.0016
1.012 0
1.014 -16.6416
1.016 14.64
1.018 -8.0634
1.02 0
1.022 -5.8331
1.024 0
1.026 -12.2953
1.028 0
1.03 20.1872
1.032 0
1.034 -1.2581
1.036 0
1.038 11.6662
1.04 0
1.042 -14.2397
1.044 0
1.046 -11.3231
1.048 -14.64
1.05 -1.1437
1.052 0
1.054 -2.6306
1.056 0
1.058 9.0356
1.06 0
1.062 1.4297
1.064 14.64
1.066 30.6525
1.068 0
1.07 15.7266
1.072 0
1.074 3.5456
1.076 0
1.078 4.9753
1.08 14.64
1.082 1.5441
1.084 14.64
1.086 -7.5487
1.088 0
1.09 0.6290625
1.092 14.64
1.094 6.5194
1.096 0
1.098 -11.0372
1.1 14.64
1.102 -4.7466
1.104 -14.64
1.106 23.9044
1.108 -14.64
1.11 2.8022
};
\end{axis}

\end{tikzpicture}
\subcaption{Ausgangsspannung}
	\label{fig:Uout}
\end{subfigure}
\label{fig:opt.Kennlinie}
\caption{Anpassung von $K$}
\end{figure}
In \figref{fig:K} ist $K_0$ die initial Kennlinie und $K_1$ ist die Kennlinie nach der ersten Anpassung. Dabei wurde $\Uquest_{, \mathrm{ideal}}$ so berechnet, dass für $\Uout_{, \mathrm{ideal}}$ gilt $V_{PP} = \SI{1,5}{\V}$
\subsubsection*{-Probleme mit der Kennlinie-}
\label{subsubsec:opt.adjusta.problem}
-Hier wird die Problematik mit der Amplitude nochmal aufgegriffen und unsere Lösungsansätze mit Plots verdeutlicht-
\end{document}
