\documentclass[../Report.tex]{subfiles}


\begin{document}

\chapter{Iterative Optimierung des Hammerstein-Modells}
\label{chap:opt}

%TODO: überprüfe Referenz zu Jens, da Ideen nicht publiziert wurden! Wie wird das eingebunden?
Ziel der Optimierung von Übertragungsfunktion $\Hcompl$ und Kennlinie $K$ mit ihren Parametern $a$ ist die Minimierung des Fehlers zwischen idealem und gemessenem Ausgangssignal, $\Uout_{, \mathrm{id}}$ und $\Uout_{, \mathrm{meas}}$ . Die Minimierung des relativen Fehlers ist also gegeben durch
\begin{align}
\label{eq:opt.relFehler}
	\min \; \mathit{f} \oft = \min \left( \frac{\Uout_{, \mathrm{meas}}  - \Uout_{, \mathrm{id}} }{ \Uout_{, \mathrm{id}} } \right) 
	= \min \left( \frac{ \Uout_{, \mathrm{meas}}}{ \Uout_{, \mathrm{id}}} -1 \right) 
	\; .
\end{align}
Für das verwendete Hammerstein-Modell liegt die in \cite{----Jens---} %TODO: Referenz Idee iterative Optimierung Jens
vorgeschlagene getrennte, iterative Optimierung von $\Hcompl$ und $K$ nahe. 
Die Auswertung der Qualität des Einzelsinus erfolgt dabei durch das RF-Tool von --- Zitat RF-Tool --- mit Entwicklungsstand vom --- --- unter Verwendung des als \lstinline{QGesamt1} geführten Qualitätswerts \footnote{\label{foot:opt.H.quality} Hierauf beziehen sich alle weiteren Angaben zur Qualität des Signals. Eine intensivere Befassung mit dem Tool hat nicht stattgefunden.}.
%TODO: Auffüllen Daten!


\section{Optimierung der linearen Übertragungsfunktion $H$}
\label{sec:opt.H}

Die Optimierung von $\Hcompl \ofomega$ beruht auf der Annahme, dass sich \eqref{eq:opt.relFehler} auf die Betragsspektren des berechneten und des gemessenen Ausgangssignals, $\Uoutc_{, \mathrm{id}} \ofomega $ und $\Uoutc_{, \mathrm{meas}} \ofomega $ fortsetzen lässt mit 

\begin{align}
\label{eq:opt.ratio}
	\fabs \ofomega :=  
				\frac{\mathrm{abs} \left( \Uoutc_{, \mathrm{meas}} \ofomega \right)}{\mathrm{abs} \left(\Uoutc_{, \mathrm{id}} \ofomega \right)} -1
				\; .
\end{align} 

Ist im Betragsspektrum des gemessenen Signals eine Frequenz mit halbem Betrag verglichen mit dem idealen Signal vertreten, wird dies entsprechend der Linearität der Übertragunsfunktion dahingehend gedeutet, dass die Verstärkung von $\Hcompl$ bei dieser Frequenz um einen Faktor $2$ zu gering ist.
Iterativ mit einer Schrittweite $\sigma_H$ ausgeführt, folgt für den $i$-ten Schritt

\begin{align}
\label{eq:opt.Hnew}
	\mathrm{abs} \left( \Hcompl^{i+1} \right)
		=\mathrm{abs} \left( \Hcompl^{i}  \right) \cdot
		\left( 1 - \sigma_H^i \: \fabs^{i}	\right)					 
\end{align}

für $\sigma_H^i \in \left[ 0 , 1 \right]$ und $\Uoutc_{, \mathrm{meas}}^{i}$ in $\fabs^{i}$ als gemessenem Ausgangssignal für das mit $\Hcompl^{i}$ berechnete Eingangssignal \footnote{Nachfolgend wird aus Gründen der Übersichtlichkeit $\fabs$ statt dem länglichen Bruch genutzt}.
Würde allerdings \eqref{eq:opt.Hnew} mit komplexen Zahlen und nicht allein den Beträgen ausgeführt, würde auch die Phase der $-1$ beachtet und folglich die durch $\sigma_H$ skalierte komplexe Zahl wesentlich verändert. Also muss für das Phasenspektrum eine andere Optimierung erfolgen.
Eine Möglichkeit hierfür wäre die simple Anpassung der Phase $ \mathrm{arg} \left( \Hcompl \right) = \varphi_H$ mit 
\begin{align}
\label{eq:opt.HnewPhase}
	\varphi_H^{i+1} = \varphi_H^{i} - \sigma_{\varphi}^{i} 
			\left( \: \mathrm{arg} \left( \Uoutc_{, \mathrm{meas}} \right)
					- \mathrm{arg} \left( \Uoutc_{, \mathrm{id}} \right) \: \right)
\end{align}
mit $\sigma_{\varphi}^i \in \left[ 0 , 1 \right]$. Diese Anpassung der Phase wurde jedoch nur kurzen Tests unterzogen und anschließend nicht weiter verfolgt. Es hat sich die Signalform des Ausgangssignals unproportional stärker verändert, als dies nur im Falle der Betrags-Anpassung der Fall war. Vermutlich %%%% TODO: kann man vermutlich hier nutzen???
liegt dies an dem aus dem Ausgangssignal gewonnenen Phasengang, der in wesentlich größerem Maße vom idealen Phasengang abweicht als im Betragsspektrum. In \figref{fig:opt.spektrum_BB_signal} sind Betrag und Phase der durch FFT erhaltenen Spektren für gemessenes und ideales Ausgangssignal vor Durchführung einer Optimierung dargestellt. Insbesondere illustriert \figref{fig:opt.tzeum} die bei gemessenem Signal auftretende Streuung der Phase. 
\\


\pgfplotstableread[col sep = comma] {opt_spect_ideal_abs.csv} \absSpectIdeal 
\pgfplotstableread[col sep = comma] {opt_spect_meas_abs.csv} \absSpectMeas
\pgfplotstableread[col sep = comma] {opt_spect_ideal_angle.csv} \angleSpectIdeal 
\pgfplotstableread[col sep = comma] {opt_spect_meas_angle.csv} \angleSpectMeas 

\begin{figure}[htb]
\begin{subfigure}{0.5 \textwidth}
\centering
    \begin{tikzpicture}
\begin{axis}[
		legend entries = {Ideales Signal, Gemessenes Signal},
		legend pos = north east,
		xlabel={Frequenz},
		ylabel={Spektraldichte },
		%xtick distance = 10000000,
%		xminorgrids,
%		xmajorgrids,
		minor x tick num =3,
		xtick pos = lower,
		ytick pos = left,
		xtick align = outside,
		ytick align = outside,
		scaled x ticks = base 10:-6,
		xtick scale label code/.code={\si{\MHz}},
		xmin = -5000000,
		xmax = 85000000,
		ymin = 0,
		ymax = 0.03,
		]
		
		\addplot[blue, mark size=3.5pt] table [ x index =0, y index=1] {\absSpectIdeal};	% plot des Idealen Betragsspektrums
		\addplot[green, mark size=3.5pt] table [ x index =0, y index=1] {\absSpectMeas};	% Plot des gemessenen Betragsspektrums
\end{axis}
\end{tikzpicture}
\caption{Betragsspektren}
	\label{fig:opt.abs_spektrum}
\end{subfigure}
\begin{subfigure}{0.5 \textwidth}
\centering
    \begin{tikzpicture}
\begin{axis}[
		legend entries = {Ideales Signal, Gemessenes Signal},
		legend pos = north east,
		xlabel={Frequenz},
		ylabel={Phase},
		y label style={at={(axis description cs:-0.1,.5)},anchor=south},
%		%xtick distance = 10000000,
%		xminorgrids,
%		xmajorgrids,
		minor x tick num =3,
		xtick pos = lower,
		ytick pos = left,
		xtick align = outside,
		ytick align = outside,
		scaled x ticks = base 10:-6,
		xtick scale label code/.code={\si{\MHz}},
		xmin = -5000000,
		xmax = 85000000,
		scaled y ticks={real:3.1415},
		ytick scale label code/.code={$\si{\radian}$},
%		ymin = 0,
%		ymax = 0.03,
		]
		
		\addplot[blue, only marks, mark size=1pt] table [ x index =0, y index=1] {\angleSpectIdeal};	% plot des Idealen Betragsspektrums
		\addplot[green, only marks, mark size=1pt] table [ x index =0, y index=1] {\angleSpectMeas};	% Plot des gemessenen Betragsspektrums
\end{axis}
\end{tikzpicture}
\caption{Phasenspektren}
	\label{fig:opt.angle_spektrum}
\end{subfigure}
\caption{Spektrum des Einzelsinus-Signals, berechnet und gemessen mit je $109$ Punkten}
\label{fig:opt.spektrum_BB_signal}
\end{figure}

 
Neben den für kontinuierliche Funktionen problemlos definierbaren iterativen Zuweisungen ergeben sich in Messung und diskreter Ausführung jedoch Fehlerquellen. Problematisch sind insbesondere solche, die in \eqref{eq:opt.Hnew} durch das Betragsverhältnis der Ausgangssignale verstärkt werden. 
Unterscheiden sich die Spektren hier um einen großen Faktor, resultiert dies in einer großen Anpassung der Übertragungsfunktion für die betreffende Frequenz. Dies ist folglich insbesondere bei kleinen Beträgen der Spektren problematisch, wenn Ungenauigkeiten und Störeinflüsse betrachtet werden. 
\\
Besondere Störeinflüsse ergeben sich also durch
\begin{itemize}
	\item Rauschen: Weißes Rauschen macht sich in allen Frequenzen bemerkbar mit kritischem Einfluss bei geringer Spektraldichte des Signals.
	
	\item Diskretisierungsfehler: Die FFT bedingt eine begrenzte Auflösung, in den Spektren von $\Hcompl$ und den gemessenen Signalen und liegt insbesondere im Allgemeinen an unterschiedlichen Frequenzen und mit unterschiedlich vielen Punkten vor.
	
	\item Interpolationsfehler: Die (hier lineare) Interpolation der Spektren zur Auswertung von $\fabs$ an den Frequenzen von $\Hcompl$ kann insbesondere den Einfluss oben genannter Punkte verstärken.
\end{itemize}

Weiterhin zeigt sich auch in der geringen Stützstellenzahl in \figref{fig:opt.spektrum_BB_signal} bereits eine erste konzeptuelle Problematik des Vorgehens. Der Frequenzabstand zwischen zwei Werten der FFT ist stets mit der Wiederholfrequenz $f_{rep}$ gegeben und lässt sich somit nicht durch eine höhere Auflösung der Messgeräte verbessern, der betrachtete Bereich bis $\SI{80}{\MHz}$ nicht besser auflösen.
Folglich setzt sich diese Ungenauigkeit auch auf die Optimierung der Kennlinie fort. Insbesondere relevant wird dies, da die Kennlinie mit nahezu der doppelten Anzahl an Werten erstellt wird und somit der Interpolationsfehler ungleich größer wird als bei ähnlicher Anzahl Stützstellen.


\subsubsection*{Ignorieren kleiner Beträge im Spektrum}
\label{subsubsec:opt.H.prom}

Um Rauscheinflüsse und Probleme durch Nulldurchgänge zu dämpfen, wurde ein erster intuitiver Ansatz vorgenommen: Bei den Betragsspektren der in $\fabs$ eingehenden Signale, des gemessenen und idealisierten Spannungssignals, wurden alle Anteile, die verglichen mit dem Maximalwert des betreffenden Spektrums besonders klein sind, auf einen vorgegebenen Wert, im Folgenden Default-Wert genannt, gesetzt. Dies führt an den betroffenen Frequenzen zu $\fabs = 0$ und damit keiner Änderung von $\Hcompl$.
Dies bedeutet also, dass alle Einträge des Betragsspektrums von $\Uout_{,\mathrm{ideal}}$ mit weniger als zum Beispiel $5 \, \promille $ der maximalen Amplitude auf den Default-Wert gesetzt werden. Insbesondere werden auch die Einträge an den Frequenzen zurückgesetzt, die im Spektrum von $\Uout_{,\mathrm{meas}}$ klein gegen das zugehörige Maximum sind.
\\
Zu beachten bei letzterem Punkt ist die notwendige Rundung, wenn die Einträge der FFT an unterschiedlichen Frequenzen vorliegen. 
\\
\\
\noindent
Mit Beschränkung auf $3 \, \promille$ und dem globalen Minimum beider Spektren als Default-Wert werden insbesondere Einträge in den höheren Frequenzen des Spektrums \ref{fig:opt.abs_spektrum} beeinflusst. Erst ab etwa $5 \promille$ werden alle Frequenzen ab ungefähr $\SI{60}{\MHz}$ auf den Default-Wert gesetzt. Mit dieser Methodik könnten folglich in erster Linie massive Korrekturen an hohen Frequenzen von $\Hcompl$ verhindert.
Verglichen mit der nachfolgend beschriebenen Anpassung hat dieses Vorgehen aufgrund der beschriebenen Bandbreite einen niedrigeren Einfluss auf die Qualität des Ausgangssignals.


\subsubsection*{Ignorieren großer Korrektur-Terme}
\label{subsubsec:opt.H.RMS}

Ein zweiter, sehr grober Ansatz liegt in der Beschränkung von $\fabs$ auf Werte unterhalb einer vorgegebenen Schwelle. Zugrunde liegt die Annahme, dass die gerade an Nulldurchgängen des Spektrums sowie bei vielen hohen Frequenzen auftretenden großen Werte durch die in obiger Aufzählung genannten Fehlerquellen entstehen. Hier bedeutet dies insbesondere, dass die Diskretisierung die Nulldurchgänge nicht korrekt darstellen kann. Die Interpolation auf Frequenzen von $\Hcompl$ ist dann aufgrund der großen Sprünge von Werten in direkter Umgebung der problematischen Frequenzen mit großer Ungenauickgeit behaftet. Dies kann zu den beschriebenen, großen Korrektur-Termen in $\fabs$ führen.

\lstset{language=Python}
\begin{lstlisting}[caption={Pseudocode zur Veranschaulichung der Anpassung des Korrekturterms}, label=code:opt.H.pseudoRMS, numbers=none]
	rms_orig = root_mean_square( f_abs )
	f_abs_to_use = f_abs[ where( abs(f_abs) >= 0.02 * rms_orig ] 
	rms_mod = root_mean_square( f_abs_to_use )
	idx_to_clear = f_abs[ where( abs(f_abs) >= rms_mod ] 
	f_abs[ ix_to_clear ] = 0
\end{lstlisting}

Vereinfacht bedeutet der verfolgte Ansatz, ausnehmend große Werte von $\fabs$ als unrealistisch abzutun. Eine Pseudo-Implementierung findet sich in \coderef{code:opt.H.pseudoRMS}, um die nachfolgende Erläuterung zu illustrieren. In der vorgenommenen Implementierung wurde $\fabs$ an den ausgewählten Frequenzen beliebig auf $0$ gesetzt, also keine Anpassung bei diesen Frequenzen ermöglicht.
Als Grenze genutzt wurde ein modifizierter Effektivwert, nachfolgend mit RMS (Root Mean Square) bezeichnet. 
Der reine RMS von $\fabs$ unterliegt der Problematik, eine unproportional große Gewichtung von kleinen Einträgen zu enthalten.
\\
Idealerweise enthält $\fabs$ mit jeder Iteration kleinere Einträge als zuvor. Es würden also bei Nutzung des reinen RMS unter Umständen mit zunehmender Schrittzahl zunehmend mehr Werte in $\fabs$ ignoriert - was der Optimierung entsprechende Grenzen setzt. 
In Kombination mit den im vorigen Abschnitt erläuterten Anpassungen wäre die Problematik unumgänglich, da Frequenzen, die explizit nicht bei der Anpassung berücksichtig werden sollen, den reinen RMS-Wert beeinflussen.
Folglich muss der RMS modifiziert werden. Hier wurden zur Berechnung des modifizierten RMS nur die Werte einbezogen, die mehr als beliebig gewählte $2 \, \%$ des reinen RMS betragen. Es handelt sich also bei der vorgenommenen Anpassung um eine sehr grobe und größtenteils willkürliche Wahl der Parameter, die zu Zwecken der Illustration jedoch brauchbare Ergebnisse liefert.

Für einen Eindruck der Tragweite der RMS-Beschränkung bietet sich die Betrachtung von $\fabs$ in \figref{fig:opt.H.RMS_fabs} an. Klar zu erkennen sind die Ausreißer bei den meisten Frequenzen der originalen Nulldurchgänge. Auch die bei höheren Frequenzen auftretenden größeren Fehlerterme fallen auf. Die Korrekturen sind an den auf 0 gezogenen Werten zu erkennen. Der Unterschied zwischen dem originalen und dem modifizierten RMS liegt hier bei ungefähr $0.04$, also nahezu $10 \%$.
Für die Einordnung der Größenordnung des Korrekturterms sei nochmals betont, dass aufgrund der Anpassung \eqref{eq:opt.Hnew} ein Korrekturterm von $\pm 0.5$ bei dem im Folgenden standardmäßig verwendeten $\sigma_H = \nicefrac{1}{2}$ zu einer Anpassung in $\Hcompl$ in Höhe eines Viertels des ursprünglichen Wertes führt.

\pgfplotstableread[col sep = comma] {f_abs_orig.csv} \fabsOrig 
\pgfplotstableread[col sep = comma] {f_abs_RMS_1.csv} \fabsRMSone 
\pgfplotstableread[col sep = comma] {f_abs_RMS_2.csv} \fabsRMStwo
\pgfplotstableread[col sep = comma] {f_abs_RMS_3.csv} \fabsRMSthree
\pgfplotstableread[col sep = comma] {H_RMS_orig.csv} \Horig
\pgfplotstableread[col sep = comma] {H_RMS_1.csv} \HrmsOne
\pgfplotstableread[col sep = comma] {H_RMS_2.csv} \HrmsTwo
\pgfplotstableread[col sep = comma] {H_RMS_3.csv} \HrmsThree
\begin{figure}[htb]
%\begin{center}
%\begin{subfigure}{\textwidth}
%%	\begin{center}
%    \begin{tikzpicture}
%\begin{axis}[
%		scale only axis,
%		width = 0.75 \textwidth,
%		height = 0.2 \textheight,
%		legend entries = {original, Step 1, Step 2, Step 3},
%		legend pos = north east,
%		legend columns = 3,
%		xlabel={Frequenz},
%		ylabel={Betrag Übertragung $\Hcompl$},
%		y label style={at={(axis description cs:-0.1,.5)},anchor=south},
%		%xtick distance = 10000000,
%%		xminorgrids,
%%		xmajorgrids,
%%		minor x tick num =3,
%		xtick pos = lower,
%		ytick pos = left,
%		xtick align = outside,
%		ytick align = outside,
%		scaled x ticks = base 10:-6,
%		xtick scale label code/.code={[MHz]},
%		xmin = 0,
%		xmax = 80000000,
%		ymin = 0,
%		%ymax = 0.03,
%		]
%		\addplot [black, sharp plot, mark size = 3pt] table [x index = 0, y index = 1] {\Horig};
%		\addplot [blue, sharp plot, mark size =3pt] table [x index = 0, y index = 1] {\HrmsOne};
%		\addplot [red, sharp plot, mark size =3pt] table [x index = 0, y index = 1] {\HrmsTwo};
%%		\addplot [red, sharp plot, mark size =3pt] table [x index = 0, y index = 1] {\HrmsThree};
%\end{axis}
%\end{tikzpicture}
%\caption{Entwicklung des Betrags der Übertragungsfunktion über mehrere Iterationen und im Anfangszustand} %alte Messung!
%	\label{fig:opt.H.RMS_H}
%%	\end{center}
%\end{subfigure}
%\\
%\begin{subfigure}{\textwidth}
%%\begin{center}
    \begin{tikzpicture}
\begin{axis}[
		scale only axis,
		width = 0.7 \textwidth,
		height = 0.17 \textheight,
		legend entries = {initial, angepasst},
		legend pos = south west,
		legend columns = 2,
		cycle list name = color list,
		xlabel={Frequenz},
		ylabel={Korrekturterm $\fabs$},
		y label style={at={(axis description cs:-0.1,.5)},anchor=south},
		%xtick distance = 10000000,
%		xminorgrids,
%%		xmajorgrids,
%		minor x tick num =3,
		xtick pos = lower,
		ytick pos = left,
		xtick align = outside,
		ytick align = outside,
		scaled x ticks = base 10:-6,
		xtick scale label code/.code={[MHz]},
		extra y ticks={0.5697153886590458, -0.5697153886590458
						},
			extra y tick labels={+RMS, -RMS},
		extra y tick style={grid=major, ytick pos=right, ytick align=outside, ticklabel pos=right},		
		xmin = 0,
		xmax = 80000000,
		ymin = -0.75,
		ymax = 0.75,
		]
		
		\addplot  table [x index = 0, y index = 1] {\fabsOrig};
		\addplot  table [x index = 0, y index = 1] {\fabsRMSone};
%		\addplot [red, sharp plot, mark size =3pt] table [x index = 0, y index = 1] {\fabsRMStwo};
%		\addplot [blue, sharp plot, mark size =3pt] table [x index = 0, y index = 1] {\fabsRMSthree};
\end{axis}
\end{tikzpicture}
\caption{Korrekturterm in initialer und in angepasster Form mit RMS- Korrektur, am Messaufbau}
\label{fig:opt.H.RMS_fabs}
%\end{center}
%\end{subfigure}
%\end{center}
%\caption[Ignorieren großer Korrektur-Terme]{Entwicklung von Übertragungsfunktion und Korrekturterm bei Beschränkung von $\fabs$ mit angepasstem RMS-Wert und Schrittweite $\sigma_H = \frac12$}
%\label{fig:opt.H.iteration}
\end{figure}



\subsubsection{Auswertung}


Eine erste Illustration bietet die Anwendung des beschriebenen Vorgehens auf das \mock-System. Hier lässt sich testen, ob unter der Annahme einer vorgegebenen Übertragungsfunktion  $\Hcompl_{ \mathrm{mock}}$ im Sinne des Hammerstein-Modells und unter Vernachlässigung der Kennlinie (entspricht einer als ideal berechneten Kennlinie) eine Verbesserung des Signals erreicht wird und wie sich die Fehlerquellen mit Ausnahme von Rauschen auswirken. 
Das Ergebnis dieses Tests ist eindeutig und anschaulich in \figref{fig:opt.H.mock} aufgetragen für jeweils 15 Iterationen. Es lässt sich feststellen, dass eine Anpassung sowohl der Nulldurchgänge als auch bei hohen Frequenzen unbedingt notwendig ist und dies durch die Beschränkung großer Korrekturterme angegangen werden kann.\footnote{Die Auswirkungen der zusätzlichen Nutzung einer Beschränkung auf $3 \promille$ verbessern die Übertragungsfunktion optisch nicht wesentlich.}
Auch zeigen sich in direkter Umgebung der Nulldurchgänge noch Frequenzen, an denen die Korrekturterme zwar geringer als der modifizierte RMS, jedoch immer noch ungenau durch die Interpolation aufgelöst sind.
Diese Fehler sind ausschlaggebend dafür, dass sich die Qualität des erhaltenen Ausgangssignals über die Iterationen verschlechtert.\footnote{Je geringer der Wert des Qualitätskriteriums, desto ähnlicher ist das Signal einem idealen Einzelsinus.} Nichtsdestotrotz nähert sich die Übertragungsfunktion außerhalb der problematischen Nulldurchgänge gerade in niedrigeren Frequenzen sehr gut an $\Hcompl_{\mathrm{mock}}$ an. 
 
 
\pgfplotstableread[col sep = comma] {opt_mock_rms/H0.csv} \Hinit
\pgfplotstableread[col sep = comma] {H_a_mock.csv} \Hmock
%\pgfplotstableread[col sep = comma] {opt_mock_rms/H1.csv} \rmsHone
%\pgfplotstableread[col sep = comma] {opt_mock_rms/H5.csv} \rmsHfive
%\pgfplotstableread[col sep = comma] {opt_mock_rms/H10.csv} \rmsHten
\pgfplotstableread[col sep = comma] {opt_mock_rms/H15_3prom.csv} \rmsHprom
\pgfplotstableread[col sep = comma] {opt_mock_rms/H15.csv} \rmsHften
%\pgfplotstableread[col sep = comma] {opt_mock_simple/H1.csv} \simpleHone
%\pgfplotstableread[col sep = comma] {opt_mock_simple/H5.csv} \simpleHfive
%\pgfplotstableread[col sep = comma] {opt_mock_simple/H10.csv} \simpleHten
\pgfplotstableread[col sep = comma] {opt_mock_simple/H15.csv} \simpleHften

\begin{figure}[htb]
\begin{subfigure}{\textwidth}
    \begin{tikzpicture}
\begin{axis}[
		scale only axis,
		width = 0.8 \textwidth,
		height = 0.2 \textheight,		
		legend style={ at={(0.5,1.1)},
				anchor=south},
		legend columns = 2,
		transpose legend, 
		cycle list name = color list,
		xlabel={Frequenz},
		ylabel={Betrag Übertragung $\Hcompl$},
		y label style={at={(axis description cs:-0.1,.5)},anchor=south},
		%xtick distance = 10000000,
%		xminorgrids,
%		xmajorgrids,
%		minor x tick num =3,
		xtick pos = lower,
		ytick pos = left,
		xtick align = outside,
		ytick align = outside,
		scaled x ticks = base 10:-6,
		xtick scale label code/.code={[MHz]},
		xmin = 0,
		xmax = 80000000,
		ymin = 0,
		ymax = 15,
		]	
		\addplot  table [x index = 0, y index = 1] {\Hmock}; \addlegendentry{$\Hcompl_{\mathrm{mock}}$}
		
		\addplot  table [x index = 0, y index = 1] {\Hinit}; \addlegendentry{$\Hcompl_{\mathrm{initial}}$}
		
%		\addplot [mark size =3pt] table [x index = 0, y index = 1] {\simpleHone}; \addlegendentry{Step 1 - keine Anpassung}
%		\addplot  table [x index = 0, y index = 1] {\simpleHfive}; \addlegendentry{Step 5 - keine Anpassung}
%		\addplot [mark size =3pt] table [x index = 0, y index = 1] {\simpleHten}; \addlegendentry{Step 10 - keine Anpassung}
		\addplot  table [x index = 0, y index = 1] {\simpleHften}; \addlegendentry{Step 15 - keine Anpassung}
		
%		\addplot [mark size =3pt] table [x index = 0, y index = 1] {\rmsHone}; \addlegendentry{Step 1 - mit RMS}
%		\addplot  table [x index = 0, y index = 1] {\rmsHfive}; \addlegendentry{Step 5 - mit RMS}
		\addplot  table [x index = 0, y index = 1] {\rmsHften}; \addlegendentry{Step 15 - mit RMS}	
		
		\addplot table [x index = 0, y index = 1] {\rmsHprom}; \addlegendentry{Step 15 - mit RMS und $3 \promille$}

		
\end{axis}
\end{tikzpicture}
\caption{Entwicklung des Betrags der Übertragungsfunktion des \mock-Systems mit und ohne Anpassung großer Korrekturterme}
	\label{subfig:opt.H.mock}
\end{subfigure}
\\
\begin{subfigure}{\textwidth}
\begin{tikzpicture}
	\begin{axis}[
		scale only axis,
		width = 0.4 \textwidth,
		height = 0.2 \textheight,		
		legend pos = outer north east,
		xlabel={Iterationsschritt},
		ylabel={Güte Ausgangssignal},
		xtick pos = lower,
		ytick pos = left,
		xtick align = outside,
		ytick align = outside,
		xmin = 0.2,
		xmax = 15.8,
		]	
		
	\addplot table [x expr=\coordindex+1, y index =0]{
		0.48726213261815016 
		0.49000679356347154
		0.5101194806372478
		0.5266448340807586
		0.5375527813320436
		0.5447108764910896
		0.5499773649457441
		0.5545875437451622
		0.5592582035611007
		0.5643452240862432
		0.5699343256576608
		0.5758895933658901
		0.5819216415470254
		0.5877172129682854
		0.5930401259075484
		}; \addlegendentry{Korrektur ohne Anpassung}	
	
	\addplot table [x expr=\coordindex+1, y index =0]{
		0.48726213261815016
		0.4759961751169583 
		0.4852543982854474 
		0.4954878236964439 
		0.5034038461346506 
		0.5086562700430226 
		0.5124108351861989 
		0.5156318810838099 
		0.517584903029569
		0.518781160261209
		0.5195457063994412 
		0.520135841606002
		0.5205906984373314 
		0.5210120621383171 
		0.5215551522937539
		}; \addlegendentry{Korrektur mit RMS}
		
		\addplot table [x expr=\coordindex+1, y index =0]{
		0.48726213261815016 
		0.4695879133604225
		0.47376436493301266 
		0.48016422206587334 
		0.4867655459013696 
		0.4916978026875846 
		0.49484977950838993 
		0.4971321854184491 
		0.4989985349115239 
		0.5021058501215027 
		0.5044349692747538 
		0.5061724548300056 
		0.5074558511064865 
		0.5085423753752575 
		0.5094731920124752
		}; \addlegendentry{Korrektur mit RMS und $3 \promille$ }			
	\end{axis}
\end{tikzpicture}
\caption{Entwicklung des Qualitätswertes des Ausgagnssigals über 15 Iterationen, mit und ohne Anpassung des Korrekturterms}
	\label{subfig:opt.H.mockQuality}
\end{subfigure}
\caption[Optimierung von $\Hcompl$ im \mock-System]{Anwendung der Optimierung von $\Hcompl$ mit und ohne Anpassung des Korrekturterms auf das \mock-System bei $\sigma_H = \nicefrac{1}{2}$}
\end{figure}


Nutzt man die Optimierung für $\Hcompl$ im nahezu linearen Bereich der Kennlinie $K$, übertragen sich die prinzipiellen Erkenntnisse aus dem \mock-System  auf die Kavität und es lässt sich auch der Einfluss von Rauschen auf die Optimierung feststellen. Genutzt wurde hierzu und im Folgenden eine Kennlinie, die mit einer Peak-to-Peak-Spannung des Eingangssignals von $\SI{0.6}{\volt}$ erzeugt wurde. Das Ausgangssignal wurde in der Optimierung mit $V_{pp} = \SI{0.6}{\volt}$ angesetzt, was zu etwa $\SI{60}{\milli\volt}$ Peak-to-Peak-Spannung am Eingang zurückgerechnet wird. 
\footnote{Hier sei auf die in ------ beschriebene Problematik der nichtlinearen Vorverzerrung bei großen Unterschieden zwischen der Amplitude des momentan betrachteten Signals und des zur Berechnung von $K$ genutzten hingewiesen. Diese Erkenntnis lag zum Zeitpunkt der hier betrachteten Messung noch nicht in dieser Schärfe vor.}
Klar erkennbar sind wieder die Problematik der Nulldurchgänge und die Auflösung hoher Frequenzen.
Einen Überblick über die Variation der Parameter der Optimierung bietet \figref{fig:opt.H.parameter}. 


%In \figref{fig:opt.H.iteration} ist die Entwicklung von Übertragungsfunktion und $\fabs$ über mehrere Iterationen aufgetragen. Der Einfluss des RMS-Cutters macht sich dabei verglichen mit ---- ABB oben, rein iteriert --- bemerkbar, es treten weniger starke Änderungen auf. 
%Gleichzeitig zeigt sich, dass nicht in jedem Schritt an exakt den gleichen Stellen am jeweiligen RMS geschnitten werden muss. Dies belegt die Zufälligkeit des Fehlers und erläutert die prinzipielle Berechtigung der Methodik.
%Es zeigt sich, dass die vorgenommene Anpassung keinen großartigen Einfluss auf die Qualität des Signals hat.
%Dies ist insofern beachtenswert, als dass die Übertragungsfunktion auch an einigen Stellen mit massiver Verstärkung stark angepasst wird, vergleiche hierzu \figref{fig:opt.H.RMS_H} bei etwa $\SI{25}{\MHz}$.
%Die Qualität des Signals bewegt sich zwischen einem Wert von ---- ---- und ---- ---- und zeigt vor allem rauschbedingte Schwankungen.
%%TODO: check!!! Mehr Durchläufe ausprobieren, unbedingt!



\pgfplotstableread[col sep = comma] {H_iteration/H_0.csv} \Hstart 
\pgfplotstableread[col sep = comma] {H_iteration/H_3prom.csv} \Hprom
\pgfplotstableread[col sep = comma] {H_iteration/H_RMS.csv} \Hrms 
\pgfplotstableread[col sep = comma] {H_iteration/H_RMS_3prom.csv} \Hrmsprom 
\pgfplotstableread[col sep = comma] {H_iteration/H_sigma0.2.csv} \Hsigma
\pgfplotstableread[col sep = comma] {H_iteration/H_simple1.csv} \HsimpleA %no number in name
\pgfplotstableread[col sep = comma] {H_iteration/H_simple2.csv} \HsimpleB

\begin{figure}[htb]
\begin{center}
	\begin{subfigure}{\textwidth}
\begin{tikzpicture}
	\begin{axis}[
		scale only axis,
		width = 0.4 \textwidth,
		height = 0.2 \textheight,		
		%legend pos = outer north east,
		legend style={ at={(1.5,0.2)},
				anchor=south},	
		xlabel={Iterationsschritt},
		ylabel={Güte Ausgangssignal},
		xtick pos = lower,
		ytick pos = left,
		xtick align = outside,
		ytick align = outside,
		xmin = 0.2,
		xmax = 5.8,
		]	
		
	\addplot [blue, mark = x] table [col sep = comma, x expr=\coordindex+1, y index =1]{			
			ohne alles, an System
			QGesamt1,2.141630873256258
			QGesamt1,2.064664590979294
			QGesamt1,2.165397345084028
			QGesamt1,1.9985538231494953
			QGesamt1,2.031694877231738
		}; \label{plot:opt.H.simple_adjustA} \addlegendentry{Ohne Anpassungen Messung 1}	
	\addplot [blue, mark = +] table [col sep = comma, x expr=\coordindex+1, y index =1]{			
			ohne alles für Rauscheinfluss, data
			QGesamt1,2.135948818933304
			QGesamt1,2.1384178165733116
			QGesamt1,2.0464488222784887
			QGesamt1,1.9973886067241382
			QGesamt1,2.0689413165209682
		}; \label{plot:opt.H.simple_adjustB} \addlegendentry{Ohne Anpassungen Messung 2}	
	\addplot [black, mark = *] table [col sep = comma, x expr=\coordindex+1, y index =1]{			simple mit sigma 0.2, data
			QGesamt1,2.0536759283401667
			QGesamt1,2.025676659927534
			QGesamt1,2.035313029109188
		}; \addlegendentry{mit $\sigma_H = 0.2$}	
	\addplot [red, mark = x] table [col sep = comma, x expr=\coordindex+1, y index =1]{			
			mit RMS, data
			QGesamt1,2.0868523560789374
			QGesamt1,2.062575354725346
			QGesamt1,1.9691672211272198
			QGesamt1,2.0299124251901866
			QGesamt1,2.0676059254911445
		}; \addlegendentry{RMS}	
	\addplot [green, mark = x] table [col sep = comma, x expr=\coordindex+1, y index =1]{			
			mit 3 prom, data
			QGesamt1,2.2592583084704576
			QGesamt1,2.1266302971337128
			QGesamt1,2.0822550911677356
			QGesamt1,2.0910260802802365
			QGesamt1,2.0850568732052337
		}; \addlegendentry{ $3 \promille$ }	
	\addplot [yellow, mark = x] table [col sep = comma, x expr=\coordindex+1, y index =1]{			
			mit 3 prom und RMS, data
			QGesamt1,2.09716411420358
			QGesamt1,2.0352919493881836
			QGesamt1,2.0714614121252213
			QGesamt1,1.9686861551413428
			QGesamt1,2.120204783654906
		}; \addlegendentry{ RMS und $3 \promille$ }	
	
	\end{axis}
\end{tikzpicture}
	\caption{Entwicklung der Qualität des Ausgangssignals bei unterschiedlicher Parameterwahl}
	\label{subfig:opt.H.qualityOverview}
	\end{subfigure}
	\\
	\begin{subfigure}{\textwidth}
    \begin{tikzpicture}
\begin{axis}[
%		scale only axis,
%		width = 0.4 \textwidth,
%		height = 0.2 \textheight,		
%		legend style={ at={(1,1.1)},
%				anchor=south},
		legend columns = -1,
%		transpose legend, 
		legend entries={initial,
						ohne Anpassung, 
%						ohne Anpassung 2,
%						$\sigma_H = 0.2$ in Step 3,
						RMS,
						$3\promille$,
%						RMS und $3\promille$
						},
		legend to name=named,
		cycle list name = color list,
		xlabel={Frequenz},
		ylabel={Betrag Übertragung $\Hcompl$},
		y label style={at={(axis description cs:-0.1,.5)},anchor=south},
		xtick distance = 5000000,
		xtick pos = lower,
		ytick pos = left,
		xtick align = outside,
		ytick align = outside,
		scaled x ticks = base 10:-6,
		xtick scale label code/.code={[MHz]},
		xmin = 0,
		xmax = 20000000,
		ymin = 5,
		ymax = 13,
		]	
		\addplot [black] table [x index = 0, y index = 1] {\Hstart};
%		\addplot  table [x index = 0, y index = 1] {\HsimpleA};
		\addplot [blue] table [x index = 0, y index = 1] {\HsimpleB};
%		\addplot  table [x index = 0, y index = 1] {\Hsigma};
		\addplot [red] table [x index = 0, y index = 1] {\Hrms};
		\addplot [green] table [x index = 0, y index = 1] {\Hprom};
%		\addplot  table [x index = 0, y index = 1] {\Hrmsprom};
\end{axis}
\end{tikzpicture}
    \begin{tikzpicture}
\begin{axis}[
%		scale only axis,
%		width = 0.35 \textwidth,
%		height = 0.2 \textheight,		
		cycle list name = color list,
		xlabel={Frequenz},
		ylabel={Betrag Übertragung $\Hcompl$},
		y label style={at={(axis description cs:-0.1,.5)},anchor=south},
		xtick distance = 5000000,
		xtick pos = lower,
		ytick pos = left,
		xtick align = outside,
		ytick align = outside,
		scaled x ticks = base 10:-6,
		xtick scale label code/.code={[MHz]},
		xmin = 60000000,
		xmax = 80000000,
		ymin = 0,
		ymax = 5,
		]	
		\addplot [black] table [x index = 0, y index = 1] {\Hstart};
%		\addplot  table [x index = 0, y index = 1] {\HsimpleA};
		\addplot [blue] table [x index = 0, y index = 1] {\HsimpleB};
%		\addplot  table [x index = 0, y index = 1] {\Hsigma};
		\addplot [red] table [x index = 0, y index = 1] {\Hrms};
		\addplot [green] table [x index = 0, y index = 1] {\Hprom};
%		\addplot  table [x index = 0, y index = 1] {\Hrmsprom};
\end{axis}
\end{tikzpicture}

\ref{named}
	\caption{Einfluss unterschiedlicher Parameterwahl auf Entwicklung der Übertragungsfunktion bei hohen Frequenzen, nach 5 Iterationen}
	\label{subfig:opt.H.paramHoverview}
	\end{subfigure}
\caption{Einfluss unterschiedlicher Parameterwahl auf Entwicklung der Qualität des Ausgangssignals und die Übertragungsfunktion}
\label{fig:opt.H.parameter}
\end{center}
\end{figure}


Klar erkennbar ist aus \figref{subfig:opt.H.qualityOverview} vor allem, dass der Einfluss zufälliger Schwankungen auf die Qualität des Signals enorm ist, man vergleiche hierzu nur die unter identischen Voraussetzungen und um wenige Minuten verzögert entstandenen Messungen mit \ref{plot:opt.H.simple_adjustA} und \ref{plot:opt.H.simple_adjustB}. 
Weiterhin lässt sich im Vergleich mit  \figref{subfig:opt.H.mockQuality} die wenig verwunderliche Aussage treffen, dass die Qualität aller Ausgangssignale wesentlich schlechter ist, als dies beim \mock-System der Fall war. 
Diese nicht eindeutige Änderung der Qualität ist insofern verwunderlich, als dass auch die starke Anpassung über die Iteration bei in $\Hcompl$ stark verstärkten niedrigen Frequenzen nicht nennenswert auf die Qualität auswirkt.
Und letztlich ist über die hier vorgenommenen Iterationsschritte keine eindeutige Verbesserung des Ausgangssignals zu erkennen. Da gleichzeitig jedoch auch keine enorme Verschlechterung eintritt und der Rauscheinfluss nicht beziffert werden kann, ist eine qualitative Aussage über die einzelne Optimierung von $\Hcompl$ mit den vorliegenden Daten für die vorgenommenen Anpassungen nicht zu treffen.
Festhalten jedoch lässt sich auch bei Messungen am Messaufbau, dass die vorgenommenen Anpassungen und insbesondere die Beschränkung mit einem modifizierten RMS wesentlich weniger als abwegig angesehene Korrekturterme erlauben und so eine kleinschrittigere Anpassung möglich machen. 


\newpage
\newpage
\section{Optimierung der nichtlinearen Kennlinie $K$}
\label{sec:opt.K}
Der Unterschied zur Optimierung von $\Hcompl$ ist, dass diese Optimierung im Zeitbereich statt findet. Deshalb kann \eqref{eq:opt.relFehler} zu
\begin{align}
	\label{eq:opt.deltaUquest}
	\Delta \Uquest \oft = \Uquest_{, \mathrm{meas}} \oft - \Uquest_{, \mathrm{ideal}} \oft
	\qquad
	\Uquest_{, \mathrm{meas}} \oft = \mathscr{F}^{-1} \left\{ \Hcompl^{-1} \ofomega \cdot \Uoutc_{, \mathrm{meas}} \ofomega \right\}
\end{align}
geändert werden. Bei den Funktionen $\Uquest_{, \mathrm{meas}} \oft$ und $\Uquest_{, \mathrm{ideal}} \oft$ handelt es sich um Polynome gleichen Grades deshalb lässt sich die Differenz ebenfalls als ein Polynom mit Grad $N$ darstellen
\begin{align}
	\Delta \Uquest \oft = \sum_{n=1}^N \, \tilde{a}_n \, \left[ U_{in}(t) \right]^n	
\end{align}
Die Berechnung der Koeffizienten $\tilde{a}_n$ stellt ebenso ein lineares Optimierungsproblem dar wie schon die Berechnung der Koeffizienten $a_n$ in \eqref{eq:Uquest} siehe \cite{harzheim}. Dabei werden $M$ Samples von ${\Delta \Uquest_{,i} = \Delta \Uquest (i \cdot \Delta t)}$ mit zugehörigen Samples des Eingangssignals ${\Uin_{,i} = \Uin (i \cdot \Delta t)}$ verglichen. Mit der Potenzreihe aus  \eqref{eq:opt.deltaUquest} ergibt sich folgendes Gleichungssystem
\begin{align}
	\left( 
	\begin{matrix}
	 	\Uin_{,1} & \Uin_{,1}^2 & \dots & \Uin_{,1}^N \\
		\Uin_{,2} & \Uin_{,2}^2 & \dots & \Uin_{,2}^N \\
		\vdots & \vdots & \ddots & \vdots \\
		\Uin_{,M} & \Uin_{,M}^2 & \dots & \Uin_{,M}^N \\
	\end{matrix}
	\right)
	\cdot
	\left(
	\begin{matrix}
		\tilde{a}_1 \\
		\tilde{a}_2 \\
		\vdots \\
		\tilde{a}_N \\	 
	\end{matrix}
	\right) = \left( 
	\begin{matrix}
		\Delta \Uquest_{,1} \\
		\Delta \Uquest_{,2} \\
		\vdots \\
		\Delta \Uquest_{,M} \\	 
	\end{matrix}
	\right)
	\label{eq:Uquest.Gleichungssystem}
\end{align}
Dieses Gleichungssystem ist mit normalerweise $M>N$ überbestimmt und wird mit der Methode der kleinsten Quadrate gelöst. Die Koeffizienten $\tilde{a}_n$ werden nun wie folgt zur Anpassung der Koeffizienten $a_n$ verwendet
\begin{align}
	\label{eq:opt.adjusta}
	a_n^{i+1} = a_n^{i} + \sigma_{a}^{i} \tilde{a}_n^{i}
\end{align}
Für die Schrittweite gilt $\sigma_{a}^i \in \left[ 0 , 1 \right]$.

\subsubsection*{Erste Ergebnisse}
\label{subsubsec:opt.adjusta.results}
Für die Berechnung der ersten Kennlinie $K_0$ wurde das ideal Ausgangssignal $\Uout_{, \mathrm{ideal}}$ über $\Hcompl^{-1}$ zurückgerechnet und als Eingangssignal verwendet $\Uin_{, \mathrm{initial }} = \Uquest_{, \mathrm{ideal}}$. Dabei wurde $V_{PP} = \SI{600}{\mV}$ gesetzt, um $K_0$ in einen größeren Bereich berechnen zu können.\\
Wenn man jetzt $\Uquest_{, \mathrm{ideal}}$ mit $V_{PP} = \SI{600}{\mV}$ über $K_0$ zurückrechnet, um das erste nichtlinear vorverzerrte Eingangssignal zu erhalten, so stellt man fest, dass die Grenzen, in denen $K_0$ invertiert werden kann, zu klein sind. Wenn also $\Uquest_{, \mathrm{ideal}}$ über die Grenzen von $K_0$ geht, so wäre ein möglicher Ansatz $V_{PP}$ auf den maximal von $K_0$ zulässigen Wert zu setzen.\\
Als andere Möglichkeit die Kennlinie anzupassen könnte man $V_{PP}$ von $\Uquest_{, \mathrm{ideal}}$ verkleinern siehe \figref{fig:K}
\begin{figure}[H]
\begin{subfigure}{0.5 \textwidth}
    \newlength\figureheight
	\newlength\figurewidth
	\setlength\figureheight{8cm}
	\setlength\figurewidth{8cm}
    % This file was created by matplotlib2tikz v0.6.17.
\begin{tikzpicture}

\definecolor{color0}{rgb}{0.12156862745098,0.466666666666667,0.705882352941177}
\definecolor{color1}{rgb}{1,0.498039215686275,0.0549019607843137}
\definecolor{color2}{rgb}{0.172549019607843,0.627450980392157,0.172549019607843}

\begin{axis}[
xlabel={$U_{in}$ in \si{\milli \volt}},
ylabel={$U_{?}$ in \si{\milli \volt}},
xmin=-330, xmax=330,
ymin=-703.9464559176, ymax=610.6443744396,
width=\figurewidth,
height=\figureheight,
tick align=outside,
tick pos=left,
x grid style={white!69.01960784313725!black},
y grid style={white!69.01960784313725!black},
legend style={at={(0.03,0.97)}, anchor=north west, draw=white!80.0!black},
legend cell align={left},
legend entries={{$K_0$},{$K_1$},{$K_2$}},
extra y ticks={87.1993445092, -71.9335154435},
extra y tick labels={\tiny{$\max(U_?)$}, \tiny{$\min(U_?)$}},
extra y tick style={grid=major, ytick pos=left, ytick align=outside, ticklabel pos=left},
]
\addlegendimage{no markers, color0}
\addlegendimage{no markers, color1}
\addlegendimage{no markers, color2}
\addplot [semithick, color0]
table {%
-300 -355.073268737
-298 -355.737870278
-296 -356.340552425
-294 -356.881745408
-292 -357.361879456
-290 -357.781384799
-288 -358.140691667
-286 -358.44023029
-284 -358.680430898
-282 -358.861723721
-280 -358.984538988
-278 -359.04930693
-276 -359.056457777
-274 -359.006421757
-272 -358.899629102
-270 -358.736510041
-268 -358.517494804
-266 -358.24301362
-264 -357.913496721
-262 -357.529374334
-260 -357.091076691
-258 -356.599034022
-256 -356.053676556
-254 -355.455434522
-252 -354.804738152
-250 -354.102017675
-248 -353.34770332
-246 -352.542225318
-244 -351.686013898
-242 -350.779499291
-240 -349.823111726
-238 -348.817281433
-236 -347.762438642
-234 -346.659013583
-232 -345.507436485
-230 -344.308137579
-228 -343.061547095
-226 -341.768095262
-224 -340.42821231
-222 -339.04232847
-220 -337.61087397
-218 -336.134279042
-216 -334.612973914
-214 -333.047388817
-212 -331.43795398
-210 -329.785099634
-208 -328.089256008
-206 -326.350853332
-204 -324.570321836
-202 -322.748091751
-200 -320.884593305
-198 -318.980256728
-196 -317.035512251
-194 -315.050790104
-192 -313.026520516
-190 -310.963133717
-188 -308.861059938
-186 -306.720729407
-184 -304.542572355
-182 -302.327019012
-180 -300.074499607
-178 -297.785444371
-176 -295.460283534
-174 -293.099447324
-172 -290.703365973
-170 -288.27246971
-168 -285.807188764
-166 -283.307953366
-164 -280.775193746
-162 -278.209340134
-160 -275.610822758
-158 -272.980071851
-156 -270.31751764
-154 -267.623590356
-152 -264.898720229
-150 -262.143337489
-148 -259.357872366
-146 -256.542755089
-144 -253.698415889
-142 -250.825284994
-140 -247.923792637
-138 -244.994369045
-136 -242.037444449
-134 -239.053449079
-132 -236.042813164
-130 -233.005966936
-128 -229.943340622
-126 -226.855364454
-124 -223.742468662
-122 -220.605083474
-120 -217.443639121
-118 -214.258565833
-116 -211.05029384
-114 -207.819253372
-112 -204.565874658
-110 -201.290587928
-108 -197.993823413
-106 -194.676011342
-104 -191.337581944
-102 -187.978965451
-100 -184.600592091
-98 -181.202892096
-96 -177.786295693
-94 -174.351233114
-92 -170.898134588
-90 -167.427430346
-88 -163.939550616
-86 -160.43492563
-84 -156.913985616
-82 -153.377160805
-80 -149.824881427
-78 -146.25757771
-76 -142.675679887
-74 -139.079618185
-72 -135.469822836
-70 -131.846724068
-68 -128.210752112
-66 -124.562337199
-64 -120.901909556
-62 -117.229899415
-60 -113.546737006
-58 -109.852852557
-56 -106.1486763
-54 -102.434638464
-52 -98.7111692787
-50 -94.9786989741
-48 -91.2376577802
-46 -87.4884759267
-44 -83.7315836437
-42 -79.9674111609
-40 -76.1963887082
-38 -72.4189465156
-36 -68.6355148129
-34 -64.84652383
-32 -61.0524037967
-30 -57.253584943
-28 -53.4504974987
-26 -49.6435716937
-24 -45.8332377578
-22 -42.0199259211
-20 -38.2040664132
-18 -34.3860894642
-16 -30.5664253039
-14 -26.7455041622
-12 -22.9237562689
-10 -19.101611854
-8 -15.2795011472
-6 -11.4578543786
-4 -7.63710177791
-2 -3.81767357509
0 0
2 3.81548871748
4 7.62836234747
6 11.4381906601
8 15.2445434255
10 19.0469904137
12 22.845101395
14 26.6384461393
16 30.4265944169
18 34.2091159979
20 37.9855806523
22 41.7555581503
24 45.5186182621
26 49.2743307576
28 53.0222654072
30 56.7619919808
32 60.4930802486
34 64.2150999808
36 67.9276209474
38 71.6302129186
40 75.3224456644
42 79.0038889551
44 82.6741125606
46 86.3326862513
48 89.9791797971
50 93.6131629681
52 97.2342055346
54 100.841877267
56 104.435747934
58 108.015387308
60 111.580365157
62 115.130251252
64 118.664615364
66 122.183027262
68 125.685056716
70 129.170273496
72 132.638247374
74 136.088548118
76 139.520745499
78 142.934409286
80 146.329109251
82 149.704415163
84 153.059896793
86 156.39512391
88 159.709666284
90 163.003093687
92 166.274975887
94 169.524882655
96 172.752383761
98 175.957048975
100 179.138448068
102 182.296150809
104 185.429726968
106 188.538746316
108 191.622778623
110 194.681393659
112 197.714161194
114 200.720650998
116 203.700432842
118 206.653076494
120 209.578151727
122 212.475228309
124 215.34387601
126 218.183664602
128 220.994163854
130 223.774943535
132 226.525573417
134 229.245623269
136 231.934662862
138 234.592261966
140 237.21799035
142 239.811417785
144 242.372114041
146 244.899648888
148 247.393592096
150 249.853513435
152 252.278982676
154 254.669569589
156 257.024843943
158 259.344375509
160 261.627734057
162 263.874489357
164 266.08421118
166 268.256469294
168 270.390833471
170 272.48687348
172 274.544159093
174 276.562260077
176 278.540746205
178 280.479187246
180 282.37715297
182 284.234213147
184 286.049937548
186 287.823895942
188 289.5556581
190 291.244793791
192 292.890872787
194 294.493464856
196 296.052139769
198 297.566467297
200 299.036017209
202 300.460359275
204 301.839063266
206 303.171698952
208 304.457836103
210 305.697044488
212 306.888893879
214 308.032954045
216 309.128794756
218 310.175985783
220 311.174096895
222 312.122697862
224 313.021358456
226 313.869648446
228 314.667137601
230 315.413395693
232 316.107992491
234 316.750497765
236 317.340481286
238 317.877512824
240 318.361162148
242 318.790999029
244 319.166593237
246 319.487514543
248 319.753332715
250 319.963617525
252 320.117938743
254 320.215866138
256 320.256969481
258 320.240818541
260 320.16698309
262 320.035032897
264 319.844537731
266 319.595067365
268 319.286191566
270 318.917480107
272 318.488502756
274 317.998829283
276 317.44802946
278 316.835673056
280 316.161329841
282 315.424569585
284 314.624962059
286 313.762077032
288 312.835484275
290 311.844753558
292 310.78945465
294 309.669157323
296 308.483431346
298 307.231846489
300 305.913972522
};
\addplot [semithick, color1]
table {%
-300 -644.192327265
-298 -639.238926493
-296 -634.296692915
-294 -629.365579415
-292 -624.445538876
-290 -619.53652418
-288 -614.63848821
-286 -609.751383851
-284 -604.875163983
-282 -600.009781491
-280 -595.155189258
-278 -590.311340166
-276 -585.478187099
-274 -580.655682938
-272 -575.843780569
-270 -571.042432872
-268 -566.251592732
-266 -561.471213032
-264 -556.701246654
-262 -551.941646481
-260 -547.192365396
-258 -542.453356283
-256 -537.724572024
-254 -533.005965502
-252 -528.2974896
-250 -523.599097202
-248 -518.91074119
-246 -514.232374447
-244 -509.563949856
-242 -504.9054203
-240 -500.256738663
-238 -495.617857827
-236 -490.988730674
-234 -486.369310089
-232 -481.759548954
-230 -477.159400152
-228 -472.568816566
-226 -467.987751079
-224 -463.416156574
-222 -458.853985934
-220 -454.301192042
-218 -449.757727781
-216 -445.223546034
-214 -440.698599684
-212 -436.182841614
-210 -431.676224707
-208 -427.178701845
-206 -422.690225913
-204 -418.210749792
-202 -413.740226366
-200 -409.278608518
-198 -404.825849131
-196 -400.381901087
-194 -395.946717271
-192 -391.520250564
-190 -387.102453849
-188 -382.693280011
-186 -378.292681931
-184 -373.900612493
-182 -369.51702458
-180 -365.141871074
-178 -360.775104859
-176 -356.416678817
-174 -352.066545832
-172 -347.724658787
-170 -343.390970564
-168 -339.065434047
-166 -334.748002118
-164 -330.438627661
-162 -326.137263559
-160 -321.843862693
-158 -317.558377949
-156 -313.280762208
-154 -309.010968353
-152 -304.748949268
-150 -300.494657835
-148 -296.248046937
-146 -292.009069458
-144 -287.77767828
-142 -283.553826287
-140 -279.337466361
-138 -275.128551385
-136 -270.927034242
-134 -266.732867816
-132 -262.546004989
-130 -258.366398645
-128 -254.194001665
-126 -250.028766934
-124 -245.870647334
-122 -241.719595748
-120 -237.575565059
-118 -233.438508151
-116 -229.308377905
-114 -225.185127206
-112 -221.068708936
-110 -216.959075978
-108 -212.856181215
-106 -208.75997753
-104 -204.670417806
-102 -200.587454926
-100 -196.511041773
-98 -192.44113123
-96 -188.37767618
-94 -184.320629506
-92 -180.269944091
-90 -176.225572818
-88 -172.18746857
-86 -168.155584229
-84 -164.12987268
-82 -160.110286804
-80 -156.096779486
-78 -152.089303607
-76 -148.087812051
-74 -144.0922577
-72 -140.102593439
-70 -136.118772149
-68 -132.140746714
-66 -128.168470017
-64 -124.20189494
-62 -120.240974367
-60 -116.285661181
-58 -112.335908264
-56 -108.3916685
-54 -104.452894772
-52 -100.519539962
-50 -96.5915569544
-48 -92.6688986309
-46 -88.751517875
-44 -84.8393675696
-42 -80.9324005978
-40 -77.0305698425
-38 -73.1338281867
-36 -69.2421285133
-34 -65.3554237055
-32 -61.4736666461
-30 -57.5968102181
-28 -53.7248073046
-26 -49.8576107884
-24 -45.9951735527
-22 -42.1374484803
-20 -38.2843884543
-18 -34.4359463576
-16 -30.5920750732
-14 -26.7527274842
-12 -22.9178564734
-10 -19.0874149239
-8 -15.2613557187
-6 -11.4396317407
-4 -7.62219587292
-2 -3.80900099836
0 0
2 3.80485423919
4 7.60560883621
6 11.4023109081
8 15.1950075719
10 18.9837459445
12 22.768573143
14 26.5495362845
16 30.3266824859
18 34.1000588643
20 37.8697125366
22 41.6356906199
24 45.3980402312
26 49.1568084876
28 52.9120425059
30 56.6637894033
32 60.4120962968
34 64.1570103034
36 67.8985785401
38 71.6368481238
40 75.3718661718
42 79.1036798008
44 82.832336128
46 86.5578822705
48 90.2803653451
50 93.9998324689
52 97.7163307589
54 101.429907332
56 105.140609306
58 108.848483797
60 112.553577922
62 116.255938798
64 119.955613543
66 123.652649273
68 127.347093105
70 131.038992157
72 134.728393546
74 138.415344387
76 142.099891799
78 145.782082899
80 149.461964803
82 153.139584628
84 156.814989492
86 160.488226511
88 164.159342803
90 167.828385485
92 171.495401673
94 175.160438485
96 178.823543037
98 182.484762447
100 186.144143831
102 189.801734307
104 193.457580992
106 197.111731002
108 200.764231455
110 204.415129468
112 208.064472157
114 211.712306641
116 215.358680035
118 219.003639456
120 222.647232023
122 226.289504851
124 229.930505058
126 233.570279761
128 237.208876077
130 240.846341123
132 244.482722015
134 248.118065872
136 251.752419809
138 255.385830944
140 259.018346394
142 262.650013277
144 266.280878708
146 269.910989805
148 273.540393685
150 277.169137465
152 280.797268262
154 284.424833194
156 288.051879376
158 291.678453926
160 295.304603962
162 298.9303766
164 302.555818956
166 306.180978149
168 309.805901295
170 313.430635512
172 317.055227915
174 320.679725623
176 324.304175752
178 327.928625419
180 331.553121742
182 335.177711837
184 338.802442821
186 342.427361811
188 346.052515925
190 349.677952279
192 353.30371799
194 356.929860176
196 360.556425953
198 364.183462439
200 367.81101675
202 371.439136004
204 375.067867317
206 378.697257806
208 382.327354589
210 385.958204782
212 389.589855503
214 393.222353869
216 396.855746996
218 400.490082002
220 404.125406003
222 407.761766117
224 411.39920946
226 415.037783151
228 418.677534304
230 422.318510039
232 425.960757471
234 429.604323718
236 433.249255897
238 436.895601124
240 440.543406517
242 444.192719193
244 447.843586269
246 451.496054861
248 455.150172087
250 458.805985064
252 462.463540909
254 466.122886739
256 469.784069671
258 473.447136822
260 477.112135308
262 480.779112248
264 484.448114757
266 488.119189954
268 491.792384954
270 495.467746875
272 499.145322835
274 502.825159949
276 506.507305336
278 510.191806111
280 513.878709393
282 517.568062298
284 521.259911942
286 524.954305444
288 528.65128992
290 532.350912488
292 536.053220263
294 539.758260364
296 543.466079907
298 547.176726009
300 550.890245787
};
\addplot [semithick, color2]
table {%
-300 -304.896190826
-298 -306.750679421
-296 -308.525735418
-294 -310.221911313
-292 -311.839759597
-290 -313.379832766
-288 -314.842683312
-286 -316.22886373
-284 -317.538926513
-282 -318.773424155
-280 -319.932909149
-278 -321.01793399
-276 -322.029051171
-274 -322.966813186
-272 -323.831772529
-270 -324.624481692
-268 -325.345493171
-266 -325.995359459
-264 -326.574633049
-262 -327.083866435
-260 -327.523612111
-258 -327.894422571
-256 -328.196850309
-254 -328.431447817
-252 -328.598767591
-250 -328.699362122
-248 -328.733783907
-246 -328.702585437
-244 -328.606319207
-242 -328.445537711
-240 -328.220793442
-238 -327.932638894
-236 -327.581626561
-234 -327.168308936
-232 -326.693238513
-230 -326.156967786
-228 -325.560049249
-226 -324.903035395
-224 -324.186478719
-222 -323.410931713
-220 -322.576946872
-218 -321.685076689
-216 -320.735873658
-214 -319.729890273
-212 -318.667679027
-210 -317.549792414
-208 -316.376782929
-206 -315.149203064
-204 -313.867605313
-202 -312.532542171
-200 -311.144566131
-198 -309.704229686
-196 -308.21208533
-194 -306.668685558
-192 -305.074582862
-190 -303.430329737
-188 -301.736478676
-186 -299.993582173
-184 -298.202192722
-182 -296.362862816
-180 -294.476144949
-178 -292.542591615
-176 -290.562755308
-174 -288.537188522
-172 -286.466443749
-170 -284.351073484
-168 -282.191630221
-166 -279.988666453
-164 -277.742734674
-162 -275.454387377
-160 -273.124177058
-158 -270.752656208
-156 -268.340377322
-154 -265.887892894
-152 -263.395755418
-150 -260.864517386
-148 -258.294731294
-146 -255.686949634
-144 -253.0417249
-142 -250.359609586
-140 -247.641156186
-138 -244.886917194
-136 -242.097445103
-134 -239.273292407
-132 -236.415011599
-130 -233.523155174
-128 -230.598275625
-126 -227.640925446
-124 -224.651657131
-122 -221.631023173
-120 -218.579576066
-118 -215.497868303
-116 -212.38645238
-114 -209.245880788
-112 -206.076706023
-110 -202.879480577
-108 -199.654756945
-106 -196.40308762
-104 -193.125025096
-102 -189.821121867
-100 -186.491930426
-98 -183.138003267
-96 -179.759892884
-94 -176.35815177
-92 -172.93333242
-90 -169.485987327
-88 -166.016668984
-86 -162.525929886
-84 -159.014322527
-82 -155.482399399
-80 -151.930712997
-78 -148.359815814
-76 -144.770260345
-74 -141.162599082
-72 -137.53738452
-70 -133.895169152
-68 -130.236505473
-66 -126.561945975
-64 -122.872043153
-62 -119.1673495
-60 -115.44841751
-58 -111.715799676
-56 -107.970048493
-54 -104.211716455
-52 -100.441356054
-50 -96.6595197848
-48 -92.866760141
-46 -89.0636296163
-44 -85.2506807045
-42 -81.4284658992
-40 -77.5975376942
-38 -73.7584485832
-36 -69.91175106
-34 -66.0579976183
-32 -62.1977407517
-30 -58.3315329541
-28 -54.4599267192
-26 -50.5834745407
-24 -46.7027289123
-22 -42.8182423278
-20 -38.9305672809
-18 -35.0402562653
-16 -31.1478617747
-14 -27.253936303
-12 -23.3590323437
-10 -19.4637023907
-8 -15.5684989377
-6 -11.6739744784
-4 -7.78068150653
-2 -3.88917251582
0 0
2 3.88628354719
4 7.76912563202
6 11.6479737608
8 15.5222754397
10 19.3914781751
12 23.2550294732
14 27.1123768402
16 30.9629677826
18 34.8062498065
20 38.6416704181
22 42.4686771239
24 46.28671743
26 50.0952388427
28 53.8936888682
30 57.681515013
32 61.4581647831
34 65.2230856849
36 68.9757252247
38 72.7155309087
40 76.4419502432
42 80.1544307345
44 83.8524198888
46 87.5353652124
48 91.2027142116
50 94.8539143926
52 98.4884132618
54 102.105658325
56 105.705097089
58 109.286177061
60 112.848345745
62 116.391050649
64 119.913739278
66 123.41585914
68 126.896857739
70 130.356182584
72 133.793281179
74 137.207601031
76 140.598589647
78 143.965694532
80 147.308363193
82 150.626043136
84 153.918181868
86 157.184226894
88 160.423625721
90 163.635825856
92 166.820274804
94 169.976420072
96 173.103709166
98 176.201589592
100 179.269508857
102 182.306914467
104 185.313253927
106 188.287974746
108 191.230524427
110 194.140350479
112 197.016900407
114 199.859621718
116 202.667961917
118 205.441368511
120 208.179289007
122 210.88117091
124 213.546461726
126 216.174608963
128 218.765060127
130 221.317262723
132 223.830664258
134 226.304712238
136 228.738854169
138 231.132537558
140 233.485209912
142 235.796318735
144 238.065311535
146 240.291635818
148 242.474739089
150 244.614068857
152 246.709072625
154 248.759197902
156 250.763892193
158 252.722603004
160 254.634777842
162 256.499864212
164 258.317309622
166 260.086561578
168 261.807067585
170 263.47827515
172 265.09963178
174 266.67058498
176 268.190582257
178 269.659071117
180 271.075499066
182 272.439313612
184 273.749962259
186 275.006892514
188 276.209551883
190 277.357387874
192 278.449847991
194 279.486379742
196 280.466430632
198 281.389448168
200 282.254879856
202 283.062173202
204 283.810775713
206 284.500134895
208 285.129698254
210 285.698913296
212 286.207227528
214 286.654088456
216 287.038943587
218 287.361240425
220 287.620426479
222 287.815949254
224 287.947256255
226 288.013794991
228 288.015012966
230 287.950357688
232 287.819276662
234 287.621217394
236 287.355627391
238 287.02195416
240 286.619645206
242 286.148148036
244 285.606910156
246 284.995379072
248 284.31300229
250 283.559227318
252 282.73350166
254 281.835272824
256 280.863988316
258 279.819095641
260 278.700042307
262 277.506275819
264 276.237243683
266 274.892393407
268 273.471172496
270 271.973028456
272 270.397408794
274 268.743761017
276 267.011532629
278 265.200171138
280 263.30912405
282 261.337838872
284 259.285763108
286 257.152344266
288 254.937029852
290 252.639267373
292 250.258504334
294 247.794188241
296 245.245766602
298 242.612686922
300 239.894396707
};
\end{axis}

\end{tikzpicture}
\subcaption{Kennlinie}
	\label{fig:K}
\end{subfigure}
\begin{subfigure}{0.5 \textwidth}
	\setlength\figureheight{8cm}
	\setlength\figurewidth{8cm}
    % This file was created by matplotlib2tikz v0.6.17.
\begin{tikzpicture}

\definecolor{color0}{rgb}{0.12156862745098,0.466666666666667,0.705882352941177}
\definecolor{color1}{rgb}{1,0.498039215686275,0.0549019607843137}

\begin{axis}[
xlabel={$t$ in \si{\micro \second}},
ylabel={$U_{out}$ in \si{\milli \volt}},
xmin=-0.0555, xmax=1.1655,
ymin=-873.008455, ymax=867.887955,
width=\figurewidth,
height=\figureheight,
tick align=outside,
tick pos=left,
x grid style={white!69.01960784313725!black},
y grid style={white!69.01960784313725!black},
legend entries={{$U_{out,0}$},{$U_{out,1}$}},
legend cell align={left},
legend style={draw=white!80.0!black}
]
\addlegendimage{no markers, color0}
\addlegendimage{no markers, color1}
\addplot [semithick, color0]
table {%
0 19.0069
0.002 13.4269
0.004 8.1956
0.006 11.6831
0.008 1.9762
0.01 18.6581
0.012 11.7412
0.014 -0.93
0.016 -24.3544
0.018 -3.1969
0.02 1.2788
0.022 15.6938
0.024 17.8444
0.026 15.8681
0.028 7.3237
0.03 -0.11625
0.032 12.2644
0.034 1.3369
0.036 -9.2419
0.038 -1.4531
0.04 -3.6619
0.042 -4.5338
0.044 6.3938
0.046 -13.3106
0.048 -6.1031
0.05 -17.205
0.052 -10.5206
0.054 -16.1006
0.056 5.115
0.058 -2.1506
0.06 -3.0225
0.062 -2.9644
0.064 13.8338
0.066 15.4613
0.068 11.5087
0.07 15.5194
0.072 -21.4481
0.074 -15.6938
0.076 -0.465
0.078 -16.2169
0.08 1.7437
0.082 -2.4994
0.084 -12.4387
0.086 -3.0225
0.088 0.58125
0.09 0.34875
0.092 -7.2075
0.094 -13.3106
0.096 -0.81375
0.098 2.2087
0.1 16.7981
0.102 -0.755625
0.104 7.2656
0.106 16.1006
0.108 16.4494
0.11 14.88
0.112 14.9963
0.114 16.275
0.116 -6.7425
0.118 -4.4175
0.12 -0.81375
0.122 13.0781
0.124 0.058125
0.126 -15.9262
0.128 -2.2669
0.13 -4.3012
0.132 -1.6856
0.134 1.5113
0.136 3.1388
0.138 1.5113
0.14 8.9513
0.142 -0.058125
0.144 -12.0319
0.146 2.4994
0.148 11.7412
0.15 2.0925
0.152 2.7319
0.154 0.174375
0.156 0.406875
0.158 1.4531
0.16 23.4244
0.162 0.11625
0.164 -10.8694
0.166 14.1244
0.168 -4.2431
0.17 14.4731
0.172 -9.0094
0.174 16.8562
0.176 12.3806
0.178 -1.9181
0.18 -5.7544
0.182 -23.8313
0.184 11.5669
0.186 0.58125
0.188 -1.1044
0.19 16.4494
0.192 4.8244
0.194 26.8537
0.196 14.5894
0.198 0
0.2 -12.7294
0.202 16.9144
0.204 12.0319
0.206 -0.871875
0.208 -9.3
0.21 -25.5169
0.212 -7.3819
0.214 -15.6356
0.216 -6.8006
0.218 -1.395
0.22 1.1625
0.222 14.3569
0.224 -1.3369
0.226 0.465
0.228 3.1969
0.23 -0.290625
0.232 -6.4519
0.234 -1.6275
0.236 -9.0675
0.238 -2.3831
0.24 -36.27
0.242 -14.9381
0.244 14.7638
0.246 1.6856
0.248 0.11625
0.25 -19.065
0.252 2.0925
0.254 0.639375
0.256 -6.045
0.258 -10.0556
0.26 -4.4756
0.262 0.988125
0.264 -5.4056
0.266 2.9063
0.268 7.905
0.27 -1.6275
0.272 3.0806
0.274 -1.86
0.276 2.9644
0.278 -13.6013
0.28 1.2788
0.282 -11.7412
0.284 -9.6487
0.286 16.9725
0.288 -4.8825
0.29 13.02
0.292 8.6025
0.294 -4.3012
0.296 -3.9525
0.298 -15.5775
0.3 -2.9644
0.302 1.8019
0.304 7.905
0.306 -2.4994
0.308 13.5431
0.31 14.8219
0.312 0.11625
0.314 16.5656
0.316 3.0225
0.318 10.1719
0.32 -23.4244
0.322 4.5338
0.324 1.6856
0.326 1.2788
0.328 5.9288
0.33 8.9513
0.332 8.8931
0.334 0.755625
0.336 1.0462
0.338 1.5113
0.34 19.1812
0.342 -5.2894
0.344 -14.0663
0.346 -12.4387
0.348 12.8456
0.35 -11.9737
0.352 2.4413
0.354 14.3569
0.356 1.4531
0.358 11.4506
0.36 -6.5681
0.362 4.8244
0.364 0
0.366 -0.34875
0.368 -19.8788
0.37 3.9525
0.372 -14.6475
0.374 -14.6475
0.376 24.8194
0.378 -2.0925
0.38 10.7531
0.382 14.415
0.384 4.7663
0.386 -10.1137
0.388 26.6213
0.39 4.4175
0.392 34.9912
0.394 0.93
0.396 -18.8906
0.398 -12.8456
0.4 5.115
0.402 -40.5713
0.404 -2.0344
0.406 -2.0925
0.408 11.8575
0.41 -2.2669
0.412 -7.4981
0.414 15.5775
0.416 -2.9644
0.418 9.6487
0.42 -5.2894
0.422 -15.0544
0.424 -2.79
0.426 -16.3331
0.428 13.2525
0.43 -1.7437
0.432 -6.1031
0.434 -0.755625
0.436 -2.6738
0.438 11.625
0.44 -14.2406
0.442 -3.9525
0.444 -20.9831
0.446 -5.2313
0.448 -24.5287
0.45 -16.1006
0.452 3.0225
0.454 -16.8562
0.456 -19.7044
0.458 -13.8338
0.46 -1.4531
0.462 -0.58125
0.464 8.9513
0.466 -15.6938
0.468 2.9063
0.47 0.871875
0.472 20.8088
0.474 -0.81375
0.476 -13.3687
0.478 27.9581
0.48 25.8075
0.482 -11.4506
0.484 16.2169
0.486 1.9181
0.488 -4.1269
0.49 6.045
0.492 -16.4494
0.494 0.523125
0.496 1.395
0.498 -3.9525
0.5 -3.72
0.502 15.1125
0.504 -8.835
0.506 -11.2762
0.508 -5.9869
0.51 17.5538
0.512 52.08
0.514 98.6381
0.516 137.2913
0.518 189.6037
0.52 225.99
0.522 286.2656
0.524 351.54
0.526 394.1456
0.528 438.9019
0.53 457.5019
0.532 505.2225
0.534 533.5294
0.536 566.0794
0.538 626.7037
0.54 623.6231
0.542 663.9038
0.544 682.62
0.546 711.3919
0.548 719.5875
0.55 728.4806
0.552 777.0731
0.554 744.93
0.556 770.5631
0.558 779.1075
0.56 788.7563
0.562 776.1431
0.564 788.0588
0.566 774.6319
0.568 760.9144
0.57 744.8719
0.572 733.0144
0.574 686.8631
0.576 671.1694
0.578 663.8456
0.58 611.8819
0.582 585.0281
0.584 550.4438
0.586 549.2812
0.588 494.1787
0.59 428.9044
0.592 388.0425
0.594 360.3169
0.596 320.5012
0.598 240.7538
0.6 214.2487
0.602 149.4394
0.604 98.8125
0.606 45.105
0.608 25.2263
0.61 -15.7519
0.612 -54.6375
0.614 -112.1231
0.616 -162.8662
0.618 -248.7169
0.62 -254.6456
0.622 -306.3188
0.624 -348.4594
0.626 -387.6938
0.628 -440.7037
0.63 -452.6194
0.632 -483.0188
0.634 -543.8175
0.636 -590.6662
0.638 -595.9556
0.64 -621.5888
0.642 -680.9925
0.644 -711.1012
0.646 -740.0475
0.648 -755.625
0.65 -737.49
0.652 -775.6781
0.654 -768.5869
0.656 -784.5712
0.658 -785.2106
0.66 -774.9806
0.662 -760.9144
0.664 -762.8325
0.666 -773.4694
0.668 -780.5025
0.67 -729.585
0.672 -730.8056
0.674 -714.2981
0.676 -663.2063
0.678 -644.49
0.68 -625.7737
0.682 -599.6757
0.684 -562.4756
0.686 -533.2969
0.688 -481.4494
0.69 -419.4881
0.692 -385.3688
0.694 -343.5769
0.696 -312.3637
0.698 -255.6919
0.7 -187.1044
0.702 -122.0044
0.704 -45.3375
0.706 -14.0663
0.708 -0.93
0.71 -13.3687
0.712 -2.9644
0.714 2.2669
0.716 26.2725
0.718 -33.3637
0.72 24.4125
0.722 2.3831
0.724 19.1231
0.726 -14.7056
0.728 -12.3225
0.73 -9.9394
0.732 -8.1375
0.734 -10.0556
0.736 7.0331
0.738 -5.8706
0.74 -17.7281
0.742 -22.9594
0.744 -16.3912
0.746 -7.0331
0.748 3.1969
0.75 16.1588
0.752 26.7375
0.754 17.3212
0.756 6.5681
0.758 -13.1362
0.76 23.3081
0.762 13.1362
0.764 5.5219
0.766 14.5313
0.768 -21.9131
0.77 16.74
0.772 7.6725
0.774 17.1469
0.776 9.765
0.778 30.69
0.78 -6.4519
0.782 2.9063
0.784 -14.7056
0.786 2.1506
0.788 -3.3131
0.79 0
0.792 -16.0425
0.794 -14.9381
0.796 2.9644
0.798 15.345
0.8 26.8537
0.802 -14.9381
0.804 1.6275
0.806 -1.9762
0.808 -22.2619
0.81 -26.505
0.812 0
0.814 -0.639375
0.816 12.4387
0.818 -1.9181
0.82 4.7081
0.822 15.7519
0.824 21.1575
0.826 -1.1044
0.828 -6.1613
0.83 -11.625
0.832 1.6856
0.834 1.9762
0.836 -0.290625
0.838 -11.7994
0.84 -0.11625
0.842 -3.4875
0.844 -16.9725
0.846 -28.6556
0.848 -4.0106
0.85 -15.9262
0.852 13.5431
0.854 -15.9844
0.856 1.395
0.858 18.0769
0.86 19.7044
0.862 -16.1588
0.864 -9.3
0.866 -3.3131
0.868 6.8587
0.87 0.2325
0.872 -14.2406
0.874 2.4413
0.876 8.835
0.878 -32.3175
0.88 -6.9169
0.882 1.8019
0.884 14.7056
0.886 1.3369
0.888 10.4044
0.89 -0.6975
0.892 9.1256
0.894 -9.3581
0.896 2.4994
0.898 -26.2725
0.9 -4.8244
0.902 -15.345
0.904 -6.8587
0.906 13.02
0.908 -21.2738
0.91 -15.5194
0.912 -23.25
0.914 -14.8219
0.916 -5.6381
0.918 15.7519
0.92 4.185
0.922 1.2788
0.924 7.2075
0.926 17.5538
0.928 22.2038
0.93 1.6275
0.932 0.34875
0.934 -10.1137
0.936 2.0344
0.938 -16.2169
0.94 -6.1031
0.942 14.2406
0.944 -10.2881
0.946 11.9737
0.948 -1.3369
0.95 -30.7481
0.952 -4.4756
0.954 -10.9856
0.956 13.5431
0.958 23.8894
0.96 9.4162
0.962 1.2788
0.964 1.3369
0.966 -15.4031
0.968 10.9275
0.97 9.7069
0.972 -6.9169
0.974 0.81375
0.976 -0.465
0.978 -0.406875
0.98 -18.4838
0.982 -1.1044
0.984 -5.2313
0.986 1.4531
0.988 -13.6013
0.99 -15.5194
0.992 -9.9394
0.994 11.5087
0.996 0.2325
0.998 15.4031
1 -25.11
1.002 1.6275
1.004 -1.7437
1.006 -15.2288
1.008 -8.1956
1.01 -17.4375
1.012 -2.2087
1.014 -5.0569
1.016 -13.95
1.018 -0.58125
1.02 0
1.022 -0.34875
1.024 5.6381
1.026 -13.8919
1.028 10.23
1.03 -3.1388
1.032 21.855
1.034 -20.8088
1.036 14.5894
1.038 -14.9963
1.04 -1.4531
1.042 -12.6131
1.044 1.7437
1.046 0.871875
1.048 11.8575
1.05 0.58125
1.052 -4.4175
1.054 -1.9762
1.056 0.93
1.058 2.6156
1.06 -10.4044
1.062 -2.0925
1.064 12.555
1.066 1.2788
1.068 10.4625
1.07 16.5075
1.072 -13.7175
1.074 0.174375
1.076 0.755625
1.078 0.6975
1.08 28.0163
1.082 14.88
1.084 11.16
1.086 2.5575
1.088 5.4056
1.09 -1.7437
1.092 8.0213
1.094 16.6238
1.096 -3.4875
1.098 19.8206
1.1 29.5856
1.102 -1.6275
1.104 -3.3713
1.106 -14.7056
1.108 5.7544
1.11 12.0319
};
\addplot [semithick, color1]
table {%
0 0
0.002 11.4375
0.004 0
0.006 7.6631
0.008 29.28
0.01 17.8425
0.012 0
0.014 15.0403
0.016 14.64
0.018 2.0016
0.02 14.64
0.022 9.0356
0.024 14.64
0.026 0
0.028 -14.64
0.03 -12.7528
0.032 0
0.034 -11.5519
0.036 -14.64
0.038 -12.2953
0.04 -14.64
0.042 5.3184
0.044 0
0.046 -19.9012
0.048 14.64
0.05 16.1269
0.052 0
0.054 -5.5472
0.056 14.64
0.058 -7.7775
0.06 0
0.062 -6.8053
0.064 -14.64
0.066 26.2491
0.068 0
0.07 -9.9506
0.072 -14.64
0.074 -11.0944
0.076 -14.64
0.078 9.0928
0.08 -14.64
0.082 -1.2009
0.084 0
0.086 -28.6509
0.088 -14.64
0.09 -2.2875
0.092 -14.64
0.094 -9.3216
0.096 0
0.098 -4.575
0.1 0
0.102 -7.0341
0.104 -14.64
0.106 -4.6322
0.108 29.28
0.11 24.3047
0.112 14.64
0.114 5.3756
0.116 0
0.118 3.0881
0.12 0
0.122 -7.6059
0.124 -14.64
0.126 6.7481
0.128 -14.64
0.13 -17.0419
0.132 -14.64
0.134 1.7156
0.136 0
0.138 1.7156
0.14 14.64
0.142 22.7034
0.144 0
0.146 2.5734
0.148 0
0.15 16.6416
0.152 -14.64
0.154 6.9197
0.156 14.64
0.158 -10.1794
0.16 -14.64
0.162 -11.6091
0.164 14.64
0.166 10.5225
0.168 -29.28
0.17 5.49
0.172 29.28
0.174 0
0.176 0
0.178 0.571875
0.18 -14.64
0.182 1.1437
0.184 0
0.186 16.0697
0.188 14.64
0.19 9.9506
0.192 0
0.194 -1.5441
0.196 -29.28
0.198 -0.915
0.2 14.64
0.202 19.2722
0.204 0
0.206 -7.32
0.208 14.64
0.21 -14.8116
0.212 0
0.214 -24.8194
0.216 -29.28
0.218 -4.6322
0.22 0
0.222 1.5441
0.224 -14.64
0.226 -35.9138
0.228 -14.64
0.23 -22.1316
0.232 0
0.234 1.7728
0.236 0
0.238 9.3787
0.24 0
0.242 0.915
0.244 14.64
0.246 -4.8037
0.248 -14.64
0.25 5.5472
0.252 14.64
0.254 -9.7791
0.256 14.64
0.258 -12.6384
0.26 -14.64
0.262 -13.4962
0.264 -14.64
0.266 16.6416
0.268 14.64
0.27 -14.5256
0.272 14.64
0.274 4.0031
0.276 0
0.278 9.3787
0.28 0
0.282 -6.6909
0.284 0
0.286 16.9275
0.288 0
0.29 -4.8037
0.292 14.64
0.294 -18.9862
0.296 0
0.298 1.4297
0.3 -14.64
0.302 -7.6059
0.304 0
0.306 -7.8919
0.308 -14.64
0.31 2.5734
0.312 0
0.314 11.8378
0.316 -14.64
0.318 16.9275
0.32 14.64
0.322 4.0031
0.324 -14.64
0.326 -17.2134
0.328 14.64
0.33 -4.3463
0.332 14.64
0.334 -0.8578125
0.336 0
0.338 15.4978
0.34 14.64
0.342 14.0109
0.344 14.64
0.346 -6.1762
0.348 14.64
0.35 22.2459
0.352 0
0.354 -7.0341
0.356 14.64
0.358 0.8578125
0.36 14.64
0.362 20.7591
0.364 0
0.366 -0.0571875
0.368 0
0.37 -8.9784
0.372 0
0.374 -4.5178
0.376 29.28
0.378 2.0588
0.38 -29.28
0.382 -33.9694
0.384 0
0.386 8.5209
0.388 0
0.39 -4.9753
0.392 14.64
0.394 -23.3325
0.396 -14.64
0.398 -1.3153
0.4 -14.64
0.402 0.4003125
0.404 14.64
0.406 -14.9259
0.408 -14.64
0.41 12.9816
0.412 14.64
0.414 12.1237
0.416 -14.64
0.418 8.9784
0.42 0
0.422 5.8331
0.424 14.64
0.426 -11.0372
0.428 -14.64
0.43 2.9166
0.432 14.64
0.434 5.0325
0.436 14.64
0.438 -7.6059
0.44 0
0.442 6.9769
0.444 0
0.446 20.7591
0.448 14.64
0.45 -1.2009
0.452 0
0.454 -2.0016
0.456 0
0.458 2.0588
0.46 0
0.462 -1.4297
0.464 -14.64
0.466 -8.5209
0.468 -14.64
0.47 -0.2859375
0.472 -14.64
0.474 -27.5072
0.476 -14.64
0.478 18.3572
0.48 0
0.482 9.0928
0.484 0
0.486 -5.8903
0.488 -14.64
0.49 18.8719
0.492 -14.64
0.494 2.1731
0.496 -29.28
0.498 -7.6631
0.5 0
0.502 14.9831
0.504 0
0.506 -6.7481
0.508 0
0.51 7.2628
0.512 43.92
0.514 100.0209
0.516 131.76
0.518 211.4794
0.52 219.6
0.522 290.3409
0.524 322.08
0.526 368.8022
0.528 409.92
0.53 464.2481
0.532 512.4
0.534 540.8221
0.536 541.68
0.538 587.8303
0.54 614.88
0.542 649.7071
0.544 688.08
0.546 700.89
0.548 702.72
0.55 719.3615
0.552 732
0.554 746.2397
0.556 746.64
0.558 764.9972
0.56 746.64
0.562 750.0141
0.564 746.64
0.566 742.1793
0.568 732
0.57 718.275
0.572 688.08
0.574 667.7212
0.576 644.16
0.578 630.435
0.58 629.52
0.582 560.9522
0.584 527.04
0.586 499.8187
0.588 468.48
0.59 435.4828
0.592 409.92
0.594 334.0894
0.596 292.8
0.598 246.764
0.6 175.68
0.602 154.5206
0.604 102.48
0.606 54.0994
0.608 29.28
0.61 -17.4994
0.612 -73.2
0.614 -116.4909
0.616 -161.04
0.618 -201.7003
0.62 -248.88
0.622 -308.6981
0.624 -366
0.626 -373.5487
0.628 -439.2
0.63 -466.7644
0.632 -497.76
0.634 -529.7278
0.636 -556.32
0.638 -587.0869
0.64 -629.52
0.642 -657.9993
0.644 -702.72
0.646 -700.0322
0.648 -732
0.65 -751.7868
0.652 -732
0.654 -765.7978
0.656 -746.64
0.658 -780.0947
0.66 -775.92
0.662 -793.8768
0.664 -732
0.666 -757.0481
0.668 -732
0.67 -685.6781
0.672 -688.08
0.674 -666.12
0.676 -644.16
0.678 -627.8616
0.68 -614.88
0.682 -565.3556
0.684 -512.4
0.686 -489.8109
0.688 -468.48
0.69 -419.3559
0.692 -380.64
0.694 -334.3181
0.696 -336.72
0.698 -253.2263
0.7 -190.32
0.702 -129.015
0.704 -43.92
0.706 13.8966
0.708 29.28
0.71 4.5178
0.712 29.28
0.714 -22.9894
0.716 14.64
0.718 -10.7512
0.72 14.64
0.722 -1.8872
0.724 0
0.726 -18.0712
0.728 0
0.73 -1.7156
0.732 14.64
0.734 0.4003125
0.736 29.28
0.738 13.4391
0.74 0
0.742 6.8625
0.744 0
0.746 5.8331
0.748 14.64
0.75 5.6616
0.752 0
0.754 -16.3556
0.756 -14.64
0.758 -2.6306
0.76 0
0.762 -12.9244
0.764 -29.28
0.766 -12.3525
0.768 -14.64
0.77 -10.4653
0.772 0
0.774 -0.0571875
0.776 -29.28
0.778 -7.6631
0.78 14.64
0.782 0.7434375
0.784 0
0.786 -9.3787
0.788 -14.64
0.79 8.6353
0.792 0
0.794 16.6987
0.796 14.64
0.798 -8.6925
0.8 0
0.802 -18.3572
0.804 0
0.806 -16.3556
0.808 -14.64
0.81 -10.8656
0.812 -14.64
0.814 -1.7728
0.816 0
0.818 -8.6925
0.82 14.64
0.822 1.6012
0.824 -14.64
0.826 -6.8053
0.828 -14.64
0.83 -3.3169
0.832 0
0.834 32.3681
0.836 0
0.838 0.7434375
0.84 -14.64
0.842 1.1437
0.844 0
0.846 -1.1437
0.848 0
0.85 -1.4869
0.852 14.64
0.854 -5.3184
0.856 -14.64
0.858 15.3834
0.86 -14.64
0.862 1.0866
0.864 -14.64
0.866 0.8578125
0.868 14.64
0.87 14.0109
0.872 -14.64
0.874 -21.3881
0.876 -14.64
0.878 -4.0603
0.88 14.64
0.882 2.1731
0.884 14.64
0.886 -18.9291
0.888 0
0.89 -9.8934
0.892 0
0.894 11.3803
0.896 0
0.898 -4.9181
0.9 14.64
0.902 -16.5272
0.904 14.64
0.906 0.68625
0.908 0
0.91 -9.0356
0.912 0
0.914 -0.571875
0.916 0
0.918 -9.0356
0.92 0
0.922 6.405
0.924 0
0.926 -14.4112
0.928 0
0.93 -6.6909
0.932 0
0.934 4.5178
0.936 14.64
0.938 1.0294
0.94 14.64
0.942 14.4112
0.944 0
0.946 6.7481
0.948 0
0.95 22.3603
0.952 14.64
0.954 14.7544
0.956 14.64
0.958 12.2953
0.96 14.64
0.962 1.8872
0.964 14.64
0.966 -17.6709
0.968 -14.64
0.97 14.2397
0.972 14.64
0.974 -16.4128
0.976 0
0.978 -9.5503
0.98 0
0.982 -7.7203
0.984 0
0.986 -24.2475
0.988 -14.64
0.99 -1.2009
0.992 0
0.994 7.6059
0.996 -14.64
0.998 -12.81
1 -29.28
1.002 -11.1516
1.004 0
1.006 4.4034
1.008 -14.64
1.01 -2.0016
1.012 0
1.014 -16.6416
1.016 14.64
1.018 -8.0634
1.02 0
1.022 -5.8331
1.024 0
1.026 -12.2953
1.028 0
1.03 20.1872
1.032 0
1.034 -1.2581
1.036 0
1.038 11.6662
1.04 0
1.042 -14.2397
1.044 0
1.046 -11.3231
1.048 -14.64
1.05 -1.1437
1.052 0
1.054 -2.6306
1.056 0
1.058 9.0356
1.06 0
1.062 1.4297
1.064 14.64
1.066 30.6525
1.068 0
1.07 15.7266
1.072 0
1.074 3.5456
1.076 0
1.078 4.9753
1.08 14.64
1.082 1.5441
1.084 14.64
1.086 -7.5487
1.088 0
1.09 0.6290625
1.092 14.64
1.094 6.5194
1.096 0
1.098 -11.0372
1.1 14.64
1.102 -4.7466
1.104 -14.64
1.106 23.9044
1.108 -14.64
1.11 2.8022
};
\end{axis}

\end{tikzpicture}
\subcaption{Ausgangsspannung}
	\label{fig:Uout}
\end{subfigure}
\label{fig:opt.Kennlinie}
\caption{Anpassung von $K$}
\end{figure}
In \figref{fig:K} ist $K_0$ die initial Kennlinie und $K_1$ ist die Kennlinie nach der ersten Anpassung und $K_2$ nach der 2. Anpassung. Dabei wurde $\Uquest_{, \mathrm{ideal}}$ so berechnet, dass für $\Uout_{, \mathrm{ideal}}$ gilt $V_{PP} = \SI{1,5}{\V}$. Die Grenzen sind in \figref{fig:K} angezeigt.
\subsubsection*{Grenzen der initial Kennlinie}
\label{subsubsec:opt.adjusta.problem}
In \figref{fig:UinUquest} sind die Spannungen eingezeichnet mit denen $K_0$ berechnet wurde. Dabei gilt für $\Uin = \Uquest_{,\mathrm{ideal}}$, was sich durch $\Uout_{,\mathrm{ideal}}$ mit $V_{PP} = \SI{3}{\V}$ berechnen lässt. Die Spannung $\Uquest_{,1}$ wurde aus dem gemessenen $\Uout_{,\mathrm{meas}}$ berechnet.\\
Das eingezeichnete Maximum von $\Uquest$ in \figref{fig:K0} liegt an der zulässigen Grenze von $K_0$, deshalb kann damit zwar ein vorverzerrtes Signal berechnen werden, aber Messungen zeigen, dass die Qualität in diesem Bereich stark abnimmt. \\
Eine Ursache dafür könnte sein, dass mit der least-square Methode, mit der die Parameter $a_n$ für $K$ bestimmt werden, die Extrema von $\Uquest$ nicht genug Gewichtung bekommen, um in diesem Bereich gültig zu sein.

\begin{figure}[H]
\begin{subfigure}{0.5 \textwidth}
	\setlength\figureheight{8cm}
	\setlength\figurewidth{8cm}
    % This file was created by matplotlib2tikz v0.6.17.
\begin{tikzpicture}

\definecolor{color0}{rgb}{0.12156862745098,0.466666666666667,0.705882352941177}

\begin{axis}[
xlabel={$U_{in}$ in \si{\milli \volt}},
ylabel={$U_{?}$ in \si{\milli \volt}},
xmin=-316.65, xmax=291.65,
ymin=-400, ymax=400,
width=\figurewidth,
height=\figureheight,
tick align=outside,
tick pos=left,
x grid style={white!69.01960784313725!black},
y grid style={white!69.01960784313725!black},
legend entries={{$K_0$}},
legend cell align={left},
legend style={at={(0.03,0.97)}, anchor=north west, draw=white!80.0!black},
extra y ticks={321.5318, -274.6144},
extra y tick labels={\tiny{$\max(U_?)$}, \tiny{$\min(U_?)$}},
extra y tick style={grid=major, ytick pos=left, ytick align=outside, ticklabel pos=left},
]
\addlegendimage{no markers, color0}
\addplot [semithick, color0]
table {%
-289 -363.379697856809
-288 -363.37698596791
-287 -363.360941852745
-286 -363.331613796975
-285 -363.289050086263
-284 -363.233299006269
-283 -363.164408842656
-282 -363.082427881085
-281 -362.987404407217
-280 -362.879386706715
-279 -362.75842306524
-278 -362.624561768454
-277 -362.477851102017
-276 -362.318339351592
-275 -362.146074802841
-274 -361.961105741425
-273 -361.763480453005
-272 -361.553247223244
-271 -361.330454337802
-270 -361.095150082342
-269 -360.847382742525
-268 -360.587200604013
-267 -360.314651952467
-266 -360.02978507355
-265 -359.732648252921
-264 -359.423289776244
-263 -359.10175792918
-262 -358.768100997391
-261 -358.422367266537
-260 -358.064605022281
-259 -357.694862550284
-258 -357.313188136209
-257 -356.919630065716
-256 -356.514236624467
-255 -356.097056098124
-254 -355.668136772348
-253 -355.227526932801
-252 -354.775274865145
-251 -354.311428855041
-250 -353.836037188152
-249 -353.349148150137
-248 -352.85081002666
-247 -352.341071103381
-246 -351.819979665963
-245 -351.287584000067
-244 -350.743932391355
-243 -350.189073125487
-242 -349.623054488127
-241 -349.045924764935
-240 -348.457732241573
-239 -347.858525203703
-238 -347.248351936986
-237 -346.627260727084
-236 -345.995299859659
-235 -345.352517620372
-234 -344.698962294885
-233 -344.034682168859
-232 -343.359725527956
-231 -342.674140657838
-230 -341.977975844166
-229 -341.271279372602
-228 -340.554099528807
-227 -339.826484598444
-226 -339.088482867173
-225 -338.340142620657
-224 -337.581512144557
-223 -336.812639724535
-222 -336.033573646251
-221 -335.244362195369
-220 -334.445053657549
-219 -333.635696318453
-218 -332.816338463743
-217 -331.98702837908
-216 -331.147814350126
-215 -330.298744662543
-214 -329.439867601992
-213 -328.571231454134
-212 -327.692884504633
-211 -326.804875039148
-210 -325.907251343342
-209 -325.000061702876
-208 -324.083354403412
-207 -323.157177730612
-206 -322.221579970136
-205 -321.276609407648
-204 -320.322314328808
-203 -319.358743019278
-202 -318.38594376472
-201 -317.403964850794
-200 -316.412854563164
-199 -315.41266118749
-198 -314.403433009435
-197 -313.385218314659
-196 -312.358065388824
-195 -311.322022517593
-194 -310.277137986626
-193 -309.223460081585
-192 -308.161037088132
-191 -307.089917291928
-190 -306.010148978636
-189 -304.921780433916
-188 -303.824859943431
-187 -302.719435792841
-186 -301.605556267809
-185 -300.483269653996
-184 -299.352624237064
-183 -298.213668302675
-182 -297.066450136489
-181 -295.911018024169
-180 -294.747420251376
-179 -293.575705103773
-178 -292.395920867019
-177 -291.208115826778
-176 -290.012338268711
-175 -288.808636478478
-174 -287.597058741743
-173 -286.377653344167
-172 -285.150468571411
-171 -283.915552709136
-170 -282.672954043005
-169 -281.422720858679
-168 -280.16490144182
-167 -278.899544078089
-166 -277.626697053148
-165 -276.346408652659
-164 -275.058727162283
-163 -273.763700867681
-162 -272.461378054516
-161 -271.151807008449
-160 -269.835036015142
-159 -268.511113360256
-158 -267.180087329453
-157 -265.842006208394
-156 -264.496918282742
-155 -263.144871838157
-154 -261.785915160302
-153 -260.420096534837
-152 -259.047464247426
-151 -257.668066583728
-150 -256.281951829406
-149 -254.889168270122
-148 -253.489764191537
-147 -252.083787879313
-146 -250.671287619111
-145 -249.252311696592
-144 -247.82690839742
-143 -246.395126007255
-142 -244.957012811758
-141 -243.512617096592
-140 -242.061987147418
-139 -240.605171249897
-138 -239.142217689692
-137 -237.673174752464
-136 -236.198090723874
-135 -234.717013889585
-134 -233.229992535257
-133 -231.737074946553
-132 -230.238309409133
-131 -228.733744208661
-130 -227.223427630796
-129 -225.707407961201
-128 -224.185733485538
-127 -222.658452489468
-126 -221.125613258653
-125 -219.587264078754
-124 -218.043453235433
-123 -216.494229014351
-122 -214.939639701171
-121 -213.379733581554
-120 -211.814558941161
-119 -210.244164065654
-118 -208.668597240694
-117 -207.087906751944
-116 -205.502140885065
-115 -203.911347925719
-114 -202.315576159566
-113 -200.71487387227
-112 -199.109289349491
-111 -197.498870876891
-110 -195.883666740131
-109 -194.263725224874
-108 -192.63909461678
-107 -191.009823201512
-106 -189.375959264732
-105 -187.7375510921
-104 -186.094646969278
-103 -184.447295181928
-102 -182.795544015712
-101 -181.139441756291
-100 -179.479036689327
-99 -177.814377100481
-98 -176.145511275416
-97 -174.472487499792
-96 -172.795354059271
-95 -171.114159239516
-94 -169.428951326187
-93 -167.739778604946
-92 -166.046689361455
-91 -164.349731881375
-90 -162.648954450368
-89 -160.944405354096
-88 -159.23613287822
-87 -157.524185308402
-86 -155.808610930304
-85 -154.089458029587
-84 -152.366774891912
-83 -150.640609802942
-82 -148.911011048338
-81 -147.178026913761
-80 -145.441705684874
-79 -143.702095647337
-78 -141.959245086813
-77 -140.213202288963
-76 -138.464015539449
-75 -136.711733123932
-74 -134.956403328074
-73 -133.198074437537
-72 -131.436794737981
-71 -129.67261251507
-70 -127.905576054464
-69 -126.135733641825
-68 -124.363133562815
-67 -122.587824103094
-66 -120.809853548326
-65 -119.029270184172
-64 -117.246122296292
-63 -115.460458170349
-62 -113.672326092005
-61 -111.881774346921
-60 -110.088851220758
-59 -108.293604999178
-58 -106.496083967844
-57 -104.696336412415
-56 -102.894410618555
-55 -101.090354871925
-54 -99.2842174581858
-53 -97.4760466629997
-52 -95.6658907720281
-51 -93.8537980709328
-50 -92.0398168453753
-49 -90.2239953810171
-48 -88.4063819635201
-47 -86.5870248785457
-46 -84.7659724117556
-45 -82.9432728488114
-44 -81.1189744753747
-43 -79.2931255771072
-42 -77.4657744396704
-41 -75.6369693487261
-40 -73.8067585899357
-39 -71.975190448961
-38 -70.1423132114635
-37 -68.3081751631049
-36 -66.4728245895467
-35 -64.6363097764507
-34 -62.7986790094784
-33 -60.9599805742914
-32 -59.1202627565514
-31 -57.27957384192
-30 -55.4379621160588
-29 -53.5954758646294
-28 -51.7521633732935
-27 -49.9080729277126
-26 -48.0632528135484
-25 -46.2177513164625
-24 -44.3716167221165
-23 -42.5248973161721
-22 -40.6776413842908
-21 -38.8298972121343
-20 -36.9817130853642
-19 -35.1331372896421
-18 -33.2842181106296
-17 -31.4350038339885
-16 -29.5855427453801
-15 -27.7358831304663
-14 -25.8860732749086
-13 -24.0361614643687
-12 -22.186195984508
-11 -20.3362251209884
-10 -18.4862971594714
-9 -16.6364603856185
-8 -14.7867630850916
-7 -12.937253543552
-6 -11.0879800466616
-5 -9.23899088008178
-4 -7.39033432947432
-3 -5.5420586805008
-2 -3.69421221882282
-1 -1.84684323010201
0 0
1 1.8462691858216
2 3.69191604170118
3 5.5368922819771
4 7.38114962098776
5 9.22463977307153
6 11.0673144525668
7 12.9091253738119
8 14.7500242511453
9 16.5899627989053
10 18.4288927314304
11 20.2667657630588
12 22.103533608129
13 23.9391479809793
14 25.7735605959482
15 27.606723167374
16 29.4385874095951
17 31.2691050369499
18 33.0982277637767
19 34.925907304414
20 36.7520953732001
21 38.5767436844734
22 40.3998039525722
23 42.2212278918351
24 44.0409672166002
25 45.8589736412061
26 47.6751988799911
27 49.4895946472936
28 51.3021126574519
29 53.1127046248045
30 54.9213222636896
31 56.7279172884458
32 58.5324414134114
33 60.3348463529247
34 62.1350838213242
35 63.9331055329482
36 65.7288632021351
37 67.5223085432233
38 69.3133932705511
39 71.102069098457
40 72.8882877412794
41 74.6720009133565
42 76.4531603290268
43 78.2317177026287
44 80.0076247485005
45 81.7808331809807
46 83.5512947144076
47 85.3189610631195
48 87.0837839414549
49 88.8457150637522
50 90.6047061443497
51 92.3607088975858
52 94.1136750377989
53 95.8635562793274
54 97.6103043365096
55 99.353870923684
56 101.094207755189
57 102.831266545363
58 104.564999008544
59 106.29535685907
60 108.022291811281
61 109.745755579514
62 111.465699878108
63 113.182076421401
64 114.894836923732
65 116.603933099439
66 118.309316662859
67 120.010939328333
68 121.708752810198
69 123.402708822792
70 125.092759080454
71 126.778855297522
72 128.460949188335
73 130.138992467231
74 131.812936848548
75 133.482734046625
76 135.1483357758
77 136.809693750411
78 138.466759684798
79 140.119485293297
80 141.767822290248
81 143.41172238999
82 145.05113730686
83 146.686018755196
84 148.316318449338
85 149.941988103623
86 151.56297943239
87 153.179244149977
88 154.790733970724
89 156.397400608967
90 157.999195779045
91 159.596071195298
92 161.187978572063
93 162.774869623678
94 164.356696064482
95 165.933409608814
96 167.504961971011
97 169.071304865412
98 170.632390006356
99 172.188169108181
100 173.738593885225
101 175.283616051827
102 176.823187322324
103 178.357259411056
104 179.885784032361
105 181.408712900577
106 182.925997730043
107 184.437590235096
108 185.943442130076
109 187.44350512932
110 188.937730947167
111 190.426071297956
112 191.908477896025
113 193.384902455712
114 194.855296691355
115 196.319612317294
116 197.777801047866
117 199.229814597409
118 200.675604680263
119 202.115123010765
120 203.548321303254
121 204.975151272068
122 206.395564631545
123 207.809513096025
124 209.216948379845
125 210.617822197344
126 212.01208626286
127 213.399692290732
128 214.780591995297
129 216.154737090895
130 217.522079291863
131 218.882570312541
132 220.236161867266
133 221.582805670377
134 222.922453436211
135 224.255056879109
136 225.580567713407
137 226.898937653445
138 228.21011841356
139 229.514061708092
140 230.810719251378
141 232.100042757757
142 233.381983941567
143 234.656494517146
144 235.923526198834
145 237.183030700968
146 238.434959737887
147 239.679265023928
148 240.915898273432
149 242.144811200735
150 243.365955520177
151 244.579282946095
152 245.784745192828
153 246.982293974715
154 248.171881006093
155 249.353458001302
156 250.526976674679
157 251.692388740563
158 252.849645913292
159 253.998699907205
160 255.139502436641
161 256.272005215936
162 257.396159959431
163 258.511918381462
164 259.61923219637
165 260.718053118491
166 261.808332862164
167 262.890023141729
168 263.963075671522
169 265.027442165883
170 266.08307433915
171 267.129923905661
172 268.167942579755
173 269.19708207577
174 270.217294108044
175 271.228530390916
176 272.230742638724
177 273.223882565807
178 274.207901886502
179 275.182752315149
180 276.148385566086
181 277.10475335365
182 278.051807392181
183 278.989499396017
184 279.917781079496
185 280.836604156957
186 281.745920342737
187 282.645681351176
188 283.535838896612
189 284.416344693383
190 285.287150455827
191 286.148207898283
192 286.99946873509
193 287.840884680585
194 288.672407449107
195 289.493988754994
196 290.305580312586
197 291.107133836219
198 291.898601040233
199 292.679933638966
200 293.451083346756
201 294.212001877941
202 294.962640946861
203 295.702952267854
204 296.432887555257
205 297.152398523409
206 297.861436886649
207 298.559954359315
208 299.247902655745
209 299.925233490277
210 300.591898577251
211 301.247849631005
212 301.893038365876
213 302.527416496204
214 303.150935736326
215 303.763547800581
216 304.365204403307
217 304.955857258843
218 305.535458081528
219 306.103958585699
220 306.661310485695
221 307.207465495854
222 307.742375330514
223 308.265991704015
224 308.778266330694
225 309.27915092489
226 309.768597200941
227 310.246556873186
228 310.712981655963
229 311.16782326361
230 311.611033410466
231 312.042563810868
232 312.462366179157
233 312.870392229669
234 313.266593676743
235 313.650922234718
236 314.023329617932
237 314.383767540723
238 314.73218771743
239 315.068541862392
240 315.392781689945
241 315.70485891443
242 316.004725250184
243 316.292332411545
244 316.567632112852
245 316.830576068444
246 317.081115992659
247 317.319203599835
248 317.54479060431
249 317.757828720424
250 317.958269662513
251 318.146065144918
252 318.321166881975
253 318.483526588024
254 318.633095977403
255 318.76982676445
256 318.893670663503
257 319.004579388902
258 319.102504654984
259 319.187398176087
260 319.259211666551
261 319.317896840713
262 319.363405412912
263 319.395689097486
264 319.414699608774
};
\end{axis}

\end{tikzpicture}	
\subcaption{Kennlinie}
	\label{fig:K0}
\end{subfigure}
\begin{subfigure}{0.5 \textwidth}
	\setlength\figureheight{8cm}
	\setlength\figurewidth{8cm}
    % This file was created by matplotlib2tikz v0.6.17.
\begin{tikzpicture}

\definecolor{color0}{rgb}{0.12156862745098,0.466666666666667,0.705882352941177}
\definecolor{color1}{rgb}{1,0.498039215686275,0.0549019607843137}

\begin{axis}[
xlabel={$t$ in \si{\micro \second}},
ylabel={$U$ in \si{\milli \volt}},
xmin=-0.0554, xmax=1.1634,
ymin=-400, ymax=400,
width=\figurewidth,
height=\figureheight,
tick align=outside,
tick pos=left,
x grid style={white!69.01960784313725!black},
y grid style={white!69.01960784313725!black},
legend entries={{$U_{?,1}$},{$U_{in}$}},
legend cell align={left},
legend style={draw=white!80.0!black}
]
\addlegendimage{no markers, color0}
\addlegendimage{no markers, color1}
\addplot [semithick, color0]
table {%
0 -6.16700467414
0.004 8.87297239424
0.008 -2.93118040018
0.012 -23.8163851912
0.016 -2.29509589584
0.02 17.9590020152
0.024 -3.42157976989
0.028 -13.3759783991
0.032 4.12471618308
0.036 3.7284124997
0.04 -7.06832535917
0.044 -0.349866227228
0.048 9.99854221489
0.052 12.889716331
0.056 4.3704799152
0.06 -14.5061078015
0.064 -9.16613084223
0.068 18.6974797225
0.072 11.5610623599
0.076 -17.0120431085
0.08 -7.03230169761
0.084 12.370431128
0.088 -0.725763776415
0.092 -8.24395890654
0.096 4.88511199809
0.1 3.75441217674
0.104 -7.59572136523
0.108 -6.85495020556
0.112 1.46209193524
0.116 8.45410539354
0.12 2.56191755975
0.124 -12.6194889514
0.128 -6.71842766824
0.132 9.82658915299
0.136 0.254579746728
0.14 -9.9807615222
0.144 4.64419196985
0.148 6.76804410604
0.152 -7.39906919397
0.156 -2.87035725348
0.16 4.79983168486
0.164 -2.40377068738
0.168 -4.05914546633
0.172 -2.25878664344
0.176 -2.77547501542
0.18 6.33745453009
0.184 4.25437134226
0.188 -14.2190580501
0.192 -6.88246756686
0.196 11.6989996759
0.2 -2.21555912692
0.204 -12.9629325556
0.208 4.38495508673
0.212 7.03512899838
0.216 -2.84758002775
0.22 2.29659368114
0.224 0.440865759814
0.228 -6.29719587254
0.232 4.20376976066
0.236 3.84482789473
0.24 -7.5820955377
0.244 5.86848023231
0.248 13.013476234
0.252 -9.0945921474
0.256 -8.68313888053
0.26 12.4127075216
0.264 6.0182832475
0.268 -5.2262572115
0.272 1.81528667109
0.276 0.548938429495
0.28 -3.44407703589
0.284 0.92076887892
0.288 -3.35606305955
0.292 -1.1227943542
0.296 9.89347329029
0.3 -6.23834159436
0.304 -19.6819038302
0.308 6.89198222387
0.312 13.5869901015
0.316 -18.011630247
0.32 -18.1864672246
0.324 6.14618158526
0.328 3.06534125928
0.332 -5.30575377734
0.336 -0.862725161993
0.34 -1.72922256544
0.344 -1.5157847936
0.348 -3.69007159795
0.352 -10.0377017381
0.356 8.60937570759
0.36 20.4048416871
0.364 -14.1541696432
0.368 -20.6409312638
0.372 21.0620766196
0.376 10.5769220254
0.38 -27.3096779204
0.384 -0.0881439858366
0.388 22.109057755
0.392 -6.5041809996
0.396 2.2426227341
0.4 17.3521472803
0.404 -27.5618077729
0.408 2.29474539457
0.412 173.168416198
0.416 305.782945567
0.42 311.52419421
0.424 291.700961606
0.428 294.45707989
0.432 306.872144172
0.436 329.955445059
0.44 327.266106615
0.444 294.94352352
0.448 300.51859561
0.452 330.71115259
0.456 313.959518135
0.46 287.566562918
0.464 310.925713374
0.468 328.707049013
0.472 297.377673471
0.476 281.807165107
0.48 312.802512061
0.484 313.916666843
0.488 246.183614014
0.492 168.839090019
0.496 101.448093928
0.5 4.37981959504
0.504 -84.8431623547
0.508 -112.801197731
0.512 -149.835620389
0.516 -248.073435444
0.52 -324.90161721
0.524 -340.470371785
0.528 -349.058393481
0.532 -357.607467971
0.536 -358.070536546
0.54 -371.046332152
0.544 -365.141081575
0.548 -332.1216135
0.552 -339.048705402
0.556 -368.367720509
0.56 -356.37296773
0.564 -345.787857767
0.568 -347.096801831
0.572 -298.758323952
0.576 -249.277325141
0.58 -232.964195745
0.584 -161.753886596
0.588 -86.2228489632
0.592 -101.57686844
0.596 -30.4681035953
0.6 206.08717246
0.604 278.001291074
0.608 46.591475904
0.612 -129.498735297
0.616 -26.6807026391
0.62 132.751778862
0.624 121.088356026
0.628 1.91820874264
0.632 -40.3439300316
0.636 17.9228145162
0.64 34.8128197569
0.644 -1.27053998479
0.648 29.4750295479
0.652 50.1232000586
0.656 -55.6661272912
0.66 -107.582913665
0.664 3.93595975267
0.668 37.7486676683
0.672 -84.1254231824
0.676 -119.591392089
0.68 -40.3298815037
0.684 -43.1401300299
0.688 -106.464798479
0.692 -76.2313409895
0.696 19.7371773976
0.7 48.3860734478
0.704 -16.2058248785
0.708 -61.6377525293
0.712 -4.19513805432
0.716 52.0714327624
0.72 -1.91189633613
0.724 -54.0961864617
0.728 -4.55214640828
0.732 33.1863404067
0.736 -1.13681267237
0.74 -11.4881159812
0.744 8.58000084579
0.748 -6.55952519398
0.752 -19.8193070496
0.756 6.81114325147
0.76 21.625110733
0.764 -1.43889385276
0.768 -27.4635198704
0.772 -24.56938331
0.776 8.3941096397
0.78 28.6975085746
0.784 3.25582546655
0.788 -24.3085079295
0.792 -14.627777041
0.796 -0.320143375498
0.8 0.0488741735688
0.804 5.28199807417
0.808 6.20974978446
0.812 -7.01062637573
0.816 -14.0309883272
0.82 -9.11605968625
0.824 0.417966840878
0.828 12.4199673598
0.832 9.74239343741
0.836 -9.43990465482
0.84 -11.9933692471
0.844 3.49857405121
0.848 4.52896245228
0.852 -3.31239330179
0.856 2.09452150758
0.86 6.404824647
0.864 -4.13773863722
0.868 -11.8989463468
0.872 -2.95022165963
0.876 9.99591846962
0.88 6.62936642993
0.884 -8.39655483887
0.888 -8.03279978226
0.892 5.99273370398
0.896 1.86748216134
0.9 -14.0971916625
0.904 -6.34739248233
0.908 12.2830045375
0.912 3.18971685909
0.916 -19.3496211568
0.92 -15.5449788933
0.924 6.74772993136
0.928 11.4334822961
0.932 -2.90609911124
0.936 -9.74133639106
0.94 -3.95539959843
0.944 -0.3809245884
0.948 1.06949370171
0.952 7.05223430113
0.956 8.15302430657
0.96 -0.748910099198
0.964 -5.03510024792
0.968 4.21250465275
0.972 14.973633586
0.976 11.5160096975
0.98 -2.47581051503
0.984 -3.75485764841
0.988 12.9364067535
0.992 19.2377981774
0.996 5.65489445083
1 0.63309973406
1.004 10.6780963067
1.008 11.8780036688
1.012 5.30318289227
1.016 8.92530010491
1.02 15.1940833732
1.024 8.86386089823
1.028 -2.56770219249
1.032 -1.63128078767
1.036 10.9630065543
1.04 13.3550952977
1.044 -0.369136363919
1.048 -5.37904646921
1.052 3.16367071575
1.056 1.31454330551
1.06 -4.68080206474
1.064 5.26325850289
1.068 9.29372926555
1.072 -10.6571690983
1.076 -24.1142930377
1.08 -8.35505674893
1.084 14.9110820951
1.088 13.5461691402
1.092 -13.9092747923
1.096 -27.1290153184
1.1 -4.43603582504
1.104 10.073576871
1.108 -4.44390272169
};
\addplot [semithick, color1]
table {%
0 -11.6987
0.004 1.5312
0.008 -12.3725
0.012 -5.0837
0.016 -4.3028
0.02 1.8987
0.024 -9.5244
0.028 -4.0731
0.032 -1.4547
0.036 0.214375
0.04 -5.4512
0.044 -0.8575
0.048 1.5006
0.052 -1.1637
0.056 6.0791
0.06 -1.1331
0.064 -10.5503
0.068 1.3628
0.072 0.9034374
0.076 0.949375
0.08 -11.6681
0.084 1.3475
0.088 -5.8341
0.092 -6.615
0.096 -5.7422
0.1 2.3887
0.104 -9.8
0.108 -8.1616
0.112 0.5512499
0.116 -2.9553
0.12 -5.0072
0.124 -13.6894
0.128 -3.0166
0.132 -1.8528
0.136 -2.0825
0.14 -6.9059
0.144 -4.9919
0.148 6.8141
0.152 -3.6597
0.156 -5.3747
0.16 -5.2981
0.164 -1.0412
0.168 -0.9953125
0.172 -1.6231
0.176 -5.8953
0.18 -2.205
0.184 -0.4746875
0.188 -8.2381
0.192 -2.3887
0.196 -7.4419
0.2 1.4394
0.204 -15.6953
0.208 -2.7562
0.212 0.3215625
0.216 -2.0825
0.22 -9.1416
0.224 -6.5078
0.228 -3.5525
0.232 -3.0625
0.236 -0.3675
0.24 -6.615
0.244 2.8481
0.248 -3.5066
0.252 -4.0884
0.256 -11.5762
0.26 3.9659
0.264 -4.4253
0.268 0.4134375
0.272 -2.5266
0.276 0.5665625
0.28 -8.3606
0.284 -2.1437
0.288 -3.6597
0.292 -3.5372
0.296 -0.2603125
0.3 0.6124999
0.304 -7.8553
0.308 -4.8694
0.312 -0.3215625
0.316 -10.1062
0.32 -6.3394
0.324 0.0459375
0.328 -2.1131
0.332 -7.9931
0.336 1.7303
0.34 -5.7422
0.344 -2.2816
0.348 -0.3675
0.352 -6.9059
0.356 -5.88
0.36 5.1909
0.364 -4.5172
0.368 -5.9719
0.372 -1.9294
0.376 -1.6537
0.38 -10.0909
0.384 1.0412
0.388 1.9141
0.392 -9.8919
0.396 1.6231
0.4 6.4925
0.404 -25.7709
0.408 -12.1734
0.412 75.1231
0.416 218.9228
0.42 196.4747
0.424 182.5097
0.428 189.875
0.432 241.5241
0.436 298.3794
0.44 282.3625
0.444 290.0034
0.448 297.5984
0.452 297.5984
0.456 277.7075
0.46 311.2112
0.464 299.834
0.468 281.4437
0.472 231.5097
0.476 239.6253
0.48 210.5775
0.484 180.8712
0.488 149.0519
0.492 106.9272
0.496 52.5984
0.5 -9.8766
0.504 -32.5697
0.508 -60.025
0.512 -75.1078
0.516 -150.9047
0.52 -179.3247
0.524 -187.4862
0.528 -192.6312
0.532 -239.0281
0.536 -255.5503
0.54 -266.6519
0.544 -268.8875
0.548 -267.1418
0.552 -267.1725
0.556 -266.4987
0.56 -246.1484
0.564 -238.1247
0.568 -236.3637
0.572 -181.2541
0.576 -148.8681
0.58 -146.7856
0.584 -124.705
0.588 -53.7316
0.592 -45.9834
0.596 -14.8225
0.6 105.84
0.604 142.7584
0.608 59.6881
0.612 -112.4856
0.616 -0.949375
0.62 50.3016
0.624 62.5516
0.628 5.9259
0.632 -26.5825
0.636 6.9519
0.64 5.6656
0.644 1.7609
0.648 11.3619
0.652 32.4625
0.656 -50.0106
0.66 -45.8303
0.664 -28.175
0.668 30.9619
0.672 -57.1309
0.676 -54.2675
0.68 -38.6181
0.684 -28.1597
0.688 -61.1734
0.692 -35.8159
0.696 12.495
0.7 17.6094
0.704 -0.5665625
0.708 -47.7291
0.712 11.2087
0.716 12.9697
0.72 0.06125
0.724 -28.4966
0.728 -1.9141
0.732 9.6928
0.736 -3.7516
0.74 -9.7694
0.744 1.2709
0.748 -4.4406
0.752 -14.945
0.756 -0.2909375
0.76 3.9353
0.764 3.6137
0.768 -19.6766
0.772 -17.9769
0.776 1.8375
0.78 7.2734
0.784 -5.6044
0.788 -13.7659
0.792 1.9447
0.796 -3.8434
0.8 -6.5231
0.804 -1.7762
0.808 4.1956
0.812 -3.4606
0.816 -7.6562
0.82 -11.0556
0.824 -3.9966
0.828 4.3334
0.832 -1.5159
0.836 -8.0391
0.84 -0.7809374
0.844 -3.5219
0.848 -1.6537
0.852 -0.5665625
0.856 -0.214375
0.86 -6.4006
0.864 -6.2475
0.868 -2.0366
0.872 -0.1684375
0.876 -0.5512499
0.88 1.0566
0.884 -6.7834
0.888 -8.4066
0.892 -0.153125
0.896 0.275625
0.9 -3.0778
0.904 -4.3947
0.908 -8.5137
0.912 -0.091875
0.916 -6.1403
0.92 -7.5031
0.924 3.0625
0.928 -0.214375
0.932 -10.1369
0.936 -9.9991
0.94 -8.0544
0.944 3.7209
0.948 -2.0978
0.952 -7.5797
0.956 -1.0412
0.96 -3.8587
0.964 -5.1756
0.968 -0.153125
0.972 0.1378125
0.976 -0.9953125
0.98 -1.2097
0.984 -4.6397
0.988 0.091875
0.992 -0.1225
0.996 0.9646874
1 -6.2781
1.004 1.8222
1.008 0.4440625
1.012 -4.9919
1.016 0.5359375
1.02 6.6916
1.024 -3.8587
1.028 -4.9459
1.032 -1.4394
1.036 -1.6691
1.04 0.79625
1.044 -9.9837
1.048 -4.8541
1.052 -1.2403
1.056 -1.6384
1.06 -2.9553
1.064 3.8741
1.068 -2.205
1.072 -8.1462
1.076 -5.635
1.08 -1.5925
1.084 3.3534
1.088 -4.6397
1.092 -7.3959
1.096 -8.1003
1.1 -4.8541
1.104 -6.3394
1.108 -12.8319
};
\end{axis}

\end{tikzpicture}
\subcaption{Berechnung von $K$}
	\label{fig:UinUquest}
\end{subfigure}
\caption{Nichtlineare Vorverzerrung}
\label{fig:Amplitudenproblem}
\end{figure}
Für die Evaluierung der Grenzen der Kennlinien wurden verschiedene Amplituden von $\Uquest_{,\textrm{ideal}}$ über die Kennlinie nichtlinear vorverzerrt. Das Ausgangssignal wurde mit dem RF-Tool bewertet und ist in \figref{fig:evaluateK.quality} gezeigt. Dabei sind die Grenzen von $\Uquest$ in \figref{fig:K0_quality} mit $\min(\Uquest)$ und $\max(\Uquest)$ beschriftet dies entspricht mit \SI{120}{\mV} etwa $34\%$ des maximal möglichen $V_{PP}$ das von $K$ im bijektiven Bereich zugelassen ist. Die anderen Grenzen sind die, bis wohin die Güte überprüft wurde.
\begin{figure}[H]
\begin{subfigure}{0.5 \textwidth}
	\setlength\figureheight{8cm}
	\setlength\figurewidth{8cm}
    % This file was created by matplotlib2tikz v0.6.17.
\begin{tikzpicture}

\definecolor{color0}{rgb}{0.12156862745098,0.466666666666667,0.705882352941177}

\begin{axis}[
xlabel={$U_{in}$ in \si{\milli \volt}},
ylabel={$U_{?}$ in \si{\milli \volt}},
xmin=-329.9, xmax=327.9,
ymin=-200, ymax=200,
width=\figurewidth,
height=\figureheight,
tick align=outside,
tick pos=left,
x grid style={white!69.01960784313725!black},
y grid style={white!69.01960784313725!black},
legend entries={{$K_0$}},
legend cell align={left},
legend style={at={(0.03,0.97)}, anchor=north west, draw=white!80.0!black},
extra y ticks={64.0888558094, -55.9111441906, 115.359940457, -100.640059543},
extra y tick labels={\tiny{$\max(U_?)$}, \tiny{$\min(U_?)$}},
extra y tick style={grid=major, ytick pos=left, ytick align=inside, ticklabel pos=left}
]
\addlegendimage{no markers, color0}
\addplot [semithick, color0]
table {%
-300 -192.694568057
-298 -192.255682485
-296 -191.79910167
-294 -191.324954187
-292 -190.833368609
-290 -190.32447351
-288 -189.798397462
-286 -189.25526904
-284 -188.695216817
-282 -188.118369367
-280 -187.524855262
-278 -186.914803078
-276 -186.288341387
-274 -185.645598763
-272 -184.986703779
-270 -184.311785009
-268 -183.620971027
-266 -182.914390406
-264 -182.192171719
-262 -181.454443541
-260 -180.701334444
-258 -179.932973003
-256 -179.14948779
-254 -178.35100738
-252 -177.537660346
-250 -176.709575261
-248 -175.866880699
-246 -175.009705234
-244 -174.138177439
-242 -173.252425888
-240 -172.352579154
-238 -171.438765811
-236 -170.511114432
-234 -169.569753591
-232 -168.614811862
-230 -167.646417818
-228 -166.664700032
-226 -165.669787078
-224 -164.66180753
-222 -163.640889961
-220 -162.607162945
-218 -161.560755056
-216 -160.501794866
-214 -159.430410949
-212 -158.34673188
-210 -157.250886231
-208 -156.143002576
-206 -155.023209489
-204 -153.891635544
-202 -152.748409312
-200 -151.59365937
-198 -150.427514289
-196 -149.250102643
-194 -148.061553007
-192 -146.861993953
-190 -145.651554055
-188 -144.430361886
-186 -143.198546021
-184 -141.956235033
-182 -140.703557494
-180 -139.44064198
-178 -138.167617063
-176 -136.884611317
-174 -135.591753316
-172 -134.289171632
-170 -132.97699484
-168 -131.655351513
-166 -130.324370225
-164 -128.984179549
-162 -127.634908059
-160 -126.276684329
-158 -124.909636931
-156 -123.53389444
-154 -122.149585429
-152 -120.756838471
-150 -119.35578214
-148 -117.94654501
-146 -116.529255655
-144 -115.104042647
-142 -113.67103456
-140 -112.230359969
-138 -110.782147445
-136 -109.326525564
-134 -107.863622898
-132 -106.393568022
-130 -104.916489508
-128 -103.43251593
-126 -101.941775862
-124 -100.444397877
-122 -98.9405105491
-120 -97.4302424516
-118 -95.9137221581
-116 -94.391078242
-114 -92.8624392771
-112 -91.3279338367
-110 -89.7876904946
-108 -88.2418378241
-106 -86.6905043989
-104 -85.1338187926
-102 -83.5719095786
-100 -82.0049053305
-98 -80.432934622
-96 -78.8561260264
-94 -77.2746081175
-92 -75.6885094687
-90 -74.0979586536
-88 -72.5030842457
-86 -70.9040148187
-84 -69.300878946
-82 -67.6938052012
-80 -66.0829221578
-78 -64.4683583895
-76 -62.8502424697
-74 -61.228702972
-72 -59.60386847
-70 -57.9758675372
-68 -56.3448287472
-66 -54.7108806735
-64 -53.0741518897
-62 -51.4347709693
-60 -49.7928664859
-58 -48.148567013
-56 -46.5020011242
-54 -44.8532973931
-52 -43.2025843931
-50 -41.5499906979
-48 -39.895644881
-46 -38.2396755159
-44 -36.5822111762
-42 -34.9233804355
-40 -33.2633118673
-38 -31.6021340452
-36 -29.9399755426
-34 -28.2769649332
-32 -26.6132307906
-30 -24.9489016882
-28 -23.2841061996
-26 -21.6189728984
-24 -19.9536303581
-22 -18.2882071523
-20 -16.6228318545
-18 -14.9576330383
-16 -13.2927392773
-14 -11.6282791449
-12 -9.96438121478
-10 -8.30117406044
-8 -6.63878625545
-6 -4.97734637336
-4 -3.31698298772
-2 -1.65782467208
0 0
2 1.65636245498
4 3.31113411929
6 4.96418641939
8 6.61539078173
10 8.26461863275
12 9.9117413989
14 11.5566305066
16 13.1991573824
18 14.8391934526
20 16.4766101438
22 18.1112788823
24 19.7430710946
26 21.3718582072
28 22.9975116465
30 24.6199028389
32 26.238903211
34 27.8543841891
36 29.4662171997
38 31.0742736693
40 32.6784250243
42 34.2785426911
44 35.8744980962
46 37.466162666
48 39.053407827
50 40.6361050056
52 42.2141256284
54 43.7873411216
56 45.3556229119
58 46.9188424255
60 48.476871089
62 50.0295803289
64 51.5768415715
66 53.1185262433
68 54.6545057708
70 56.1846515804
72 57.7088350985
74 59.2269277517
76 60.7388009663
78 62.2443261688
80 63.7433747856
82 65.2358182432
84 66.7215279681
86 68.2003753867
88 69.6722319254
90 71.1369690106
92 72.594458069
94 74.0445705267
96 75.4871778105
98 76.9221513465
100 78.3493625615
102 79.7686828816
104 81.1799837335
106 82.5831365436
108 83.9780127382
110 85.364483744
112 86.7424209872
114 88.1116958944
116 89.472179892
118 90.8237444064
120 92.1662608641
122 93.4996006916
124 94.8236353153
126 96.1382361616
128 97.443274657
130 98.7386222279
132 100.024150301
134 101.299730302
136 102.565233658
138 103.820531796
140 105.065496141
142 106.299998121
144 107.523909161
146 108.737100688
148 109.939444129
150 111.13081091
152 112.311072457
154 113.480100197
156 114.637765557
158 115.783939962
160 116.91849484
162 118.041301616
164 119.152231718
166 120.251156571
168 121.337947602
170 122.412476237
172 123.474613904
174 124.524232028
176 125.561202036
178 126.585395354
180 127.596683408
182 128.594937626
184 129.580029434
186 130.551830257
188 131.510211523
190 132.455044658
192 133.386201089
194 134.303552241
196 135.206969541
198 136.096324417
200 136.971488293
202 137.832332597
204 138.678728756
206 139.510548195
208 140.32766234
210 141.12994262
212 141.917260459
214 142.689487284
216 143.446494522
218 144.1881536
220 144.914335943
222 145.624912978
224 146.319756132
226 146.998736831
228 147.661726501
230 148.308596569
232 148.939218462
234 149.553463605
236 150.151203426
238 150.73230935
240 151.296652804
242 151.844105215
244 152.374538009
246 152.887822613
248 153.383830452
250 153.862432954
252 154.323501545
254 154.766907651
256 155.192522699
258 155.600218115
260 155.989865325
262 156.361335757
264 156.714500836
266 157.049231989
268 157.365400642
270 157.662878223
272 157.941536156
274 158.20124587
276 158.441878789
278 158.663306342
280 158.865399953
282 159.04803105
284 159.211071059
286 159.354391406
288 159.477863518
290 159.581358822
292 159.664748743
294 159.727904709
296 159.770698145
298 159.793000478
};
\end{axis}

\end{tikzpicture}	
\subcaption{Kennlinie}
	\label{fig:K0_quality}
\end{subfigure}
\begin{subfigure}{0.5 \textwidth}
	\setlength\figureheight{8cm}
	\setlength\figurewidth{8cm}
    % This file was created by matplotlib2tikz v0.6.17.
\begin{tikzpicture}

\definecolor{color0}{rgb}{0.12156862745098,0.466666666666667,0.705882352941177}

\begin{axis}[
xlabel={$\frac{V_{PP, U_{?}}}{V_{PP, \textrm{max}}}$},
ylabel={QGesamt1},
xmin=0.326820036459133, xmax=0.626405069880006,
ymin=1.2425958340305, ymax=1.6843073440395,
width=\figurewidth,
height=\figureheight,
tick align=outside,
tick pos=left,
x grid style={white!69.01960784313725!black},
y grid style={white!69.01960784313725!black},
legend cell align={left},
legend entries={{Güte}},
legend style={draw=white!80.0!black}
]
\addlegendimage{no markers, color0}
\addplot [semithick, color0, mark=x, mark size=3, mark options={solid}, only marks]
table {%
0.340437537978264 1.66422954813
0.408525045573917 1.51978724502
0.47661255316957 1.40006187247
0.544700060765222 1.26267362994
0.612787568360875 1.29629730078
};
\end{axis}

\end{tikzpicture}	
\subcaption{Berechnung von $K$}
	\label{fig:evaluateK.quality}
\end{subfigure}
\caption{Bewertung des Ausgangssignals}
\label{fig:Amplitudenproblem}
\end{figure}
Dabei ist der kleinste und damit der beste Wert der Güte mit \SI{190}{\mV} etwa $54\%$ des maximal möglichen $V_{PP}$ das von $K$ aufgezeichnet worden. Diese Messung lässt sich nicht auf alle Kennlinien übertragen und sollte im Vornherein festgestellt werden.

\end{document}
