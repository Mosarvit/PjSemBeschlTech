\documentclass[../Report.tex]{subfiles}


\begin{document}

\chapter{Iterative Optimierung des Hammerstein-Modells}
\label{chap:opt}

%TODO: überprüfe Referenz zu Jens, da Ideen nicht publiziert wurden! Wie wird das eingebunden?
Ziel der Optimierung von Übertragungsfunktion $\Hcompl$ und Kennlinie $K$ mit ihren Parametern $a$ ist die Minimierung des Fehlers zwischen idealem und gemessenem Ausgangssignal, $\Uout_{, \mathrm{id}}$ und $\Uout_{, \mathrm{meas}}$ . Die Minimierung des relativen Fehlers ist also gegeben durch
\begin{align}
\label{eq:opt.relFehler}
	\min \; \mathit{f} \oft = \min \left( \frac{\Uout_{, \mathrm{meas}}  - \Uout_{, \mathrm{id}} }{ \Uout_{, \mathrm{id}} } \right) 
	= \min \left( \frac{ \Uout_{, \mathrm{meas}}}{ \Uout_{, \mathrm{id}}} -1 \right) 
	\; .
\end{align}
Für das verwendete Hammerstein-Modell liegt die in \cite{----Jens---} %TODO: Referenz Idee iterative Optimierung Jens
vorgeschlagene getrennte, iterative Optimierung von $\Hcompl$ und $K$ nahe. 
Die Auswertung der Qualität des Einzelsinus erfolgt dabei durch das RF-Tool von --- Zitat RF-Tool --- mit Entwicklungsstand vom --- --- unter Verwendung des als \lstinline{QGesamt1} geführten Qualitätswerts \footnote{\label{foot:opt.H.quality} Hierauf beziehen sich alle weiteren Angaben zur Qualität des Signals. Eine intensivere Befassung mit dem Tool hat nicht stattgefunden.}.
%TODO: Auffüllen Daten!


\section{Optimierung der linearen Übertragungsfunktion $H$}
\label{sec:opt.H}

Die Optimierung von $\Hcompl \ofomega$ beruht auf der Annahme, dass sich \eqref{eq:opt.relFehler} auf die Betragsspektren des berechneten und des gemessenen Ausgangssignals, $\Uoutc_{, \mathrm{id}} \ofomega $ und $\Uoutc_{, \mathrm{meas}} \ofomega $ fortsetzen lässt mit 

\begin{align}
\label{eq:opt.ratio}
	\fabs \ofomega :=  
				\frac{\mathrm{abs} \left( \Uoutc_{, \mathrm{meas}} \ofomega \right)}{\mathrm{abs} \left(\Uoutc_{, \mathrm{id}} \ofomega \right)} -1
				\; .
\end{align} 

Ist im Betragsspektrum des gemessenen Signals eine Frequenz mit halbem Betrag verglichen mit dem idealen Signal vertreten, wird dies entsprechend der Linearität der Übertragunsfunktion dahingehend gedeutet, dass die Verstärkung von $\Hcompl$ bei dieser Frequenz um einen Faktor $2$ zu gering ist.
Iterativ mit einer Schrittweite $\sigma_H$ ausgeführt, folgt für den $i$-ten Schritt

\begin{align}
\label{eq:opt.Hnew}
	\mathrm{abs} \left( \Hcompl^{i+1} \right)
		=\mathrm{abs} \left( \Hcompl^{i}  \right) \cdot
		\left( 1 + \sigma_H^i \: \fabs^{i}	\right)					 
\end{align}

für $\sigma_H^i \in \left[ 0 , 1 \right]$ und $\Uoutc_{, \mathrm{meas}}^{i}$ in $\fabs^{i}$ als gemessenem Ausgangssignal für das mit $\Hcompl^{i}$ berechnete Eingangssignal \footnote{Nachfolgend wird aus Gründen der Übersichtlichkeit $\fabs$ statt dem länglichen Bruch genutzt}.
Würde allerdings \eqref{eq:opt.Hnew} mit komplexen Zahlen und nicht allein den Beträgen ausgeführt, würde auch die Phase der $-1$ beachtet und folglich die durch $\sigma_H$ skalierte komplexe Zahl wesentlich verändert. Also muss für das Phasenspektrum eine andere Optimierung erfolgen.
Eine Möglichkeit hierfür wäre die simple Anpassung der Phase $ \mathrm{arg} \left( \Hcompl \right) = \varphi_H$ mit 
\begin{align}
\label{eq:opt.HnewPhase}
	\varphi_H^{i+1} = \varphi_H^{i} - \sigma_{\varphi}^{i} 
			\left( \: \mathrm{arg} \left( \Uoutc_{, \mathrm{meas}} \right)
					- \mathrm{arg} \left( \Uoutc_{, \mathrm{id}} \right) \: \right)
\end{align}
mit $\sigma_{\varphi}^i \in \left[ 0 , 1 \right]$. Diese Anpassung der Phase wurde jedoch nur kurzen Tests unterzogen und anschließend nicht weiter verfolgt. Es hat sich die Signalform des Ausgangssignals unproportional stärker verändert, als dies nur im Falle der Betrags-Anpassung der Fall war. Vermutlich %%%% TODO: kann man vermutlich hier nutzen???
liegt dies an dem aus dem Ausgangssignal gewonnenen Phasengang, der in wesentlich größerem Maße vom idealen Phasengang abweicht als im Betragsspektrum. 
%Allerdings ist zu erwähnen, dass sich nicht ausschließen lässt, dass im simpelsten Fall das Vorzeichen des Iterationsschrittes schlicht falsch gewählt wurde. Aufgrund des späten Auftretens dieser möglichen Fehlerquellen konnten jedoch keine Tests diesbezüglich mehr dürchgeführt werden.
%%%%%%%%%%%%%%%%%%%%%%%% TODO: Check Phase am Mock-System!!!! %%%%%%
In \figref{fig:opt.spektrum_BB_signal} sind Betrag und Phase der durch FFT erhaltenen Spektren für gemessenes und ideales Ausgangssignal vor Durchführung einer Optimierung dargestellt. Insbesondere illustriert \figref{subfig:opt.angle_spektrum} die bei gemessenem Signal auftretende Streuung der Phase. 
\\


\pgfplotstableread[col sep = comma] {opt_spect_ideal_abs.csv} \absSpectIdeal 
\pgfplotstableread[col sep = comma] {opt_spect_meas_abs.csv} \absSpectMeas
\pgfplotstableread[col sep = comma] {opt_spect_ideal_angle.csv} \angleSpectIdeal 
\pgfplotstableread[col sep = comma] {opt_spect_meas_angle.csv} \angleSpectMeas 

\begin{figure}[htb]
\begin{subfigure}{0.5 \textwidth}
\centering
    \begin{tikzpicture}
\begin{axis}[
		legend entries = {Ideales Signal, Gemessenes Signal},
		legend pos = north east,
		xlabel={Frequenz},
		ylabel={Spektraldichte },
		%xtick distance = 10000000,
%		xminorgrids,
%		xmajorgrids,
		minor x tick num =3,
		xtick pos = lower,
		ytick pos = left,
		xtick align = outside,
		ytick align = outside,
		scaled x ticks = base 10:-6,
		xtick scale label code/.code={\si{\MHz}},
		xmin = -5000000,
		xmax = 85000000,
		ymin = 0,
		ymax = 0.03,
		]
		
		\addplot[blue, mark size=3.5pt] table [ x index =0, y index=1] {\absSpectIdeal};	% plot des Idealen Betragsspektrums
		\addplot[green, mark size=3.5pt] table [ x index =0, y index=1] {\absSpectMeas};	% Plot des gemessenen Betragsspektrums
\end{axis}
\end{tikzpicture}
\caption{Betragsspektren}
	\label{subfig:opt.abs_spektrum}
\end{subfigure}
\begin{subfigure}{0.5 \textwidth}
\centering
    \begin{tikzpicture}
\begin{axis}[
		legend entries = {Ideales Signal, Gemessenes Signal},
		legend pos = north east,
		xlabel={Frequenz},
		ylabel={Phase},
		y label style={at={(axis description cs:-0.1,.5)},anchor=south},
%		%xtick distance = 10000000,
%		xminorgrids,
%		xmajorgrids,
		minor x tick num =3,
		xtick pos = lower,
		ytick pos = left,
		xtick align = outside,
		ytick align = outside,
		scaled x ticks = base 10:-6,
		xtick scale label code/.code={\si{\MHz}},
		xmin = -5000000,
		xmax = 85000000,
		scaled y ticks={real:3.1415},
		ytick scale label code/.code={$\si{\radian}$},
%		ymin = 0,
%		ymax = 0.03,
		]
		
		\addplot[blue, only marks, mark size=1pt] table [ x index =0, y index=1] {\angleSpectIdeal};	% plot des Idealen Betragsspektrums
		\addplot[green, only marks, mark size=1pt] table [ x index =0, y index=1] {\angleSpectMeas};	% Plot des gemessenen Betragsspektrums
\end{axis}
\end{tikzpicture}
\caption{Phasenspektren}
	\label{subfig:opt.angle_spektrum}
\end{subfigure}
\caption{Spektrum des Einzelsinus-Signals, berechnet und gemessen mit je $109$ Punkten}
\label{fig:opt.spektrum_BB_signal}
\end{figure}

 
Neben den für kontinuierliche Funktionen problemlos definierbaren iterativen Zuweisungen ergeben sich in Messung und diskreter Ausführung jedoch Fehlerquellen. Problematisch sind insbesondere solche, die in \eqref{eq:opt.Hnew} durch das Betragsverhältnis der Ausgangssignale verstärkt werden. 
Unterscheiden sich die Spektren hier um einen großen Faktor, resultiert dies in einer großen Anpassung der Übertragungsfunktion für die betreffende Frequenz. Dies ist folglich insbesondere bei kleinen Beträgen der Spektren problematisch, wenn Ungenauigkeiten und Störeinflüsse betrachtet werden. 
\\
Besondere Störeinflüsse ergeben sich also durch
\begin{itemize}
	\item Rauschen: Weißes Rauschen macht sich in allen Frequenzen bemerkbar mit kritischem Einfluss bei geringer Spektraldichte des Signals.
	
	\item Diskretisierungsfehler: Die FFT bedingt eine begrenzte Auflösung, in den Spektren von $\Hcompl$ und den gemessenen Signalen und liegt insbesondere im Allgemeinen an unterschiedlichen Frequenzen und mit unterschiedlich vielen Punkten vor.
	
	\item Interpolationsfehler: Die (hier lineare) Interpolation der Spektren zur Auswertung von $\fabs$ an den Frequenzen von $\Hcompl$ kann insbesondere den Einfluss oben genannter Punkte verstärken.
\end{itemize}

Weiterhin zeigt sich auch in der geringen Stützstellenzahl in \figref{fig:opt.spektrum_BB_signal} bereits eine erste konzeptuelle Problematik des Vorgehens. Der Frequenzabstand zwischen zwei Werten der FFT ist stets mit der Wiederholfrequenz $f_{rep}$ gegeben und lässt sich somit nicht durch eine höhere Auflösung der Messgeräte verbessern, der betrachtete Bereich bis $\SI{80}{\MHz}$ nicht besser auflösen.
Folglich setzt sich diese Ungenauigkeit auch auf die Optimierung der Kennlinie fort. Insbesondere relevant wird dies, da die Kennlinie mit nahezu der doppelten Anzahl an Werten erstellt wird und somit der Interpolationsfehler ungleich größer wird als bei ähnlicher Anzahl Stützstellen.


\subsection{Umgang mit Fehlerquellen}
\subsubsection*{Ignorieren kleiner Beträge im Spektrum}
\label{subsubsec:opt.H.prom}

Um Rauscheinflüsse und Probleme durch Nulldurchgänge zu dämpfen, wurde ein erster intuitiver Ansatz vorgenommen: Bei den Betragsspektren der in $\fabs$ eingehenden Signale, des gemessenen und idealisierten Spannungssignals, wurden alle Anteile, die verglichen mit dem Maximalwert des betreffenden Spektrums besonders klein sind, auf einen vorgegebenen Wert, im Folgenden Default-Wert genannt, gesetzt. Dies führt an den betroffenen Frequenzen zu $\fabs = 0$ und damit keiner Änderung von $\Hcompl$.
Dies bedeutet also, dass alle Einträge des Betragsspektrums von $\Uout_{,\mathrm{ideal}}$ mit weniger als zum Beispiel $5 \, \promille $ der maximalen Amplitude auf den Default-Wert gesetzt werden. Insbesondere werden auch die Einträge an den Frequenzen zurückgesetzt, die im Spektrum von $\Uout_{,\mathrm{meas}}$ klein gegen das zugehörige Maximum sind.
\\
Zu beachten bei letzterem Punkt ist die notwendige Rundung, wenn die Einträge der FFT an unterschiedlichen Frequenzen vorliegen. 
\\
\\
\noindent
Mit Beschränkung auf $3 \, \promille$ und dem globalen Minimum beider Spektren als Default-Wert werden insbesondere Einträge in den höheren Frequenzen des Spektrums \ref{subfig:opt.abs_spektrum} beeinflusst. Erst ab etwa $5 \promille$ werden alle Frequenzen ab ungefähr $\SI{60}{\MHz}$ auf den Default-Wert gesetzt. Mit dieser Methodik könnten folglich in erster Linie massive Korrekturen an hohen Frequenzen von $\Hcompl$ verhindert.
Verglichen mit der nachfolgend beschriebenen Anpassung hat dieses Vorgehen aufgrund der beschriebenen Bandbreite einen niedrigeren Einfluss auf die Qualität des Ausgangssignals.


\subsubsection*{Ignorieren großer Korrektur-Terme}
\label{subsubsec:opt.H.RMS}

Ein zweiter, sehr grober Ansatz liegt in der Beschränkung von $\fabs$ auf Werte unterhalb einer vorgegebenen Schwelle. Zugrunde liegt die Annahme, dass die gerade an Nulldurchgängen des Spektrums sowie bei vielen hohen Frequenzen auftretenden großen Werte durch die in obiger Aufzählung genannten Fehlerquellen entstehen. Hier bedeutet dies insbesondere, dass die Diskretisierung die Nulldurchgänge nicht korrekt darstellen kann. Die Interpolation auf Frequenzen von $\Hcompl$ ist dann aufgrund der großen Sprünge von Werten in direkter Umgebung der problematischen Frequenzen mit großer Ungenauickgeit behaftet. Dies kann zu den beschriebenen, großen Korrektur-Termen in $\fabs$ führen.

\lstset{language=Python}
\begin{lstlisting}[caption={Pseudocode zur Veranschaulichung der Anpassung des Korrekturterms}, label=code:opt.H.pseudoRMS, numbers=none]
	rms_orig = root_mean_square( f_abs )
	f_abs_to_use = f_abs[ where( abs(f_abs) >= 0.02 * rms_orig ] 
	rms_mod = root_mean_square( f_abs_to_use )
	idx_to_clear = f_abs[ where( abs(f_abs) >= rms_mod ] 
	f_abs[ ix_to_clear ] = 0
\end{lstlisting}

Vereinfacht bedeutet der verfolgte Ansatz, ausnehmend große Werte von $\fabs$ als unrealistisch abzutun. Eine Pseudo-Implementierung findet sich in \coderef{code:opt.H.pseudoRMS}, um die nachfolgende Erläuterung zu illustrieren. In der vorgenommenen Implementierung wurde $\fabs$ an den ausgewählten Frequenzen beliebig auf $0$ gesetzt, also keine Anpassung bei diesen Frequenzen ermöglicht.
Als Grenze genutzt wurde ein modifizierter Effektivwert, nachfolgend mit RMS (Root Mean Square) bezeichnet. 
Der reine RMS von $\fabs$ unterliegt der Problematik, eine unproportional große Gewichtung von kleinen Einträgen zu enthalten.
\\
Idealerweise enthält $\fabs$ mit jeder Iteration kleinere Einträge als zuvor. Es würden also bei Nutzung des reinen RMS unter Umständen mit zunehmender Schrittzahl zunehmend mehr Werte in $\fabs$ ignoriert - was der Optimierung entsprechende Grenzen setzt. 
In Kombination mit den im vorigen Abschnitt erläuterten Anpassungen wäre die Problematik unumgänglich, da Frequenzen, die explizit nicht bei der Anpassung berücksichtig werden sollen, den reinen RMS-Wert beeinflussen.
Folglich muss der RMS modifiziert werden. Hier wurden zur Berechnung des modifizierten RMS nur die Werte einbezogen, die mehr als beliebig gewählte $2 \, \%$ des reinen RMS betragen. Es handelt sich also bei der vorgenommenen Anpassung um eine sehr grobe und größtenteils willkürliche Wahl der Parameter, die zu Zwecken der Illustration jedoch brauchbare Ergebnisse liefert.

Für einen Eindruck der Tragweite der RMS-Beschränkung bietet sich die Betrachtung von $\fabs$ in \figref{fig:opt.H.RMS_fabs} an. Klar zu erkennen sind die Ausreißer bei den meisten Frequenzen der originalen Nulldurchgänge. Auch die bei höheren Frequenzen auftretenden größeren Fehlerterme fallen auf. Die Korrekturen sind an den auf 0 gezogenen Werten zu erkennen. Der Unterschied zwischen dem originalen und dem modifizierten RMS liegt hier bei ungefähr $0.04$, also nahezu $10 \%$.
Für die Einordnung der Größenordnung des Korrekturterms sei nochmals betont, dass aufgrund der Anpassung \eqref{eq:opt.Hnew} ein Korrekturterm von $\pm 0.5$ bei dem im Folgenden standardmäßig verwendeten $\sigma_H = \nicefrac{1}{2}$ zu einer Anpassung in $\Hcompl$ in Höhe eines Viertels des ursprünglichen Wertes führt.

\pgfplotstableread[col sep = comma] {f_abs_orig.csv} \fabsOrig 
\pgfplotstableread[col sep = comma] {f_abs_RMS_1.csv} \fabsRMSone 
\pgfplotstableread[col sep = comma] {f_abs_RMS_2.csv} \fabsRMStwo
\pgfplotstableread[col sep = comma] {f_abs_RMS_3.csv} \fabsRMSthree
\pgfplotstableread[col sep = comma] {H_RMS_orig.csv} \Horig
\pgfplotstableread[col sep = comma] {H_RMS_1.csv} \HrmsOne
\pgfplotstableread[col sep = comma] {H_RMS_2.csv} \HrmsTwo
\pgfplotstableread[col sep = comma] {H_RMS_3.csv} \HrmsThree
\begin{figure}[htb]
%\begin{center}
%\begin{subfigure}{\textwidth}
%%	\begin{center}
%    \begin{tikzpicture}
%\begin{axis}[
%		scale only axis,
%		width = 0.75 \textwidth,
%		height = 0.2 \textheight,
%		legend entries = {original, Step 1, Step 2, Step 3},
%		legend pos = north east,
%		legend columns = 3,
%		xlabel={Frequenz},
%		ylabel={Betrag Übertragung $\Hcompl$},
%		y label style={at={(axis description cs:-0.1,.5)},anchor=south},
%		%xtick distance = 10000000,
%%		xminorgrids,
%%		xmajorgrids,
%%		minor x tick num =3,
%		xtick pos = lower,
%		ytick pos = left,
%		xtick align = outside,
%		ytick align = outside,
%		scaled x ticks = base 10:-6,
%		xtick scale label code/.code={[MHz]},
%		xmin = 0,
%		xmax = 80000000,
%		ymin = 0,
%		%ymax = 0.03,
%		]
%		\addplot [black, sharp plot, mark size = 3pt] table [x index = 0, y index = 1] {\Horig};
%		\addplot [blue, sharp plot, mark size =3pt] table [x index = 0, y index = 1] {\HrmsOne};
%		\addplot [red, sharp plot, mark size =3pt] table [x index = 0, y index = 1] {\HrmsTwo};
%%		\addplot [red, sharp plot, mark size =3pt] table [x index = 0, y index = 1] {\HrmsThree};
%\end{axis}
%\end{tikzpicture}
%\caption{Entwicklung des Betrags der Übertragungsfunktion über mehrere Iterationen und im Anfangszustand} %alte Messung!
%	\label{fig:opt.H.RMS_H}
%%	\end{center}
%\end{subfigure}
%\\
%\begin{subfigure}{\textwidth}
%%\begin{center}
    \begin{tikzpicture}
\begin{axis}[
		scale only axis,
		width = 0.7 \textwidth,
		height = 0.17 \textheight,
		legend entries = {initial, angepasst},
		legend pos = south west,
		legend columns = 2,
		cycle list name = color list,
		xlabel={Frequenz},
		ylabel={Korrekturterm $\fabs$},
		y label style={at={(axis description cs:-0.1,.5)},anchor=south},
		%xtick distance = 10000000,
%		xminorgrids,
%%		xmajorgrids,
%		minor x tick num =3,
		xtick pos = lower,
		ytick pos = left,
		xtick align = outside,
		ytick align = outside,
		scaled x ticks = base 10:-6,
		xtick scale label code/.code={[MHz]},
		extra y ticks={0.5697153886590458, -0.5697153886590458
						},
			extra y tick labels={+RMS, -RMS},
		extra y tick style={grid=major, ytick pos=right, ytick align=outside, ticklabel pos=right},		
		xmin = 0,
		xmax = 80000000,
		ymin = -0.75,
		ymax = 0.75,
		]
		
		\addplot  table [x index = 0, y index = 1] {\fabsOrig};
		\addplot  table [x index = 0, y index = 1] {\fabsRMSone};
%		\addplot [red, sharp plot, mark size =3pt] table [x index = 0, y index = 1] {\fabsRMStwo};
%		\addplot [blue, sharp plot, mark size =3pt] table [x index = 0, y index = 1] {\fabsRMSthree};
\end{axis}
\end{tikzpicture}
\caption{Korrekturterm in initialer und in angepasster Form mit RMS- Korrektur, am Messaufbau}
\label{fig:opt.H.RMS_fabs}
%\end{center}
%\end{subfigure}
%\end{center}
%\caption[Ignorieren großer Korrektur-Terme]{Entwicklung von Übertragungsfunktion und Korrekturterm bei Beschränkung von $\fabs$ mit angepasstem RMS-Wert und Schrittweite $\sigma_H = \frac12$}
%\label{fig:opt.H.iteration}
\end{figure}



\subsection{Auswertung}
\label{subsec:opt.H.auswertung}
Eine erste Illustration bietet die Anwendung des beschriebenen Vorgehens auf das \mock-System. Hier lässt sich testen, ob unter der Annahme einer vorgegebenen Übertragungsfunktion  $\Hcompl_{ \mathrm{mock}}$ im Sinne des Hammerstein-Modells und unter Vernachlässigung der Kennlinie (entspricht einer als ideal berechneten Kennlinie) eine Verbesserung des Signals erreicht wird und wie sich die Fehlerquellen mit Ausnahme von Rauschen auswirken. 
Das Ergebnis dieses Tests ist eindeutig und anschaulich in \figref{subfig:opt.H.mock} aufgetragen für jeweils 15 Iterationen. Es lässt sich feststellen, dass eine Anpassung sowohl der Nulldurchgänge als auch bei hohen Frequenzen unbedingt notwendig ist und dies durch die Beschränkung großer Korrekturterme angegangen werden kann.\footnote{Die Auswirkungen der zusätzlichen Nutzung einer Beschränkung auf $3 \promille$ verbessern die Übertragungsfunktion optisch nicht wesentlich.}
Auch zeigen sich in direkter Umgebung der Nulldurchgänge noch Frequenzen, an denen die Korrekturterme zwar geringer als der modifizierte RMS, jedoch immer noch ungenau durch die Interpolation aufgelöst sind.
Diese Fehler sind ausschlaggebend dafür, dass sich die Qualität des erhaltenen Ausgangssignals über die Iterationen verschlechtert.\footnote{Je geringer der Wert des Qualitätskriteriums, desto ähnlicher ist das Signal einem idealen Einzelsinus.} Nichtsdestotrotz nähert sich die Übertragungsfunktion außerhalb der problematischen Nulldurchgänge gerade in niedrigeren Frequenzen sehr gut an $\Hcompl_{\mathrm{mock}}$ an. 
 
 
\pgfplotstableread[col sep = comma] {opt_mock_rms/H0.csv} \Hinit
\pgfplotstableread[col sep = comma] {H_a_mock.csv} \Hmock
%\pgfplotstableread[col sep = comma] {opt_mock_rms/H1.csv} \rmsHone
%\pgfplotstableread[col sep = comma] {opt_mock_rms/H5.csv} \rmsHfive
%\pgfplotstableread[col sep = comma] {opt_mock_rms/H10.csv} \rmsHten
\pgfplotstableread[col sep = comma] {opt_mock_rms/H15_3prom.csv} \rmsHprom
\pgfplotstableread[col sep = comma] {opt_mock_rms/H15.csv} \rmsHften
%\pgfplotstableread[col sep = comma] {opt_mock_simple/H1.csv} \simpleHone
%\pgfplotstableread[col sep = comma] {opt_mock_simple/H5.csv} \simpleHfive
%\pgfplotstableread[col sep = comma] {opt_mock_simple/H10.csv} \simpleHten
\pgfplotstableread[col sep = comma] {opt_mock_simple/H15.csv} \simpleHften

\begin{figure}[H]
\begin{subfigure}{\textwidth}
    \begin{tikzpicture}
\begin{axis}[
		scale only axis,
		width = 0.8 \textwidth,
		height = 0.2 \textheight,		
		legend style={ at={(0.5,1.1)},
				anchor=south},
		legend columns = 2,
		transpose legend, 
		cycle list name = color list,
		xlabel={Frequenz},
		ylabel={Betrag Übertragung $\Hcompl$},
		y label style={at={(axis description cs:-0.1,.5)},anchor=south},
		%xtick distance = 10000000,
%		xminorgrids,
%		xmajorgrids,
%		minor x tick num =3,
		xtick pos = lower,
		ytick pos = left,
		xtick align = outside,
		ytick align = outside,
		scaled x ticks = base 10:-6,
		xtick scale label code/.code={[MHz]},
		xmin = 0,
		xmax = 80000000,
		ymin = 0,
		ymax = 15,
		]	
		\addplot  table [x index = 0, y index = 1] {\Hmock}; \addlegendentry{$\Hcompl_{\mathrm{mock}}$}
		
		\addplot  table [x index = 0, y index = 1] {\Hinit}; \addlegendentry{$\Hcompl_{\mathrm{initial}}$}
		
%		\addplot [mark size =3pt] table [x index = 0, y index = 1] {\simpleHone}; \addlegendentry{Step 1 - keine Anpassung}
%		\addplot  table [x index = 0, y index = 1] {\simpleHfive}; \addlegendentry{Step 5 - keine Anpassung}
%		\addplot [mark size =3pt] table [x index = 0, y index = 1] {\simpleHten}; \addlegendentry{Step 10 - keine Anpassung}
		\addplot  table [x index = 0, y index = 1] {\simpleHften}; \addlegendentry{Step 15 - keine Anpassung}
		
%		\addplot [mark size =3pt] table [x index = 0, y index = 1] {\rmsHone}; \addlegendentry{Step 1 - mit RMS}
%		\addplot  table [x index = 0, y index = 1] {\rmsHfive}; \addlegendentry{Step 5 - mit RMS}
		\addplot  table [x index = 0, y index = 1] {\rmsHften}; \addlegendentry{Step 15 - mit RMS}	
		
		\addplot table [x index = 0, y index = 1] {\rmsHprom}; \addlegendentry{Step 15 - mit RMS und $3 \promille$}

		
\end{axis}
\end{tikzpicture}
\caption{Entwicklung des Betrags der Übertragungsfunktion des \mock-Systems mit und ohne Anpassung großer Korrekturterme}
	\label{subfig:opt.H.mock}
\end{subfigure}
\\
\begin{subfigure}{\textwidth}
\begin{tikzpicture}
	\begin{axis}[
		scale only axis,
		width = 0.4 \textwidth,
		height = 0.2 \textheight,		
		legend pos = outer north east,
		xlabel={Iterationsschritt},
		ylabel={Güte Ausgangssignal},
		xtick pos = lower,
		ytick pos = left,
		xtick align = outside,
		ytick align = outside,
		xmin = 0.2,
		xmax = 15.8,
		]	
		
	\addplot table [x expr=\coordindex+1, y index =0]{
		0.48726213261815016 
		0.49000679356347154
		0.5101194806372478
		0.5266448340807586
		0.5375527813320436
		0.5447108764910896
		0.5499773649457441
		0.5545875437451622
		0.5592582035611007
		0.5643452240862432
		0.5699343256576608
		0.5758895933658901
		0.5819216415470254
		0.5877172129682854
		0.5930401259075484
		}; \addlegendentry{Korrektur ohne Anpassung}	
	
	\addplot table [x expr=\coordindex+1, y index =0]{
		0.48726213261815016
		0.4759961751169583 
		0.4852543982854474 
		0.4954878236964439 
		0.5034038461346506 
		0.5086562700430226 
		0.5124108351861989 
		0.5156318810838099 
		0.517584903029569
		0.518781160261209
		0.5195457063994412 
		0.520135841606002
		0.5205906984373314 
		0.5210120621383171 
		0.5215551522937539
		}; \addlegendentry{Korrektur mit RMS}
		
		\addplot table [x expr=\coordindex+1, y index =0]{
		0.48726213261815016 
		0.4695879133604225
		0.47376436493301266 
		0.48016422206587334 
		0.4867655459013696 
		0.4916978026875846 
		0.49484977950838993 
		0.4971321854184491 
		0.4989985349115239 
		0.5021058501215027 
		0.5044349692747538 
		0.5061724548300056 
		0.5074558511064865 
		0.5085423753752575 
		0.5094731920124752
		}; \addlegendentry{Korrektur mit RMS und $3 \promille$ }			
	\end{axis}
\end{tikzpicture}
\caption{Entwicklung des Qualitätswertes des Ausgagnssigals über 15 Iterationen, mit und ohne Anpassung des Korrekturterms}
	\label{subfig:opt.H.mockQuality}
\end{subfigure}
\caption[Optimierung von $\Hcompl$ im \mock-System]{Anwendung der Optimierung von $\Hcompl$ mit und ohne Anpassung des Korrekturterms auf das \mock-System bei $\sigma_H = \nicefrac{1}{2}$}
\end{figure}


Nutzt man die Optimierung für $\Hcompl$ im nahezu linearen Bereich der Kennlinie $K$, übertragen sich die prinzipiellen Erkenntnisse aus dem \mock-System  auf die Kavität und es lässt sich auch der Einfluss von Rauschen auf die Optimierung feststellen. Genutzt wurde hierzu und im Folgenden eine Kennlinie, die mit einer Peak-to-Peak-Spannung des Eingangssignals von $\SI{0.6}{\volt}$ erzeugt wurde. Das Ausgangssignal wurde in der Optimierung mit $V_{pp} = \SI{0.6}{\volt}$ angesetzt, was zu etwa $\SI{60}{\milli\volt}$ Peak-to-Peak-Spannung am Eingang zurückgerechnet wird. 
\footnote{Hier sei auf die unter\textit{\nameref{subsubsec:opt.adjusta.problem}} beschriebene Problematik der nichtlinearen Vorverzerrung bei großen Unterschieden zwischen der Amplitude des momentan betrachteten Signals und des zur Berechnung von $K$ genutzten hingewiesen. Diese Erkenntnis lag zum Zeitpunkt der hier betrachteten Messung noch nicht in dieser Schärfe vor.}
Klar erkennbar sind wieder die Problematik der Nulldurchgänge und die Auflösung hoher Frequenzen.
Einen Überblick über die Variation der Parameter der Optimierung bietet \figref{fig:opt.H.parameter}. 


\pgfplotstableread[col sep = comma] {H_iteration/H_0.csv} \Hstart 
\pgfplotstableread[col sep = comma] {H_iteration/H_3prom.csv} \Hprom
\pgfplotstableread[col sep = comma] {H_iteration/H_RMS.csv} \Hrms 
\pgfplotstableread[col sep = comma] {H_iteration/H_RMS_3prom.csv} \Hrmsprom 
\pgfplotstableread[col sep = comma] {H_iteration/H_sigma0.2.csv} \Hsigma
\pgfplotstableread[col sep = comma] {H_iteration/H_simple1.csv} \HsimpleA %no number in name
\pgfplotstableread[col sep = comma] {H_iteration/H_simple2.csv} \HsimpleB

\begin{figure}[H]
\begin{center}
	\begin{subfigure}{\textwidth}
\begin{tikzpicture}
	\begin{axis}[
		scale only axis,
		width = 0.4 \textwidth,
		height = 0.2 \textheight,		
		%legend pos = outer north east,
		legend style={ at={(1.5,0.2)},
				anchor=south},	
		xlabel={Iterationsschritt},
		ylabel={Güte Ausgangssignal},
		xtick pos = lower,
		ytick pos = left,
		xtick align = outside,
		ytick align = outside,
		xmin = 0.2,
		xmax = 5.8,
		]	
		
	\addplot [blue, mark = x] table [col sep = comma, x expr=\coordindex+1, y index =1]{			
			ohne alles, an System
			QGesamt1,2.141630873256258
			QGesamt1,2.064664590979294
			QGesamt1,2.165397345084028
			QGesamt1,1.9985538231494953
			QGesamt1,2.031694877231738
		}; \label{plot:opt.H.simple_adjustA} \addlegendentry{Ohne Anpassungen Messung 1}	
	\addplot [blue, mark = +] table [col sep = comma, x expr=\coordindex+1, y index =1]{			
			ohne alles für Rauscheinfluss, data
			QGesamt1,2.135948818933304
			QGesamt1,2.1384178165733116
			QGesamt1,2.0464488222784887
			QGesamt1,1.9973886067241382
			QGesamt1,2.0689413165209682
		}; \label{plot:opt.H.simple_adjustB} \addlegendentry{Ohne Anpassungen Messung 2}	
	\addplot [black, mark = *] table [col sep = comma, x expr=\coordindex+1, y index =1]{			simple mit sigma 0.2, data
			QGesamt1,2.0536759283401667
			QGesamt1,2.025676659927534
			QGesamt1,2.035313029109188
		}; \addlegendentry{mit $\sigma_H = 0.2$}	
	\addplot [red, mark = x] table [col sep = comma, x expr=\coordindex+1, y index =1]{			
			mit RMS, data
			QGesamt1,2.0868523560789374
			QGesamt1,2.062575354725346
			QGesamt1,1.9691672211272198
			QGesamt1,2.0299124251901866
			QGesamt1,2.0676059254911445
		}; \addlegendentry{RMS}	
	\addplot [green, mark = x] table [col sep = comma, x expr=\coordindex+1, y index =1]{			
			mit 3 prom, data
			QGesamt1,2.2592583084704576
			QGesamt1,2.1266302971337128
			QGesamt1,2.0822550911677356
			QGesamt1,2.0910260802802365
			QGesamt1,2.0850568732052337
		}; \addlegendentry{ $3 \promille$ }	
	\addplot [yellow, mark = x] table [col sep = comma, x expr=\coordindex+1, y index =1]{			
			mit 3 prom und RMS, data
			QGesamt1,2.09716411420358
			QGesamt1,2.0352919493881836
			QGesamt1,2.0714614121252213
			QGesamt1,1.9686861551413428
			QGesamt1,2.120204783654906
		}; \addlegendentry{ RMS und $3 \promille$ }	
	
	\end{axis}
\end{tikzpicture}
	\caption{Entwicklung der Qualität des Ausgangssignals bei unterschiedlicher Parameterwahl}
	\label{subfig:opt.H.qualityOverview}
	\end{subfigure}
	\\
	\begin{subfigure}{\textwidth}
    \begin{tikzpicture}
\begin{axis}[
%		scale only axis,
%		width = 0.4 \textwidth,
%		height = 0.2 \textheight,		
%		legend style={ at={(1,1.1)},
%				anchor=south},
		legend columns = -1,
%		transpose legend, 
		legend entries={initial,
						ohne Anpassung, 
%						ohne Anpassung 2,
%						$\sigma_H = 0.2$ in Step 3,
						RMS,
						$3\promille$,
%						RMS und $3\promille$
						},
		legend to name=named,
		cycle list name = color list,
		xlabel={Frequenz},
		ylabel={Betrag Übertragung $\Hcompl$},
		y label style={at={(axis description cs:-0.1,.5)},anchor=south},
		xtick distance = 5000000,
		xtick pos = lower,
		ytick pos = left,
		xtick align = outside,
		ytick align = outside,
		scaled x ticks = base 10:-6,
		xtick scale label code/.code={[MHz]},
		xmin = 0,
		xmax = 20000000,
		ymin = 5,
		ymax = 13,
		]	
		\addplot [black] table [x index = 0, y index = 1] {\Hstart};
%		\addplot  table [x index = 0, y index = 1] {\HsimpleA};
		\addplot [blue] table [x index = 0, y index = 1] {\HsimpleB};
%		\addplot  table [x index = 0, y index = 1] {\Hsigma};
		\addplot [red] table [x index = 0, y index = 1] {\Hrms};
		\addplot [green] table [x index = 0, y index = 1] {\Hprom};
%		\addplot  table [x index = 0, y index = 1] {\Hrmsprom};
\end{axis}
\end{tikzpicture}
    \begin{tikzpicture}
\begin{axis}[
%		scale only axis,
%		width = 0.35 \textwidth,
%		height = 0.2 \textheight,		
		cycle list name = color list,
		xlabel={Frequenz},
		ylabel={Betrag Übertragung $\Hcompl$},
		y label style={at={(axis description cs:-0.1,.5)},anchor=south},
		xtick distance = 5000000,
		xtick pos = lower,
		ytick pos = left,
		xtick align = outside,
		ytick align = outside,
		scaled x ticks = base 10:-6,
		xtick scale label code/.code={[MHz]},
		xmin = 60000000,
		xmax = 80000000,
		ymin = 0,
		ymax = 5,
		]	
		\addplot [black] table [x index = 0, y index = 1] {\Hstart};
%		\addplot  table [x index = 0, y index = 1] {\HsimpleA};
		\addplot [blue] table [x index = 0, y index = 1] {\HsimpleB};
%		\addplot  table [x index = 0, y index = 1] {\Hsigma};
		\addplot [red] table [x index = 0, y index = 1] {\Hrms};
		\addplot [green] table [x index = 0, y index = 1] {\Hprom};
%		\addplot  table [x index = 0, y index = 1] {\Hrmsprom};
\end{axis}
\end{tikzpicture}

\ref{named}
	\caption{Einfluss unterschiedlicher Parameterwahl auf Entwicklung der Übertragungsfunktion bei hohen Frequenzen, nach 5 Iterationen}
	\label{subfig:opt.H.paramHoverview}
	\end{subfigure}
\caption{Einfluss unterschiedlicher Parameterwahl auf Entwicklung der Qualität des Ausgangssignals und die Übertragungsfunktion}
\label{fig:opt.H.parameter}
\end{center}
\end{figure}


Klar erkennbar ist aus \figref{subfig:opt.H.qualityOverview} vor allem, dass der Einfluss zufälliger Schwankungen auf die Qualität des Signals enorm ist, man vergleiche hierzu nur die unter identischen Voraussetzungen und um wenige Minuten verzögert entstandenen Messungen mit \ref{plot:opt.H.simple_adjustA} und \ref{plot:opt.H.simple_adjustB}. 
Weiterhin lässt sich im Vergleich mit  \figref{subfig:opt.H.mockQuality} die wenig verwunderliche Aussage treffen, dass die Qualität aller Ausgangssignale wesentlich schlechter ist, als dies beim \mock-System der Fall war. 
Diese nicht eindeutige Änderung der Qualität ist insofern verwunderlich, als dass auch die starke Anpassung über die Iteration bei in $\Hcompl$ stark verstärkten niedrigen Frequenzen nicht nennenswert auf die Qualität auswirkt.
Und letztlich ist über die hier vorgenommenen Iterationsschritte keine eindeutige Verbesserung des Ausgangssignals zu erkennen. Da gleichzeitig jedoch auch keine enorme Verschlechterung eintritt und der Rauscheinfluss nicht beziffert werden kann, ist eine qualitative Aussage über die einzelne Optimierung von $\Hcompl$ mit den vorliegenden Daten für die vorgenommenen Anpassungen nicht zu treffen.
Festhalten jedoch lässt sich auch bei Messungen am Messaufbau, dass die vorgenommenen Anpassungen und insbesondere die Beschränkung mit einem modifizierten RMS wesentlich weniger als abwegig angesehene Korrekturterme erlauben und so eine kleinschrittigere Anpassung möglich machen. 


\section{Optimierung der nichtlinearen Kennlinie $K$}
\label{sec:opt.K}
Die erste Kennlinie wird über ein linear vorverzerrtes Signal bestimmt. Dabei ist es möglich, dass sich die Kennlinie für ein nichtlinear vorverzerrtes Signal oder für eine Eingangssignal mit verändertem Frequenzspektrum ändert. \\
Für die Anpassung der Kennlinie $K$ werden die beiden berechneten Signal $\Uquest_{, \mathrm{meas}} \oft$ und $\Uquest_{, \mathrm{ideal}} \oft$ mit einander verglichen. Diese Signale lassen sich wie folgt aus der Parametrisierung der nichtlinearen Kennlinie als Potenzreihe berechnen.
\begin{align}
	\Uquest_{, \mathrm{meas}} \oft = \sum_{n=1}^N \, \overline{a}_n \left[ U_{in}(t) \right]^n
	\quad
	\Uquest_{, \mathrm{ideal}} \oft = \sum_{n=1}^N \, a_n \left[ U_{in}(t) \right]^n
\end{align}
Dabei sind $\overline{a}_n$ die Koeffizienten der neuen Kennlinie. Diese können mit der schon aus \cite{harzheim} bekannten Matrixmultiplikation berechnet werden. Bei diesem Verfahren werden einzelne Sample des Eingangssignals $\Uin$ und der Größe $\Uquest$ verglichen und das entstehende lineare Optimierungsproblem mit der Methode der kleinsten Quadrate gelöst. \\
Für eine Optimierung ist es notwendig eine Schrittweiter zu definieren. Die Differenz der beiden Signale $\Uquest_{, \mathrm{meas}} \oft$ und $\Uquest_{, \mathrm{ideal}} \oft$ wird ebenfalls durch eine Potenzreihe der nichtlinearen Kennlinie parametrisiert.
\begin{align}
\label{eq:opt.deltaUquest}
	\Delta \Uquest \oft = \Uquest_{, \mathrm{meas}} \oft - \Uquest_{, \mathrm{ideal}} \oft
	=
	\sum_{n=1}^N \, \left( \overline{a}_n -  a_n\right) \left[ U_{in}(t) \right]^n
	=
	\sum_{n=1}^N \, \tilde{a}_n \, \left[ U_{in}(t) \right]^n	
\end{align}
Dafür können die Koeffizienten $\tilde{a}_n$ direkt ohne die Berechnung von $\overline{a}_n$ bestimmt werden. Die Berechnung der Koeffizienten $\tilde{a}_n$ stellt ebenso ein lineares Optimierungsproblem dar. Dabei werden $M$ Samples von ${\Delta \Uquest_{,i} = \Delta \Uquest (i \cdot \Delta t)}$ mit zugehörigen Samples des Eingangssignals ${\Uin_{,i} = \Uin (i \cdot \Delta t)}$ verglichen. Dieses Lösungsverfahren wird in \cite{harzheim} vorgestellt. Mit der Potenzreihe aus \eqref{eq:opt.deltaUquest} ergibt sich folgendes Gleichungssystem
\begin{align}
	\left( 
	\begin{matrix}
	 	\Uin_{,1} & \Uin_{,1}^2 & \dots & \Uin_{,1}^N \\
		\Uin_{,2} & \Uin_{,2}^2 & \dots & \Uin_{,2}^N \\
		\vdots & \vdots & \ddots & \vdots \\
		\Uin_{,M} & \Uin_{,M}^2 & \dots & \Uin_{,M}^N \\
	\end{matrix}
	\right)
	\cdot
	\left(
	\begin{matrix}
		\tilde{a}_1 \\
		\tilde{a}_2 \\
		\vdots \\
		\tilde{a}_N \\	 
	\end{matrix}
	\right) = \left( 
	\begin{matrix}
		\Delta \Uquest_{,1} \\
		\Delta \Uquest_{,2} \\
		\vdots \\
		\Delta \Uquest_{,M} \\	 
	\end{matrix}
	\right)
	\label{eq:Uquest.Gleichungssystem}
\end{align}
Dieses Gleichungssystem ist mit normalerweise $M>N$ überbestimmt und wird mit der Methode der kleinsten Quadrate gelöst. Die Koeffizienten $\tilde{a}_n$ werden nun wie folgt zur Anpassung der Koeffizienten $a_n$ verwendet
\begin{align}
	\label{eq:opt.adjusta}
	a_n^{i+1} = a_n^{i} + \sigma_{a}^{i} \tilde{a}_n^{i}
\end{align}
Für die Schrittweite gilt $\sigma_{a}^i \in \left[ 0 , 1 \right]$. Der Ablauf des Algorithmus ist in \figref{fig:Algorithmus.K} gezeigt. Dabei stellen die grünen Schritt die Initialisierung dar, blaue Schritte sind Berechnungen oder Messungen von Signalen und im roten Block findet die eigentliche Anpassung der Kennlinie statt. Durchgezogene Pfeile folgen dem Programmablauf und gestrichelte Pfeile stellen Parameterübergaben dar.

\begin{figure}[H]
\centering
\tikzstyle{decision} = [diamond, draw, fill=blue!20, 
    text width=4.5em, text badly centered, node distance=3cm, inner sep=0pt]
\tikzstyle{block_g} = [rectangle, draw, fill=green!20, 
    text width=6.5em, text centered, rounded corners, node distance=2cm, minimum height=4em]
\tikzstyle{block_b} = [rectangle, draw, fill=blue!20, 
    text width=6em, text centered, rounded corners, node distance=4.5cm, minimum height=4em]
\tikzstyle{block_r} = [rectangle, draw, fill=red!20, 
    text width=6em, text centered, rounded corners, node distance=4.5cm, minimum height=4em]
\tikzstyle{line} = [draw, -latex']
\tikzstyle{cloud} = [draw, ellipse,fill=red!20, node distance=3cm,
    minimum height=2em]
\begin{tikzpicture}[node distance = 5cm, auto]
    % Place nodes
    \node [block_g] (ideal) {$\Uout_{,\textrm{ideal}}$ festlegen};
    \node [block_g, below of=ideal] (H) {$\Hcompl$ bestimmen};
    \node [block_g, below of=H] (K) {$K_0$ bestimmen \newline Ref.: \newline ${V_{PP,?} = \SI{578}{\mV}}$};
    \node [block_b, right of=H] (Uquest) {$\Uquest_{,\textrm{ideal}}$ berechnen};
    \node [block_b, below of=Uquest] (Uin) {$\Uin$ mit $K_i$ berechnen};
    \node [block_r, right of=Uquest] (adjust) {$a_n$ anpassen $K_i$ neu berechnen};
    \node [block_b, right of=adjust] (Uquest_meas) {$\Uquest_{,\textrm{meas}}$ berechnen};
    \node [block_b, below of=Uquest_meas] (Uout) {$\Uout$ messen};
    
    % Draw edges
    \path [line, thick] (ideal) --  (H);
    \path [line, dashed] (H) -- node {$\Hcompl$} (Uquest);
    \path [line, thick] (H) -- (K);
    \path [line, thick] (K) -- node {$K_0$} (Uin);
    \path [line, thick] (Uin) -- (Uout);
    \path [line, dashed] (Uin) -- ++(45mm,0) -- ++(0,20mm) node [xshift=1em] {$\Uin$} -- (adjust);
    \path [line, thick] (Uout) -- (Uquest_meas);
    \path [line, thick] (Uquest_meas) -- node {$\Uquest_{,\textrm{meas}}$} (adjust);
    \path [line, thick] (adjust) -- (Uquest);
    \path [line, dashed] (adjust) -- node {$K_i$} (Uin);
    \path [line, thick] (Uquest) -- (Uin);
    \path [line, dashed] (Uquest) -- ++(0,15mm) -- node {$\Uquest_{,\textrm{ideal}}$} ++(45mm,0) -- (adjust);
\end{tikzpicture}
\caption{Algorithmus zur Optimierung von $K$}
  	\label{fig:Algorithmus.K}
\end{figure}

\subsection{Erster Ansatz}
\label{subsec:opt.adjusta.results}
Für die Berechnung der ersten Kennlinie $K_0$ wurde das ideale Ausgangssignal $\Uout_{, \mathrm{ideal}}$ mit $V_{PP} = \SI{6}{\V}$ über $\Hcompl^{-1}$ zurückgerechnet und als Eingangssignal verwendet $\Uin_{, \mathrm{initial }} = \Uquest_{, \mathrm{ideal}}$. Dieser Wert von $V_{PP} = \SI{6}{\V}$ wurde nicht gemessen, sondern lediglich zur Berechnung des ersten Eingangssignals verwendet, damit die Kennlinie im nichtlinearen Spannungsbereich des System berechnet werden kann. Für das berechnete Eingangssignal ergibt sich $V_{PP} \approx \SI{587}{\mV}$.\\
Als erster Ansatz wurde versucht $K$ über das selbe $\Uquest_{, \mathrm{ideal}}$ zu optimieren, mit dem es zuvor zur Berechnung verwendet wurde. Das heißt, dass die beiden Signale die gleiche Amplitude haben. Beim Zurückrechnen über die Kennlinie wurde festgestellt, dass $K$ in diesem Bereich nicht mehr bijektiv ist. Für den Fall, dass das Signal zu groß für den gültigen Bereich ist wird es auf beliebig gewählte $80\%$ des maximal zulässigen Wertes gesetzt. Zu diesem Zeitpunkt war schon bekannt, dass die Kennlinie nicht für $100\%$ ihres gültigen Bereiches gültig ist, deshalb wurde ein kleinerer Wert gesetzt. Dabei könnten sich in jedem Iterationsschritt unterschiedliche Amplituden von $\Uquest$ ergeben. Ziel hierbei ist es den Bereich, in dem $K$ bijektiv ist, schrittweise zu vergrößern.
\subsubsection*{Ergebnisse und Erkenntnisse}
\label{subsubsec:opt.adjusta.results}
Das Ergebnis ist eine Schwankung der Qualität siehe \figref{fig:evaluate30Q}. Die Kennlinien, zu den markierten Iterationen sind in \figref{fig:evaluate30K} eingezeichnet. Zu erkennen ist eine alternierende Anpassung der Kennlinie, die schon in den ersten drei Iterationen angefangen hat. Ergänzend wurden auch die letzten beiden Kennlinien eingezeichnet, um die große Anpassung innerhalb einer Iteration zu verdeutlichen. Ursache dafür könnte sein, dass die Kennlinie in einen nicht mehr sinnvollen Bereich genutzt wurde, um das Eingangssignal zu berechnen. Die Details dazu folgen im nächsten Abschitt. Dadurch könnten zu große Änderungen in einem Iterationsschritt vorgenommen worden sein, die nicht mehr zielführend waren.\\
Ein weiteres Ergebnis ist auch, dass das hier verwendete Eingangssignal nicht die zur Berechnung verwendeten $V_{PP} = \SI{6}{\V}$ über dem Gapspannungsteiler erreicht. Mit dem daraus berechneten Eingangssignal lassen sich $V_{PP} = \SI{4,7}{\V}$ über dem Gapspannungsteiler erreichen, deshalb sollte im nächsten Ansatz zur Optimierung von einem Ausgangssignal mit höchstens diesem Wert von $V_{PP}$ ausgegangen werden. Diese Wahl der Amplitunden am Ausgang dienen lediglich der Referenz auf einen Wert. Man kann analog dazu auch die Amplitude des Eingangssignal festlegen und damit als Referenz arbeiten.\\
Im nachfolgenden Diagramm der Güte wurden die Iterationen 1, 2, 3, 29 und 30 anders markiert, weil die dazu gehörigen Kennlinien $K_i$ im nebenstehenden Diagramm eingezeichnet wurden. Die anderen Kennlinien hatten sich über dem Verlauf der Messung den Kennlinien $K_{29}$ und $K_{30}$ abwechselnd angenähert.
\begin{figure}[H]
\begin{subfigure}{0.5 \textwidth}
    \newlength\figureheight
	\newlength\figurewidth
	\setlength\figureheight{7.5cm}
	\setlength\figurewidth{7.5cm}
    % This file was created by matplotlib2tikz v0.6.17.
\begin{tikzpicture}

\definecolor{color0}{rgb}{0.12156862745098,0.466666666666667,0.705882352941177}
\definecolor{color1}{rgb}{1,0.498039215686275,0.0549019607843137}
\definecolor{color2}{rgb}{0.172549019607843,0.627450980392157,0.172549019607843}
\definecolor{color3}{rgb}{0.83921568627451,0.152941176470588,0.156862745098039}
\definecolor{color4}{rgb}{0.580392156862745,0.403921568627451,0.741176470588235}
\definecolor{color5}{rgb}{0,0,0}

\begin{axis}[
xlabel={$U_{in}$ in \si{\milli \volt}},
ylabel={$U_{?}$ in \si{\milli \volt}},
xmin=-330, xmax=330,
ymin=-331.26776831825, ymax=277.36020454125,
width=\figurewidth,
height=\figureheight,
tick align=outside,
tick pos=left,
x grid style={white!69.01960784313725!black},
y grid style={white!69.01960784313725!black},
legend style={at={(0.03,0.97)}, anchor=north west, draw=white!80.0!black},
legend cell align={left},
legend entries={{initial},{1},{2},{3},{29},{30}}
]
\addlegendimage{no markers, color5}
\addlegendimage{no markers, color0}
\addlegendimage{no markers, color1}
\addlegendimage{no markers, color2}
\addlegendimage{no markers, color3}
\addlegendimage{no markers, color4}
\addplot [semithick, color5]
table {%
-300 -192.694568057
-298 -192.255682485
-296 -191.79910167
-294 -191.324954187
-292 -190.833368609
-290 -190.32447351
-288 -189.798397462
-286 -189.25526904
-284 -188.695216817
-282 -188.118369367
-280 -187.524855262
-278 -186.914803078
-276 -186.288341387
-274 -185.645598763
-272 -184.986703779
-270 -184.311785009
-268 -183.620971027
-266 -182.914390406
-264 -182.192171719
-262 -181.454443541
-260 -180.701334444
-258 -179.932973003
-256 -179.14948779
-254 -178.35100738
-252 -177.537660346
-250 -176.709575261
-248 -175.866880699
-246 -175.009705234
-244 -174.138177439
-242 -173.252425888
-240 -172.352579154
-238 -171.438765811
-236 -170.511114432
-234 -169.569753591
-232 -168.614811862
-230 -167.646417818
-228 -166.664700032
-226 -165.669787078
-224 -164.66180753
-222 -163.640889961
-220 -162.607162945
-218 -161.560755056
-216 -160.501794866
-214 -159.430410949
-212 -158.34673188
-210 -157.250886231
-208 -156.143002576
-206 -155.023209489
-204 -153.891635544
-202 -152.748409312
-200 -151.59365937
-198 -150.427514289
-196 -149.250102643
-194 -148.061553007
-192 -146.861993953
-190 -145.651554055
-188 -144.430361886
-186 -143.198546021
-184 -141.956235033
-182 -140.703557494
-180 -139.44064198
-178 -138.167617063
-176 -136.884611317
-174 -135.591753316
-172 -134.289171632
-170 -132.97699484
-168 -131.655351513
-166 -130.324370225
-164 -128.984179549
-162 -127.634908059
-160 -126.276684329
-158 -124.909636931
-156 -123.53389444
-154 -122.149585429
-152 -120.756838471
-150 -119.35578214
-148 -117.94654501
-146 -116.529255655
-144 -115.104042647
-142 -113.67103456
-140 -112.230359969
-138 -110.782147445
-136 -109.326525564
-134 -107.863622898
-132 -106.393568022
-130 -104.916489508
-128 -103.43251593
-126 -101.941775862
-124 -100.444397877
-122 -98.9405105491
-120 -97.4302424516
-118 -95.9137221581
-116 -94.391078242
-114 -92.8624392771
-112 -91.3279338367
-110 -89.7876904946
-108 -88.2418378241
-106 -86.6905043989
-104 -85.1338187926
-102 -83.5719095786
-100 -82.0049053305
-98 -80.432934622
-96 -78.8561260264
-94 -77.2746081175
-92 -75.6885094687
-90 -74.0979586536
-88 -72.5030842457
-86 -70.9040148187
-84 -69.300878946
-82 -67.6938052012
-80 -66.0829221578
-78 -64.4683583895
-76 -62.8502424697
-74 -61.228702972
-72 -59.60386847
-70 -57.9758675372
-68 -56.3448287472
-66 -54.7108806735
-64 -53.0741518897
-62 -51.4347709693
-60 -49.7928664859
-58 -48.148567013
-56 -46.5020011242
-54 -44.8532973931
-52 -43.2025843931
-50 -41.5499906979
-48 -39.895644881
-46 -38.2396755159
-44 -36.5822111762
-42 -34.9233804355
-40 -33.2633118673
-38 -31.6021340452
-36 -29.9399755426
-34 -28.2769649332
-32 -26.6132307906
-30 -24.9489016882
-28 -23.2841061996
-26 -21.6189728984
-24 -19.9536303581
-22 -18.2882071523
-20 -16.6228318545
-18 -14.9576330383
-16 -13.2927392773
-14 -11.6282791449
-12 -9.96438121478
-10 -8.30117406044
-8 -6.63878625545
-6 -4.97734637336
-4 -3.31698298772
-2 -1.65782467208
0 0
2 1.65636245498
4 3.31113411929
6 4.96418641939
8 6.61539078173
10 8.26461863275
12 9.9117413989
14 11.5566305066
16 13.1991573824
18 14.8391934526
20 16.4766101438
22 18.1112788823
24 19.7430710946
26 21.3718582072
28 22.9975116465
30 24.6199028389
32 26.238903211
34 27.8543841891
36 29.4662171997
38 31.0742736693
40 32.6784250243
42 34.2785426911
44 35.8744980962
46 37.466162666
48 39.053407827
50 40.6361050056
52 42.2141256284
54 43.7873411216
56 45.3556229119
58 46.9188424255
60 48.476871089
62 50.0295803289
64 51.5768415715
66 53.1185262433
68 54.6545057708
70 56.1846515804
72 57.7088350985
74 59.2269277517
76 60.7388009663
78 62.2443261688
80 63.7433747856
82 65.2358182432
84 66.7215279681
86 68.2003753867
88 69.6722319254
90 71.1369690106
92 72.594458069
94 74.0445705267
96 75.4871778105
98 76.9221513465
100 78.3493625615
102 79.7686828816
104 81.1799837335
106 82.5831365436
108 83.9780127382
110 85.364483744
112 86.7424209872
114 88.1116958944
116 89.472179892
118 90.8237444064
120 92.1662608641
122 93.4996006916
124 94.8236353153
126 96.1382361616
128 97.443274657
130 98.7386222279
132 100.024150301
134 101.299730302
136 102.565233658
138 103.820531796
140 105.065496141
142 106.299998121
144 107.523909161
146 108.737100688
148 109.939444129
150 111.13081091
152 112.311072457
154 113.480100197
156 114.637765557
158 115.783939962
160 116.91849484
162 118.041301616
164 119.152231718
166 120.251156571
168 121.337947602
170 122.412476237
172 123.474613904
174 124.524232028
176 125.561202036
178 126.585395354
180 127.596683408
182 128.594937626
184 129.580029434
186 130.551830257
188 131.510211523
190 132.455044658
192 133.386201089
194 134.303552241
196 135.206969541
198 136.096324417
200 136.971488293
202 137.832332597
204 138.678728756
206 139.510548195
208 140.32766234
210 141.12994262
212 141.917260459
214 142.689487284
216 143.446494522
218 144.1881536
220 144.914335943
222 145.624912978
224 146.319756132
226 146.998736831
228 147.661726501
230 148.308596569
232 148.939218462
234 149.553463605
236 150.151203426
238 150.73230935
240 151.296652804
242 151.844105215
244 152.374538009
246 152.887822613
248 153.383830452
250 153.862432954
252 154.323501545
254 154.766907651
256 155.192522699
258 155.600218115
260 155.989865325
262 156.361335757
264 156.714500836
266 157.049231989
268 157.365400642
270 157.662878223
272 157.941536156
274 158.20124587
276 158.441878789
278 158.663306342
280 158.865399953
282 159.04803105
284 159.211071059
286 159.354391406
288 159.477863518
290 159.581358822
292 159.664748743
294 159.727904709
296 159.770698145
298 159.793000478
};
\addplot [semithick, color0]
table {%
-300 -246.946115979
-298 -245.611994829
-296 -244.270382281
-294 -242.921345248
-292 -241.564950642
-290 -240.201265373
-288 -238.830356354
-286 -237.452290497
-284 -236.067134713
-282 -234.674955914
-280 -233.275821011
-278 -231.869796917
-276 -230.456950543
-274 -229.0373488
-272 -227.6110586
-270 -226.178146856
-268 -224.738680478
-266 -223.292726379
-264 -221.84035147
-262 -220.381622663
-260 -218.916606869
-258 -217.445371001
-256 -215.967981969
-254 -214.484506686
-252 -212.995012064
-250 -211.499565013
-248 -209.998232446
-246 -208.491081275
-244 -206.97817841
-242 -205.459590765
-240 -203.935385249
-238 -202.405628776
-236 -200.870388257
-234 -199.329730604
-232 -197.783722728
-230 -196.23243154
-228 -194.675923954
-226 -193.11426688
-224 -191.547527229
-222 -189.975771915
-220 -188.399067848
-218 -186.81748194
-216 -185.231081103
-214 -183.639932249
-212 -182.044102289
-210 -180.443658134
-208 -178.838666698
-206 -177.22919489
-204 -175.615309624
-202 -173.99707781
-200 -172.374566361
-198 -170.747842188
-196 -169.116972203
-194 -167.482023317
-192 -165.843062443
-190 -164.200156491
-188 -162.553372374
-186 -160.902777003
-184 -159.248437291
-182 -157.590420147
-180 -155.928792486
-178 -154.263621217
-176 -152.594973253
-174 -150.922915506
-172 -149.247514887
-170 -147.568838307
-168 -145.886952679
-166 -144.201924915
-164 -142.513821925
-162 -140.822710622
-160 -139.128657918
-158 -137.431730723
-156 -135.73199595
-154 -134.029520511
-152 -132.324371316
-150 -130.616615279
-148 -128.90631931
-146 -127.193550321
-144 -125.478375224
-142 -123.760860931
-140 -122.041074353
-138 -120.319082402
-136 -118.59495199
-134 -116.868750028
-132 -115.140543429
-130 -113.410399103
-128 -111.678383962
-126 -109.944564919
-124 -108.209008885
-122 -106.471782771
-120 -104.732953489
-118 -102.992587952
-116 -101.25075307
-114 -99.507515755
-112 -97.7629429194
-110 -96.0171014746
-108 -94.2700583323
-106 -92.5218804041
-104 -90.7726346017
-102 -89.0223878369
-100 -87.2712070213
-98 -85.5191590666
-96 -83.7663108846
-94 -82.0127293869
-92 -80.2584814852
-90 -78.5036340912
-88 -76.7482541166
-86 -74.9924084731
-84 -73.2361640724
-82 -71.4795878262
-80 -69.7227466462
-78 -67.9657074441
-76 -66.2085371316
-74 -64.4513026204
-72 -62.6940708222
-70 -60.9369086486
-68 -59.1798830114
-66 -57.4230608223
-64 -55.6665089929
-62 -53.910294435
-60 -52.1544840603
-58 -50.3991447804
-56 -48.644343507
-54 -46.8901471519
-52 -45.1366226268
-50 -43.3838368433
-48 -41.6318567131
-46 -39.8807491479
-44 -38.1305810595
-42 -36.3814193595
-40 -34.6333309597
-38 -32.8863827716
-36 -31.1406417071
-34 -29.3961746777
-32 -27.6530485953
-30 -25.9113303715
-28 -24.171086918
-26 -22.4323851465
-24 -20.6952919686
-22 -18.9598742962
-20 -17.2261990408
-18 -15.4943331142
-16 -13.7643434281
-14 -12.0362968942
-12 -10.3102604241
-10 -8.58630092963
-8 -6.86448532239
-6 -5.14488051412
-4 -3.42755341649
-2 -1.71257094122
0 0
2 1.71009249548
4 3.41763963352
6 5.12257450242
8 6.82483019049
10 8.52433978603
12 10.2210363773
14 11.9148530527
16 13.6057229005
18 15.293579009
20 16.9783544664
22 18.6599823612
24 20.3383957815
26 22.0135278158
28 23.6853115522
30 25.3536800791
32 27.0185664849
34 28.6799038578
36 30.3376252861
38 31.9916638581
40 33.6419526622
42 35.2884247865
44 36.9310133195
46 38.5696513495
48 40.2042719647
50 41.8348082534
52 43.461193304
54 45.0833602047
56 46.7012420439
58 48.3147719098
60 49.9238828908
62 51.5285080752
64 53.1285805513
66 54.7240334073
68 56.3147997316
70 57.9008126125
72 59.4820051382
74 61.0583103972
76 62.6296614776
78 64.1959914678
80 65.7572334562
82 67.3133205309
84 68.8641857804
86 70.4097622928
88 71.9499831566
90 73.48478146
92 75.0140902913
94 76.5378427388
96 78.0559718909
98 79.5684108358
100 81.0750926618
102 82.5759504573
104 84.0709173105
106 85.5599263098
108 87.0429105434
110 88.5198030997
112 89.9905370669
114 91.4550455334
116 92.9132615874
118 94.3651183174
120 95.8105488115
122 97.249486158
124 98.6818634454
126 100.107613762
128 101.526670196
130 102.938965835
132 104.344433769
134 105.743007085
136 107.134618871
138 108.519202216
140 109.896690209
142 111.267015937
144 112.630112489
146 113.985912953
148 115.334350417
150 116.67535797
152 118.0088687
154 119.334815696
156 120.653132045
158 121.963750836
160 123.266605157
162 124.561628097
164 125.848752744
166 127.127912186
168 128.399039511
170 129.662067808
172 130.916930166
174 132.163559671
176 133.401889413
178 134.631852481
180 135.853381961
182 137.066410943
184 138.270872515
186 139.466699765
188 140.653825782
190 141.832183653
192 143.001706468
194 144.162327314
196 145.31397928
198 146.456595453
200 147.590108923
202 148.714452778
204 149.829560106
206 150.935363994
208 152.031797533
210 153.118793809
212 154.196285911
214 155.264206928
216 156.322489948
218 157.371068058
220 158.409874348
222 159.438841906
224 160.457903819
226 161.466993177
228 162.466043067
230 163.454986579
232 164.433756799
234 165.402286817
236 166.360509721
238 167.308358599
240 168.245766539
242 169.17266663
244 170.08899196
246 170.994675617
248 171.88965069
250 172.773850266
252 173.647207435
254 174.509655285
256 175.361126903
258 176.201555378
260 177.030873799
262 177.849015254
264 178.65591283
266 179.451499617
268 180.235708703
270 181.008473176
272 181.769726123
274 182.519400635
276 183.257429798
278 183.983746701
280 184.698284433
282 185.400976082
284 186.091754735
286 186.770553482
288 187.437305411
290 188.09194361
292 188.734401167
294 189.364611171
296 189.982506709
298 190.588020871
300 191.181086744
};
\addplot [semithick, color1]
table {%
-300 -194.738097639
-298 -194.402448855
-296 -194.046497468
-294 -193.670393107
-292 -193.274285404
-290 -192.858323986
-288 -192.422658485
-286 -191.967438528
-284 -191.492813747
-282 -190.998933771
-280 -190.485948228
-278 -189.95400675
-276 -189.403258965
-274 -188.833854503
-272 -188.245942993
-270 -187.639674066
-268 -187.015197351
-266 -186.372662478
-264 -185.712219075
-262 -185.034016773
-260 -184.338205202
-258 -183.624933991
-256 -182.894352769
-254 -182.146611166
-252 -181.381858812
-250 -180.600245337
-248 -179.801920369
-246 -178.987033539
-244 -178.155734477
-242 -177.308172811
-240 -176.444498172
-238 -175.564860189
-236 -174.669408491
-234 -173.758292709
-232 -172.831662471
-230 -171.889667409
-228 -170.93245715
-226 -169.960181325
-224 -168.972989563
-222 -167.971031495
-220 -166.954456749
-218 -165.923414955
-216 -164.878055743
-214 -163.818528742
-212 -162.744983582
-210 -161.657569893
-208 -160.556437305
-206 -159.441735446
-204 -158.313613946
-202 -157.172222436
-200 -156.017710544
-198 -154.850227901
-196 -153.669924135
-194 -152.476948877
-192 -151.271451756
-190 -150.053582402
-188 -148.823490445
-186 -147.581325513
-184 -146.327237237
-182 -145.061375246
-180 -143.783889169
-178 -142.494928638
-176 -141.19464328
-174 -139.883182726
-172 -138.560696605
-170 -137.227334547
-168 -135.883246181
-166 -134.528581138
-164 -133.163489046
-162 -131.788119535
-160 -130.402622236
-158 -129.007146777
-156 -127.601842788
-154 -126.186859899
-152 -124.762347739
-150 -123.328455938
-148 -121.885334126
-146 -120.433131931
-144 -118.971998985
-142 -117.502084916
-140 -116.023539354
-138 -114.536511929
-136 -113.04115227
-134 -111.537610007
-132 -110.02603477
-130 -108.506576187
-128 -106.979383889
-126 -105.444607506
-124 -103.902396666
-122 -102.352901
-120 -100.796270137
-118 -99.2326537066
-116 -97.6622013387
-114 -96.0850626628
-112 -94.5013873085
-110 -92.9113249054
-108 -91.3150250831
-106 -89.7126374712
-104 -88.1043116993
-102 -86.4901973971
-100 -84.8704441941
-98 -83.24520172
-96 -81.6146196044
-94 -79.9788474768
-92 -78.338034967
-90 -76.6923317044
-88 -75.0418873188
-86 -73.3868514397
-84 -71.7273736968
-82 -70.0636037195
-80 -68.3956911377
-78 -66.7237855808
-76 -65.0480366786
-74 -63.3685940605
-72 -61.6856073562
-70 -59.9992261953
-68 -58.3096002075
-66 -56.6168790223
-64 -54.9212122694
-62 -53.2227495784
-60 -51.5216405788
-58 -49.8180349003
-56 -48.1120821725
-54 -46.403932025
-52 -44.6937340874
-50 -42.9816379894
-48 -41.2677933605
-46 -39.5523498304
-44 -37.8354570286
-42 -36.1172645849
-40 -34.3979221287
-38 -32.6775792897
-36 -30.9563856975
-34 -29.2344909817
-32 -27.512044772
-30 -25.7891966979
-28 -24.0660963891
-26 -22.3428934752
-24 -20.6197375857
-22 -18.8967783503
-20 -17.1741653986
-18 -15.4520483603
-16 -13.7305768648
-14 -12.0099005419
-12 -10.2901690212
-10 -8.57153193219
-8 -6.85413890457
-6 -5.13813956793
-4 -3.42368355188
-2 -1.71092048603
0 0
2 1.7089282766
4 3.41571471417
6 5.12020968309
8 6.82226355374
10 8.52172669653
12 10.2184494818
14 11.91228228
16 13.6030754615
18 15.2906793967
20 16.974944456
22 18.6557210097
24 20.3328594283
26 22.0062100821
28 23.6756233415
30 25.3409495769
32 27.0020391588
34 28.6587424574
36 30.3109098433
38 31.9583916867
40 33.601038358
42 35.2387002277
44 36.8712276661
46 38.4984710437
48 40.1202807308
50 41.7365070978
52 43.347000515
54 44.951611353
56 46.550189982
58 48.1425867725
60 49.7286520948
62 51.3082363194
64 52.8811898165
66 54.4473629567
68 56.0066061103
70 57.5587696477
72 59.1037039393
74 60.6412593554
76 62.1712862665
78 63.6936350429
80 65.2081560551
82 66.7146996733
84 68.2131162681
86 69.7032562098
88 71.1849698688
90 72.6581076155
92 74.1225198202
94 75.5780568534
96 77.0245690854
98 78.4619068866
100 79.8899206275
102 81.3084606784
104 82.7173774097
106 84.1165211917
108 85.505742395
110 86.8848913898
112 88.2538185465
114 89.6123742356
116 90.9604088275
118 92.2977726924
120 93.6243162009
122 94.9398897232
124 96.2443436299
126 97.5375282912
128 98.8192940775
130 100.089491359
132 101.347970507
134 102.594581891
136 103.829175881
138 105.051602849
140 106.261713164
142 107.459357197
144 108.644385317
146 109.816647897
148 110.975995305
150 112.122277913
152 113.25534609
154 114.375050208
156 115.481240636
158 116.573767745
160 117.652481905
162 118.717233487
164 119.767872861
166 120.804250397
168 121.826216467
170 122.833621439
172 123.826315685
174 124.804149575
176 125.76697348
178 126.714637769
180 127.646992814
182 128.563888984
184 129.465176649
186 130.350706182
188 131.220327951
190 132.073892327
192 132.91124968
194 133.732250382
196 134.536744802
198 135.32458331
200 136.095616278
202 136.849694075
204 137.586667071
206 138.306385638
208 139.008700146
210 139.693460965
212 140.360518465
214 141.009723016
216 141.64092499
218 142.253974757
220 142.848722686
222 143.425019149
224 143.982714515
226 144.521659156
228 145.041703441
230 145.542697741
232 146.024492426
234 146.486937867
236 146.929884434
238 147.353182498
240 147.756682428
242 148.140234595
244 148.503689371
246 148.846897124
248 149.169708225
250 149.471973045
252 149.753541955
254 150.014265324
256 150.253993522
258 150.472576922
260 150.669865892
262 150.845710803
264 150.999962025
266 151.13246993
268 151.243084886
270 151.331657266
272 151.398037438
274 151.442075774
276 151.463622644
};
\addplot [semithick, color2]
table {%
-300 -248.148712258
-298 -246.756987043
-296 -245.358861338
-294 -243.954393435
-292 -242.543641624
-290 -241.126664196
-288 -239.70351944
-286 -238.274265648
-284 -236.838961109
-282 -235.397664115
-280 -233.950432956
-278 -232.497325922
-276 -231.038401304
-274 -229.573717392
-272 -228.103332477
-270 -226.627304849
-268 -225.145692798
-266 -223.658554616
-264 -222.165948592
-262 -220.667933017
-260 -219.164566181
-258 -217.655906375
-256 -216.14201189
-254 -214.622941016
-252 -213.098752043
-250 -211.569503261
-248 -210.035252962
-246 -208.496059436
-244 -206.951980973
-242 -205.403075864
-240 -203.849402398
-238 -202.291018868
-236 -200.727983562
-234 -199.160354772
-232 -197.588190788
-230 -196.0115499
-228 -194.430490399
-226 -192.845070576
-224 -191.25534872
-222 -189.661383123
-220 -188.063232074
-218 -186.460953865
-216 -184.854606785
-214 -183.244249126
-212 -181.629939177
-210 -180.01173523
-208 -178.389695574
-206 -176.763878499
-204 -175.134342298
-202 -173.501145259
-200 -171.864345674
-198 -170.224001833
-196 -168.580172026
-194 -166.932914544
-192 -165.282287678
-190 -163.628349717
-188 -161.971158952
-186 -160.310773674
-184 -158.647252173
-182 -156.98065274
-180 -155.311033665
-178 -153.638453238
-176 -151.962969751
-174 -150.284641493
-172 -148.603526754
-170 -146.919683827
-168 -145.233171
-166 -143.544046565
-164 -141.852368811
-162 -140.15819603
-160 -138.461586511
-158 -136.762598546
-156 -135.061290424
-154 -133.357720437
-152 -131.651946874
-150 -129.944028026
-148 -128.234022184
-146 -126.521987637
-144 -124.807982678
-142 -123.092065595
-140 -121.374294679
-138 -119.654728222
-136 -117.933424513
-134 -116.210441842
-132 -114.485838501
-130 -112.75967278
-128 -111.032002969
-126 -109.302887358
-124 -107.572384239
-122 -105.840551901
-120 -104.107448635
-118 -102.373132732
-116 -100.637662482
-114 -98.9010961753
-112 -97.1634921026
-110 -95.4249085542
-108 -93.6854038207
-106 -91.9450361926
-104 -90.2038639602
-102 -88.4619454142
-100 -86.7193388448
-98 -84.9761025426
-96 -83.2322947981
-94 -81.4879739017
-92 -79.7431981439
-90 -77.9980258152
-88 -76.252515206
-86 -74.5067246068
-84 -72.760712308
-82 -71.0145366001
-80 -69.2682557737
-78 -67.5219281191
-76 -65.7756119268
-74 -64.0293654873
-72 -62.283247091
-70 -60.5373150285
-68 -58.7916275901
-66 -57.0462430664
-64 -55.3012197478
-62 -53.5566159248
-60 -51.8124898878
-58 -50.0688999274
-56 -48.3259043339
-54 -46.5835613978
-52 -44.8419294097
-50 -43.1010666599
-48 -41.361031439
-46 -39.6218820374
-44 -37.8836767455
-42 -36.1464738539
-40 -34.410331653
-38 -32.6753084333
-36 -30.9414624852
-34 -29.2088520992
-32 -27.4775355657
-30 -25.7475711753
-28 -24.0190172184
-26 -22.2919319855
-24 -20.566373767
-22 -18.8424008534
-20 -17.1200715351
-18 -15.3994441027
-16 -13.6805768466
-14 -11.9635280572
-12 -10.2483560251
-10 -8.53511904061
-8 -6.82387539431
-6 -5.11468337663
-4 -3.40760127805
-2 -1.70268738901
0 0
2 1.70040259853
4 3.3984621161
6 5.09412026226
8 6.78731874653
10 8.47799927846
12 10.1661035676
14 11.8515733234
16 13.5343502555
18 15.2143760733
20 16.8915924865
22 18.5659412046
24 20.237363937
26 21.9058023934
28 23.5711982832
30 25.233493316
32 26.8926292013
34 28.5485476487
36 30.2011903677
38 31.8504990678
40 33.4964154586
42 35.1388812496
44 36.7778381503
46 38.4132278703
48 40.0449921191
50 41.6730726062
52 43.2974110412
54 44.9179491336
56 46.5346285929
58 48.1473911287
60 49.7561784505
62 51.3609322678
64 52.9615942902
66 54.5581062272
68 56.1504097884
70 57.7384466832
72 59.3221586212
74 60.901487312
76 62.4763744651
78 64.0467617899
80 65.6125909961
82 67.1738037932
84 68.7303418908
86 70.2821469982
88 71.8291608252
90 73.3713250811
92 74.9085814756
94 76.4408717182
96 77.9681375185
98 79.4903205858
100 81.0073626299
102 82.5192053602
104 84.0257904862
106 85.5270597175
108 87.0229547637
110 88.5134173342
112 89.9983891386
114 91.4778118864
116 92.9516272872
118 94.4197770505
120 95.8822028858
122 97.3388465027
124 98.7896496107
126 100.234553919
128 101.673501138
130 103.106432977
132 104.533291144
134 105.954017351
136 107.368553306
138 108.776840718
140 110.178821298
142 111.574436755
144 112.963628798
146 114.346339138
148 115.722509482
150 117.092081542
152 118.454997027
154 119.811197645
156 121.160625107
158 122.503221123
160 123.838927401
162 125.167685651
164 126.489437583
166 127.804124907
168 129.111689331
170 130.412072566
172 131.70521632
174 132.991062304
176 134.269552227
178 135.540627799
180 136.804230728
182 138.060302726
184 139.3087855
186 140.549620761
188 141.782750218
190 143.008115581
192 144.225658559
194 145.435320862
196 146.637044199
198 147.83077028
200 149.016440815
202 150.193997512
204 151.363382082
206 152.524536234
208 153.677401677
210 154.821920122
212 155.958033277
214 157.085682852
216 158.204810557
218 159.315358101
220 160.417267194
222 161.510479545
224 162.594936864
226 163.67058086
228 164.737353243
230 165.795195723
232 166.844050008
234 167.883857809
236 168.914560835
238 169.936100796
240 170.9484194
242 171.951458359
244 172.94515938
246 173.929464174
248 174.90431445
250 175.869651918
252 176.825418288
254 177.771555268
256 178.708004568
258 179.634707898
260 180.551606968
262 181.458643487
264 182.355759164
266 183.242895709
268 184.119994832
270 184.986998242
272 185.843847649
274 186.690484761
276 187.52685129
278 188.352888943
280 189.168539431
282 189.973744464
284 190.768445751
286 191.552585
288 192.326103923
290 193.088944228
292 193.841047625
294 194.582355824
296 195.312810534
298 196.032353464
300 196.740926324
};
\addplot [semithick, color3]
table {%
-300 -303.602860461
-298 -301.108989718
-296 -298.62336151
-294 -296.145936598
-292 -293.676675743
-290 -291.215539705
-288 -288.762489246
-286 -286.317485127
-284 -283.880488107
-282 -281.451458948
-280 -279.030358411
-278 -276.617147256
-276 -274.211786244
-274 -271.814236136
-272 -269.424457693
-270 -267.042411676
-268 -264.668058844
-266 -262.30135996
-264 -259.942275784
-262 -257.590767076
-260 -255.246794597
-258 -252.910319109
-256 -250.581301372
-254 -248.259702146
-252 -245.945482193
-250 -243.638602274
-248 -241.339023148
-246 -239.046705577
-244 -236.761610322
-242 -234.483698144
-240 -232.212929802
-238 -229.949266059
-236 -227.692667674
-234 -225.443095409
-232 -223.200510024
-230 -220.964872281
-228 -218.736142939
-226 -216.51428276
-224 -214.299252504
-222 -212.091012933
-220 -209.889524807
-218 -207.694748886
-216 -205.506645932
-214 -203.325176705
-212 -201.150301967
-210 -198.981982477
-208 -196.820178998
-206 -194.664852288
-204 -192.51596311
-202 -190.373472224
-200 -188.237340391
-198 -186.107528372
-196 -183.983996927
-194 -181.866706817
-192 -179.755618803
-190 -177.650693645
-188 -175.551892106
-186 -173.459174944
-184 -171.372502922
-182 -169.291836799
-180 -167.217137337
-178 -165.148365297
-176 -163.085481439
-174 -161.028446523
-172 -158.977221312
-170 -156.931766565
-168 -154.892043044
-166 -152.858011508
-164 -150.829632719
-162 -148.806867439
-160 -146.789676426
-158 -144.778020443
-156 -142.77186025
-154 -140.771156607
-152 -138.775870277
-150 -136.785962018
-148 -134.801392593
-146 -132.822122761
-144 -130.848113285
-142 -128.879324924
-140 -126.915718439
-138 -124.957254591
-136 -123.003894141
-134 -121.055597849
-132 -119.112326477
-130 -117.174040785
-128 -115.240701534
-126 -113.312269485
-124 -111.388705399
-122 -109.469970035
-120 -107.556024156
-118 -105.646828521
-116 -103.742343893
-114 -101.84253103
-112 -99.947350695
-110 -98.0567636478
-108 -96.1707306494
-106 -94.2892124605
-104 -92.412169842
-102 -90.5395635545
-100 -88.6713543589
-98 -86.807503016
-96 -84.9479702866
-94 -83.0927169314
-92 -81.2417037112
-90 -79.3948913868
-88 -77.5522407189
-86 -75.7137124684
-84 -73.8792673961
-82 -72.0488662627
-80 -70.222469829
-78 -68.4000388557
-76 -66.5815341037
-74 -64.7669163338
-72 -62.9561463067
-70 -61.1491847831
-68 -59.345992524
-66 -57.5465302901
-64 -55.750758842
-62 -53.9586389408
-60 -52.170131347
-58 -50.3851968215
-56 -48.6037961251
-54 -46.8258900185
-52 -45.0514392626
-50 -43.2804046181
-48 -41.5127468458
-46 -39.7484267064
-44 -37.9874049609
-42 -36.2296423698
-40 -34.4750996941
-38 -32.7237376945
-36 -30.9755171318
-34 -29.2303987667
-32 -27.4883433601
-30 -25.7493116727
-28 -24.0132644653
-26 -22.2801624987
-24 -20.5499665337
-22 -18.822637331
-20 -17.0981356514
-18 -15.3764222558
-16 -13.6574579048
-14 -11.9412033594
-12 -10.2276193802
-10 -8.51666672799
-8 -6.80830616363
-6 -5.10249844787
-4 -3.39920434149
-2 -1.69838460527
0 0
2 1.69598871355
4 3.3896207746
6 5.08093542236
8 6.76997189605
10 8.4567694349
12 10.1413672781
14 11.8238046649
16 13.5041208345
18 15.1823550262
20 16.8585464791
22 18.5327344324
24 20.2049581255
26 21.8752567974
28 23.5436696875
30 25.2102360349
32 26.8749950789
34 28.5379860586
36 30.1992482134
38 31.8588207823
40 33.5167430047
42 35.1730541198
44 36.8277933667
46 38.4809999847
48 40.1327132131
50 41.7829722909
52 43.4318164575
54 45.0792849521
56 46.7254170139
58 48.3702518821
60 50.0138287959
62 51.6561869945
64 53.2973657172
66 54.9374042032
68 56.5763416917
70 58.2142174219
72 59.8510706331
74 61.4869405644
76 63.121866455
78 64.7558875443
80 66.3890430714
82 68.0213722755
84 69.6529143959
86 71.2837086718
88 72.9137943423
90 74.5432106468
92 76.1719968243
94 77.8001921142
96 79.4278357557
98 81.054966988
100 82.6816250503
102 84.3078491818
104 85.9336786217
106 87.5591526093
108 89.1843103838
110 90.8091911844
112 92.4338342503
114 94.0582788207
116 95.6825641349
118 97.306729432
120 98.9308139514
122 100.554856932
124 102.178897614
126 103.802975235
128 105.427129035
130 107.051398254
132 108.67582213
134 110.300439903
136 111.925290811
138 113.550414095
140 115.175848994
142 116.801634746
144 118.42781059
146 120.054415767
148 121.681489515
150 123.309071074
152 124.937199682
154 126.565914579
156 128.195255004
158 129.825260197
160 131.455969396
162 133.087421841
164 134.719656771
166 136.352713425
168 137.986631043
170 139.621448863
172 141.257206125
174 142.893942069
176 144.531695932
178 146.170506955
180 147.810414377
182 149.451457437
184 151.093675374
186 152.737107428
188 154.381792837
190 156.027770841
192 157.675080679
194 159.32376159
196 160.973852814
198 162.62539359
200 164.278423157
202 165.932980753
204 167.589105619
206 169.246836994
208 170.906214117
210 172.567276226
212 174.230062562
214 175.894612364
216 177.56096487
218 179.22915932
220 180.899234953
222 182.571231008
224 184.245186725
226 185.921141343
228 187.599134101
230 189.279204238
232 190.961390993
234 192.645733607
236 194.332271317
238 196.021043363
240 197.712088984
242 199.405447421
244 201.10115791
246 202.799259693
248 204.499792008
250 206.202794095
252 207.908305192
254 209.616364539
256 211.327011375
258 213.040284939
260 214.756224471
262 216.47486921
264 218.196258394
266 219.920431264
268 221.647427058
270 223.377285016
272 225.110044376
274 226.845744379
276 228.584424263
278 230.326123267
280 232.070880631
282 233.818735594
284 235.569727395
286 237.323895274
288 239.081278469
290 240.84191622
292 242.605847765
294 244.373112346
296 246.143749199
298 247.917797565
300 249.695296684
};
\addplot [semithick, color4]
table {%
-282 -164.113617255
-280 -164.096723699
-278 -164.054407356
-276 -163.986863259
-274 -163.894286445
-272 -163.776871946
-270 -163.634814799
-268 -163.468310037
-266 -163.277552695
-264 -163.062737809
-262 -162.824060411
-260 -162.561715538
-258 -162.275898223
-256 -161.966803502
-254 -161.634626409
-252 -161.279561979
-250 -160.901805245
-248 -160.501551244
-246 -160.078995009
-244 -159.634331575
-242 -159.167755977
-240 -158.67946325
-238 -158.169648427
-236 -157.638506544
-234 -157.086232636
-232 -156.513021736
-230 -155.919068881
-228 -155.304569103
-226 -154.669717438
-224 -154.014708921
-222 -153.339738585
-220 -152.645001467
-218 -151.930692599
-216 -151.197007018
-214 -150.444139758
-212 -149.672285852
-210 -148.881640337
-208 -148.072398246
-206 -147.244754615
-204 -146.398904477
-202 -145.535042867
-200 -144.653364821
-198 -143.754065372
-196 -142.837339556
-194 -141.903382406
-192 -140.952388959
-190 -139.984554247
-188 -139.000073306
-186 -137.999141171
-184 -136.981952876
-182 -135.948703456
-180 -134.899587945
-178 -133.834801378
-176 -132.754538789
-174 -131.658995214
-172 -130.548365687
-170 -129.422845243
-168 -128.282628915
-166 -127.12791174
-164 -125.958888751
-162 -124.775754983
-160 -123.57870547
-158 -122.367935248
-156 -121.143639351
-154 -119.906012814
-152 -118.65525067
-150 -117.391547956
-148 -116.115099705
-146 -114.826100953
-144 -113.524746733
-142 -112.21123208
-140 -110.88575203
-138 -109.548501616
-136 -108.199675873
-134 -106.839469837
-132 -105.468078541
-130 -104.08569702
-128 -102.692520309
-126 -101.288743442
-124 -99.8745614544
-122 -98.4501693806
-120 -97.0157622552
-118 -95.5715351127
-116 -94.1176829879
-114 -92.6544009153
-112 -91.1818839295
-110 -89.7003270653
-108 -88.2099253573
-106 -86.71087384
-104 -85.2033675481
-102 -83.6876015162
-100 -82.1637707791
-98 -80.6320703712
-96 -79.0926953273
-94 -77.5458406819
-92 -75.9917014698
-90 -74.4304727254
-88 -72.8623494836
-86 -71.2875267788
-84 -69.7061996458
-82 -68.1185631191
-80 -66.5248122335
-78 -64.9251420234
-76 -63.3197475236
-74 -61.7088237687
-72 -60.0925657934
-70 -58.4711686322
-68 -56.8448273197
-66 -55.2137368907
-64 -53.5780923798
-62 -51.9380888215
-60 -50.2939212506
-58 -48.6457847016
-56 -46.9938742091
-54 -45.3383848079
-52 -43.6795115326
-50 -42.0174494177
-48 -40.3523934979
-46 -38.6845388079
-44 -37.0140803822
-42 -35.3412132556
-40 -33.6661324625
-38 -31.9890330378
-36 -30.310110016
-34 -28.6295584317
-32 -26.9475733195
-30 -25.2643497142
-28 -23.5800826503
-26 -21.8949671624
-24 -20.2091982853
-22 -18.5229710535
-20 -16.8364805016
-18 -15.1499216644
-16 -13.4634895763
-14 -11.7773792722
-12 -10.0917857865
-10 -8.40690415389
-8 -6.72292940908
-6 -5.04005658665
-4 -3.35848072124
-2 -1.67839684748
0 0
2 1.67651478656
4 3.35095247758
6 5.02311803842
8 6.69281643444
10 8.35985263102
12 10.0240315935
14 11.6851582873
16 13.3430376778
18 14.9974747303
20 16.6482744102
22 18.2952416828
24 19.9381815136
26 21.5768988678
28 23.211198711
30 24.8408860084
32 26.4657657253
34 28.0856428273
36 29.7003222796
38 31.3096090476
40 32.9133080966
42 34.5112243921
44 36.1031628995
46 37.6889285839
48 39.268326411
50 40.8411613459
52 42.4072383542
54 43.966362401
56 45.5183384519
58 47.0629714722
60 48.6000664272
62 50.1294282824
64 51.650862003
66 53.1641725545
68 54.6691649022
70 56.1656440115
72 57.6534148478
74 59.1322823764
76 60.6020515627
78 62.062527372
80 63.5135147698
82 64.9548187214
84 66.3862441921
86 67.8075961474
88 69.2186795526
90 70.619299373
92 72.0092605741
94 73.3883681211
96 74.7564269796
98 76.1132421148
100 77.4586184921
102 78.7923610769
104 80.1142748345
106 81.4241647303
108 82.7218357297
110 84.0070927981
112 85.2797409008
114 86.5395850031
116 87.7864300705
118 89.0200810683
120 90.2403429619
122 91.4470207167
124 92.639919298
126 93.8188436712
128 94.9835988016
130 96.1339896546
132 97.2698211957
134 98.3908983901
136 99.4970262032
138 100.5880096
140 101.663653547
142 102.723763009
144 103.76814295
146 104.796598338
148 105.808934136
150 106.80495531
152 107.784466827
154 108.74727365
156 109.693180746
158 110.621993079
160 111.533515616
162 112.427553321
164 113.30391116
166 114.162394098
168 115.002807101
170 115.824955133
172 116.628643161
174 117.41367615
176 118.179859065
178 118.926996872
180 119.654894535
182 120.36335702
184 121.052189293
186 121.721196319
188 122.370183063
190 122.998954491
192 123.607315568
194 124.195071259
196 124.76202653
198 125.307986346
200 125.832755673
202 126.336139476
204 126.817942719
206 127.27797037
208 127.716027392
210 128.131918752
212 128.525449414
214 128.896424344
216 129.244648508
218 129.569926871
220 129.872064398
222 130.150866054
224 130.406136805
226 130.637681617
228 130.845305454
230 131.028813282
232 131.188010067
234 131.322700773
236 131.432690367
238 131.517783813
240 131.577786077
242 131.612502124
};
\end{axis}

\end{tikzpicture}
\subcaption{Kennlinien}
	\label{fig:evaluate30K}
\end{subfigure}
\begin{subfigure}{0.5 \textwidth}
	\setlength\figureheight{7.5cm}
	\setlength\figurewidth{7.5cm}
    % This file was created by matplotlib2tikz v0.6.17.
\begin{tikzpicture}

\begin{axis}[
xlabel={Iteration},
ylabel={QGesamt1},
xmin=-0.45, xmax=31.45,
ymin=1.04829029440141, ymax=3.10021311484252,
width=\figurewidth,
height=\figureheight,
tick align=outside,
tick pos=left,
x grid style={white!69.01960784313725!black},
y grid style={white!69.01960784313725!black},
legend entries={{Güte}},
legend cell align={left},
legend style={at={(0.03,0.97)}, anchor=north west, draw=white!80.0!black}
]
\addlegendimage{no markers, blue}
\addplot [semithick, blue, mark=*, mark size=3, mark options={solid}, only marks]
table {%
1 1.33933657581415
};
\addplot [semithick, blue, mark=*, mark size=3, mark options={solid}, only marks, forget plot]
table {%
2 1.71408593424802
};
\addplot [semithick, blue, mark=*, mark size=3, mark options={solid}, only marks, forget plot]
table {%
3 1.14155951351237
};
\addplot [semithick, blue, mark=x, mark size=3, mark options={solid}, only marks, forget plot]
table {%
4 1.92563871584933
};
\addplot [semithick, blue, mark=x, mark size=3, mark options={solid}, only marks, forget plot]
table {%
5 1.20828266898654
};
\addplot [semithick, blue, mark=x, mark size=3, mark options={solid}, only marks, forget plot]
table {%
6 1.83819854338923
};
\addplot [semithick, blue, mark=x, mark size=3, mark options={solid}, only marks, forget plot]
table {%
7 1.15509998393241
};
\addplot [semithick, blue, mark=x, mark size=3, mark options={solid}, only marks, forget plot]
table {%
8 2.43128738947511
};
\addplot [semithick, blue, mark=x, mark size=3, mark options={solid}, only marks, forget plot]
table {%
9 1.24003365394859
};
\addplot [semithick, blue, mark=x, mark size=3, mark options={solid}, only marks, forget plot]
table {%
10 2.47474461493424
};
\addplot [semithick, blue, mark=x, mark size=3, mark options={solid}, only marks, forget plot]
table {%
11 1.20493465742107
};
\addplot [semithick, blue, mark=x, mark size=3, mark options={solid}, only marks, forget plot]
table {%
12 2.97380032968469
};
\addplot [semithick, blue, mark=x, mark size=3, mark options={solid}, only marks, forget plot]
table {%
13 1.27797746368757
};
\addplot [semithick, blue, mark=x, mark size=3, mark options={solid}, only marks, forget plot]
table {%
14 2.93080134092079
};
\addplot [semithick, blue, mark=x, mark size=3, mark options={solid}, only marks, forget plot]
table {%
15 1.21602915220021
};
\addplot [semithick, blue, mark=x, mark size=3, mark options={solid}, only marks, forget plot]
table {%
16 2.77292007210168
};
\addplot [semithick, blue, mark=x, mark size=3, mark options={solid}, only marks, forget plot]
table {%
17 1.21379443133402
};
\addplot [semithick, blue, mark=x, mark size=3, mark options={solid}, only marks, forget plot]
table {%
18 2.72757644489784
};
\addplot [semithick, blue, mark=x, mark size=3, mark options={solid}, only marks, forget plot]
table {%
19 1.23951736331941
};
\addplot [semithick, blue, mark=x, mark size=3, mark options={solid}, only marks, forget plot]
table {%
20 3.00614253023569
};
\addplot [semithick, blue, mark=x, mark size=3, mark options={solid}, only marks, forget plot]
table {%
21 1.23130025809298
};
\addplot [semithick, blue, mark=x, mark size=3, mark options={solid}, only marks, forget plot]
table {%
22 2.66341712746378
};
\addplot [semithick, blue, mark=x, mark size=3, mark options={solid}, only marks, forget plot]
table {%
23 1.21301024650504
};
\addplot [semithick, blue, mark=x, mark size=3, mark options={solid}, only marks, forget plot]
table {%
24 2.74936004060199
};
\addplot [semithick, blue, mark=x, mark size=3, mark options={solid}, only marks, forget plot]
table {%
25 1.21843227630116
};
\addplot [semithick, blue, mark=x, mark size=3, mark options={solid}, only marks, forget plot]
table {%
26 2.94636297359222
};
\addplot [semithick, blue, mark=x, mark size=3, mark options={solid}, only marks, forget plot]
table {%
27 1.27435230337997
};
\addplot [semithick, blue, mark=x, mark size=3, mark options={solid}, only marks, forget plot]
table {%
28 2.96054792478101
};
\addplot [semithick, blue, mark=*, mark size=3, mark options={solid}, only marks, forget plot]
table {%
29 1.26596961123896
};
\addplot [semithick, blue, mark=*, mark size=3, mark options={solid}, only marks, forget plot]
table {%
30 3.00694389573156
};
\end{axis}

\end{tikzpicture}
\subcaption{Qualität}
	\label{fig:evaluate30Q}
\end{subfigure}
\label{fig:opt.evaluate30}
\caption{Erster Ansatz zur Anpassung von $K$}
\end{figure}

\subsubsection*{Grenzen der initial Kennlinie}
\label{subsubsec:opt.adjusta.problem}
Das Signal $\Uquest_{,\mathrm{meas}}$, wurde aus dem gemessenen $\Uout_{,\mathrm{meas}}$ berechnet und ist in \figref{fig:Amplitudenproblem} eingezeichnet. Die sich daraus ergebende Kennlinie ist in \figref{fig:K0} eingezeichnet. Das dazu gehörige Eingangssignal hat dabei die Form von $\Uquest_{, \mathrm{ideal}}$ mit $V_{PP} = \SI{578}{\mV}$ und ist damit etwa doppelt so groß wie das in \figref{fig:UinUquest} eingezeichnete $\Uquest_{, \mathrm{ideal}}$.\\
Wenn man jetzt $\Uquest_{, \mathrm{ideal}}$ mit $V_{PP} = \SI{300}{\mV}$ über $K_0$ zurückrechnet, um das erste nichtlinear vorverzerrte Eingangssignal zu erhalten, so stellt man fest, dass dies an der Grenze von $K_0$ stößt, siehe auch das eingezeichnete Maximum von $\Uquest$ in \figref{fig:K0}. \\
Deshalb kann damit zwar ein vorverzerrtes Signal berechnen werden, aber Messungen zeigen, dass die Qualität in diesem Bereich abnimmt. \\
Eine Ursache dafür könnte sein, dass die Methode der kleinsten Quadrate, mit der die Parameter $a_n$ für $K$ bestimmt werden, die Extrema von $\Uquest$ nicht genug gewichtet, um in diesem Bereich noch gültig zu sein.
\begin{figure}[H]
\begin{subfigure}{0.5 \textwidth}
	\setlength\figureheight{7.5cm}
	\setlength\figurewidth{7.5cm}
    % This file was created by matplotlib2tikz v0.6.17.
\begin{tikzpicture}

\definecolor{color0}{rgb}{0.12156862745098,0.466666666666667,0.705882352941177}

\begin{axis}[
xlabel={$U_{in}$ in \si{\milli \volt}},
ylabel={$U_{?}$ in \si{\milli \volt}},
xmin=-308, xmax=286,
ymin=-200, ymax=200,
width=\figurewidth,
height=\figureheight,
tick align=outside,
tick pos=left,
x grid style={white!69.01960784313725!black},
y grid style={white!69.01960784313725!black},
legend style={at={(0.03,0.97)}, anchor=north west, draw=white!80.0!black},
legend cell align={left},
legend entries={{$K_0$}},
extra y ticks={156, -137.2},
extra y tick labels={\tiny{$\max(U_?)$}, \tiny{$\min(U_?)$}},
extra y tick style={grid=major, ytick pos=left, ytick align=outside, ticklabel pos=left}
]
\addlegendimage{no markers, color0}
\addplot [semithick, color0]
table {%
-281 -179.700028168458
-279 -179.683789155593
-277 -179.639957605421
-275 -179.568739042489
-273 -179.470338991345
-271 -179.344962976534
-269 -179.192816522604
-267 -179.0141051541
-265 -178.80903439557
-263 -178.577809771561
-261 -178.320636806618
-259 -178.037721025289
-257 -177.72926795212
-255 -177.395483111658
-253 -177.03657202845
-251 -176.652740227042
-249 -176.24419323198
-247 -175.811136567812
-245 -175.353775759084
-243 -174.872316330343
-241 -174.366963806135
-239 -173.837923711007
-237 -173.285401569506
-235 -172.709602906179
-233 -172.110733245571
-231 -171.48899811223
-229 -170.844603030702
-227 -170.177753525534
-225 -169.488655121272
-223 -168.777513342464
-221 -168.044533713656
-219 -167.289921759394
-217 -166.513883004225
-215 -165.716622972696
-213 -164.898347189354
-211 -164.059261178744
-209 -163.199570465414
-207 -162.319480573911
-205 -161.419197028781
-203 -160.49892535457
-201 -159.558871075826
-199 -158.599239717095
-197 -157.620236802923
-195 -156.622067857858
-193 -155.604938406445
-191 -154.569053973232
-189 -153.514620082765
-187 -152.441842259592
-185 -151.350926028257
-183 -150.242076913309
-181 -149.115500439293
-179 -147.971402130757
-177 -146.809987512247
-175 -145.63146210831
-173 -144.436031443492
-171 -143.22390104234
-169 -141.995276429401
-167 -140.750363129221
-165 -139.489366666347
-163 -138.212492565326
-161 -136.919946350704
-159 -135.611933547028
-157 -134.288659678844
-155 -132.9503302707
-153 -131.597150847142
-151 -130.229326932715
-149 -128.847064051969
-147 -127.450567729447
-145 -126.040043489699
-143 -124.615696857269
-141 -123.177733356705
-139 -121.726358512554
-137 -120.261777849361
-135 -118.784196891674
-133 -117.293821164039
-131 -115.790856191004
-129 -114.275507497114
-127 -112.747980606916
-125 -111.208481044957
-123 -109.657214335784
-121 -108.094386003943
-119 -106.52020157398
-117 -104.934866570444
-115 -103.338586517879
-113 -101.731566940833
-111 -100.114013363852
-109 -98.4861313114839
-107 -96.8481263082742
-105 -95.2002038787698
-103 -93.5425695475174
-101 -91.8754288390637
-99 -90.1989872779553
-97 -88.5134503887389
-95 -86.819023695961
-93 -85.1159127241684
-91 -83.4043229979076
-89 -81.6844600417254
-87 -79.9565293801684
-85 -78.2207365377832
-83 -76.4772870391165
-81 -74.7263864087149
-79 -72.968240171125
-77 -71.2030538508936
-75 -69.4310329725673
-73 -67.6523830606926
-71 -65.8673096398163
-69 -64.076018234485
-67 -62.2787143692453
-65 -60.475603568644
-63 -58.6668913572276
-61 -56.8527832595427
-59 -55.0334848001361
-57 -53.2092015035544
-55 -51.3801388943442
-53 -49.5465024970522
-51 -47.708497836225
-49 -45.8663304364093
-47 -44.0202058221517
-45 -42.1703295179988
-43 -40.3169070484974
-41 -38.460143938194
-39 -36.6002457116353
-37 -34.737417893368
-35 -32.8718660079386
-33 -31.0037955798939
-31 -29.1334121337805
-29 -27.260921194145
-27 -25.3865282855341
-25 -23.5104389324944
-23 -21.6328586595726
-21 -19.7539929913153
-19 -17.8740474522692
-17 -15.9932275669809
-15 -14.111738859997
-13 -12.2297868558643
-11 -10.3475770791293
-9 -8.46531505433872
-7 -6.58320630603918
-5 -4.70145635877733
-3 -2.82027073709983
-1 -0.939854965553311
1 0.939585431315572
3 2.81784492896017
5 4.69471800283385
7 6.56999912838995
9 8.44348278108184
11 10.3149634363629
13 12.1842355696864
15 14.0510936565057
17 15.9153321722742
19 17.7767455924453
21 19.6351283924723
23 21.4902750478085
25 23.3419800339074
27 25.1900378262222
29 27.0342429002063
31 28.8743897313131
33 30.7102727949959
35 32.541686566708
37 34.3684255219029
39 36.1902841360339
41 38.0070568845543
43 39.8185382429175
45 41.6245226865769
47 43.4248046909857
49 45.2191787315974
51 47.0074392838652
53 48.7893808232427
55 50.564797825183
57 52.3334847651396
59 54.0952361185658
61 55.849846360915
63 57.5971099676405
65 59.3368214141957
67 61.0687751760339
69 62.7927657286085
71 64.5085875473728
73 66.2160351077802
75 67.914902885284
77 69.6049853553376
79 71.2860769933944
81 72.9579722749077
83 74.6204656753308
85 76.2733516701172
87 77.91642473472
89 79.5494793445929
91 81.1723099751889
93 82.7847111019616
95 84.3864772003643
97 85.9774027458503
99 87.557282213873
101 89.1259100798857
103 90.6830808193418
105 92.2285889076946
107 93.7622288203976
109 95.283795032904
111 96.7930820206671
113 98.2898842591405
115 99.7739962237773
117 101.245212390031
119 102.703327233355
121 104.148135229202
123 105.579430853027
125 106.997008580281
127 108.40066288642
129 109.790188246895
131 111.165379137161
133 112.52603003267
135 113.871935408876
137 115.202889741233
139 116.518687505194
141 117.819123176211
143 119.103991229739
145 120.373086141231
147 121.62620238614
149 122.86313443992
151 124.083676778023
153 125.287623875904
155 126.474770209015
157 127.64491025281
159 128.797838482742
161 129.933349374265
163 131.051237402832
165 132.151297043897
167 133.233322772912
169 134.297109065331
171 135.342450396607
173 136.369141242194
175 137.376976077546
177 138.365749378115
179 139.335255619354
181 140.285289276718
183 141.215644825659
185 142.126116741632
187 143.016499500088
189 143.886587576482
191 144.736175446267
193 145.565057584896
195 146.373028467823
197 147.159882570501
199 147.925414368383
201 148.669418336923
203 149.391688951574
205 150.092020687789
207 150.770208021022
209 151.426045426727
211 152.059327380355
213 152.669848357362
215 153.257402833199
217 153.821785283322
219 154.362790183182
221 154.880212008233
223 155.373845233929
225 155.843484335723
227 156.288923789068
229 156.709958069418
231 157.106381652226
233 157.477989012945
235 157.824574627029
237 158.145932969931
239 158.441858517104
241 158.712145744002
243 158.956589126078
245 159.174983138786
247 159.367122257579
249 159.532800957909
251 159.671813715232
253 159.783955004999
255 159.869019302664
257 159.926801083681
259 159.957094823503
};
\end{axis}

\end{tikzpicture}	
\subcaption{Resultierende Kennlinie}
	\label{fig:K0}
\end{subfigure}
\begin{subfigure}{0.5 \textwidth}
	\setlength\figureheight{7.5cm}
	\setlength\figurewidth{7.5cm}
    % This file was created by matplotlib2tikz v0.6.17.
\begin{tikzpicture}

\definecolor{color0}{rgb}{0.12156862745098,0.466666666666667,0.705882352941177}
\definecolor{color1}{rgb}{1,0.498039215686275,0.0549019607843137}

\begin{axis}[
xlabel={$t$ in \si{\micro \second}},
ylabel={$U$ in \si{\milli \volt}},
xmin=-0.05554, xmax=1.16634,
ymin=-400, ymax=400,
width=\figurewidth,
height=\figureheight,
tick align=outside,
tick pos=left,
x grid style={white!69.01960784313725!black},
y grid style={white!69.01960784313725!black},
legend style={draw=white!80.0!black},
legend cell align={left},
legend entries={{$U_{?,1}$},{$U_{in}$}}
]
\addlegendimage{no markers, color0}
\addlegendimage{no markers, color1}
\addplot [semithick, color0]
table {%
0 -6.16700467414
0.0004 -4.96285981485
0.0008 -3.51778837084
0.0012 -1.88801518427
0.0016 -0.137795113788
0.002 1.66277470235
0.0024 3.44017461828
0.0028 5.12003373718
0.0032 6.62976579266
0.0036 7.9011794294
0.004 8.87297239424
0.0044 9.49302203677
0.0048 9.72039049547
0.0052 9.52697183845
0.0056 8.89871995875
0.006 7.83640981696
0.0064 6.35590021898
0.0068 4.48788318434
0.0072 2.27712252199
0.0076 -0.218798125589
0.008 -2.93118040018
0.0084 -5.78231617085
0.0088 -8.68792470989
0.0092 -11.5598424803
0.0096 -14.308876551
0.01 -16.847726307
0.0104 -19.0938750346
0.0108 -20.9723532624
0.0112 -22.4182794402
0.0116 -23.3790905362
0.012 -23.8163851912
0.0124 -23.7073148616
0.0128 -23.0454734697
0.0132 -21.8412529458
0.0136 -20.1216501057
0.014 -17.9295289362
0.0144 -15.3223609076
0.0148 -12.3704837443
0.0152 -9.15493554631
0.0156 -5.7649356748
0.016 -2.29509589584
0.0164 1.15754551937
0.0168 4.49656802608
0.0172 7.62919005381
0.0176 10.4691668468
0.018 12.9394688305
0.0184 14.9746521007
0.0188 16.5228433342
0.0192 17.547274769
0.0196 18.02732044
0.02 17.9590020152
0.0204 17.3549507654
0.0208 16.2438307606
0.0212 14.6692466819
0.0216 12.6881770086
0.022 10.3689891972
0.0224 7.78910725019
0.0228 5.0324133111
0.0232 2.18647323596
0.0236 -0.660318791117
0.024 -3.42157976989
0.0244 -6.01566632356
0.0248 -8.36820180094
0.0252 -10.4143198124
0.0256 -12.1005527955
0.026 -13.3863074077
0.0264 -14.2448836281
0.0268 -14.6640108611
0.0272 -14.6458914709
0.0276 -14.2067594188
0.028 -13.3759783991
0.0284 -12.1947194955
0.0288 -10.7142723391
0.0292 -8.99405558358
0.0296 -7.09940180617
0.03 -5.09919840527
0.0304 -3.06346949924
0.0308 -1.06098415608
0.0312 0.8430264624
0.0316 2.5889661294
0.032 4.12471618308
0.0324 5.40716584719
0.0328 6.4033848827
0.0332 7.0914071557
0.0336 7.4606085063
0.034 7.51167739247
0.0344 7.25619159566
0.0348 6.71582826292
0.0352 5.92124721594
0.0356 4.91069832993
0.036 3.7284124997
0.0364 2.42284197931
0.0368 1.04481951591
0.0372 -0.354293387215
0.0376 -1.72440052842
0.038 -3.01834319621
0.0384 -4.19339301087
0.0388 -5.21251669687
0.0392 -6.04537447758
0.0396 -6.6690254127
0.04 -7.06832535917
0.0404 -7.23601582445
0.0408 -7.17251428994
0.0412 -6.88542812739
0.0416 -6.38882456315
0.042 -5.7022978721
0.0424 -4.84988178343
0.0428 -3.85885971455
0.0432 -2.75852776991
0.0436 -1.57896539685
0.044 -0.349866227228
0.0444 0.900522908342
0.0448 2.14631329786
0.0452 3.36470263511
0.0456 4.53649631994
0.046 5.64640807147
0.0464 6.68313680877
0.0468 7.63922734385
0.0472 8.51073249834
0.0476 9.29670336664
0.048 9.99854221489
0.0484 10.6192585979
0.0488 11.1626734375
0.0492 11.6326178602
0.0496 12.0321734619
0.05 12.3629983511
0.0504 12.6247789393
0.0508 12.8148411702
0.0512 12.9279470029
0.0516 12.9562928093
0.052 12.889716331
0.0524 12.7161083856
0.0528 12.4220150904
0.0532 11.9934064336
0.0536 11.4165780162
0.054 10.67914512
0.0544 9.77108228399
0.0548 8.68575757913
0.0552 7.42090896519
0.0556 5.97951061465
0.056 4.3704799152
0.0564 2.6091809406
0.0568 0.717687340157
0.0572 -1.27522342136
0.0576 -3.33436207518
0.058 -5.41871186966
0.0584 -7.48229909237
0.0588 -9.47533138518
0.0592 -11.3455735137
0.0596 -13.0399191878
0.06 -14.5061078015
0.0604 -15.6945269901
0.0608 -16.5600360547
0.0612 -17.0637418617
0.0616 -17.1746579721
0.062 -16.8711796029
0.0624 -16.1423115354
0.0628 -14.98859317
0.0632 -13.4226743474
0.0636 -11.4695070078
0.064 -9.16613084223
0.0644 -6.56104533878
0.0648 -3.71317551626
0.0652 -0.690453624224
0.0656 2.43194639933
0.066 5.57367143503
0.0664 8.65148644122
0.0668 11.5817343408
0.0672 14.282878928
0.0676 16.6780464474
0.068 18.6974797225
0.0684 20.2808200082
0.0688 21.3791361131
0.0692 21.9566276596
0.0696 21.9919393999
0.07 21.4790359557
0.0704 20.4276007587
0.0708 18.8629388671
0.0712 16.8253801272
0.0716 14.3691962713
0.072 11.5610623599
0.0724 8.47810889224
0.0728 5.20562533399
0.0732 1.83448823261
0.0736 -1.54160296802
0.074 -4.82899219693
0.0744 -7.9369409014
0.0748 -10.7804151378
0.0752 -13.2826932642
0.0756 -15.377707705
0.076 -17.0120431085
0.0764 -18.146524784
0.0768 -18.757345201
0.0772 -18.836692045
0.0776 -18.392858313
0.078 -17.4498325865
0.0784 -16.0463853118
0.0788 -14.2346840252
0.0792 -12.0784863715
0.0796 -9.6509739214
0.08 -7.03230169761
0.0804 -4.30694755513
0.0808 -1.56095181622
0.0812 1.12085937946
0.0816 3.6575737692
0.082 5.97472594394
0.0824 8.00658043258
0.0828 9.69809116229
0.0832 11.0064755018
0.0836 11.9023550843
0.084 12.370431128
0.0844 12.409678462
0.0848 12.0330593154
0.0852 11.2667745426
0.0856 10.1490857344
0.086 8.72875605447
0.0864 7.06317013881
0.0868 5.21620357833
0.0872 3.25592003311
0.0876 1.25217866806
0.088 -0.725763776415
0.0884 -2.61157337318
0.0888 -4.34400329293
0.0892 -5.86883049655
0.0896 -7.14050285395
0.09 -8.12344626019
0.0904 -8.79299446169
0.0908 -9.13591852848
0.0912 -9.15054767251
0.0916 -8.84648786511
0.092 -8.24395890654
0.0924 -7.37278373373
0.0928 -6.27107535457
0.0932 -4.98367647031
0.0936 -3.56041427171
0.094 -2.05423784109
0.0944 -0.519307930962
0.0948 0.990891409942
0.0952 2.42535275898
0.0956 3.73719545212
0.096 4.88511199809
0.0964 5.83456606641
0.0968 6.55870797722
0.0972 7.03898523056
0.0976 7.26543772556
0.098 7.23667945644
0.0984 6.95958016257
0.0988 6.44867121071
0.0992 5.72530950238
0.0996 4.81664108905
0.1 3.75441217674
0.1004 2.5736791246
0.1008 1.3114707914
0.1012 0.00545615093579
0.1016 -1.30733243808
0.102 -2.59167439657
0.1024 -3.81519478892
0.1028 -4.94922472502
0.1032 -5.96946475984
0.1036 -6.85643702636
0.104 -7.59572136523
0.1044 -8.17798008283
0.1048 -8.59878478657
0.1052 -8.8582666496
0.1056 -8.96061813227
0.106 -8.91347937724
0.1064 -8.72724601174
0.1068 -8.41433682081
0.1072 -7.98845966808
0.1076 -7.46391218053
0.108 -6.85495020556
0.1084 -6.17525208632
0.1088 -5.43750063913
0.1092 -4.65309766123
0.1096 -3.83201818849
0.11 -2.98280392298
0.1104 -2.1126876275
0.1108 -1.22783319338
0.1112 -0.333669859365
0.1116 0.564706018237
0.112 1.46209193524
0.1124 2.35273537325
0.1128 3.22991799506
0.1132 4.08559958844
0.1136 4.91015059977
0.114 5.69219774497
0.1144 6.41860151735
0.1148 7.07457768464
0.1152 7.64396738623
0.1156 8.10965254844
0.116 8.45410539354
0.1164 8.66005319811
0.1168 8.71123251813
0.1172 8.59320117585
0.1176 8.29417169258
0.118 7.80582679114
0.1184 7.12407625853
0.1188 6.24971495651
0.1192 5.1889441191
0.1196 3.95372222655
0.12 2.56191755975
0.1204 1.03724180796
0.1208 -0.591047449582
0.1212 -2.28867827354
0.1216 -4.01712302515
0.122 -5.73456744209
0.1224 -7.39706392315
0.1228 -8.95983407784
0.1232 -10.378677997
0.1236 -11.6114415848
0.124 -12.6194889514
0.1244 -13.3691245289
0.1248 -13.8329093778
0.1252 -13.990818176
0.1256 -13.8311875858
0.126 -13.3514129714
0.1264 -12.5583585898
0.1268 -11.4684561146
0.1272 -10.1074773389
0.1276 -8.50997872263
0.128 -6.71842766824
0.1284 -4.78203254714
0.1288 -2.75531009276
0.1292 -0.696434355373
0.1296 1.3345794444
0.13 3.27779550481
0.1304 5.07531663012
0.1308 6.67321335644
0.1312 8.02335406313
0.1316 9.08506939784
0.132 9.82658915299
0.1324 10.2261969269
0.1328 10.2730572473
0.1332 9.96768100603
0.1336 9.32200765133
0.134 8.35909616096
0.1344 7.11243086324
0.1348 5.6248621752
0.1352 3.94721575169
0.1356 2.13661588106
0.136 0.254579746728
0.1364 -1.63505201906
0.1368 -3.46827658059
0.1372 -5.18310613991
0.1376 -6.72168423011
0.138 -8.03226579552
0.1384 -9.07099024817
0.1388 -9.80338377414
0.1392 -10.2055366293
0.1396 -10.2649126731
0.14 -9.9807615222
0.1404 -9.36411798393
0.1408 -8.43738832385
0.1412 -7.23353787874
0.1416 -5.79490899325
0.142 -4.17171168919
0.1424 -2.42024137134
0.1428 -0.600887783968
0.1432 1.22399301473
0.1436 2.99226698274
0.144 4.64419196985
0.1444 6.1244912065
0.1448 7.38425617812
0.1452 8.38261110157
0.1456 9.0880801603
0.146 9.47960975536
0.1464 9.54721087018
0.1468 9.29220074454
0.1472 8.72703787522
0.1476 7.87475934928
0.148 6.76804410604
0.1484 5.44793936765
0.1488 3.96229966467
0.1492 2.36399815722
0.1496 0.708977933628
0.15 -0.945783635403
0.1504 -2.54432179107
0.1508 -4.03366707911
0.1512 -5.36569147727
0.1516 -6.49875138806
0.152 -7.39906919397
0.1524 -8.04180548053
0.1528 -8.41178617317
0.1532 -8.50386222978
0.1536 -8.32289366148
0.154 -7.88336397059
0.1544 -7.20864504614
0.1548 -6.32994560677
0.1552 -5.28498793261
0.1556 -4.11646743924
0.156 -2.87035725348
0.1564 -1.59412507228
0.1568 -0.334932043133
0.1572 0.862116877375
0.1576 1.95560461878
0.158 2.90925483309
0.1584 3.69315284309
0.1588 4.28469155554
0.1592 4.66921066578
0.1596 4.84031007867
0.16 4.79983168486
0.1604 4.55751693003
0.1608 4.13036046777
0.1612 3.54169208275
0.1616 2.82002954183
0.162 1.99775366553
0.1624 1.10966337798
0.1628 0.191472544334
0.1632 -0.721688106598
0.1636 -1.59670179143
0.164 -2.40377068738
0.1644 -3.11754266131
0.1648 -3.71801536939
0.1652 -4.19118330272
0.1656 -4.52940462559
0.166 -4.73147680543
0.1664 -4.80242256332
0.1668 -4.75300005893
0.1672 -4.59896295334
0.1676 -4.36010658048
0.168 -4.05914546633
0.1684 -3.72047449122
0.1688 -3.36887080324
0.1692 -3.02819595967
0.1696 -2.72015760014
0.17 -2.46318724503
0.1704 -2.2714856769
0.1708 -2.15428001208
0.1712 -2.11532730699
0.1716 -2.1526887532
0.172 -2.25878664344
0.1724 -2.42074383265
0.1728 -2.62099289362
0.1732 -2.83813010236
0.1736 -3.04797829281
0.174 -3.22481296765
0.1744 -3.34269825595
0.1748 -3.37687371322
0.1752 -3.30512981734
0.1756 -3.10910947989
0.176 -2.77547501542
0.1764 -2.29688472999
0.1768 -1.67273043793
0.1772 -0.909596523457
0.1776 -0.0214122712495
0.178 0.970718338378
0.1784 2.03901183653
0.1788 3.14981427472
0.1792 4.26469939588
0.1796 5.341818668
0.18 6.33745453009
0.1804 7.20771863788
0.1808 7.91032953756
0.1812 8.40639936937
0.1816 8.66215714186
0.182 8.65053694456
0.1824 8.35256319299
0.1828 7.75847151728
0.1832 6.86851299714
0.1836 5.69340078624
0.184 4.25437134226
0.1844 2.58284698972
0.1848 0.719701837305
0.1852 -1.28585143977
0.1856 -3.37772136664
0.186 -5.49469733189
0.1864 -7.57245000036
0.1868 -9.54570291628
0.1872 -11.3504978404
0.1876 -12.9264713217
0.188 -14.2190580501
0.1884 -15.1815377825
0.1888 -15.7768470794
0.1892 -15.9790845923
0.1896 -15.7746489624
0.19 -15.1629611566
0.1904 -14.1567378223
0.1908 -12.7817984498
0.1912 -11.0764061755
0.1916 -9.09015932101
0.192 -6.88246756686
0.1924 -4.52066238109
0.1928 -2.07780534815
0.1932 0.369730172483
0.1936 2.7448195474
0.194 4.9721990071
0.1944 6.98103620507
0.1948 8.7073667338
0.1952 10.0963083259
0.1956 11.1039719655
0.196 11.6989996759
0.1964 11.8636719579
0.1968 11.5945432422
0.1972 10.9025807306
0.1976 9.8127999964
0.198 8.36340902819
0.1984 6.60449034888
0.1988 4.59626774337
0.1992 2.40701935614
0.1996 0.110711891083
0.2 -2.21555912692
0.2004 -4.49423097587
0.2008 -6.6499415054
0.2012 -8.61210716969
0.2016 -10.31731886
0.202 -11.7114699199
0.2024 -12.7515407481
0.2028 -13.4069772125
0.2032 -13.6606152367
0.2036 -13.5091207968
0.204 -12.9629325556
0.2044 -12.0457127745
0.2048 -10.7933302916
0.2052 -9.25241655186
0.2056 -7.47855126968
0.206 -5.53414771534
0.2064 -3.4861183402
0.2068 -1.40340908765
0.2072 0.645504995494
0.2076 2.59506948698
0.208 4.38495508673
0.2084 5.9621325221
0.2088 7.28264287533
0.2092 8.31300052171
0.2096 9.03118036123
0.21 9.42715727673
0.2104 9.5029831255
0.2108 9.27240439654
0.2112 8.76004125412
0.2116 8.00016536159
0.212 7.03512899838
0.2124 5.91351096939
0.2128 4.68805516565
0.2132 3.4134849786
0.2136 2.14428082506
0.214 0.932508663005
0.2144 -0.174215438725
0.2148 -1.13454574655
0.2152 -1.91494258733
0.2156 -2.49086549696
0.216 -2.84758002775
0.2164 -2.98055018823
0.2168 -2.89540524805
0.2172 -2.60748675592
0.2176 -2.14099841303
0.218 -1.52779725446
0.2184 -0.805878795057
0.2188 -0.0176208340336
0.2192 0.792139985738
0.2196 1.57811837051
0.22 2.29659368114
0.2204 2.90734293567
0.2208 3.37541367188
0.2212 3.67266361787
0.2216 3.7790038353
0.222 3.6832942568
0.2224 3.38385484359
0.2228 2.88857136907
0.2232 2.21459145577
0.2236 1.38762329238
0.224 0.440865759814
0.2244 -0.586386151367
0.2248 -1.65040335603
0.2252 -2.7048837489
0.2256 -3.70293561505
0.226 -4.5991013888
0.2264 -5.35133781261
0.2268 -5.92286997711
0.2272 -6.28384149443
0.2276 -6.41269098879
0.228 -6.29719587254
0.2284 -5.93513759141
0.2288 -5.3345576487
0.2292 -4.51359015862
0.2296 -3.49987378059
0.23 -2.32956297093
0.2304 -1.04597487225
0.2308 0.30207681108
0.2312 1.66219653739
0.2316 2.98049526412
0.232 4.20376976066
0.2324 5.28168585978
0.2328 6.1688776464
0.2332 6.8268771161
0.2336 7.22579478551
0.234 7.34568088992
0.2344 7.17750881549
0.2348 6.72373683017
0.2352 5.99842044444
0.2356 5.02686522235
0.236 3.84482789473
0.2364 2.49729149629
0.2368 1.03685725378
0.2372 -0.478188577012
0.2376 -1.98606140769
0.238 -3.42383639344
0.2384 -4.72988821622
0.2388 -5.84631881244
0.2392 -6.72127880303
0.2396 -7.31109155626
0.24 -7.5820955377
0.2404 -7.51213063841
0.2408 -7.09160715463
0.2412 -6.32411153479
0.2416 -5.22652033182
0.242 -3.82861234369
0.2424 -2.17218798563
0.2428 -0.309723784092
0.2432 1.69739221031
0.2436 3.78098201625
0.244 5.86848023231
0.2444 7.88553017006
0.2448 9.75867835968
0.2452 11.4180677922
0.2456 12.8000296268
0.246 13.8494762478
0.2464 14.5220053943
0.2468 14.7856353649
0.2472 14.6221046355
0.2476 14.0276851282
0.248 13.013476234
0.2484 11.605165854
0.2488 9.8422644559
0.2492 7.77683770405
0.2496 5.47178186794
0.25 2.99870324464
0.2504 0.435477595287
0.2508 -2.13642246111
0.2512 -4.63473548528
0.2516 -6.97925301099
0.252 -9.0945921474
0.2524 -10.912798613
0.2528 -12.3756861348
0.2532 -13.4368279574
0.2536 -14.0631291419
0.254 -14.2359239064
0.2544 -13.9515598706
0.2548 -13.2214500533
0.2552 -12.0715931116
0.2556 -10.5415818661
0.256 -8.68313888053
0.2564 -6.5582350702
0.2568 -4.23686233979
0.2572 -1.79454355096
0.2576 0.690327761127
0.258 3.13921989913
0.2584 5.47616781941
0.2588 7.63035218837
0.2592 9.53846014003
0.2596 11.1467441559
0.26 12.4127075216
0.2604 13.306359443
0.2608 13.810999556
0.2612 13.9235095856
0.2616 13.654148609
0.262 13.0258670276
0.2624 12.0731722508
0.2628 10.8405955488
0.2632 9.38082393206
0.2636 7.75257272079
0.264 6.0182832475
0.2644 4.24173557869
0.2648 2.4856680653
0.2652 0.809493896919
0.2656 -0.732800264606
0.266 -2.09449682743
0.2664 -3.23766545088
0.2668 -4.13430494795
0.2672 -4.76707246593
0.2676 -5.12958020632
0.268 -5.2262572115
0.2684 -5.07179088797
0.2688 -4.6901792791
0.2692 -4.11344000537
0.2696 -3.38003468091
0.27 -2.53307799799
0.2704 -1.61840815794
0.2708 -0.682599645922
0.2712 0.228999643921
0.2716 1.07412723704
0.272 1.81528667109
0.2724 2.42118411349
0.2728 2.86786629276
0.2732 3.13951816732
0.2736 3.22889344721
0.274 3.13736672151
0.2744 2.87461188118
0.2748 2.45792711099
0.2752 1.91124131715
0.2756 1.26384986821
0.276 0.548938429495
0.2764 -0.198037970987
0.2768 -0.941048025348
0.2772 -1.64517436396
0.2776 -2.27822667474
0.278 -2.81220739593
0.2784 -3.22456651426
0.2788 -3.49919151755
0.2792 -3.62709032218
0.2796 -3.60673852372
0.28 -3.44407703589
0.2804 -3.15216147086
0.2808 -2.75047983236
0.2812 -2.26396960303
0.2816 -1.72177850042
0.282 -1.15582449506
0.2824 -0.599219651074
0.2828 -0.0846285804641
0.2832 0.357364472772
0.2836 0.69980583144
0.284 0.92076887892
0.2844 1.00453201594
0.2848 0.94248221994
0.2852 0.733699775473
0.2856 0.3851931675
0.286 -0.0882319986605
0.2864 -0.664470478106
0.2868 -1.31496579743
0.2872 -2.00587870077
0.2876 -2.69953043015
0.288 -3.35606305955
0.2884 -3.935248788
0.2888 -4.39837250031
0.2892 -4.71010736943
0.2896 -4.84030202531
0.29 -4.76559994147
0.2904 -4.47081714754
0.2908 -3.95001298134
0.2912 -3.20720003786
0.2916 -2.25665332364
0.292 -1.1227943542
0.2924 0.160357075051
0.2928 1.55015312675
0.2932 2.99668237169
0.2936 4.44448398837
0.294 5.83455207701
0.2944 7.10655740446
0.2948 8.20120352145
0.2952 9.06262694582
0.2956 9.64074731617
0.296 9.89347329029
0.2964 9.78867354928
0.2968 9.30582948316
0.2972 8.43729674481
0.2976 7.18911649933
0.298 5.58133337364
0.2984 3.64779523175
0.2988 1.43542928727
0.2992 -0.996991018746
0.2996 -3.5805542616
0.3 -6.23834159436
0.3004 -8.88790985721
0.3008 -11.4440428973
0.3012 -13.8216751783
0.3016 -15.9388836378
0.302 -17.7198394426
0.3024 -19.0976110012
0.3028 -20.0167133766
0.3032 -20.435307001
0.3036 -20.3269600744
0.304 -19.6819038302
0.3044 -18.5077274621
0.3048 -16.8294792722
0.3052 -14.6891618145
0.3056 -12.1446306805
0.306 -9.26792829485
0.3064 -6.1431048465
0.3068 -2.86359749589
0.3072 0.470744448613
0.3076 3.75688723157
0.308 6.89198222387
0.3084 9.77676050549
0.3088 12.3188339656
0.3092 14.4357871062
0.3096 16.0579506375
0.31 17.130758842
0.3104 17.6166071611
0.3108 17.4961440102
0.3112 16.7689507981
0.3116 15.4535858009
0.312 13.5869901015
0.3124 11.2232764462
0.3128 8.43194373155
0.3132 5.29558013854
0.3136 1.90713591054
0.314 -1.6331382187
0.3144 -5.22097944994
0.3148 -8.75151314412
0.3152 -12.1226464392
0.3156 -15.2383377997
0.316 -18.011630247
0.3164 -20.3673452666
0.3168 -22.2443473208
0.3172 -23.5973050052
0.3176 -24.3978935711
0.318 -24.6354040718
0.3184 -24.316746013
0.3188 -23.4658522882
0.3192 -22.1225165257
0.3196 -20.3407130022
0.32 -18.1864672246
0.3204 -15.7353605188
0.3208 -13.0697639376
0.3212 -10.2759050895
0.3216 -7.4408758422
0.322 -4.64968912206
0.3224 -1.98248927732
0.3228 0.48798713483
0.3232 2.69861438805
0.3236 4.59767286228
0.324 6.14618158526
0.3244 7.31873149115
0.3248 8.10380233629
0.3252 8.50356600667
0.3256 8.53319828509
0.326 8.21973927773
0.3264 7.60055894478
0.3268 6.72149793409
0.3272 5.63476468084
0.3276 4.39667713358
0.328 3.06534125928
0.3284 1.69835857131
0.3288 0.350651370289
0.3292 -0.927512615835
0.3296 -2.09222462483
0.33 -3.10730949971
0.3304 -3.9453266764
0.3308 -4.58816998244
0.3312 -5.02725585944
0.3316 -5.26330684846
0.332 -5.30575377734
0.3324 -5.17179538822
0.3328 -4.88516753115
0.3332 -4.47468499838
0.3336 -3.97262716151
0.334 -3.41304350114
0.3344 -2.83005671978
0.3348 -2.25623937752
0.3352 -1.72113499976
0.3356 -1.24998662511
0.336 -0.862725161993
0.3364 -0.573257182285
0.3368 -0.389077467341
0.3372 -0.311216365191
0.3376 -0.334516484154
0.338 -0.448218113652
0.3384 -0.63681868354
0.3388 -0.881159158352
0.3392 -1.15968004945
0.3396 -1.44978215821
0.34 -1.72922256544
0.3404 -1.97747495726
0.3408 -2.17698518979
0.3412 -2.31425797053
0.3416 -2.38071846071
0.342 -2.37330314211
0.3424 -2.29474698985
0.3428 -2.15354829975
0.3432 -1.96360781406
0.3436 -1.74355440183
0.344 -1.5157847936
0.3444 -1.30525906334
0.3448 -1.13810605687
0.3452 -1.04010320827
0.3456 -1.03510268337
0.346 -1.14348017026
0.3464 -1.38068365812
0.3468 -1.75595710765
0.3472 -2.2713080616
0.3476 -2.92077916568
0.348 -3.69007159795
0.3484 -4.55655399954
0.3488 -5.48967423095
0.3492 -6.45177380506
0.3496 -7.39928689252
0.35 -8.28428811141
0.3504 -9.05633665921
0.3508 -9.66454945039
0.3512 -10.0598234572
0.3516 -10.1971179972
0.352 -10.0377017381
0.3524 -9.55126703644
0.3528 -8.7178160707
0.3532 -7.52922910705
0.3536 -5.9904349976
0.354 -4.12011736418
0.3544 -1.95090640906
0.3548 0.470974669809
0.3552 3.08661880258
0.3556 5.82588792591
0.356 8.60937570759
0.3564 11.350824768
0.3568 13.9599231623
0.3572 16.3453873822
0.3576 18.4182247667
0.358 20.0950576135
0.3584 21.3013849812
0.3588 21.9746564966
0.3592 22.0670356304
0.3596 21.5477378578
0.36 20.4048416871
0.3604 18.646487347
0.3608 16.3013984186
0.3612 13.41868518
0.3616 10.066914073
0.362 6.33245456948
0.3624 2.31714181478
0.3628 -1.86468026198
0.3632 -6.08964616917
0.3636 -10.2292754568
0.364 -14.1541696432
0.3644 -17.7383566818
0.3648 -20.8636314545
0.3652 -23.4237374563
0.3656 -25.328237155
0.366 -26.5059264796
0.3664 -26.9076623265
0.3668 -26.5084904651
0.3672 -25.3089841775
0.3676 -23.3357306152
0.368 -20.6409312638
0.3684 -17.301114049
0.3688 -13.4149863662
0.3692 -9.1004895192
0.3696 -4.491144557
0.37 0.268193806639
0.3704 5.02603562911
0.3708 9.62924880769
0.3712 13.9281763328
0.3716 17.7817005275
0.372 21.0620766196
0.3724 23.6593639298
0.3728 25.4852951444
0.3732 26.47644221
0.3736 26.5965607479
0.374 25.8380227415
0.3744 24.2222786507
0.3748 21.799323937
0.3752 18.6461800483
0.3756 14.8644349495
0.376 10.5769220254
0.3764 5.92364739668
0.3768 1.05710321386
0.3772 -3.86287268508
0.3776 -8.67451312728
0.378 -13.2196253423
0.3784 -17.3489891505
0.3788 -20.9274592548
0.3792 -23.8385907282
0.3796 -25.9886227816
0.38 -27.3096779204
0.3804 -27.7620608772
0.3808 -27.3355733081
0.3812 -26.0497950274
0.3816 -23.9533193018
0.382 -21.1219671025
0.3824 -17.6560418822
0.3828 -13.6767210626
0.3832 -9.32171172902
0.3836 -4.74032488163
0.384 -0.0881439858366
0.3844 4.47852129124
0.3848 8.80819696755
0.3852 12.7596219235
0.3856 16.2067461372
0.386 19.0431688938
0.3864 21.1858556433
0.3868 22.5779978478
0.3872 23.1909108514
0.3876 23.0248994025
0.388 22.109057755
0.3884 20.5000099016
0.3888 18.2796340778
0.3892 15.5518528088
0.3896 12.4386041111
0.39 9.07513974761
0.3904 5.60482156217
0.3908 2.17360595626
0.3912 -1.07558119275
0.3916 -4.00837190456
0.392 -6.5041809996
0.3924 -8.46093369227
0.3928 -9.79911803418
0.3932 -10.4650078658
0.3936 -10.4329311067
0.394 -9.70649203462
0.3944 -8.31869346795
0.3948 -6.33094416189
0.3952 -3.83097683635
0.3956 -0.929741656794
0.396 2.2426227341
0.3964 5.54160471533
0.3968 8.81375665209
0.3972 11.9025648375
0.3976 14.6545470709
0.398 16.9253628653
0.3984 18.5857191514
0.3988 19.5268598597
0.3992 19.6654407289
0.3996 18.9476106655
0.4 17.3521472803
0.4004 14.8925259226
0.4008 11.6178375334
0.4012 7.61250967001
0.4016 2.99482576828
0.402 -2.08572132384
0.4024 -7.45216571043
0.4028 -12.9047816317
0.4032 -18.2266269863
0.4036 -23.189742296
0.404 -27.5618077729
0.4044 -31.1130454193
0.4048 -33.6231446203
0.4052 -34.8879887189
0.4056 -34.725966618
0.406 -32.9836672581
0.4064 -29.5407753713
0.4068 -24.3140134687
0.4072 -17.260006623
0.4076 -8.3769821591
0.408 2.29474539457
0.4084 14.6735141327
0.4088 28.6382498099
0.4092 44.0315282618
0.4096 60.6636617735
0.41 78.3176741754
0.4104 96.7550025407
0.4108 115.721742178
0.4112 134.955236828
0.4116 154.190807976
0.412 173.168416198
0.4124 191.639053348
0.4128 209.370676926
0.4132 226.153516533
0.4136 241.804606234
0.414 256.171424986
0.4144 269.134558965
0.4148 280.609333505
0.4152 290.546397274
0.4156 298.931275911
0.416 305.782945567
0.4164 311.151507432
0.4168 315.115071384
0.4172 317.775979515
0.4176 319.25651777
0.418 319.694275796
0.4184 319.237321031
0.4188 318.039353048
0.4192 316.25499829
0.4196 314.035394036
0.42 311.52419421
0.4204 308.854109191
0.4208 306.144068048
0.4212 303.49706545
0.4216 300.998727992
0.422 298.716606846
0.4224 296.700176554
0.4228 294.981494385
0.4232 293.576451907
0.4236 292.48653105
0.424 291.700961606
0.4244 291.199166245
0.4248 290.953373118
0.4252 290.931274962
0.4256 291.098617269
0.426 291.421606309
0.4264 291.869040088
0.4268 292.414081149
0.4272 293.035608744
0.4276 293.719108556
0.428 294.45707989
0.4284 295.248962299
0.4288 296.100604944
0.4292 297.023321917
0.4296 298.032594335
0.43 299.146494767
0.4304 300.383920741
0.4308 301.762731465
0.4312 303.297885159
0.4316 304.999673435
0.432 306.872144172
0.4324 308.911795474
0.4328 311.106610993
0.4332 313.435491712
0.4336 315.868121833
0.434 318.365287408
0.4344 320.879646595
0.4348 323.356930712
0.4352 325.737536392
0.4356 327.958451821
0.436 329.955445059
0.4364 331.665430285
0.4368 333.028918993
0.4372 333.992458093
0.4376 334.510955616
0.438 334.549797473
0.4384 334.08666527
0.4388 333.112975313
0.4392 331.634872289
0.4396 329.673727121
0.44 327.266106615
0.4404 324.463201968
0.4408 321.329723403
0.4412 317.942288183
0.4416 314.387348457
0.442 310.758722982
0.4424 307.154812101
0.4428 303.675587933
0.4432 300.419460968
0.4436 297.480129924
0.444 294.94352352
0.4444 292.884940747
0.4448 291.366490296
0.4452 290.43492028
0.4456 290.119916586
0.446 290.432932611
0.4464 291.366595246
0.4468 292.894712556
0.4472 294.97288817
0.4476 297.539726851
0.448 300.51859561
0.4484 303.819885938
0.4488 307.343705699
0.4492 310.982914766
0.4496 314.626406869
0.45 318.162531916
0.4504 321.482548346
0.4508 324.4839942
0.4512 327.073868351
0.4516 329.171519787
0.452 330.71115259
0.4524 331.643867058
0.4528 331.939172741
0.4532 331.58592651
0.4536 330.592667507
0.454 328.98734039
0.4544 326.816417874
0.4548 324.143452698
0.4552 321.047107088
0.4556 317.618723963
0.456 313.959518135
0.4564 310.177477013
0.4568 306.384068662
0.4572 302.69086013
0.4576 299.206150744
0.458 296.031723476
0.4584 293.259812723
0.4588 290.970379012
0.4592 289.22877064
0.4596 288.083839409
0.46 287.566562918
0.4604 287.68920976
0.4608 288.445067049
0.4612 289.808732474
0.4616 291.736956077
0.462 294.170000721
0.4624 297.033475233
0.4628 300.240580889
0.4632 303.694700609
0.4636 307.292251307
0.464 310.925713374
0.4644 314.486747557
0.4648 317.869308427
0.4652 320.972665223
0.4656 323.704245043
0.466 325.982219846
0.4664 327.737767313
0.4668 328.916946044
0.4672 329.482137368
0.4676 329.413018959
0.468 328.707049013
0.4684 327.379453569
0.4688 325.462723302
0.4692 323.005639314
0.4696 320.071859854
0.47 316.738111169
0.4704 313.092035485
0.4708 309.229757382
0.4712 305.253236231
0.4716 301.267476894
0.472 297.377673471
0.4724 293.686361442
0.4728 290.290652286
0.4732 287.279621426
0.4736 284.731915522
0.474 282.713638687
0.4744 281.276569415
0.4748 280.456751096
0.4752 280.273489156
0.4756 280.728777383
0.476 281.807165107
0.4764 283.476065868
0.4768 285.686497268
0.4772 288.374231138
0.4776 291.461323145
0.478 294.857981752
0.4784 298.464728294
0.4788 302.174792858
0.4792 305.876685022
0.4796 309.456874247
0.48 312.802512061
0.4804 315.804127128
0.4808 318.358224895
0.4812 320.369725806
0.4816 321.754179936
0.482 322.439701382
0.4824 322.368572636
0.4828 321.498477387
0.4832 319.80332957
0.4836 317.273676733
0.484 313.916666843
0.4844 309.755579031
0.4848 304.828930402
0.4852 299.189182512
0.4856 292.901082117
0.486 286.039681144
0.4864 278.688090085
0.4868 270.935026987
0.4872 262.872230642
0.4876 254.59181119
0.488 246.183614014
0.4884 237.73267339
0.4888 229.316830726
0.4892 221.004588435
0.4896 212.853264532
0.49 204.907505072
0.4904 197.198201654
0.4908 189.741849799
0.4912 182.540371159
0.4916 175.581408743
0.492 168.839090019
0.4924 162.275238146
0.4928 155.840997414
0.4932 149.478825426
0.4936 143.124792198
0.494 136.71111566
0.4944 130.168854255
0.4948 123.430670893
0.4952 116.433578665
0.4956 109.12157765
0.496 101.448093928
0.4964 93.3781366381
0.4968 84.8900964368
0.4972 75.977118889
0.4976 66.6479988948
0.498 56.9275568387
0.4984 46.8564733446
0.4988 36.4905768182
0.4992 25.8995958154
0.4996 15.1654061141
0.5 4.37981959504
0.5004 -6.35802193591
0.5008 -16.9445705788
0.5012 -27.2750244236
0.5016 -37.2466092103
0.502 -46.7618847042
0.5024 -55.7319680977
0.5028 -64.0795674447
0.5032 -71.741722395
0.5036 -78.6721572435
0.504 -84.8431623547
0.5044 -90.2469340422
0.5048 -94.8963195664
0.5052 -98.8249325312
0.5056 -102.086624032
0.506 -104.754315761
0.5064 -106.918222218
0.5068 -108.683509533
0.5072 -110.167457395
0.5076 -111.496207674
0.508 -112.801197731
0.5084 -114.215387828
0.5088 -115.86939983
0.5092 -117.887688418
0.5096 -120.384865976
0.51 -123.46229828
0.5104 -127.205080037
0.5108 -131.679487631
0.5112 -136.930991293
0.5116 -142.982891028
0.512 -149.835620389
0.5124 -157.466740465
0.5128 -165.831623812
0.5132 -174.864805423
0.5136 -184.481955883
0.514 -194.582411405
0.5144 -205.052177207
0.5148 -215.767305275
0.5152 -226.597535535
0.5156 -237.410081264
0.516 -248.073435444
0.5164 -258.461074932
0.5168 -268.454943681
0.5172 -277.948604749
0.5176 -286.849963092
0.518 -295.083476729
0.5184 -302.591792286
0.5188 -309.336761426
0.5192 -315.299816574
0.5196 -320.481706824
0.52 -324.90161721
0.5204 -328.595715738
0.5208 -331.615192055
0.5212 -334.023868588
0.5216 -335.895478865
0.522 -337.310717998
0.5224 -338.35417668
0.5228 -339.111272176
0.5232 -339.665287802
0.5236 -340.094626211
0.524 -340.470371785
0.5244 -340.854244011
0.5248 -341.297007257
0.5252 -341.837383747
0.5256 -342.501496209
0.526 -343.302845695
0.5264 -344.242809009
0.5268 -345.311620031
0.5272 -346.48978065
0.5276 -347.749830757
0.528 -349.058393481
0.5284 -350.378401948
0.5288 -351.671407855
0.5292 -352.899870135
0.5296 -354.02932411
0.53 -355.030337675
0.5304 -355.88017094
0.5308 -356.564069017
0.5312 -357.076133706
0.5316 -357.419738122
0.532 -357.607467971
0.5324 -357.660593592
0.5328 -357.608097039
0.5332 -357.485297708
0.5336 -357.332137476
0.534 -357.191201283
0.5344 -357.105560979
0.5348 -357.116538561
0.5352 -357.261489281
0.5356 -357.57170529
0.536 -358.070536546
0.5364 -358.771817643
0.5368 -359.678677452
0.5372 -360.782793371
0.5376 -362.064134118
0.538 -363.491215164
0.5384 -365.021869745
0.5388 -366.604516898
0.5392 -368.179886828
0.5396 -369.683144188
0.54 -371.046332152
0.5404 -372.201045383
0.5408 -373.081228647
0.5412 -373.625990521
0.5416 -373.782318591
0.542 -373.507584051
0.5424 -372.771729629
0.5428 -371.559045069
0.5432 -369.869448748
0.5436 -367.719211779
0.544 -365.141081575
0.5444 -362.183784487
0.5448 -358.910910989
0.5452 -355.399210976
0.5456 -351.736350165
0.546 -348.018200424
0.5464 -344.34575619
0.5468 -340.821785205
0.5472 -337.547333977
0.5476 -334.618215977
0.548 -332.1216135
0.5484 -330.13292191
0.5488 -328.712957888
0.5492 -327.905641358
0.5496 -327.736244498
0.55 -328.210281141
0.5504 -329.31308669
0.5508 -331.010113283
0.5512 -333.247938256
0.5516 -335.955957036
0.552 -339.048705402
0.5524 -342.428731676
0.5528 -345.989917767
0.5532 -349.621129932
0.5536 -353.21006643
0.554 -356.647160443
0.5544 -359.829393167
0.5548 -362.663873965
0.5552 -365.071052006
0.5556 -366.987436536
0.556 -368.367720509
0.5564 -369.186224098
0.5568 -369.437599752
0.5572 -369.136768165
0.5576 -368.318083612
0.558 -367.033756544
0.5584 -365.35159003
0.5588 -363.352113361
0.5592 -361.125219989
0.5596 -358.766436896
0.56 -356.37296773
0.5604 -354.039661962
0.5608 -351.85506646
0.5612 -349.897714016
0.5616 -348.232795576
0.562 -346.909349306
0.5624 -345.958080818
0.5628 -345.389905401
0.5632 -345.195275883
0.5636 -345.344329777
0.564 -345.787857767
0.5644 -346.459063557
0.5648 -347.276053878
0.5652 -348.14496825
0.5656 -348.963632032
0.566 -349.625594454
0.5664 -350.024396605
0.5668 -350.057903389
0.5672 -349.632528837
0.5676 -348.667186087
0.568 -347.096801831
0.5684 -344.875249834
0.5688 -341.977578787
0.5692 -338.401435527
0.5696 -334.167614619
0.57 -329.31969838
0.5704 -323.922786361
0.5708 -318.061348819
0.5712 -311.836273439
0.5716 -305.361207139
0.572 -298.758323952
0.5724 -292.153674593
0.5728 -285.672292281
0.5732 -279.43324202
0.5736 -273.544806113
0.574 -268.099997041
0.5744 -263.172579751
0.5748 -258.81376925
0.5752 -255.049746578
0.5756 -251.880107558
0.576 -249.277325141
0.5764 -247.187268896
0.5768 -245.530785559
0.5772 -244.20630402
0.5776 -243.093388173
0.578 -242.057123263
0.5784 -240.953187047
0.5788 -239.63342778
0.5792 -237.951747726
0.5796 -235.770074703
0.58 -232.964195745
0.5804 -229.429226765
0.5808 -225.084500345
0.5812 -219.877670195
0.5816 -213.787855095
0.582 -206.827676387
0.5824 -199.044080396
0.5828 -190.517879236
0.5832 -181.361988826
0.5836 -171.718390101
0.584 -161.753886596
0.5844 -151.654777139
0.5848 -141.620604682
0.5852 -131.857179667
0.5856 -122.569107449
0.586 -113.9520729
0.5864 -106.185150414
0.5868 -99.4234135464
0.5872 -93.7911149559
0.5876 -89.3756942442
0.588 -86.2228489632
0.5884 -84.3328731323
0.5888 -83.6584290087
0.5892 -84.1038728128
0.5896 -85.5262050906
0.59 -87.7376630281
0.5904 -90.5099171261
0.5908 -93.5797800324
0.5912 -96.6562828874
0.5916 -99.4289260719
0.592 -101.57686844
0.5924 -102.778783485
0.5928 -102.723083704
0.5932 -101.118196673
0.5936 -97.7025687635
0.594 -92.2540752805
0.5944 -84.5985291809
0.5948 -74.6170039982
0.5952 -62.2517195155
0.5956 -47.5102800434
0.596 -30.4681035953
0.5964 -11.268934252
0.5968 9.87661214708
0.5972 32.6944590261
0.5976 56.8523669666
0.598 81.9669906328
0.5984 107.612512625
0.5988 133.33062338
0.5992 158.641574511
0.5996 183.055999594
0.6 206.08717246
0.6004 227.263359219
0.6008 246.139916967
0.6012 262.31079943
0.6016 275.419147396
0.602 285.166669092
0.6024 291.321551638
0.6028 293.724688326
0.6032 292.294056059
0.6036 287.027131468
0.604 278.001291074
0.6044 265.372198659
0.6048 249.370239938
0.6052 230.295118933
0.6056 208.508780433
0.606 184.42686725
0.6064 158.50895822
0.6068 131.247862223
0.6072 103.158263992
0.6076 74.7650288853
0.608 46.591475904
0.6084 19.1479211933
0.6088 -7.07922134473
0.6092 -31.6375192611
0.6096 -54.1175479914
0.61 -74.160668134
0.6104 -91.4653978763
0.6108 -105.79226856
0.6112 -116.967096633
0.6116 -124.882650561
0.612 -129.498735297
0.6124 -130.840758282
0.6128 -128.996878476
0.6132 -124.11387272
0.6136 -116.391880818
0.614 -106.078211758
0.6144 -93.4604079167
0.6148 -78.8587720282
0.6152 -62.6185630686
0.6156 -45.1020625743
0.616 -26.6807026391
0.6164 -7.72743175026
0.6168 11.3905244933
0.6172 30.3183729409
0.6176 48.7198222078
0.618 66.2824266597
0.6184 82.7221455715
0.6188 97.7870828125
0.6192 111.260395716
0.6196 122.962382868
0.62 132.751778862
0.6204 140.52629924
0.6208 146.222490651
0.6212 149.814949647
0.6216 151.314978647
0.622 150.768749614
0.6224 148.255045319
0.6228 143.882645163
0.6232 137.787417931
0.6236 130.129178097
0.624 121.088356026
0.6244 110.862526136
0.6248 99.6628312986
0.6252 87.7103369504
0.6256 75.2323448351
0.626 62.4586942366
0.6264 49.618078134
0.6268 36.9344028316
0.6272 24.6232221801
0.6276 12.8882812347
0.628 1.91820874264
0.6284 -8.11659723525
0.6288 -17.066841459
0.6292 -24.8069081278
0.6296 -31.2370856998
0.63 -36.2853822619
0.6304 -39.9088745051
0.6308 -42.0945364846
0.6312 -42.8595001231
0.6316 -42.2507079699
0.632 -40.3439300316
0.6324 -37.2421303733
0.6328 -33.0731853499
0.6332 -27.9869733149
0.6336 -22.1518749048
0.634 -15.7507428332
0.6344 -8.97641981211
0.6348 -2.02690193664
0.6352 4.89973817723
0.6356 11.6105398315
0.636 17.9228145162
0.6364 23.669131158
0.6368 28.7019768825
0.6372 32.8979461044
0.6376 36.161318141
0.638 38.426896057
0.6384 39.6619969375
0.6388 39.8675058697
0.6392 39.0779320501
0.6396 37.360434878
0.64 34.8128197569
0.6404 31.5605365783
0.6408 27.7527473959
0.6412 23.5575624281
0.6416 19.1565740747
0.642 14.7388459426
0.6424 10.4945368574
0.6428 6.60835754012
0.6432 3.25306923457
0.6436 0.58323847792
0.644 -1.27053998479
0.6444 -2.20674958105
0.6448 -2.1572016082
0.6452 -1.09026563815
0.6456 0.987052438625
0.646 4.02828293034
0.6464 7.94797448627
0.6468 12.6235027597
0.6472 17.8981786593
0.6476 23.585585236
0.648 29.4750295479
0.6484 35.3379559509
0.6488 40.9351316103
0.6492 46.0243849379
0.6496 50.3686542693
0.65 53.7440883602
0.6504 55.9479328796
0.6508 56.8059384401
0.6512 56.1790359439
0.6516 53.9690439668
0.652 50.1232000586
0.6524 44.6373424383
0.6528 37.5576095575
0.6532 28.9805711246
0.6536 19.0517539452
0.654 7.96257772456
0.6544 -4.0542319122
0.6548 -16.7306352454
0.6552 -29.7705096609
0.6556 -42.8579119933
0.656 -55.6661272912
0.6564 -67.8672303485
0.6568 -79.1418660235
0.6572 -89.1889434236
0.6576 -97.7349379275
0.658 -104.542503888
0.6584 -109.418119487
0.6588 -112.218513098
0.6592 -112.855656758
0.6596 -111.300155875
0.66 -107.582913665
0.6604 -101.795002481
0.6608 -94.0857303926
0.6612 -84.6589482351
0.6616 -73.7676981363
0.662 -61.7073572108
0.6624 -48.8074781472
0.6628 -35.4225700902
0.6632 -21.9220972132
0.6636 -8.67999755712
0.664 3.93595975267
0.6644 15.5746554195
0.6648 25.9116204688
0.6652 34.6582528693
0.6656 41.5699715721
0.666 46.4530628223
0.6664 49.1700148651
0.6668 49.6431847779
0.6672 47.8566935288
0.6676 43.8565007881
0.668 37.7486676683
0.6684 29.6958716139
0.6688 19.9122912876
0.6692 8.65702878348
0.6696 -3.77371972053
0.67 -17.055497051
0.6704 -30.8451455338
0.6708 -44.7907990777
0.6712 -58.542001115
0.6716 -71.7596425234
0.672 -84.1254231824
0.6724 -95.3505595658
0.6728 -105.183488035
0.6732 -113.416348246
0.6736 -119.890072114
0.674 -124.497949625
0.6744 -127.187591951
0.6748 -127.961263062
0.6752 -126.874601825
0.6756 -124.033805581
0.676 -119.591392089
0.6764 -113.740697853
0.6768 -106.709306202
0.6772 -98.7516268629
0.6776 -90.1408695079
0.678 -81.1606663633
0.6784 -72.0966031525
0.6788 -63.2279135891
0.6792 -54.8195806077
0.6796 -47.1150681554
0.68 -40.3298815037
0.6804 -34.6461226932
0.6808 -30.2081720911
0.6812 -27.119588407
0.6816 -25.4412792463
0.682 -25.1909537384
0.6824 -26.3438293105
0.6828 -28.8345275353
0.6832 -32.5600603025
0.6836 -37.3837783301
0.684 -43.1401300299
0.6844 -49.6400605679
0.6848 -56.6768689639
0.6852 -64.0323354057
0.6856 -71.482931511
0.686 -78.8059327644
0.6864 -85.7852642852
0.6868 -92.2169277857
0.6872 -97.9138782499
0.6876 -102.710242604
0.688 -106.464798479
0.6884 -109.063658084
0.6888 -110.422129244
0.6892 -110.485751831
0.6896 -109.230532309
0.69 -106.662421132
0.6904 -102.816096713
0.6908 -97.7531350594
0.6912 -91.5596557769
0.6916 -84.3435426949
0.692 -76.2313409895
0.6924 -67.3649325142
0.6928 -57.8980874113
0.6932 -47.9929834134
0.6936 -37.8167750913
0.694 -27.5382842622
0.6944 -17.3248704743
0.6948 -7.33952759869
0.6952 2.26176029981
0.6956 11.3323828795
0.696 19.7371773976
0.6964 27.3542122713
0.6968 34.0762877336
0.6972 39.8121666115
0.6976 44.4875517291
0.698 48.0458278149
0.6984 50.448585229
0.6988 51.6759405835
0.6992 51.7266657599
0.6996 50.6181323308
0.7 48.3860734478
0.7004 45.084160317
0.7008 40.7833859312
0.7012 35.5712452012
0.7016 29.5506984014
0.702 22.8389042428
0.7024 15.5657101091
0.7028 7.87189017621
0.7032 -0.092872744238
0.7036 -8.17225882058
0.704 -16.2058248785
0.7044 -24.0317388628
0.7048 -31.4897030532
0.7052 -38.4240228651
0.7056 -44.6867670742
0.706 -50.1409550354
0.7064 -54.6636974827
0.7068 -58.149210339
0.7072 -60.5116161362
0.7076 -61.6874455508
0.708 -61.6377525293
0.7084 -60.3497607111
0.7088 -57.8379664382
0.7092 -54.1446345011
0.7096 -49.3396367275
0.71 -43.5196002195
0.7104 -36.8063510405
0.7108 -29.3446598575
0.7112 -21.2993177857
0.7116 -12.8515927286
0.712 -4.19513805432
0.7124 4.46855429002
0.7128 12.9350459417
0.7132 21.0020639124
0.7136 28.4746944639
0.714 35.1705136727
0.7144 40.9245058922
0.7148 45.5936226688
0.7152 49.0608408742
0.7156 51.2385898273
0.716 52.0714327624
0.7164 51.5379077679
0.7168 49.6514567004
0.7172 46.4603968906
0.7176 42.0469188674
0.718 36.5251229356
0.7184 30.0381372716
0.7188 22.7543892586
0.7192 14.8631290677
0.7196 6.5693290714
0.72 -1.91189633613
0.7204 -10.3611892817
0.7208 -18.5609924018
0.7212 -26.3015590877
0.7216 -33.3867502252
0.722 -39.6394428751
0.7224 -44.9063872191
0.7228 -49.0623643353
0.7232 -52.0135184761
0.7236 -53.6997628313
0.724 -54.0961864617
0.7244 -53.2134212419
0.7248 -51.0969602424
0.7252 -47.8254519208
0.7256 -43.5080266931
0.726 -38.2807428469
0.7264 -32.3022663163
0.7268 -25.7489226511
0.7272 -18.8092787882
0.7276 -11.6784263202
0.728 -4.55214640828
0.7284 2.3788609789
0.7288 8.93450426392
0.7292 14.9504139495
0.7296 20.2824876509
0.73 24.810731111
0.7304 28.4422839229
0.7308 31.1135462318
0.7312 32.7913527501
0.7316 33.4731719683
0.732 33.1863404067
0.7324 31.9863730428
0.7328 29.9544206306
0.7332 27.1939715009
0.7336 23.826918731
0.734 19.9891325358
0.7344 15.8256917719
0.7348 11.4859371433
0.7352 7.11851181425
0.7356 2.86655264573
0.736 -1.13681267237
0.7364 -4.77252037142
0.7368 -7.9390526556
0.7372 -10.5552440342
0.7376 -12.5623335977
0.738 -13.9252118116
0.7384 -14.632838584
0.7388 -14.6978378212
0.7392 -14.1553014403
0.7396 -13.0608619113
0.74 -11.4881159812
0.7404 -9.52550251869
0.7408 -7.27275376894
0.7412 -4.83705122765
0.7416 -2.32902450662
0.742 0.141266193021
0.7424 2.46822603417
0.7428 4.55436931534
0.7432 6.31359216142
0.7436 7.67397079619
0.744 8.58000084579
0.7444 8.99421440691
0.7448 8.89813459372
0.7452 8.29255110784
0.7456 7.19712420035
0.746 5.64934737323
0.7464 3.70292051254
0.7468 1.42560414099
0.7472 -1.10335848755
0.7476 -3.79615214905
0.748 -6.55952519398
0.7484 -9.29803705796
0.7488 -11.9173042582
0.7492 -14.3271347305
0.7496 -16.4444474633
0.75 -18.1958848949
0.7504 -19.5200389815
0.7508 -20.3692276522
0.7512 -20.710775903
0.7516 -20.5277743583
0.752 -19.8193070496
0.7524 -18.6001587274
0.7528 -16.9000295729
0.7532 -14.7623011116
0.7536 -12.2424109182
0.754 -9.40590491964
0.7544 -6.32624442654
0.7548 -3.08245024613
0.7552 0.243331714744
0.7556 3.56826003109
0.756 6.81114325147
0.7564 9.8946287827
0.7568 12.7471717498
0.7572 15.304731417
0.7576 17.5121551676
0.758 19.3242233237
0.7584 20.7063416999
0.7588 21.6348821778
0.7592 22.0971842565
0.7596 22.0912420063
0.76 21.625110733
0.7604 20.7160756231
0.7608 19.3896304597
0.7612 17.6783180329
0.7616 15.6204850838
0.762 13.2590035806
0.7624 10.6400069726
0.7628 7.81168505195
0.7632 4.82317446217
0.7636 1.72357411063
0.764 -1.43889385276
0.7644 -4.61756579417
0.7648 -7.7678583197
0.7652 -10.8476356607
0.7656 -13.8174360539
0.766 -16.6405769266
0.7664 -19.2831637521
0.7668 -21.7140308689
0.7672 -23.9046442565
0.7676 -25.8289961857
0.768 -27.4635198704
0.7684 -28.7870488636
0.7688 -29.7808411664
0.7692 -30.4286821194
0.7696 -30.717073429
0.77 -30.6355085015
0.7704 -30.1768269809
0.7708 -29.3376343791
0.7712 -28.1187663197
0.7716 -26.5257714993
0.772 -24.56938331
0.7724 -22.2659473688
0.7728 -19.6377711476
0.7732 -16.7133625611
0.7736 -13.5275267752
0.774 -10.1212945642
0.7744 -6.54166113767
0.7748 -2.8411212558
0.7752 0.923005616031
0.7756 4.68945772396
0.776 8.3941096397
0.7764 11.9712267514
0.7768 15.3548667903
0.7772 18.4803918341
0.7776 21.2860476643
0.778 23.7145622528
0.7784 25.7147117627
0.7788 27.2428009661
0.7792 28.264005505
0.7796 28.7535259996
0.78 28.6975085746
0.7804 28.0936928134
0.7808 26.9517562433
0.7812 25.2933339318
0.7816 23.1517022972
0.782 20.5711274276
0.7824 17.6058896477
0.7828 14.31900734
0.7832 10.7806937033
0.7836 7.06658979247
0.784 3.25582546655
0.7844 -0.571033551697
0.7848 -4.33408969589
0.7852 -7.95620340351
0.7856 -11.3650033669
0.786 -14.4947538919
0.7864 -17.2880204349
0.7868 -19.697080853
0.7872 -21.6850382373
0.7876 -23.2266011127
0.788 -24.3085079295
0.7884 -24.9295847373
0.7888 -25.1004372895
0.7892 -24.8427911368
0.7896 -24.1885050757
0.79 -23.1782942036
0.7904 -21.8602084021
0.7908 -20.2879199811
0.7912 -18.5188801878
0.7916 -16.6124081089
0.792 -14.627777041
0.7924 -12.6223626385
0.7928 -10.649914094
0.7932 -8.75900441717
0.7936 -6.99170873101
0.794 -5.38255069684
0.7944 -3.95774703877
0.7948 -2.73476904616
0.7952 -1.72222830984
0.7956 -0.920082222072
0.796 -0.320143375498
0.7964 0.0931336524713
0.7968 0.341624161922
0.7972 0.452307922148
0.7976 0.455896878497
0.798 0.385370704443
0.7984 0.274477179209
0.7988 0.156254472483
0.7992 0.0616303961895
0.7996 0.0181496450182
0.8 0.0488741735688
0.8004 0.171494396273
0.8008 0.39768015229
0.8012 0.732690700647
0.8016 1.17525278206
0.802 1.71770539955
0.8024 2.34639982744
0.8028 3.04233383998
0.8032 3.7819906102
0.8036 4.5383454773
0.804 5.28199807417
0.8044 5.98238333888
0.8048 6.6090128299
0.8052 7.13269757367
0.8056 7.52670537109
0.806 7.7678089785
0.8064 7.83718669321
0.8068 7.72114338748
0.8072 7.41162766908
0.8076 6.90652928529
0.808 6.20974978446
0.8084 5.33104844907
0.8088 4.28567425964
0.8092 3.09380279984
0.8096 1.77980425674
0.81 0.371374734958
0.8104 -1.10143223188
0.8108 -2.60723398686
0.8112 -4.11436544545
0.8116 -5.5919030028
0.812 -7.01062637573
0.8124 -8.34388255395
0.8128 -9.56832087039
0.8132 -10.6644742043
0.8136 -11.6171682388
0.814 -12.4157482124
0.8144 -13.0541204167
0.8148 -13.5306134731
0.8152 -13.8476718518
0.8156 -14.0114008784
0.816 -14.0309883272
0.8164 -13.9180324074
0.8168 -13.685809313
0.8172 -13.3485154187
0.8176 -12.9205195859
0.818 -12.4156599039
0.8184 -11.846616589
0.8188 -11.2243888192
0.8192 -10.5578981738
0.8196 -9.85373529895
0.82 -9.11605968625
0.8204 -8.34665532842
0.8208 -7.54513779832
0.8212 -6.70930129571
0.8216 -5.83558771079
0.822 -4.91965404175
0.8224 -3.95700981323
0.8228 -2.94369267351
0.8232 -1.87694824724
0.8236 -0.755879679676
0.824 0.417966840878
0.8244 1.64011198286
0.8248 2.90277630785
0.8252 4.19465924422
0.8256 5.50088085324
0.826 6.80309650325
0.8264 8.07978730551
0.8268 9.30672167362
0.8272 10.4575759515
0.8276 11.5046950128
0.828 12.4199673598
0.8284 13.1757837973
0.8288 13.7460444682
0.8292 14.1071760772
0.8296 14.2391196466
0.83 14.1262492023
0.8304 13.7581834053
0.8308 13.1304552665
0.8312 12.2450096104
0.8316 11.1105037247
0.832 9.74239343741
0.8324 8.16279444433
0.8328 6.40011679696
0.8332 4.48847873996
0.8336 2.46691425249
0.834 0.378396389833
0.8344 -1.73129445379
0.8348 -3.81482220091
0.8352 -5.82462668094
0.8356 -7.7142917504
0.836 -9.43990465482
0.8364 -10.9613610747
0.8368 -12.2435716599
0.8372 -13.2575288087
0.8376 -13.9811968824
0.838 -14.4001948169
0.8384 -14.5082469796
0.8388 -14.30738589
0.8392 -13.8078987868
0.8396 -13.028018689
0.84 -11.9933692471
0.8404 -10.7361810164
0.8408 -9.2943044901
0.8412 -7.71005206077
0.8416 -6.02890676632
0.842 -4.29814005022
0.8424 -2.56538365732
0.8428 -0.87720211444
0.8432 0.72228803316
0.8436 2.19270791999
0.844 3.49857405121
0.8444 4.61028778628
0.8448 5.50492635409
0.8452 6.1668120518
0.8456 6.58784391809
0.846 6.76758425708
0.8464 6.71310063608
0.8468 6.43857209749
0.8472 5.96467603169
0.8476 5.31777919124
0.848 4.52896245228
0.8484 3.63291394789
0.8488 2.66672895516
0.8492 1.6686573032
0.8496 0.676840027221
0.85 -0.271923483152
0.8504 -1.14333845825
0.8508 -1.90667162265
0.8512 -2.53567006325
0.8516 -3.00931548293
0.852 -3.31239330179
0.8524 -3.4358626832
0.8528 -3.37702047571
0.8532 -3.13945902828
0.8536 -2.7328246103
0.854 -2.17238951979
0.8544 -1.47845668452
0.8548 -0.67562047038
0.8552 0.208088634927
0.8556 1.14214490522
0.856 2.09452150758
0.8564 3.03270495398
0.8568 3.9246974655
0.8572 4.73997643425
0.8576 5.45038273404
0.858 6.03091304213
0.8584 6.46039543611
0.8588 6.72203214748
0.8592 6.80379830189
0.8596 6.69869056418
0.86 6.404824647
0.8604 5.92538545812
0.8608 5.26843809174
0.8612 4.44661177608
0.8616 3.47667215867
0.862 2.37899986286
0.8624 1.17699503022
0.8628 -0.103571441271
0.8632 -1.43523903308
0.8636 -2.78955069504
0.864 -4.13773863722
0.8644 -5.45138531322
0.8648 -6.70304428128
0.8652 -7.86680816412
0.8656 -8.91881367225
0.866 -9.83767648
0.8664 -10.6048515248
0.8668 -11.2049169267
0.8672 -11.6257820971
0.8676 -11.8588226466
0.868 -11.8989463468
0.8684 -11.7445956297
0.8688 -11.3976928861
0.8692 -10.8635351904
0.8696 -10.1506450421
0.87 -9.27058334369
0.8704 -8.23773018698
0.8708 -7.06903818152
0.8712 -5.78376211055
0.8716 -4.40316773429
0.872 -2.95022165963
0.8724 -1.44926344041
0.8728 0.0743394705349
0.8732 1.59455357478
0.8736 3.08505762327
0.874 4.51963665337
0.8744 5.87259042436
0.8748 7.11915397934
0.8752 8.23592689649
0.8756 9.20130645101
0.876 9.99591846962
0.8764 10.6030382091
0.8768 11.0089922225
0.8772 11.2035309901
0.8776 11.1801611725
0.878 10.9364257887
0.8784 10.4741204878
0.8788 9.79943444222
0.8792 8.92300526796
0.8796 7.85987879207
0.88 6.62936642993
0.8804 5.25479536619
0.8808 3.76314959863
0.8812 2.18460311888
0.8816 0.551949966402
0.882 -1.10006051549
0.8824 -2.7354713037
0.8828 -4.31793283216
0.8832 -5.81156710684
0.8836 -7.18185951475
0.884 -8.39655483887
0.8844 -9.42653244535
0.8848 -10.2466348138
0.8852 -10.8364236055
0.8856 -11.1808383478
0.886 -11.2707345647
0.8864 -11.1032807866
0.8868 -10.6821972656
0.8872 -10.0178233304
0.8876 -9.12700501787
0.888 -8.03279978226
0.8884 -6.76400054494
0.8888 -5.35448693928
0.8892 -3.84241713468
0.8896 -2.26927891056
0.89 -0.678823507063
0.8904 0.884089967207
0.8908 2.37470829686
0.8912 3.74962221848
0.8916 4.96799330747
0.892 5.99273370398
0.8924 6.79160328611
0.8928 7.33819061671
0.8932 7.6127467935
0.8936 7.60284515182
0.894 7.30384449687
0.8944 6.71913904221
0.8948 5.86018433679
0.8952 4.74629498852
0.8956 3.40421673515
0.896 1.86748216134
0.8964 0.175565901068
0.8968 -1.62713871127
0.8972 -3.49249406592
0.8976 -5.36991158149
0.898 -7.20772790975
0.8984 -8.95462534777
0.8988 -10.5610562021
0.8992 -11.9806304919
0.8996 -13.1714272411
0.9 -14.0971916625
0.9004 -14.7283837256
0.9008 -15.043047812
0.9012 -15.0274782822
0.9016 -14.6766616343
0.902 -13.994482352
0.9024 -12.9936863188
0.9028 -11.6956026136
0.9032 -10.1296313759
0.9036 -8.33251205154
0.904 -6.34739248233
0.9044 -4.22272482493
0.9048 -2.011019004
0.9052 0.232511802017
0.9056 2.45137638054
0.906 4.58926248249
0.9064 6.59148613549
0.9068 8.40639440476
0.9072 9.98668442241
0.9076 11.2906042195
0.908 12.2830045375
0.9084 12.9362152418
0.9088 13.2307250686
0.9092 13.1556490402
0.9096 12.7089738204
0.91 11.8975773653
0.9104 10.737025285
0.9108 9.25115219399
0.9112 7.47144183758
0.9116 5.43622478456
0.912 3.18971685909
0.9124 0.78092513336
0.9128 -1.73754885525
0.9132 -4.31077701859
0.9136 -6.88281102729
0.914 -9.39799407454
0.9144 -11.8022458121
0.9148 -14.0442896509
0.9152 -16.0767938389
0.9156 -17.8574006127
0.916 -19.3496211568
0.9164 -20.523577999
0.9168 -21.3565806868
0.9172 -21.8335250191
0.9176 -21.9471106262
0.918 -21.6978761725
0.9184 -21.0940558065
0.9188 -20.1512645846
0.9192 -18.8920243668
0.9196 -17.3451450495
0.92 -15.5449788933
0.9204 -13.5305680814
0.9208 -11.3447074738
0.9212 -9.03294578283
0.9216 -6.64254908803
0.922 -4.22145074373
0.9224 -1.81721132478
0.9228 0.52398865036
0.9232 2.75830185415
0.9236 4.84509356464
0.924 6.74772993136
0.9244 8.43426011432
0.9248 9.87798001032
0.9252 11.0578681516
0.9256 11.9588873492
0.926 12.5721486941
0.9264 12.8949375663
0.9268 12.9306042716
0.9272 12.6883247807
0.9276 12.1827397284
0.928 11.4334822961
0.9284 10.4646078065
0.9288 9.30393976913
0.9292 7.98234868646
0.9296 6.53298114896
0.93 4.99045758095
0.9304 3.39005743428
0.9308 1.76691065829
0.9312 0.155213895713
0.9316 -1.41251092632
0.932 -2.90609911124
0.9324 -4.29835185296
0.9328 -5.56555795643
0.9332 -6.68791771216
0.9336 -7.64986045215
0.934 -8.44024995236
0.9344 -9.05247463055
0.9348 -9.48442235337
0.9352 -9.73834254501
0.9356 -9.82060111191
0.936 -9.74133639106
0.9364 -9.51402682419
0.9368 -9.15498328774
0.9372 -8.68278090769
0.9376 -8.11764670222
0.938 -7.48082047759
0.9384 -6.79390701758
0.9388 -6.07823773017
0.9392 -5.35425953607
0.9396 -4.64096790627
0.94 -3.95539959843
0.9404 -3.31219883659
0.9408 -2.72326847338
0.9412 -2.19751512786
0.9416 -1.74069447728
0.942 -1.35535987761
0.9424 -1.04091438549
0.9428 -0.793763144547
0.9432 -0.607560077601
0.9436 -0.47353998613
0.944 -0.3809245884
0.9444 -0.317388809328
0.9448 -0.269571840082
0.9452 -0.223616171775
0.9456 -0.165717019283
0.946 -0.0826643151014
0.9464 0.0376402215044
0.9468 0.205707556868
0.9472 0.430042570473
0.9476 0.716795367425
0.948 1.06949370171
0.9484 1.48886488719
0.9488 1.97275266576
0.9492 2.51613143148
0.9496 3.11121707659
0.95 3.74767063313
0.9504 4.41288793413
0.9508 5.09236580719
0.9512 5.7701329291
0.9516 6.42923149001
0.952 7.05223430113
0.9524 7.62178097856
0.9528 8.12111637508
0.9532 8.53461452162
0.9536 8.84827197087
0.954 9.05015558002
0.9544 9.13079138048
0.9548 9.08348320101
0.9552 8.90455206009
0.9556 8.59348994006
0.956 8.15302430657
0.9564 7.5890925465
0.9568 6.91072826816
0.9572 6.12986404779
0.9576 5.26105762776
0.958 4.32115069894
0.9584 3.32887116715
0.9588 2.30439116233
0.9592 1.26885396512
0.9596 0.243883482149
0.96 -0.748910099198
0.9604 -1.68841528774
0.9608 -2.5544825556
0.9612 -3.32835476025
0.9616 -3.99305509118
0.962 -4.53372464523
0.9624 -4.93790362364
0.9628 -5.1957521091
0.9632 -5.30020834465
0.9636 -5.24708432409
0.964 -5.03510024792
0.9644 -4.66586094069
0.9648 -4.14377861874
0.9652 -3.47594740683
0.9656 -2.67197570947
0.966 -1.74378294161
0.9664 -0.70536722459
0.9668 0.427449521134
0.9672 1.63729306854
0.9676 2.90551955237
0.968 4.21250465275
0.9684 5.53793546532
0.9688 6.86110208388
0.9692 8.16118673953
0.9696 9.41754906667
0.97 10.6100066538
0.9704 11.7191104457
0.9708 12.7264147672
0.9712 13.6147417231
0.9716 14.3684394916
0.972 14.973633586
0.9724 15.4184695354
0.9728 15.6933446722
0.9732 15.7911258583
0.9736 15.7073490973
0.974 15.440396122
0.9744 14.9916422915
0.9748 14.3655695374
0.9752 13.5698377287
0.9756 12.6153077352
0.976 11.5160096975
0.9764 10.2890505949
0.9768 8.95445614874
0.9772 7.53494340889
0.9776 6.05562202502
0.978 4.54362416356
0.9784 3.02766523991
0.9788 1.53754002623
0.9792 0.103561181273
0.9796 -1.24405026109
0.98 -2.47581051503
0.9804 -3.56364065551
0.9808 -4.48154020787
0.9812 -5.20624920866
0.9816 -5.71788022362
0.982 -6.00050135554
0.9824 -6.04265143215
0.9828 -5.83776936567
0.9832 -5.38452113159
0.9836 -4.68700991188
0.984 -3.75485764841
0.9844 -2.60314949304
0.9848 -1.25223633579
0.9852 0.272605364343
0.9856 1.9416529549
0.986 3.72134832737
0.9864 5.57502787828
0.9868 7.46375547155
0.9872 9.34723845196
0.9876 11.18480353
0.988 12.9364067535
0.9884 14.5636499118
0.9888 16.0307746705
0.9892 17.305605573
0.9896 18.360413794
0.99 19.1726752025
0.9904 19.7256988457
0.9908 20.0091053441
0.9912 20.0191387978
0.9916 19.7588005298
0.992 19.2377981774
0.9924 18.4723091458
0.9928 17.4845630593
0.9932 16.3022534224
0.9936 14.9577940362
0.994 13.4874406302
0.9944 11.9303025003
0.9948 10.3272725337
0.9952 8.71990672936
0.9956 7.14928608468
0.996 5.65489445083
0.9964 4.27354562489
0.9968 3.03839155174
0.9972 1.97804108919
0.9976 1.11581542192
0.998 0.469161998766
0.9984 0.0492439500037
0.9988 -0.139283523143
0.9992 -0.098304126169
0.9996 0.163920837737
1 0.63309973406
1.0004 1.28943172919
1.0008 2.10828434599
1.0012 3.06100523609
1.0016 4.11584194043
1.002 5.23894025972
1.0024 6.39538967742
1.0028 7.55028313634
1.0032 8.66975839476
1.0036 9.72198917333
1.004 10.6780963067
1.0044 11.5129520566
1.0048 12.2058545178
1.0052 12.7410535175
1.0056 13.1081144161
1.006 13.3021115865
1.0064 13.3236488954
1.0068 13.1787100455
1.0072 12.878346975
1.0076 12.4382194819
1.008 11.8780036688
1.0084 11.2206905766
1.0088 10.4917993536
1.0092 9.71853142237
1.0096 8.92889331143
1.01 8.150816077
1.0104 7.41129858826
1.0108 6.7356004177
1.0112 6.14650775148
1.0116 5.66369271209
1.012 5.30318289227
1.0124 5.07695387462
1.0128 4.99265320895
1.0132 5.05345989421
1.0136 5.25807902162
1.014 5.60086702949
1.0144 6.07207913545
1.0148 6.65822707079
1.0152 7.34253234415
1.0156 8.10545798413
1.016 8.92530010491
1.0164 9.77881972775
1.0168 10.6418950727
1.0172 11.4901749797
1.0176 12.2997151734
1.018 13.0475806813
1.0184 13.7123997557
1.0188 14.2748570378
1.0192 14.7181163269
1.0196 15.0281660611
1.02 15.1940833732
1.0204 15.2082152433
1.0208 15.0662777457
1.0212 14.7673765821
1.0216 14.3139539678
1.022 13.71166842
1.0224 12.9692150902
1.0228 12.0980949564
1.0232 11.1123414821
1.0236 10.0282132753
1.024 8.86386089823
1.0244 7.63897534435
1.0248 6.3744248846
1.0252 5.09188606085
1.0256 3.81347364891
1.026 2.56137349467
1.0264 1.35748131089
1.0268 0.22304986181
1.0272 -0.821653501017
1.0276 -1.75767723034
1.028 -2.56770219249
1.0284 -3.23634257354
1.0288 -3.75042875328
1.0292 -4.09926925104
1.0296 -4.2748881764
1.03 -4.27223390372
1.0304 -4.08935400403
1.0308 -3.72753088171
1.0312 -3.19137215053
1.0316 -2.48884960806
1.032 -1.63128078767
1.0324 -0.633247528056
1.0328 0.487553171779
1.0332 1.7105295366
1.0336 3.01248858655
1.034 4.36800292962
1.0344 5.74984860491
1.0348 7.12950938353
1.0352 8.47774037781
1.0356 9.76518121971
1.036 10.9630065543
1.0364 12.04359927
1.0368 12.9812298655
1.0372 13.7527237497
1.0376 14.3380971754
1.038 14.7211420176
1.0384 14.8899397744
1.0388 14.8372860487
1.0392 14.5610083629
1.0396 14.0641624703
1.04 13.3550952977
1.0404 12.4473662272
1.0408 11.3595224927
1.0412 10.1147289138
1.0416 8.74025686616
1.042 7.26684213817
1.0424 5.72792597895
1.0428 4.15879802528
1.0432 2.59566373742
1.0436 1.07466231283
1.044 -0.369136363919
1.0444 -1.7027254594
1.0448 -2.89611459289
1.0452 -3.92321215951
1.0456 -4.76260717407
1.046 -5.39822362503
1.0464 -5.81982371622
1.0468 -6.02334067331
1.0472 -6.01102687872
1.0476 -5.79140881592
1.048 -5.37904646921
1.0484 -4.79410123186
1.0488 -4.06172280617
1.0492 -3.21127180673
1.0496 -2.27540057965
1.05 -1.28901991176
1.0504 -0.288183629344
1.0508 0.691073594594
1.0512 1.61390772833
1.0516 2.44780773381
1.052 3.16367071575
1.0524 3.73677347177
1.0528 4.14760699387
1.0532 4.38254460751
1.0536 4.43431963705
1.054 4.30229458693
1.0544 3.99251062867
1.0548 3.517513454
1.0552 2.89595904878
1.0556 2.15201039951
1.056 1.31454330551
1.0564 0.416186081195
1.0568 -0.507776241313
1.0572 -1.42059776863
1.0576 -2.28532856301
1.058 -3.06609097166
1.0584 -3.72933307028
1.0588 -4.2450173136
1.0592 -4.58770440829
1.0596 -4.73749571772
1.06 -4.68080206474
1.0604 -4.41091246268
1.0608 -3.92834288363
1.0612 -3.24095243711
1.0616 -2.36382203302
1.062 -1.3188984696
1.0624 -0.134414649664
1.0628 1.15589598465
1.0632 2.51376595083
1.0636 3.89751041596
1.064 5.26325850289
1.0644 6.56626003052
1.0648 7.7622262416
1.0652 8.80866221742
1.0656 9.66614925414
1.066 10.2995374469
1.0664 10.6790120106
1.0668 10.7810013339
1.0672 10.5889002408
1.0676 10.0935882229
1.068 9.29372926555
1.0684 8.19584708351
1.0688 6.81417683675
1.0692 5.17030147141
1.0696 3.29258746864
1.07 1.21544076738
1.0704 -1.02159124314
1.0708 -3.37484165038
1.0712 -5.79763687358
1.0716 -8.24146568437
1.072 -10.6571690983
1.0724 -12.9961165285
1.0728 -15.2113354349
1.0732 -17.2585645903
1.0736 -19.097204878
1.074 -20.6911460562
1.0744 -22.0094529763
1.0748 -23.0269001103
1.0752 -23.7243487113
1.0756 -24.0889662798
1.076 -24.1142930377
1.0764 -23.8001646372
1.0768 -23.1525041936
1.0772 -22.1829998104
1.0776 -20.9086859735
1.078 -19.3514484855
1.0784 -17.5374729894
1.0788 -15.4966566247
1.0792 -13.262001057
1.0796 -10.8690031186
1.08 -8.35505674893
1.0804 -5.75887697638
1.0808 -3.11995352156
1.0812 -0.478038406094
1.0816 2.12733110644
1.082 4.65727565577
1.0824 7.07398672347
1.0828 9.34111478193
1.0832 11.424139111
1.0836 13.2907260576
1.084 14.9110820951
1.0844 16.2583069068
1.0848 17.308749942
1.0852 18.042371568
1.0856 18.4431072048
1.086 18.4992298401
1.0864 18.2037032544
1.0868 17.5545153319
1.0872 16.5549781832
1.0876 15.2139796366
1.088 13.5461691402
1.0884 11.5720603881
1.0888 9.31803316298
1.0892 6.81621802013
1.0896 4.10424957963
1.09 1.22487729188
1.0904 -1.77457344755
1.0908 -4.84289216504
1.0912 -7.92582306664
1.0916 -10.9670447398
1.092 -13.9092747923
1.0924 -16.6954784058
1.0928 -19.2701540695
1.0932 -21.5806646155
1.0936 -23.5785773579
1.094 -25.2209738648
1.0944 -26.4716878808
1.0948 -27.3024293135
1.0952 -27.6937531212
1.0956 -27.6358344349
1.096 -27.1290153184
1.0964 -26.1840941384
1.0968 -24.8223354495
1.0972 -23.0751864146
1.0976 -20.9836948181
1.098 -18.597633406
1.0984 -15.9743452694
1.0988 -13.1773349157
1.0992 -10.274639189
1.0996 -7.33702092563
1.1 -4.43603582504
1.1004 -1.64202915208
1.1008 0.977876715469
1.1012 3.36174043844
1.1016 5.45464178985
1.102 7.21034175479
1.1024 8.59270485271
1.1028 9.57683210595
1.1032 10.1498612881
1.1036 10.3114010431
1.104 10.073576871
1.1044 9.46067946161
1.1048 8.50841901016
1.1052 7.26280252413
1.1056 5.77866424964
1.106 4.11789174869
1.1064 2.34740137863
1.1068 0.536926541969
1.1072 -1.24331029223
1.1076 -2.92496545846
1.108 -4.44390272169
1.1084 -5.7424034186
1.1088 -6.7711815137
1.1092 -7.49113232628
1.1096 -7.87475161692
1.11 -7.90717200076
1.1104 -7.58677596065
1.1108 -6.92535866863
};
\addplot [semithick, color1]
table {%
0 -11.6987
0.0004 -12.0662
0.0008 -6.5537
0.0012 -0.398125
0.0016 1.1791
0.002 0.0459375
0.0024 0.6890625
0.0028 3.6137
0.0032 5.1756
0.0036 4.1191
0.004 1.5312
0.0044 0.030625
0.0048 -0.091875
0.0052 0.030625
0.0056 -1.2709
0.006 -3.675
0.0064 -5.8494
0.0068 -7.6409
0.0072 -9.5703
0.0076 -11.5916
0.008 -12.3725
0.0084 -11.8672
0.0088 -10.0297
0.0092 -8.0084
0.0096 -6.8447
0.01 -7.6409
0.0104 -9.8766
0.0108 -11.6834
0.0112 -11.025
0.0116 -8.1769
0.012 -5.0837
0.0124 -3.8894
0.0128 -5.2675
0.0132 -7.6256
0.0136 -8.7894
0.014 -7.9931
0.0144 -5.8647
0.0148 -4.0272
0.0152 -3.43
0.0156 -3.8434
0.016 -4.3028
0.0164 -3.9966
0.0168 -2.6031
0.0172 -0.2603125
0.0176 2.2203
0.018 3.8281
0.0184 3.7975
0.0188 3.8587
0.0192 4.2875
0.0196 4.0425
0.02 1.8987
0.0204 0.06125
0.0208 -0.030625
0.0212 0.091875
0.0216 -2.4041
0.022 -7.3194
0.0224 -11.3619
0.0228 -11.9131
0.0232 -9.2181
0.0236 -7.7787
0.024 -9.5244
0.0244 -11.6834
0.0248 -10.4737
0.0252 -7.9931
0.0256 -7.3959
0.026 -7.8553
0.0264 -6.4925
0.0268 -4.1037
0.0272 -3.3381
0.0276 -3.8587
0.028 -4.0731
0.0284 -3.92
0.0288 -4.0731
0.0292 -3.9659
0.0296 -3.3994
0.03 -3.7669
0.0304 -5.8647
0.0308 -7.7634
0.0312 -7.0284
0.0316 -4.2262
0.032 -1.4547
0.0324 -0.06125
0.0328 0.153125
0.0332 0
0.0336 -0.1378125
0.034 -0.0153125
0.0344 0.030625
0.0348 0
0.0352 -0.1990625
0.0356 -0.030625
0.036 0.214375
0.0364 0.06125
0.0368 -1.225
0.0372 -3.6291
0.0376 -6.2628
0.038 -7.7787
0.0384 -7.7022
0.0388 -7.7634
0.0392 -8.4219
0.0396 -8.0084
0.04 -5.4512
0.0404 -3.8894
0.0408 -5.4666
0.0412 -7.7328
0.0416 -7.0284
0.042 -4.1803
0.0424 -2.6491
0.0428 -3.7209
0.0432 -5.2828
0.0436 -4.1956
0.044 -0.8575
0.0444 0.1684375
0.0448 -3.5525
0.0452 -7.6409
0.0456 -7.1509
0.046 -4.1497
0.0464 -3.1544
0.0468 -3.8894
0.0472 -3.0319
0.0476 -0.275625
0.048 1.5006
0.0484 0.275625
0.0488 -2.8481
0.0492 -4.0119
0.0496 -1.7762
0.05 0.0153125
0.0504 -1.8069
0.0508 -3.8894
0.0512 -2.3275
0.0516 -0.0459375
0.052 -1.1637
0.0524 -3.7669
0.0528 -3.3534
0.0532 -0.2909375
0.0536 1.4241
0.054 0.214375
0.0544 -1.5772
0.0548 -0.3521875
0.0552 4.1191
0.0556 7.7175
0.056 6.0791
0.0564 0.581875
0.0568 -3.9812
0.0572 -4.1344
0.0576 -1.2709
0.058 0.06125
0.0584 -2.0366
0.0588 -3.92
0.0592 -2.3428
0.0596 -0.0765625
0.06 -1.1331
0.0604 -3.7516
0.0608 -4.2109
0.0612 -3.8587
0.0616 -5.5584
0.062 -7.7175
0.0624 -7.8094
0.0628 -7.7328
0.0632 -9.6316
0.0636 -11.6834
0.064 -10.5503
0.0644 -8.0238
0.0648 -7.0284
0.0652 -7.7481
0.0656 -7.9472
0.066 -7.84
0.0664 -7.3347
0.0668 -4.3947
0.0672 1.1178
0.0676 3.9659
0.068 1.3628
0.0684 -0.1684375
0.0688 3.92
0.0692 7.7941
0.0696 4.655
0.07 0.1378125
0.0704 1.9906
0.0708 7.4419
0.0712 8.2381
0.0716 4.3487
0.072 0.9034374
0.0724 0
0.0728 0.06125
0.0732 0
0.0736 -0.091875
0.074 -0.0153125
0.0744 0.1378125
0.0748 0.030625
0.0752 -0.42875
0.0756 -0.1071875
0.076 0.949375
0.0764 0.245
0.0768 -3.675
0.0772 -7.595
0.0776 -8.2994
0.078 -7.7941
0.0784 -9.0956
0.0788 -11.5609
0.0792 -12.2041
0.0796 -11.7753
0.08 -11.6681
0.0804 -11.7906
0.0808 -10.2594
0.0812 -8.0391
0.0816 -6.0637
0.082 -4.1344
0.0824 -1.7762
0.0828 -0.0765625
0.0832 -0.091875
0.0836 -0.0765625
0.084 1.3475
0.0844 3.7056
0.0848 4.5631
0.0852 4.0119
0.0856 2.45
0.086 0.2603125
0.0864 -2.3581
0.0868 -3.8587
0.0872 -3.6444
0.0876 -3.7822
0.088 -5.8341
0.0884 -7.7634
0.0888 -6.8294
0.0892 -4.1803
0.0896 -1.9753
0.09 -0.18375
0.0904 1.3322
0.0908 0.3215625
0.0912 -4.2109
0.0916 -7.7328
0.092 -6.615
0.0924 -4.0272
0.0928 -4.6856
0.0932 -7.5797
0.0936 -8.8047
0.094 -7.9472
0.0944 -7.0744
0.0948 -7.7175
0.0952 -8.5137
0.0956 -7.9931
0.096 -5.7422
0.0964 -3.9812
0.0968 -4.0119
0.0972 -4.0119
0.0976 -2.1131
0.098 -0.091875
0.0984 -0.153125
0.0988 -0.1225
0.0992 1.9906
0.0996 3.8894
0.1 2.3887
0.1004 0.1071875
0.1008 -0.1684375
0.1012 0.06125
0.1016 -1.715
0.102 -3.8281
0.1024 -3.1237
0.1028 -0.2603125
0.1032 0.6890625
0.1036 -3.2769
0.104 -9.8
0.1044 -11.9744
0.1048 -6.615
0.1052 -0.336875
0.1056 -0.1684375
0.106 -3.6291
0.1064 -4.655
0.1068 -3.9047
0.1072 -5.1603
0.1076 -7.6562
0.108 -8.1616
0.1084 -7.8094
0.1088 -8.4066
0.1092 -8.0391
0.1096 -4.4559
0.11 -0.2909375
0.1104 1.0872
0.1108 0.1225
0.1112 -0.6890625
0.1116 -0.091875
0.112 0.5512499
0.1124 0.1071875
0.1128 -1.0259
0.1132 -0.2296875
0.1136 2.6031
0.114 4.0119
0.1144 1.0566
0.1148 -3.5525
0.1152 -5.2828
0.1156 -4.0731
0.116 -2.9553
0.1164 -3.7669
0.1168 -5.0837
0.1172 -4.1956
0.1176 -0.2603125
0.118 3.6597
0.1184 4.655
0.1188 3.9812
0.1192 3.1084
0.1196 0.4440625
0.12 -5.0072
0.1204 -7.9012
0.1208 -4.6397
0.1212 -0.214375
0.1216 -0.6737499
0.122 -3.7056
0.1224 -4.2569
0.1228 -3.8281
0.1232 -6.4466
0.1236 -11.3159
0.124 -13.6894
0.1244 -12.0816
0.1248 -8.0391
0.1252 -4.2109
0.1256 -2.45
0.126 -3.6597
0.1264 -6.37
0.1268 -7.84
0.1272 -6.5537
0.1276 -4.1191
0.128 -3.0166
0.1284 -3.7822
0.1288 -4.8387
0.1292 -4.1037
0.1296 -1.6691
0.13 -0.0153125
0.1304 -1.0872
0.1308 -3.675
0.1312 -5.0837
0.1316 -4.1037
0.132 -1.8528
0.1324 -0.091875
0.1328 0.2603125
0.1332 0.0153125
0.1336 -0.091875
0.134 0
0.1344 -0.1990625
0.1348 -0.0459375
0.1352 0.5053125
0.1356 0.1378125
0.136 -2.0825
0.1364 -3.8587
0.1368 -3.5831
0.1372 -3.7516
0.1376 -6.1403
0.138 -7.8553
0.1384 -6.0025
0.1388 -3.9353
0.1392 -5.2828
0.1396 -7.7328
0.14 -6.9059
0.1404 -4.1344
0.1408 -3.0931
0.1412 -3.8281
0.1416 -4.2262
0.142 -3.9353
0.1424 -3.8587
0.1428 -3.92
0.1432 -3.7669
0.1436 -3.8587
0.144 -4.9919
0.1444 -7.5337
0.1448 -9.5244
0.1452 -8.1922
0.1456 -3.3228
0.146 -0.0459375
0.1464 -1.5619
0.1468 -3.8741
0.1472 -1.7762
0.1476 3.4147
0.148 6.8141
0.1484 7.7787
0.1488 7.9012
0.1492 7.8859
0.1496 6.2475
0.15 4.1037
0.1504 2.1591
0.1508 0.214375
0.1512 -2.3275
0.1516 -3.8741
0.152 -3.6597
0.1524 -3.7975
0.1528 -5.6656
0.1532 -7.7175
0.1536 -7.7481
0.154 -7.7328
0.1544 -9.4631
0.1548 -11.6375
0.1552 -11.025
0.1556 -8.1462
0.156 -5.3747
0.1564 -3.9812
0.1568 -3.5525
0.1572 -3.8434
0.1576 -4.3794
0.158 -4.0119
0.1584 -3.0625
0.1588 -3.7056
0.1592 -6.3853
0.1596 -7.8859
0.16 -5.2981
0.1604 -0.42875
0.1608 2.1744
0.1612 0.3675
0.1616 -3.3994
0.162 -4.1191
0.1624 -0.2603125
0.1628 3.7209
0.1632 3.2616
0.1636 0.245
0.164 -1.0412
0.1644 -0.1071875
0.1648 0.4134375
0.1652 0.030625
0.1656 -0.1990625
0.166 -0.0153125
0.1664 0
0.1668 0
0.1672 0.1225
0.1676 0.0459375
0.168 -0.9953125
0.1684 -3.5831
0.1688 -6.7681
0.1692 -7.9012
0.1696 -6.0178
0.17 -4.0119
0.1704 -3.8894
0.1708 -4.0119
0.1712 -2.0825
0.1716 -0.0459375
0.172 -1.6231
0.1724 -7.1816
0.1728 -12.4491
0.1732 -12.1734
0.1736 -5.8953
0.174 -0.2603125
0.1744 -0.6278125
0.1748 -3.7056
0.1752 -4.1956
0.1756 -3.8128
0.176 -5.8953
0.1764 -7.84
0.1768 -6.0484
0.1772 -3.92
0.1776 -5.6503
0.178 -7.8247
0.1784 -5.3441
0.1788 -0.398125
0.1792 1.715
0.1796 0.2296875
0.18 -2.205
0.1804 -3.7975
0.1808 -4.3947
0.1812 -4.0119
0.1816 -2.5266
0.182 -0.2296875
0.1824 1.3628
0.1828 0.275625
0.1832 -3.0012
0.1836 -4.0731
0.184 -0.4746875
0.1844 3.6903
0.1848 3.2769
0.1852 0.245
0.1856 -1.0412
0.186 -0.1071875
0.1864 0.3828125
0.1868 0.0459375
0.1872 -0.8421874
0.1876 -3.5066
0.188 -8.2381
0.1884 -11.6222
0.1888 -10.5656
0.1892 -8.0084
0.1896 -7.2428
0.19 -7.8094
0.1904 -7.4266
0.1908 -7.6869
0.1912 -9.1569
0.1916 -8.2228
0.192 -2.3887
0.1924 3.5678
0.1928 3.8587
0.1932 0.336875
0.1936 -1.6691
0.194 -0.2296875
0.1944 1.5159
0.1948 0.3215625
0.1952 -3.9659
0.1956 -7.6409
0.196 -7.4419
0.1964 -4.2722
0.1968 -1.1484
0.1972 -0.0153125
0.1976 -0.06125
0.198 -0.0153125
0.1984 0.153125
0.1988 0.030625
0.1992 -0.4134375
0.1996 -0.1071875
0.2 1.4394
0.2004 3.7209
0.2008 3.8741
0.2012 0.4899999
0.2016 -4.9306
0.202 -7.8094
0.2024 -6.2322
0.2028 -3.9659
0.2032 -5.6197
0.2036 -11.1475
0.204 -15.6953
0.2044 -15.8944
0.2048 -12.5409
0.2052 -11.5762
0.2056 -14.3937
0.206 -15.8484
0.2064 -10.8719
0.2068 -4.3487
0.2072 -2.6491
0.2076 -3.8741
0.208 -2.7562
0.2084 -0.153125
0.2088 0.1378125
0.2092 -0.0765625
0.2096 1.715
0.21 3.8434
0.2104 2.7562
0.2108 0.1990625
0.2112 -0.7809374
0.2116 -0.0765625
0.212 0.3215625
0.2124 0.030625
0.2128 -0.214375
0.2132 -0.030625
0.2136 0.1378125
0.214 0.0153125
0.2144 -0.275625
0.2148 -0.06125
0.2152 0.581875
0.2156 0.1378125
0.216 -2.0825
0.2164 -3.8587
0.2168 -3.6444
0.2172 -3.7669
0.2176 -6.0025
0.218 -7.8247
0.2184 -6.3087
0.2188 -4.0119
0.2192 -4.6244
0.2196 -7.5491
0.22 -9.1416
0.2204 -8.0544
0.2208 -5.5584
0.2212 -3.9812
0.2216 -3.8894
0.222 -3.9506
0.2224 -3.4147
0.2228 -3.7669
0.2232 -6.0637
0.2236 -7.8094
0.224 -6.5078
0.2244 -4.0731
0.2248 -3.4912
0.2252 -3.9047
0.2256 -3.5066
0.226 -3.7669
0.2264 -6.0025
0.2268 -7.8094
0.2272 -6.4925
0.2276 -4.0731
0.228 -3.5525
0.2284 -3.9353
0.2288 -2.7256
0.2292 -0.2296875
0.2296 1.0719
0.23 0.18375
0.2304 -1.9141
0.2308 -3.7669
0.2312 -4.5937
0.2316 -4.0272
0.232 -3.0625
0.2324 -3.7362
0.2328 -5.4666
0.2332 -4.3028
0.2336 0.7196875
0.234 3.9353
0.2344 0.826875
0.2348 -3.675
0.2352 -3.2922
0.2356 -0.18375
0.236 -0.3675
0.2364 -3.5831
0.2368 -6.2169
0.2372 -7.6869
0.2376 -8.82
0.238 -8.085
0.2384 -4.0884
0.2388 -0.1990625
0.2392 -0.245
0.2396 -3.5372
0.24 -6.615
0.2404 -7.8247
0.2408 -6.86
0.2412 -4.2416
0.2416 -1.1484
0.242 0.0459375
0.2424 -1.6384
0.2428 -3.7975
0.2432 -3.4147
0.2436 -0.3675
0.244 2.8481
0.2444 3.9506
0.2448 2.3428
0.2452 0.153125
0.2456 -0.8575
0.246 -0.1225
0.2464 0.9034374
0.2468 0.18375
0.2472 -2.2662
0.2476 -3.8741
0.248 -3.5066
0.2484 -3.7516
0.2488 -6.2016
0.2492 -7.8859
0.2496 -5.2216
0.25 -0.398125
0.2504 1.7303
0.2508 0.245
0.2512 -2.3734
0.2516 -3.8587
0.252 -4.0884
0.2524 -3.9047
0.2528 -4.7316
0.2532 -7.4725
0.2536 -10.7494
0.254 -11.8212
0.2544 -9.6928
0.2548 -7.8553
0.2552 -8.9119
0.2556 -11.5609
0.256 -11.5762
0.2564 -8.2841
0.2568 -3.5678
0.2572 -0.1990625
0.2576 0.79625
0.258 0.1071875
0.2584 -0.826875
0.2588 -0.1684375
0.2592 1.8987
0.2596 3.7975
0.26 3.9659
0.2604 3.8741
0.2608 4.1956
0.2612 4.0272
0.2616 2.0366
0.262 0.091875
0.2624 -0.1071875
0.2628 0.06125
0.2632 -1.4853
0.2636 -3.7362
0.264 -4.4253
0.2644 -3.9506
0.2648 -3.675
0.2652 -3.8894
0.2656 -3.7669
0.266 -3.8741
0.2664 -4.2875
0.2668 -4.0272
0.2672 -2.1744
0.2676 -0.1378125
0.268 0.4134375
0.2684 0.030625
0.2688 -0.091875
0.2692 0
0.2696 -0.214375
0.27 -0.06125
0.2704 0.581875
0.2708 0.153125
0.2712 -2.2203
0.2716 -3.9047
0.272 -2.5266
0.2724 -0.153125
0.2728 0.4134375
0.2732 0
0.2736 0.336875
0.274 0.1225
0.2744 -2.0672
0.2748 -3.8894
0.2752 -2.7103
0.2756 -0.18375
0.276 0.5665625
0.2764 0.030625
0.2768 -0.1071875
0.2772 0.0153125
0.2776 -1.0719
0.278 -3.6137
0.2784 -6.3394
0.2788 -7.7787
0.2792 -7.7022
0.2796 -7.7634
0.28 -8.3606
0.2804 -7.9931
0.2808 -5.5891
0.2812 -3.92
0.2816 -5.1603
0.282 -7.6716
0.2824 -7.5797
0.2828 -4.3181
0.2832 -0.704375
0.2836 0.1071875
0.284 -2.1437
0.2844 -3.9047
0.2848 -2.6797
0.2852 -0.18375
0.2856 0.4440625
0.286 0
0.2864 0.18375
0.2868 0.1071875
0.2872 -1.9294
0.2876 -3.8434
0.288 -3.6597
0.2884 -3.7669
0.2888 -6.1403
0.2892 -7.8553
0.2896 -5.9412
0.29 -3.92
0.2904 -5.3747
0.2908 -7.7481
0.2912 -6.6916
0.2916 -4.0884
0.292 -3.5372
0.2924 -3.9506
0.2928 -2.3734
0.2932 -0.1225
0.2936 0.18375
0.294 -0.030625
0.2944 0.7196875
0.2948 0.214375
0.2952 -2.8941
0.2956 -4.0884
0.296 -0.2603125
0.2964 3.7669
0.2968 2.7256
0.2972 0.091875
0.2976 1.0412
0.298 3.7669
0.2984 3.1697
0.2988 0.245
0.2992 -1.0872
0.2996 -0.1225
0.3 0.6124999
0.3004 0.1071875
0.3008 -1.4394
0.3012 -3.675
0.3016 -6.0637
0.302 -7.7328
0.3024 -8.0697
0.3028 -7.8553
0.3032 -7.6716
0.3036 -7.8094
0.304 -7.8553
0.3044 -7.84
0.3048 -7.6409
0.3052 -7.7941
0.3056 -7.9319
0.306 -7.8553
0.3064 -7.4572
0.3068 -7.7481
0.3072 -8.4831
0.3076 -8.0238
0.308 -4.8694
0.3084 -0.4134375
0.3088 2.7716
0.3092 3.8894
0.3096 3.6291
0.31 3.8434
0.3104 4.3028
0.3108 4.0272
0.3112 1.9906
0.3116 0.091875
0.312 -0.3215625
0.3124 0
0.3128 -1.1025
0.3132 -3.6444
0.3136 -6.0637
0.314 -7.7175
0.3144 -8.3147
0.3148 -7.9166
0.3152 -6.9519
0.3156 -7.6256
0.316 -10.1062
0.3164 -11.7447
0.3168 -10.2747
0.3172 -7.9778
0.3176 -7.4112
0.318 -7.84
0.3184 -6.7375
0.3188 -4.165
0.3192 -2.6031
0.3196 -3.675
0.32 -6.3394
0.3204 -7.8247
0.3208 -6.5844
0.3212 -4.1191
0.3216 -3.0319
0.322 -3.7975
0.3224 -4.7469
0.3228 -4.0884
0.3232 -1.8834
0.3236 -0.091875
0.324 0.0459375
0.3244 -0.030625
0.3248 0.5512499
0.3252 0.1684375
0.3256 -2.5112
0.326 -3.9812
0.3264 -1.9141
0.3268 0
0.3272 -1.8834
0.3276 -3.92
0.328 -2.1131
0.3284 0
0.3288 -2.5725
0.3292 -7.4572
0.3296 -9.1262
0.33 -7.9625
0.3304 -7.2122
0.3308 -7.7634
0.3312 -7.8706
0.3316 -7.8094
0.332 -7.9931
0.3324 -7.9012
0.3328 -6.3853
0.3332 -4.1344
0.3336 -2.0519
0.334 -0.18375
0.3344 1.6844
0.3348 3.7209
0.3352 4.7622
0.3356 4.0884
0.336 1.7303
0.3364 0.06125
0.3368 -0.0765625
0.3372 0.0459375
0.3376 -1.47
0.338 -3.7362
0.3384 -4.7009
0.3388 -4.0272
0.3392 -3.0931
0.3396 -3.7516
0.34 -5.7422
0.3404 -7.7022
0.3408 -8.1616
0.3412 -7.8553
0.3416 -7.6716
0.342 -7.84
0.3424 -6.9978
0.3428 -4.2722
0.3432 -0.826875
0.3436 0.1071875
0.344 -2.2816
0.3444 -3.9506
0.3448 -1.9447
0.3452 0
0.3456 -1.9294
0.346 -3.9353
0.3464 -2.0059
0.3468 0.0153125
0.3472 -2.2356
0.3476 -4.0425
0.348 -0.3675
0.3484 3.7975
0.3488 1.5925
0.3492 -3.5372
0.3496 -5.0837
0.35 -3.9659
0.3504 -4.7928
0.3508 -7.5797
0.3512 -8.9119
0.3516 -7.9778
0.352 -6.9059
0.3524 -7.6716
0.3528 -8.9731
0.3532 -8.1156
0.3536 -3.9506
0.354 -0.153125
0.3544 -1.225
0.3548 -7.1356
0.3552 -12.5103
0.3556 -12.1581
0.356 -5.88
0.3564 -0.2603125
0.3568 -0.398125
0.3572 -3.6597
0.3576 -4.0731
0.358 -0.459375
0.3584 4.1344
0.3588 7.5797
0.3592 8.9425
0.3596 8.0544
0.36 5.1909
0.3604 3.8587
0.3608 5.2828
0.3612 7.6716
0.3616 7.3041
0.362 4.2722
0.3624 0.8421874
0.3628 -0.06125
0.3632 0.9799999
0.3636 0.2909375
0.364 -4.5172
0.3644 -11.1628
0.3648 -15.1134
0.3652 -15.7412
0.3656 -14.1487
0.366 -11.9897
0.3664 -9.2641
0.3668 -7.8553
0.3672 -7.9931
0.3676 -7.9319
0.368 -5.9719
0.3684 -4.0119
0.3688 -3.92
0.3692 -3.9966
0.3696 -2.3428
0.37 -0.1378125
0.3704 0.2603125
0.3708 0
0.3712 0.2909375
0.3716 0.1225
0.372 -1.9294
0.3724 -3.8434
0.3728 -3.1084
0.3732 -0.275625
0.3736 1.5312
0.374 0.275625
0.3744 -2.8634
0.3748 -4.0119
0.3752 -1.8375
0.3756 0
0.376 -1.6537
0.3764 -3.8587
0.3768 -2.8328
0.3772 -0.1684375
0.3776 -0.2603125
0.378 -3.5219
0.3784 -6.8906
0.3788 -7.8706
0.3792 -7.1203
0.3796 -7.6409
0.38 -10.0909
0.3804 -11.76
0.3808 -9.9531
0.3812 -7.8706
0.3816 -8.7128
0.382 -11.5303
0.3824 -11.7141
0.3828 -8.2994
0.3832 -3.3381
0.3836 -0.1684375
0.384 1.0412
0.3844 3.5066
0.3848 8.085
0.3852 11.6222
0.3856 10.3666
0.386 7.9319
0.3864 8.33
0.3868 11.4691
0.3872 12.0356
0.3876 8.3912
0.388 1.9141
0.3884 -3.5219
0.3888 -6.2016
0.3892 -7.6562
0.3896 -9.555
0.39 -11.6222
0.3904 -11.8978
0.3908 -11.7294
0.3912 -11.8519
0.3916 -11.8366
0.392 -9.8919
0.3924 -7.9472
0.3928 -7.4266
0.3932 -7.8247
0.3936 -6.86
0.394 -4.2109
0.3944 -1.3934
0.3948 -0.0153125
0.3952 -0.2603125
0.3956 -0.091875
0.396 1.6231
0.3964 3.7669
0.3968 4.2722
0.3972 3.9353
0.3976 3.9047
0.398 3.9506
0.3984 3.3381
0.3988 3.7516
0.3992 5.8953
0.3996 7.7787
0.4 6.4925
0.4004 4.0731
0.4008 2.8634
0.4012 3.7516
0.4016 4.3334
0.402 3.9966
0.4024 1.4241
0.4028 -3.3228
0.4032 -10.5809
0.4036 -18.8037
0.404 -25.7709
0.4044 -30.9312
0.4048 -33.7487
0.4052 -38.5262
0.4056 -43.5334
0.406 -46.9175
0.4064 -43.4569
0.4068 -39.4297
0.4072 -35.1881
0.4076 -28.4812
0.408 -12.1734
0.4084 9.5856
0.4088 30.0737
0.4092 45.7231
0.4096 57.9884
0.41 69.5034
0.4104 78.645
0.4108 82.4272
0.4112 79.3647
0.4116 74.8475
0.412 75.1231
0.4124 81.5084
0.4128 92.6406
0.4132 107.9837
0.4136 128.7628
0.414 154.1662
0.4144 179.0337
0.4148 198.45
0.4152 208.3112
0.4156 214.9569
0.416 218.9228
0.4164 219.8262
0.4168 212.0781
0.4172 204.33
0.4176 201.39
0.418 203.6103
0.4184 203.3959
0.4188 200.2722
0.4192 196.9953
0.4196 195.9694
0.42 196.4747
0.4204 196.1225
0.4208 194.2084
0.4212 192.1872
0.4216 191.8044
0.422 192.0953
0.4224 190.8856
0.4228 188.405
0.4232 186.2766
0.4236 184.4237
0.424 182.5097
0.4244 180.5037
0.4248 179.1103
0.4252 176.7216
0.4256 174.195
0.426 172.5106
0.4264 174.9606
0.4268 179.7994
0.4272 185.3884
0.4276 188.0681
0.428 189.875
0.4284 195.1425
0.4288 206.4584
0.4292 218.5247
0.4296 226.0278
0.43 230.79
0.4304 236.5169
0.4308 242.6112
0.4312 244.8009
0.4316 243.3156
0.432 241.5241
0.4324 242.7491
0.4328 247.2662
0.4332 254.0803
0.4336 261.5834
0.434 269.6378
0.4344 278.4425
0.4348 288.9928
0.4352 297.9506
0.4356 301.84
0.436 298.3794
0.4364 294.2297
0.4368 293.1119
0.4372 294.0153
0.4376 291.3662
0.438 286.604
0.4384 283.0822
0.4388 282.24
0.4392 282.3625
0.4396 282.2553
0.44 282.3625
0.4404 282.2553
0.4408 282.3319
0.4412 282.2553
0.4416 282.3625
0.442 282.2553
0.4424 282.1787
0.4428 282.2094
0.4432 283.1587
0.4436 285.7925
0.444 290.0034
0.4444 293.7397
0.4448 294.934
0.4452 294.1378
0.4456 292.3922
0.446 290.2943
0.4464 288.7019
0.4468 289.7737
0.4472 293.8622
0.4476 297.7056
0.448 297.5984
0.4484 294.3522
0.4488 291.9787
0.4492 293.6019
0.4496 298.3028
0.45 301.7175
0.4504 300.6609
0.4508 298.0884
0.4512 297.185
0.4516 297.8587
0.452 297.5984
0.4524 297.7822
0.4528 299.0684
0.4532 298.2569
0.4536 292.5606
0.454 286.5428
0.4544 284.5062
0.4548 282.6228
0.4552 277.9372
0.4556 274.4306
0.456 277.7075
0.4564 282.0562
0.4568 281.6887
0.4572 278.5037
0.4576 280.0962
0.458 285.5628
0.4584 291.55
0.4588 297.2922
0.4592 303.8765
0.4596 309.3125
0.46 311.2112
0.4604 313.2478
0.4608 317.1219
0.4612 321.195
0.4616 320.6437
0.462 317.7956
0.4624 314.9168
0.4628 313.7072
0.4632 311.1653
0.4636 306.3419
0.464 299.834
0.4644 294.3828
0.4648 291.5347
0.4652 290.1872
0.4656 288.4722
0.466 286.2978
0.4664 285.0575
0.4668 285.9303
0.4672 287.3544
0.4676 286.4356
0.468 281.4437
0.4684 274.9972
0.4688 268.5047
0.4692 259.8072
0.4696 247.6184
0.47 236.0575
0.4704 231.3106
0.4708 231.2034
0.4712 232.0456
0.4716 231.3565
0.472 231.5097
0.4724 231.3106
0.4728 231.9231
0.4732 234.7253
0.4736 240.3144
0.474 243.1012
0.4744 239.7325
0.4748 235.3991
0.4752 235.6441
0.4756 238.8444
0.476 239.6253
0.4764 239.0894
0.4768 239.3344
0.4772 239.2578
0.4776 235.7972
0.478 231.5709
0.4784 229.3812
0.4788 227.6356
0.4792 222.9806
0.4796 216.2431
0.48 210.5775
0.4804 207.8519
0.4808 207.4078
0.4812 207.6987
0.4816 207.6528
0.482 207.6987
0.4824 207.1781
0.4828 204.2994
0.4832 197.5772
0.4836 188.9869
0.484 180.8712
0.4844 173.2456
0.4848 165.3137
0.4852 157.4584
0.4856 152.6809
0.486 152.6044
0.4864 156.0344
0.4868 156.9531
0.4872 153.2016
0.4876 149.1131
0.488 149.0519
0.4884 149.1744
0.4888 143.7691
0.4892 134.2141
0.4896 126.1903
0.49 121.8109
0.4904 118.6412
0.4908 114.17
0.4912 108.9025
0.4916 105.8706
0.492 106.9272
0.4924 109.5303
0.4928 109.6222
0.4932 106.2841
0.4936 100.205
0.494 91.2166
0.4944 79.3341
0.4948 67.62
0.4952 59.5503
0.4956 55.1556
0.496 52.5984
0.4964 51.0059
0.4968 50.5159
0.4972 50.8528
0.4976 50.5312
0.498 47.4994
0.4984 40.1953
0.4988 28.7262
0.4992 14.8837
0.4996 1.2403
0.5 -9.8766
0.5004 -18.8344
0.5008 -24.7603
0.5012 -27.4094
0.5016 -25.9394
0.502 -23.7037
0.5024 -22.6166
0.5028 -23.4281
0.5032 -24.5919
0.5036 -27.0725
0.504 -32.5697
0.5044 -42.0175
0.5048 -53.1037
0.5052 -62.0616
0.5056 -65.905
0.506 -66.6247
0.5064 -66.4409
0.5068 -66.6553
0.5072 -65.4456
0.5076 -63.0109
0.508 -60.025
0.5084 -58.8
0.5088 -59.9178
0.5092 -62.4444
0.5096 -65.7519
0.51 -70.0547
0.5104 -75.4753
0.5108 -78.3694
0.5112 -77.4353
0.5116 -74.6944
0.512 -75.1078
0.5124 -78.0784
0.5128 -82.0903
0.5132 -85.8419
0.5136 -91.6759
0.514 -100.8175
0.5144 -113.5116
0.5148 -127.9512
0.5152 -140.5841
0.5156 -148.5006
0.516 -150.9047
0.5164 -152.635
0.5168 -155.2075
0.5172 -156.8612
0.5176 -154.5644
0.518 -152.8341
0.5184 -155.9884
0.5188 -163.7978
0.5192 -171.1784
0.5196 -176.0325
0.52 -179.3247
0.5204 -183.704
0.5208 -188.6959
0.5212 -191.9575
0.5216 -190.9316
0.522 -188.3591
0.5224 -187.0728
0.5228 -188.0222
0.5232 -188.65
0.5236 -188.2365
0.524 -187.4862
0.5244 -188.0069
0.5248 -190.0128
0.5252 -191.9575
0.5256 -192.7231
0.526 -192.1719
0.5264 -191.9881
0.5268 -192.08
0.5272 -192.4781
0.5276 -192.1412
0.528 -192.6312
0.5284 -195.5253
0.5288 -202.3087
0.5292 -210.8378
0.5296 -218.9687
0.53 -226.5944
0.5304 -233.73
0.5308 -238.8137
0.5312 -239.7172
0.5316 -239.1506
0.532 -239.0281
0.5324 -239.1812
0.5328 -238.0481
0.5332 -238.8444
0.5336 -242.8869
0.534 -246.7609
0.5344 -246.96
0.5348 -246.7916
0.5352 -250.1603
0.5356 -254.5091
0.536 -255.5503
0.5364 -254.8153
0.5368 -255.9484
0.5372 -258.5209
0.5376 -259.4397
0.538 -258.7965
0.5384 -258.3372
0.5388 -258.6587
0.5392 -259.5775
0.5396 -262.2419
0.54 -266.6519
0.5404 -270.2656
0.5408 -270.8322
0.5412 -270.4647
0.5416 -270.8781
0.542 -270.6331
0.5424 -268.0912
0.5428 -266.5141
0.5432 -268.3975
0.5436 -270.4494
0.544 -268.8875
0.5444 -266.6212
0.5448 -267.7084
0.5452 -270.2962
0.5456 -270.8322
0.546 -270.4341
0.5464 -271.8887
0.5468 -274.2469
0.5472 -274.109
0.5476 -270.8934
0.548 -267.1418
0.5484 -266.4069
0.5488 -268.9334
0.5492 -270.5259
0.5496 -268.5659
0.55 -266.56
0.5504 -268.1372
0.5508 -270.4034
0.5512 -269.255
0.5516 -266.6825
0.552 -267.1725
0.5524 -270.189
0.5528 -271.7969
0.5532 -270.6791
0.5536 -268.3516
0.554 -266.6519
0.5544 -266.2231
0.5548 -266.5141
0.5552 -266.6059
0.5556 -266.5447
0.556 -266.4987
0.5564 -266.5447
0.5568 -266.1465
0.5572 -266.4375
0.5576 -267.0959
0.558 -266.7131
0.5584 -263.5281
0.5588 -259.1334
0.5592 -255.1981
0.5596 -251.3241
0.56 -246.1484
0.5604 -239.7478
0.5608 -234.3884
0.5612 -231.4025
0.5616 -232.1222
0.562 -234.8478
0.5624 -238.7678
0.5628 -242.6572
0.5632 -245.2144
0.5636 -243.4687
0.564 -238.1247
0.5644 -235.1387
0.5648 -238.3697
0.5652 -242.8256
0.5656 -241.2944
0.566 -235.7206
0.5664 -231.4025
0.5668 -231.1269
0.5672 -232.6275
0.5676 -234.9244
0.568 -236.3637
0.5684 -235.4756
0.5688 -229.8406
0.5692 -220.5459
0.5696 -209.8731
0.57 -200.6703
0.5704 -194.1778
0.5708 -188.6806
0.5712 -183.6428
0.5716 -180.4119
0.572 -181.2541
0.5724 -183.9491
0.5728 -185.6028
0.5732 -184.439
0.5736 -181.3153
0.574 -176.8747
0.5744 -171.4081
0.5748 -165.2525
0.5752 -159.1275
0.5756 -153.4006
0.576 -148.8681
0.5764 -145.3309
0.5768 -143.2331
0.5772 -141.2884
0.5776 -140.5381
0.578 -140.9669
0.5784 -143.4015
0.5788 -144.9941
0.5792 -144.9175
0.5796 -144.9175
0.58 -146.7856
0.5804 -148.8375
0.5808 -148.4853
0.5812 -148.7456
0.5816 -150.5525
0.582 -152.7575
0.5824 -150.2769
0.5828 -145.4841
0.5832 -139.6194
0.5836 -133.9078
0.584 -124.705
0.5844 -111.2759
0.5848 -94.9987
0.5852 -79.7169
0.5856 -68.6613
0.586 -59.6422
0.5864 -52.5219
0.5868 -47.3003
0.5872 -47.8822
0.5876 -50.6537
0.588 -53.7316
0.5884 -54.8034
0.5888 -55.7222
0.5892 -55.0331
0.5896 -52.9659
0.59 -51.0519
0.5904 -50.8069
0.5908 -50.9906
0.5912 -49.4287
0.5916 -47.1931
0.592 -45.9834
0.5924 -46.8562
0.5928 -47.8822
0.5932 -47.2237
0.5936 -43.7937
0.594 -39.5675
0.5944 -35.7241
0.5948 -31.8041
0.5952 -25.97
0.5956 -20.0747
0.596 -14.8225
0.5964 -11.8825
0.5968 -8.9884
0.5972 -7.7634
0.5976 -6.0484
0.598 -4.1497
0.5984 3.9506
0.5988 21.2691
0.5992 49.0153
0.5996 79.4566
0.6 105.84
0.6004 130.7534
0.6008 154.0744
0.6012 174.7615
0.6016 182.7241
0.602 184.24
0.6024 179.83
0.6028 176.7216
0.6032 168.4222
0.6036 157.9637
0.604 142.7584
0.6044 126.9866
0.6048 110.6175
0.6052 95.4734
0.6056 82.0137
0.606 71.3409
0.6064 65.0016
0.6068 62.6741
0.6072 62.5822
0.6076 62.6741
0.608 59.6881
0.6084 51.94
0.6088 37.5922
0.6092 17.8237
0.6096 -6.0331
0.61 -32.5697
0.6104 -58.6316
0.6108 -83.8819
0.6112 -101.9506
0.6116 -113.0216
0.612 -112.4856
0.6124 -109.9591
0.6128 -104.4312
0.6132 -98.7503
0.6136 -85.995
0.614 -71.8922
0.6144 -57.6975
0.6148 -44.4828
0.6152 -29.1703
0.6156 -13.1994
0.616 -0.949375
0.6164 7.2734
0.6168 10.9637
0.6172 11.8519
0.6176 10.5656
0.618 11.4997
0.6184 16.0475
0.6188 22.8616
0.6192 29.5225
0.6196 38.1434
0.62 50.3016
0.6204 65.17
0.6208 77.3587
0.6212 85.6275
0.6216 88.3684
0.622 89.9916
0.6224 88.9044
0.6228 86.5922
0.6232 79.5791
0.6236 71.3409
0.624 62.5516
0.6244 55.5231
0.6248 47.6525
0.6252 36.5663
0.6256 23.1678
0.626 12.4491
0.6264 8.9578
0.6268 7.9778
0.6272 5.9412
0.6276 3.8894
0.628 5.9259
0.6284 7.84
0.6288 5.4819
0.6292 0.42875
0.6296 -3.1391
0.63 -7.2581
0.6304 -14.5162
0.6308 -22.8309
0.6312 -26.9041
0.6316 -27.4553
0.632 -26.5825
0.6324 -27.2869
0.6328 -27.7156
0.6332 -27.5319
0.6336 -24.4694
0.634 -20.0594
0.6344 -14.5162
0.6348 -8.4678
0.6352 -2.0059
0.6356 3.4759
0.636 6.9519
0.6364 11.2241
0.6368 16.3691
0.6372 19.5541
0.6376 16.6753
0.638 12.1122
0.6384 9.2947
0.6388 8.0084
0.6392 5.39
0.6396 3.8741
0.64 5.6656
0.6404 7.7941
0.6408 5.88
0.6412 0.5512499
0.6416 -3.4759
0.642 -4.0425
0.6424 -1.7456
0.6428 -0.0459375
0.6432 -0.030625
0.6436 -0.0765625
0.644 1.7609
0.6444 3.7975
0.6448 4.2416
0.6452 3.8894
0.6456 5.1756
0.646 7.6256
0.6464 8.9119
0.6468 8.0084
0.6472 6.7528
0.6476 7.595
0.648 11.3619
0.6484 15.3891
0.6488 16.8744
0.6492 15.8178
0.6496 15.3891
0.65 18.9875
0.6504 25.5719
0.6508 31.0231
0.6512 31.4825
0.6516 31.2222
0.652 32.4625
0.6524 35.0809
0.6528 33.1362
0.6532 27.9912
0.6536 20.5647
0.654 12.6022
0.6544 1.96
0.6548 -10.4737
0.6552 -23.7803
0.6556 -37.8066
0.656 -50.0106
0.6564 -58.2947
0.6568 -59.4278
0.6572 -58.8
0.6576 -58.2181
0.658 -58.8
0.6584 -56.84
0.6588 -54.9872
0.6592 -53.4253
0.6596 -51.3122
0.66 -45.8303
0.6604 -39.6747
0.6608 -36.9031
0.6612 -38.8019
0.6616 -42.1553
0.662 -43.1966
0.6624 -39.935
0.6628 -35.6016
0.6632 -32.6462
0.6636 -31.4519
0.664 -28.175
0.6644 -20.58
0.6648 -9.0497
0.6652 2.8481
0.6656 11.8366
0.666 18.9262
0.6664 24.3009
0.6668 30.7016
0.6672 34.4684
0.6676 35.3719
0.668 30.9619
0.6684 27.5931
0.6688 24.1172
0.6692 16.8284
0.6696 1.1484
0.67 -14.4856
0.6704 -23.275
0.6708 -30.3953
0.6712 -41.9409
0.6716 -54.0378
0.672 -57.1309
0.6724 -55.0791
0.6728 -53.5784
0.6732 -54.8034
0.6736 -53.6703
0.674 -51.1897
0.6744 -49.4747
0.6748 -50.715
0.6752 -53.2875
0.6756 -54.8494
0.676 -54.2675
0.6764 -54.6656
0.6768 -57.2841
0.6772 -58.8459
0.6776 -56.2581
0.678 -54.7422
0.6784 -57.2994
0.6788 -58.9684
0.6792 -52.9047
0.6796 -43.855
0.68 -38.6181
0.6804 -35.6628
0.6808 -30.4106
0.6812 -24.0253
0.6816 -21.2537
0.682 -19.8144
0.6824 -17.8697
0.6828 -15.7412
0.6832 -18.1453
0.6836 -23.0147
0.684 -28.1597
0.6844 -31.1456
0.6848 -33.9019
0.6852 -38.5722
0.6856 -45.9069
0.686 -54.1297
0.6864 -59.6881
0.6868 -62.5516
0.6872 -62.8119
0.6876 -62.7506
0.688 -61.1734
0.6884 -59.045
0.6888 -55.4772
0.6892 -51.3734
0.6896 -47.5453
0.69 -46.9022
0.6904 -48.51
0.6908 -47.4381
0.6912 -41.1447
0.6916 -35.4944
0.692 -35.8159
0.6924 -38.9703
0.6928 -38.3731
0.6932 -35.4791
0.6936 -33.4731
0.694 -31.6662
0.6944 -25.8475
0.6948 -16.6447
0.6952 -6.7069
0.6956 2.9553
0.696 12.495
0.6964 22.5553
0.6968 29.6603
0.6972 31.5284
0.6976 26.7969
0.698 23.5353
0.6984 24.7756
0.6988 27.3941
0.6992 24.5306
0.6996 19.9216
0.7 17.6094
0.7004 16.0016
0.7008 10.7187
0.7012 4.3641
0.7016 2.0672
0.702 3.6597
0.7024 5.0991
0.7028 4.1037
0.7032 1.5159
0.7036 0.0153125
0.704 -0.5665625
0.7044 -3.3994
0.7048 -10.7034
0.7052 -18.9262
0.7056 -23.765
0.706 -27.0266
0.7064 -31.7428
0.7068 -38.5109
0.7072 -43.2884
0.7076 -46.7337
0.708 -47.7291
0.7084 -47.2237
0.7088 -42.5228
0.7092 -35.9691
0.7096 -28.1597
0.71 -20.3809
0.7104 -12.2041
0.7108 -4.5937
0.7112 1.5772
0.7116 7.3041
0.712 11.2087
0.7124 11.9437
0.7128 7.9625
0.7132 4.1191
0.7136 3.9506
0.714 7.4419
0.7144 10.5503
0.7148 11.7294
0.7152 11.4691
0.7156 11.6681
0.716 12.9697
0.7164 15.3891
0.7168 17.5175
0.7172 19.4162
0.7176 19.9062
0.718 19.6459
0.7184 17.7166
0.7188 15.8331
0.7192 13.0616
0.7196 8.4525
0.72 0.06125
0.7204 -10.6575
0.7208 -20.5034
0.7212 -27.0112
0.7216 -29.3387
0.722 -31.115
0.7224 -32.9984
0.7228 -35.1422
0.7232 -34.1775
0.7236 -31.6203
0.724 -28.4966
0.7244 -27.44
0.7248 -26.5978
0.7252 -23.9334
0.7256 -18.2219
0.726 -12.2194
0.7264 -9.0037
0.7268 -7.9166
0.7272 -6.6456
0.7276 -4.165
0.728 -1.9141
0.7284 -0.1225
0.7288 0.6278125
0.7292 0.1225
0.7296 -1.2403
0.73 -0.2909375
0.7304 3.8434
0.7308 7.6409
0.7312 7.9319
0.7316 7.7175
0.732 9.6928
0.7324 11.7141
0.7328 10.045
0.7332 7.8706
0.7336 8.7894
0.734 11.5609
0.7344 11.2853
0.7348 8.1922
0.7352 4.2262
0.7356 0.3828125
0.736 -3.7516
0.7364 -7.5184
0.7368 -9.7694
0.7372 -11.5609
0.7376 -13.3831
0.738 -15.4962
0.7384 -16.1853
0.7388 -15.7719
0.7392 -13.9497
0.7396 -11.9437
0.74 -9.7694
0.7404 -7.9931
0.7408 -6.2322
0.7412 -4.1497
0.7416 -1.9753
0.742 -0.1378125
0.7424 0.5359375
0.7428 0.091875
0.7432 -0.826875
0.7436 -0.153125
0.744 1.2709
0.7444 0.3215625
0.7448 -4.3181
0.7452 -7.7787
0.7456 -5.5584
0.746 -0.42875
0.7464 1.8222
0.7468 0.245
0.7472 -2.2969
0.7476 -3.8281
0.748 -4.4406
0.7484 -4.0119
0.7488 -3.1391
0.7492 -3.7362
0.7496 -6.2628
0.75 -7.8247
0.7504 -7.1356
0.7508 -7.5797
0.7512 -11.4231
0.7516 -15.4656
0.752 -14.945
0.7524 -11.9897
0.7528 -10.6422
0.7532 -11.6222
0.7536 -12.0356
0.754 -11.7753
0.7544 -10.9025
0.7548 -8.2381
0.7552 -3.4912
0.7556 -0.1225
0.756 -0.2909375
0.7564 -0.18375
0.7568 3.3994
0.7572 7.595
0.7576 7.3959
0.758 4.2416
0.7584 1.6691
0.7588 0.153125
0.7592 -1.1944
0.7596 -0.2909375
0.76 3.9353
0.7604 7.6716
0.7608 6.9978
0.7612 4.1191
0.7616 3.7209
0.762 7.3347
0.7624 11.4384
0.7628 11.9591
0.7632 7.8094
0.7636 4.0731
0.764 3.6137
0.7644 4.0119
0.7648 1.0259
0.7652 -3.5372
0.7656 -6.125
0.766 -7.6103
0.7664 -10.7341
0.7668 -15.2512
0.7672 -18.2219
0.7676 -19.5234
0.768 -19.6766
0.7684 -19.6306
0.7688 -18.4516
0.7692 -15.9862
0.7696 -12.7553
0.77 -11.6681
0.7704 -13.5516
0.7708 -15.5881
0.7712 -15.3584
0.7716 -15.4962
0.772 -17.9769
0.7724 -19.6459
0.7728 -16.6447
0.7732 -12.0662
0.7736 -10.4125
0.774 -11.6222
0.7744 -11.2853
0.7748 -8.2228
0.7752 -3.8587
0.7756 -0.2603125
0.776 1.8375
0.7764 3.6903
0.7768 6.5078
0.7772 11.2241
0.7776 15.9709
0.778 19.3856
0.7784 19.1253
0.7788 16.0781
0.7792 11.2087
0.7796 7.9931
0.78 7.2734
0.7804 7.8094
0.7808 6.8753
0.7812 4.2416
0.7816 0.6737499
0.782 -3.4912
0.7824 -7.0437
0.7828 -7.9472
0.7832 -5.5125
0.7836 -3.8894
0.784 -5.6044
0.7844 -7.7787
0.7848 -6.615
0.7852 -4.0731
0.7856 -4.1803
0.786 -7.4419
0.7864 -10.6881
0.7868 -11.76
0.7872 -11.1322
0.7876 -11.5762
0.788 -13.7659
0.7884 -15.6187
0.7888 -14.3784
0.7892 -11.9437
0.7896 -10.2441
0.79 -8.1616
0.7904 -3.8434
0.7908 -0.18375
0.7912 0.06125
0.7916 -0.1071875
0.792 1.9447
0.7924 3.8894
0.7928 2.1744
0.7932 0.0459375
0.7936 1.0106
0.794 3.7362
0.7944 3.5066
0.7948 0.3521875
0.7952 -2.695
0.7956 -3.8894
0.796 -3.8434
0.7964 -3.8741
0.7968 -4.8847
0.7972 -7.5184
0.7976 -9.8
0.798 -8.2687
0.7984 -2.7409
0.7988 0.0765625
0.7992 -3.4606
0.7996 -7.6869
0.8 -6.5231
0.8004 -3.9506
0.8008 -5.5431
0.8012 -7.84
0.8016 -5.0378
0.802 -0.2909375
0.8024 -0.06125
0.8028 -3.5678
0.8032 -5.4053
0.8036 -4.1344
0.804 -1.7762
0.8044 -0.1071875
0.8048 0.4746875
0.8052 0.0765625
0.8056 -0.6737499
0.806 -0.153125
0.8064 1.8834
0.8068 3.7975
0.8072 4.0425
0.8076 3.8894
0.808 4.1956
0.8084 4.0272
0.8088 2.0366
0.8092 0.091875
0.8096 -0.1684375
0.81 0.0459375
0.8104 -1.47
0.8108 -3.7362
0.8112 -4.5631
0.8116 -3.9966
0.812 -3.4606
0.8124 -3.8281
0.8128 -5.0684
0.8132 -7.5337
0.8136 -10.4891
0.814 -11.7906
0.8144 -10.1828
0.8148 -7.9778
0.8152 -7.4112
0.8156 -7.8247
0.816 -7.6562
0.8164 -7.7481
0.8168 -9.3406
0.8172 -11.5762
0.8176 -12.2653
0.818 -11.7906
0.8184 -11.515
0.8188 -11.7294
0.8192 -11.7447
0.8196 -11.76
0.82 -11.0556
0.8204 -8.2534
0.8208 -3.4453
0.8212 -0.1071875
0.8216 -0.9953125
0.822 -3.6903
0.8224 -4.7316
0.8228 -3.9966
0.8232 -3.5831
0.8236 -3.8741
0.824 -3.9966
0.8244 -3.9047
0.8248 -4.0578
0.8252 -3.9659
0.8256 -2.6797
0.826 -0.275625
0.8264 2.2816
0.8268 3.8434
0.8272 3.8741
0.8276 3.8741
0.828 4.3334
0.8284 4.0425
0.8288 1.8375
0.8292 0.0459375
0.8296 0.245
0.83 0.1378125
0.8304 -2.2816
0.8308 -3.9506
0.8312 -2.0672
0.8316 -0.0153125
0.832 -1.5159
0.8324 -3.8434
0.8328 -2.7716
0.8332 -0.153125
0.8336 -0.336875
0.834 -3.5372
0.8344 -6.5997
0.8348 -7.7941
0.8352 -7.7634
0.8356 -7.8094
0.836 -8.0391
0.8364 -7.9012
0.8368 -6.3547
0.8372 -4.1191
0.8376 -2.8328
0.838 -3.7209
0.8384 -5.9566
0.8388 -7.7481
0.8392 -7.3041
0.8396 -4.3028
0.84 -0.7809374
0.8404 0.1071875
0.8408 -2.2356
0.8412 -3.92
0.8416 -2.3122
0.842 -0.0765625
0.8424 -0.9799999
0.8428 -3.6903
0.8432 -4.7928
0.8436 -4.0119
0.844 -3.5219
0.8444 -3.8741
0.8448 -3.9966
0.8452 -3.92
0.8456 -3.9812
0.846 -3.9353
0.8464 -3.5525
0.8468 -3.8281
0.8472 -4.5937
0.8476 -4.0884
0.848 -1.6537
0.8484 0
0.8488 -1.3781
0.8492 -3.7669
0.8496 -3.5678
0.85 -0.3828125
0.8504 2.9706
0.8508 3.9812
0.8512 2.2662
0.8516 0.1378125
0.852 -0.5665625
0.8524 -0.0459375
0.8528 0.245
0.8532 0.0153125
0.8536 -0.153125
0.854 -0.0153125
0.8544 0
0.8548 0
0.8552 0.0765625
0.8556 0.0153125
0.856 -0.214375
0.8564 -0.0459375
0.8568 0.3675
0.8572 0.091875
0.8576 -1.5772
0.858 -3.7362
0.8584 -4.7622
0.8588 -4.0425
0.8592 -2.9247
0.8596 -3.6903
0.86 -6.4006
0.8604 -7.8706
0.8608 -6.0178
0.8612 -3.9659
0.8616 -4.9919
0.862 -7.6256
0.8624 -8.4219
0.8628 -7.8706
0.8632 -7.8706
0.8636 -7.9012
0.864 -6.2475
0.8644 -4.0731
0.8648 -3.2922
0.8652 -3.8434
0.8656 -4.2262
0.866 -3.9506
0.8664 -3.5984
0.8668 -3.8434
0.8672 -4.4866
0.8676 -4.0578
0.868 -2.0366
0.8684 -0.1071875
0.8688 0.1071875
0.8692 -0.030625
0.8696 0.42875
0.87 0.1378125
0.8704 -2.0978
0.8708 -3.8894
0.8712 -2.6797
0.8716 -0.1684375
0.872 -0.1684375
0.8724 -3.4912
0.8728 -7.1356
0.8732 -7.9625
0.8736 -4.9
0.874 -0.3828125
0.8744 1.96
0.8748 0.336875
0.8752 -3.185
0.8756 -4.0884
0.876 -0.5512499
0.8764 3.6444
0.8768 3.92
0.8772 0.4134375
0.8776 -3.3381
0.878 -4.0272
0.8784 -1.96
0.8788 -0.0765625
0.8792 -0.1378125
0.8796 -0.091875
0.88 1.0566
0.8804 0.30625
0.8808 -4.2722
0.8812 -7.7787
0.8816 -6.0331
0.882 -3.8741
0.8824 -6.8447
0.8828 -11.4844
0.8832 -11.3466
0.8836 -8.1003
0.884 -6.7834
0.8844 -7.7328
0.8848 -8.0084
0.8852 -7.8094
0.8856 -8.2687
0.886 -7.9778
0.8864 -5.6656
0.8868 -3.92
0.8872 -5.1756
0.8876 -7.6409
0.888 -8.4066
0.8884 -7.8706
0.8888 -7.7175
0.8892 -7.8553
0.8896 -6.6916
0.89 -4.1956
0.8904 -1.6231
0.8908 -0.0765625
0.8912 0.2296875
0.8916 0.0153125
0.892 -0.153125
0.8924 -0.0153125
0.8928 0.0153125
0.8932 0
0.8936 -0.1225
0.894 -0.0153125
0.8944 0.0153125
0.8948 0
0.8952 -0.1378125
0.8956 -0.030625
0.896 0.275625
0.8964 0.0765625
0.8968 -1.3628
0.8972 -3.675
0.8976 -5.9719
0.898 -7.7022
0.8984 -8.4372
0.8988 -7.9625
0.8992 -6.1097
0.8996 -4.0731
0.9 -3.0778
0.9004 -3.7669
0.9008 -5.5891
0.9012 -7.6562
0.9016 -8.6669
0.902 -7.9931
0.9024 -5.9872
0.9028 -4.0578
0.9032 -3.3687
0.9036 -3.8434
0.904 -4.3947
0.9044 -3.9966
0.9048 -3.2922
0.9052 -3.7669
0.9056 -5.8647
0.906 -7.7481
0.9064 -7.1969
0.9068 -4.2416
0.9072 -1.8528
0.9076 -3.4912
0.908 -8.5137
0.9084 -11.7294
0.9088 -9.1875
0.9092 -4.3028
0.9096 -2.2969
0.91 -3.7209
0.9104 -5.1297
0.9108 -4.1344
0.9112 -1.715
0.9116 -0.06125
0.912 -0.091875
0.9124 -0.06125
0.9128 0.6890625
0.9132 0.214375
0.9136 -3.5066
0.914 -7.5644
0.9144 -8.6362
0.9148 -7.8859
0.9152 -7.8706
0.9156 -7.9166
0.916 -6.1403
0.9164 -4.0272
0.9168 -4.3641
0.9172 -7.4572
0.9176 -10.7341
0.918 -11.8059
0.9184 -10.045
0.9188 -7.9625
0.9192 -7.4878
0.9196 -7.8247
0.92 -7.5031
0.9204 -7.7175
0.9208 -8.8047
0.9212 -8.1156
0.9216 -3.92
0.922 -0.153125
0.9224 -0.888125
0.9228 -3.7209
0.9232 -3.8128
0.9236 -0.4134375
0.924 3.0625
0.9244 3.9812
0.9248 2.205
0.9252 0.1378125
0.9256 -0.5665625
0.926 -0.0459375
0.9264 0.18375
0.9268 0
0.9272 -0.06125
0.9276 0
0.928 -0.214375
0.9284 -0.0459375
0.9288 0.4440625
0.9292 0.1071875
0.9296 -1.8681
0.93 -3.7975
0.9304 -4.1803
0.9308 -3.8894
0.9312 -4.9612
0.9316 -7.5491
0.932 -10.1369
0.9324 -11.6681
0.9328 -11.8672
0.9332 -11.76
0.9336 -11.6681
0.934 -11.76
0.9344 -11.4844
0.9348 -11.6834
0.9352 -12.1428
0.9356 -11.8672
0.936 -9.9991
0.9364 -7.9778
0.9368 -7.35
0.9372 -7.7941
0.9376 -7.2428
0.938 -4.3028
0.9384 -0.643125
0.9388 0.1684375
0.9392 -3.2922
0.9396 -7.5491
0.94 -8.0544
0.9404 -4.3794
0.9408 -0.30625
0.9412 0.18375
0.9416 -2.6644
0.942 -4.0272
0.9424 -0.8728125
0.9428 3.5984
0.9432 4.9153
0.9436 3.9812
0.944 3.7209
0.9444 3.9659
0.9448 2.3581
0.9452 0.153125
0.9456 -0.5359375
0.946 -0.0459375
0.9464 -0.030625
0.9468 -0.030625
0.9472 0.5512499
0.9476 0.153125
0.948 -2.0978
0.9484 -3.8741
0.9488 -3.4912
0.9492 -3.7362
0.9496 -6.3087
0.95 -7.9012
0.9504 -4.8847
0.9508 -0.30625
0.9512 0.42875
0.9516 -3.4147
0.952 -7.5797
0.9524 -8.0391
0.9528 -4.4406
0.9532 -0.2909375
0.9536 0.9953125
0.954 0.091875
0.9544 -0.5053125
0.9548 -0.0459375
0.9552 0.2909375
0.9556 0.0459375
0.956 -1.0412
0.9564 -3.5678
0.9568 -6.9366
0.9572 -7.9625
0.9576 -4.8387
0.958 -0.3521875
0.9584 1.6231
0.9588 0.245
0.9592 -2.3734
0.9596 -3.8587
0.96 -3.8587
0.9604 -3.8587
0.9608 -4.5325
0.9612 -4.0884
0.9616 -1.7456
0.962 0
0.9624 -1.3322
0.9628 -3.7516
0.9632 -4.2109
0.9636 -3.8741
0.964 -5.1756
0.9644 -7.6256
0.9648 -8.6822
0.9652 -7.9625
0.9656 -6.2322
0.966 -4.1344
0.9664 -1.7762
0.9668 -0.091875
0.9672 0.30625
0.9676 0.030625
0.968 -0.153125
0.9684 -0.0153125
0.9688 0.0153125
0.9692 0
0.9696 -0.06125
0.97 0
0.9704 0
0.9708 0
0.9712 -0.06125
0.9716 0
0.972 0.1378125
0.9724 0.030625
0.9728 -0.3521875
0.9732 -0.0765625
0.9736 0.735
0.974 0.18375
0.9744 -2.3887
0.9748 -3.9353
0.9752 -2.2203
0.9756 -0.06125
0.976 -0.9953125
0.9764 -3.6903
0.9768 -4.655
0.9772 -3.9812
0.9776 -3.7209
0.978 -3.92
0.9784 -3.6137
0.9788 -3.8281
0.9792 -4.7469
0.9796 -4.1344
0.98 -1.2097
0.9804 0.1071875
0.9808 -3.1084
0.9812 -7.5491
0.9816 -7.9778
0.982 -4.3487
0.9824 -0.6890625
0.9828 0.0765625
0.9832 -1.5925
0.9836 -3.7362
0.984 -4.6397
0.9844 -4.0119
0.9848 -3.0625
0.9852 -3.7516
0.9856 -5.1909
0.986 -4.2262
0.9864 0.1225
0.9868 3.7669
0.9872 3.0166
0.9876 0.18375
0.988 0.091875
0.9884 3.4912
0.9888 6.86
0.9892 7.8706
0.9896 6.3394
0.99 4.1037
0.9904 2.8175
0.9908 3.7362
0.9912 5.2828
0.9916 4.2262
0.992 -0.1225
0.9924 -3.7669
0.9928 -3.0778
0.9932 -0.2296875
0.9936 0.8421874
0.994 0.06125
0.9944 -0.18375
0.9948 0
0.9952 -0.2909375
0.9956 -0.091875
0.996 0.9646874
0.9964 0.2603125
0.9968 -3.7516
0.9972 -7.6256
0.9976 -7.3806
0.998 -4.2262
0.9984 -2.2203
0.9988 -3.6291
0.9992 -6.5078
0.9996 -7.8553
1 -6.2781
1.0004 -4.0578
1.0008 -3.3228
1.0012 -3.8741
1.0016 -3.3228
1.002 -0.3828125
1.0024 3.1697
1.0028 4.0425
1.0032 1.5466
1.0036 -0.0459375
1.004 1.8222
1.0044 3.8894
1.0048 2.2662
1.0052 0.0459375
1.0056 1.225
1.006 3.7822
1.0064 3.0778
1.0068 0.2296875
1.0072 -0.888125
1.0076 -0.091875
1.008 0.4440625
1.0084 0.06125
1.0088 -1.1178
1.0092 -3.5984
1.0096 -6.6609
1.01 -7.9012
1.0104 -5.2828
1.0108 -0.459375
1.0112 2.5266
1.0116 0.4746875
1.012 -4.9919
1.0124 -7.8859
1.0128 -4.7316
1.0132 -0.245
1.0136 -0.18375
1.014 -3.5831
1.0144 -5.4359
1.0148 -4.1344
1.0152 -1.715
1.0156 -0.091875
1.016 0.5359375
1.0164 0.091875
1.0168 -0.7503125
1.0172 -0.1684375
1.0176 2.1131
1.018 3.8434
1.0184 3.6597
1.0188 3.7822
1.0192 5.7728
1.0196 7.7634
1.02 6.6916
1.0204 4.1191
1.0208 2.9706
1.0212 3.7975
1.0216 4.4559
1.022 4.0119
1.0224 2.5266
1.0228 0.245
1.0232 -2.2816
1.0236 -3.8434
1.024 -3.8587
1.0244 -3.8587
1.0248 -4.4866
1.0252 -4.0884
1.0256 -1.6844
1.026 0
1.0264 -1.3322
1.0268 -3.7516
1.0272 -4.3334
1.0276 -3.9047
1.028 -4.9459
1.0284 -7.5644
1.0288 -9.1569
1.0292 -8.0697
1.0296 -5.2062
1.03 -3.8741
1.0304 -5.4666
1.0308 -7.7175
1.0312 -7.1969
1.0316 -4.2416
1.032 -1.4394
1.0324 -0.06125
1.0328 0.2909375
1.0332 0.0459375
1.0336 -0.398125
1.034 -0.0765625
1.0344 0.7503125
1.0348 0.18375
1.0352 -2.3734
1.0356 -3.9506
1.036 -1.6691
1.0364 3.3841
1.0368 7.4419
1.0372 7.9625
1.0376 5.6503
1.038 3.9506
1.0384 4.0425
1.0388 4.0272
1.0392 1.9906
1.0396 0.0459375
1.04 0.79625
1.0404 3.6444
1.0408 4.4712
1.0412 0.5971875
1.0416 -5.5278
1.042 -7.9625
1.0424 -4.2109
1.0428 -0.1225
1.0432 -1.8528
1.0436 -7.3347
1.044 -9.9837
1.0444 -8.1462
1.0448 -4.9459
1.0452 -3.8741
1.0456 -5.1144
1.046 -7.595
1.0464 -9.1416
1.0468 -8.085
1.0472 -5.1603
1.0476 -3.8741
1.048 -4.8541
1.0484 -4.2109
1.0488 0.1071875
1.0492 3.8128
1.0496 1.8069
1.05 -3.4453
1.0504 -6.0331
1.0508 -4.2262
1.0512 -1.225
1.0516 0
1.052 -1.2403
1.0524 -3.675
1.0528 -5.0991
1.0532 -4.1344
1.0536 -1.4853
1.054 0
1.0544 -1.1637
1.0548 -3.6903
1.0552 -5.1297
1.0556 -4.1344
1.056 -1.6384
1.0564 -0.0459375
1.0568 -0.18375
1.0572 -0.0765625
1.0576 0.91875
1.058 0.2603125
1.0584 -3.8128
1.0588 -7.6562
1.0592 -7.0744
1.0596 -4.1497
1.06 -2.9553
1.0604 -3.8128
1.0608 -4.1803
1.0612 -3.92
1.0616 -3.9047
1.062 -3.9506
1.0624 -2.7562
1.0628 -0.275625
1.0632 2.2816
1.0636 3.8434
1.064 3.8741
1.0644 3.8741
1.0648 4.3334
1.0652 4.0425
1.0656 1.8375
1.066 0.0459375
1.0664 0.245
1.0668 0.1378125
1.0672 -2.2203
1.0676 -3.9353
1.068 -2.205
1.0684 -0.0459375
1.0688 -1.2097
1.0692 -3.7669
1.0696 -3.5066
1.07 -0.336875
1.0704 1.8375
1.0708 0.336875
1.0712 -4.0731
1.0716 -7.6409
1.072 -8.1462
1.0724 -7.7787
1.0728 -9.1569
1.0732 -11.5609
1.0736 -12.2653
1.074 -11.7906
1.0744 -11.5916
1.0748 -11.7753
1.0752 -10.4891
1.0756 -8.085
1.076 -5.635
1.0764 -4.0119
1.0768 -3.5066
1.0772 -3.8587
1.0776 -4.4253
1.078 -4.0272
1.0784 -2.3275
1.0788 -0.1684375
1.0792 0.704375
1.0796 0.1225
1.08 -1.5925
1.0804 -3.7056
1.0808 -4.9919
1.0812 -4.1344
1.0816 -1.4853
1.082 0.0153125
1.0824 -1.3934
1.0828 -3.7669
1.0832 -3.7056
1.0836 -0.42875
1.084 3.3534
1.0844 4.0578
1.0848 1.3169
1.0852 -0.0765625
1.0856 2.0519
1.086 3.9506
1.0864 1.0412
1.0868 -3.5831
1.0872 -4.9919
1.0876 -3.9812
1.088 -4.6397
1.0884 -7.5184
1.0888 -10.0603
1.0892 -11.6375
1.0896 -12.3572
1.09 -11.8978
1.0904 -9.7234
1.0908 -7.9319
1.0912 -7.6716
1.0916 -7.8553
1.092 -7.3959
1.0924 -7.7022
1.0928 -9.7081
1.0932 -11.6681
1.0936 -11.025
1.094 -8.1309
1.0944 -6.0791
1.0948 -7.5031
1.0952 -11.0097
1.0956 -11.9437
1.096 -8.1003
1.0964 -4.0884
1.0968 -4.6397
1.0972 -7.6256
1.0976 -7.7634
1.098 -4.3181
1.0984 -0.8575
1.0988 0.0459375
1.0992 -1.5466
1.0996 -3.7362
1.1 -4.8541
1.1004 -4.0731
1.1008 -2.205
1.1012 -0.1684375
1.1016 0.9034374
1.102 0.18375
1.1024 -2.2662
1.1028 -3.8741
1.1032 -3.5066
1.1036 -3.7516
1.104 -6.3394
1.1044 -7.8859
1.1048 -5.8494
1.1052 -3.9047
1.1056 -5.7422
1.106 -7.8247
1.1064 -5.9719
1.1068 -3.8894
1.1072 -6.5691
1.1076 -11.3925
1.108 -12.8319
1.1084 -11.8212
1.1088 -11.5762
1.1092 -11.8059
1.1096 -10.045
1.11 -7.9472
1.1104 -7.5031
1.1108 -7.8553
};
\end{axis}

\end{tikzpicture}
\subcaption{Ideales und gemessenes $U_?$}
	\label{fig:UinUquest}
\end{subfigure}
\caption{Nichtlineare Vorverzerrung}
\label{fig:Amplitudenproblem}
\end{figure}

Für die Evaluierung der Grenzen wurden verschiedene Amplituden von $\Uquest_{,\textrm{ideal}}$ über die Kennlinie nichtlinear vorverzerrt. Das Ausgangssignal wurde mit dem RF-Tool \cite{RF-Tool} bewertet und ist in \figref{fig:evaluateK.quality} gezeigt. Dabei sind die Grenzen von $\Uquest$ in \figref{fig:K0_quality} mit $\min(\Uquest)$ und $\max(\Uquest)$ beschriftet dies entspricht mit $V_{PP} = \SI{120}{\mV}$ etwa $34\%$ des maximal möglichen $V_{PP}$, das von $K$ im bijektiven Bereich zugelassen ist. Die anderen eingezeichneten Grenzen, die nicht beschriftet sind, stellen die Grenzen dar, bis zu denen die Güte es Ausgangssignals überprüft wurde.

\begin{figure}[H]
\begin{subfigure}{0.5 \textwidth}
	\setlength\figureheight{7.5cm}
	\setlength\figurewidth{7.5cm}
    \input{images/plots/evaluate_K_bearbeitet.tikz}	
\subcaption{Kennlinie}
	\label{fig:K0_quality}
\end{subfigure}
\begin{subfigure}{0.5 \textwidth}
	\setlength\figureheight{7.5cm}
	\setlength\figurewidth{7.5cm}
    \input{images/plots/evaluate_K_quality_bearbeitet.tikz}	
\subcaption{Berechnung von $K$}
	\label{fig:evaluateK.quality}
\end{subfigure}
\caption{Bewertung des Ausgangssignals}
\label{fig:evaluateK}
\end{figure}
Dabei ist der kleinste und damit der beste Wert der Güte mit \SI{190}{\mV} etwa $54\%$ des maximal möglichen $V_{PP}$ das von $K$ aufgezeichnet worden. Diese Messung wurde nur für diese Kennlinie durchgeführt und lässt keine Schlüsse auf die allgemeine Gültigkeit für alle Kennlinien zu, deshalb sollte der optimale Wert für $V_{PP}$ sicherheitshalbe für alle Kennlinien einzeln bestimmt werden.

\subsection{Zweiter Ansatz}
\label{subsubsec:opt.adjusta.kleiner}
Aus den Ergebnissen aus \ref{subsec:opt.adjusta.results} und der Evaluierung der Grenzen aus dem vorherigen Abschnitt wurde ein neuer Ansatz zur Optimierung aufgestellt, bei dem die Kennlinie $K$ in einem kleineren Bereich optimiert wird als der, aus dem sie vorher zur Initialisierung berechnet wurde.\\
Als zweiter Ansatz wurde folgendes Setup verwendet:
\begin{itemize}
	\item $K_0$ bestimmt über $\Uout_{,\textrm{ideal}}$ mit $V_{PP} = \SI{6}{\V}$
	\item $K$ angepasst über $\Uout_{,\textrm{ideal}}$ mit $V_{PP} = \SI{3}{\V}$
\end{itemize}
Damit wurde $K$ mit einem $\Uquest_{,\textrm{ideal}}$ optimiert, bei dem $V_{PP}$ auf $66\%$ des maximal zulässigen Wertes gesetzt wurde. Der Bereich ist in \figref{fig:opt.kleinerBereich.K} eingezeichnet.
\subsubsection*{Ergebnisse und Erkenntnisse}
\label{subsubsec:opt.adjusta.kleiner.results}
Für diesen Ansatz wurden aufgrund der späten Erkenntnis keine ausführlichen Tests durchgeführt, deshalb lässt sich darüber nur bedingt eine Aussage über die Konvergenz treffen. Aber diese niedrigen Werte der Güte konnten bisher nur mit diesem Ansatz erreicht werden. Im Vergleich zu den Ergebnissen des vorigen Ansatzes in \figref{fig:evaluate30Q} fällt jedoch auf, dass in den ersten Iterationen keine Verschlechterung der Qualität verglichen mit dem Initalwert auftritt.
\\
Die Schrittweite wurde bei all diesen Messungen auf $\sigma_a = 0.5$ gelassen.
\begin{figure}[H]
\begin{subfigure}{0.5 \textwidth}
	\setlength\figureheight{7.5cm}
	\setlength\figurewidth{7.5cm}
    % This file was created by matplotlib2tikz v0.6.17.
\begin{tikzpicture}

\definecolor{color0}{rgb}{0.12156862745098,0.466666666666667,0.705882352941177}
\definecolor{color1}{rgb}{1,0.498039215686275,0.0549019607843137}
\definecolor{color2}{rgb}{0.172549019607843,0.627450980392157,0.172549019607843}
\definecolor{color3}{rgb}{0.83921568627451,0.152941176470588,0.156862745098039}
\definecolor{color4}{rgb}{0.580392156862745,0.403921568627451,0.741176470588235}

\begin{axis}[
xlabel={$U_{in}$ in \si{\milli \volt}},
ylabel={$U_{?}$ in \si{\milli \volt}},
xmin=-330, xmax=330,
ymin=-275.0452008775, ymax=220.9364653795,
width=\figurewidth,
height=\figureheight,
tick align=outside,
tick pos=left,
x grid style={white!69.01960784313725!black},
y grid style={white!69.01960784313725!black},
legend entries={{$K_{initial}$},{$K_1$},{$K_2$},{$K_3$},{$K_4$}},
legend style={at={(0.03,0.97)}, anchor=north west, draw=white!80.0!black},
legend cell align={left},
extra y ticks={124.8, -108.9},
extra y tick labels={},
extra y tick style={grid=major, ytick pos=left, ytick align=inside, ticklabel pos=left}
]
\addlegendimage{no markers, color0}
\addlegendimage{no markers, color1}
\addlegendimage{no markers, color2}
\addlegendimage{no markers, color3}
\addlegendimage{no markers, color4}
\addplot [semithick, color0]
table {%
-300 -192.934095584
-298 -192.557733192
-296 -192.162380535
-294 -191.748174993
-292 -191.315253943
-290 -190.863754762
-288 -190.39381483
-286 -189.905571523
-284 -189.39916222
-282 -188.874724298
-280 -188.332395137
-278 -187.772312113
-276 -187.194612604
-274 -186.599433989
-272 -185.986913646
-270 -185.357188952
-268 -184.710397285
-266 -184.046676024
-264 -183.366162546
-262 -182.668994229
-260 -181.955308451
-258 -181.22524259
-256 -180.478934024
-254 -179.716520132
-252 -178.93813829
-250 -178.143925877
-248 -177.334020271
-246 -176.50855885
-244 -175.667678991
-242 -174.811518074
-240 -173.940213474
-238 -173.053902572
-236 -172.152722744
-234 -171.236811368
-232 -170.306305823
-230 -169.361343486
-228 -168.402061735
-226 -167.428597949
-224 -166.441089505
-222 -165.439673781
-220 -164.424488155
-218 -163.395670005
-216 -162.353356709
-214 -161.297685645
-212 -160.228794191
-210 -159.146819725
-208 -158.051899624
-206 -156.944171268
-204 -155.823772033
-202 -154.690839297
-200 -153.54551044
-198 -152.387922837
-196 -151.218213869
-194 -150.036520911
-192 -148.842981343
-190 -147.637732542
-188 -146.420911887
-186 -145.192656755
-184 -143.953104524
-182 -142.702392572
-180 -141.440658277
-178 -140.168039017
-176 -138.88467217
-174 -137.590695113
-172 -136.286245226
-170 -134.971459886
-168 -133.64647647
-166 -132.311432357
-164 -130.966464924
-162 -129.611711551
-160 -128.247309613
-158 -126.87339649
-156 -125.49010956
-154 -124.0975862
-152 -122.695963788
-150 -121.285379702
-148 -119.865971321
-146 -118.437876022
-144 -117.001231183
-142 -115.556174182
-140 -114.102842397
-138 -112.641373206
-136 -111.171903987
-134 -109.694572117
-132 -108.209514976
-130 -106.71686994
-128 -105.216774388
-126 -103.709365698
-124 -102.194781247
-122 -100.673158413
-120 -99.1446345755
-118 -97.609347111
-116 -96.0674333979
-114 -94.5190308142
-112 -92.9642767377
-110 -91.4033085464
-108 -89.8362636183
-106 -88.2632793313
-104 -86.6844930633
-102 -85.1000421923
-100 -83.5100640963
-98 -81.9146961532
-96 -80.314075741
-94 -78.7083402375
-92 -77.0976270208
-90 -75.4820734687
-88 -73.8618169593
-86 -72.2369948705
-84 -70.6077445802
-82 -68.9742034664
-80 -67.3365089071
-78 -65.6947982801
-76 -64.0492089634
-74 -62.399878335
-72 -60.7469437728
-70 -59.0905426547
-68 -57.4308123588
-66 -55.767890263
-64 -54.1019137451
-62 -52.4330201832
-60 -50.7613469552
-58 -49.0870314391
-56 -47.4102110127
-54 -45.7310230541
-52 -44.0496049412
-50 -42.366094052
-48 -40.6806277643
-46 -38.9933434561
-44 -37.3043785055
-42 -35.6138702902
-40 -33.9219561884
-38 -32.2287735779
-36 -30.5344598366
-34 -28.8391523426
-32 -27.1429884737
-30 -25.446105608
-28 -23.7486411233
-26 -22.0507323976
-24 -20.3525168089
-22 -18.6541317351
-20 -16.9557145541
-18 -15.257402644
-16 -13.5593333826
-14 -11.8616441479
-12 -10.1644723178
-10 -8.46795527033
-8 -6.77223038339
-6 -5.07743503495
-4 -3.38370660293
-2 -1.6911824653
0 0
2 1.68970341503
4 3.37779040183
6 5.06412358245
8 6.74856557896
10 8.43097901341
12 10.1112265078
14 11.7891706843
16 13.4646741649
18 15.1375995715
20 16.8078095264
22 18.4751666516
24 20.139533569
26 21.8007729008
28 23.458747269
30 25.1133192957
32 26.7643516028
34 28.4117068126
36 30.0552475469
38 31.6948364279
40 33.3303360776
42 34.9616091181
44 36.5885181715
46 38.2109258596
48 39.8286948048
50 41.4416876289
52 43.049766954
54 44.6527954023
56 46.2506355956
58 47.8431501562
60 49.430201706
62 51.0116528671
64 52.5873662615
66 54.1572045114
68 55.7210302387
70 57.2787060655
72 58.8300946139
74 60.3750585059
76 61.9134603635
78 63.4451628089
80 64.970028464
82 66.4879199509
84 67.9986998918
86 69.5022309085
88 70.9983756232
90 72.486996658
92 73.9679566348
94 75.4411181758
96 76.9063439029
98 78.3634964383
100 79.812438404
102 81.2530324221
104 82.6851411145
106 84.1086271034
108 85.5233530108
110 86.9291814587
112 88.3259750693
114 89.7135964645
116 91.0919082664
118 92.4607730971
120 93.8200535786
122 95.1696123329
124 96.5093119822
126 97.8390151485
128 99.1585844538
130 100.46788252
132 101.76677197
134 103.055115424
136 104.332775506
138 105.599614837
140 106.85549604
142 108.100281736
144 109.333834547
146 110.556017096
148 111.766692005
150 112.965721895
152 114.152969388
154 115.328297108
156 116.491567675
158 117.642643712
160 118.781387841
162 119.907662684
164 121.021330862
166 122.122254999
168 123.210297716
170 124.285321635
172 125.347189378
174 126.395763567
176 127.430906825
178 128.452481773
180 129.460351034
182 130.454377228
184 131.43442298
186 132.40035091
188 133.35202364
190 134.289303793
192 135.212053991
194 136.120136856
196 137.013415009
198 137.891751073
200 138.75500767
202 139.603047423
204 140.435732952
206 141.25292688
208 142.054491829
210 142.840290422
212 143.61018528
214 144.364039025
216 145.101714279
218 145.823073665
220 146.527979804
222 147.216295319
224 147.887882831
226 148.542604963
228 149.180324336
230 149.800903573
232 150.404205296
234 150.990092127
236 151.558426688
238 152.1090716
240 152.641889487
242 153.156742969
244 153.65349467
246 154.13200721
248 154.592143213
250 155.0337653
252 155.456736094
254 155.860918215
256 156.246174287
258 156.612366932
260 156.959358771
262 157.287012426
264 157.595190521
266 157.883755675
268 158.152570513
270 158.401497655
272 158.630399724
274 158.839139342
276 159.027579131
278 159.195581712
280 159.343009709
282 159.469725743
284 159.575592436
286 159.66047241
288 159.724228287
290 159.76672269
292 159.78781824
};
\addplot [semithick, color1]
table {%
-300 -252.500579684
-298 -251.113170617
-296 -249.718632841
-294 -248.317030339
-292 -246.908427096
-290 -245.492887093
-288 -244.070474316
-286 -242.641252747
-284 -241.205286369
-282 -239.762639168
-280 -238.313375125
-278 -236.857558224
-276 -235.39525245
-274 -233.926521784
-272 -232.451430212
-270 -230.970041716
-268 -229.482420281
-266 -227.988629888
-264 -226.488734522
-262 -224.982798167
-260 -223.470884806
-258 -221.953058422
-256 -220.429382999
-254 -218.899922521
-252 -217.364740971
-250 -215.823902332
-248 -214.277470587
-246 -212.725509722
-244 -211.168083718
-242 -209.60525656
-240 -208.03709223
-238 -206.463654714
-236 -204.885007993
-234 -203.301216051
-232 -201.712342873
-230 -200.118452441
-228 -198.519608739
-226 -196.91587575
-224 -195.307317458
-222 -193.693997847
-220 -192.0759809
-218 -190.4533306
-216 -188.826110931
-214 -187.194385877
-212 -185.558219421
-210 -183.917675546
-208 -182.272818236
-206 -180.623711474
-204 -178.970419245
-202 -177.313005531
-200 -175.651534316
-198 -173.986069584
-196 -172.316675317
-194 -170.6434155
-192 -168.966354116
-190 -167.285555149
-188 -165.601082581
-186 -163.913000397
-184 -162.22137258
-182 -160.526263114
-180 -158.827735981
-178 -157.125855166
-176 -155.420684652
-174 -153.712288423
-172 -152.000730462
-170 -150.286074752
-168 -148.568385277
-166 -146.84772602
-164 -145.124160966
-162 -143.397754097
-160 -141.668569398
-158 -139.93667085
-156 -138.202122439
-154 -136.464988147
-152 -134.725331959
-150 -132.983217857
-148 -131.238709824
-146 -129.491871846
-144 -127.742767904
-142 -125.991461984
-140 -124.238018067
-138 -122.482500137
-136 -120.724972179
-134 -118.965498175
-132 -117.20414211
-130 -115.440967965
-128 -113.676039726
-126 -111.909421376
-124 -110.141176897
-122 -108.371370274
-120 -106.600065491
-118 -104.827326529
-116 -103.053217374
-114 -101.277802009
-112 -99.5011444161
-110 -97.7233085801
-108 -95.9443584843
-106 -94.164358112
-104 -92.3833714468
-102 -90.6014624721
-100 -88.8186951716
-98 -87.0351335287
-96 -85.2508415268
-94 -83.4658831495
-92 -81.6803223804
-90 -79.8942232028
-88 -78.1076496003
-86 -76.3206655565
-84 -74.5333350547
-82 -72.7457220785
-80 -70.9578906115
-78 -69.169904637
-76 -67.3818281387
-74 -65.5937251
-72 -63.8056595044
-70 -62.0176953354
-68 -60.2298965765
-66 -58.4423272113
-64 -56.6550512231
-62 -54.8681325956
-60 -53.0816353123
-58 -51.2956233565
-56 -49.5101607119
-54 -47.7253113619
-52 -45.9411392901
-50 -44.1577084799
-48 -42.3750829148
-46 -40.5933265784
-44 -38.8125034541
-42 -37.0326775254
-40 -35.253912776
-38 -33.4762731891
-36 -31.6998227484
-34 -29.9246254374
-32 -28.1507452395
-30 -26.3782461383
-28 -24.6071921172
-26 -22.8376471598
-24 -21.0696752495
-22 -19.3033403699
-20 -17.5387065045
-18 -15.7758376368
-16 -14.0147977502
-14 -12.2556508283
-12 -10.4984608546
-10 -8.74329181253
-8 -6.99020768566
-6 -5.23927245747
-4 -3.49055011146
-2 -1.74410463114
0 0
2 1.74169979845
4 3.4809307807
6 5.21762896327
8 6.95173036264
10 8.68317099531
12 10.4118868778
14 12.1378140266
16 13.8608884581
18 15.581046189
20 17.2982232356
22 19.0123556146
24 20.7233793423
26 22.4312304354
28 24.1358449102
30 25.8371587833
32 27.5351080711
34 29.2296287903
36 30.9206569572
38 32.6081285884
40 34.2919797004
42 35.9721463097
44 37.6485644327
46 39.321170086
48 40.989899286
50 42.6546880494
52 44.3154723924
54 45.9721883318
56 47.6247718839
58 49.2731590652
60 50.9172858923
62 52.5570883817
64 54.1925025498
66 55.8234644131
68 57.4499099882
70 59.0717752916
72 60.6889963397
74 62.301509149
76 63.909249736
78 65.5121541173
80 67.1101583094
82 68.7031983286
84 70.2912101916
86 71.8741299148
88 73.4518935148
90 75.0244370079
92 76.5916964108
94 78.1536077399
96 79.7101070118
98 81.2611302428
100 82.8066134496
102 84.3464926485
104 85.8807038562
106 87.4091830891
108 88.9318663637
110 90.4486896965
112 91.9595891039
114 93.4645006026
116 94.963360209
118 96.4561039396
120 97.9426678109
122 99.4229878394
124 100.897000042
126 102.364640434
128 103.825845033
130 105.280549855
132 106.728690917
134 108.170204235
136 109.605025826
138 111.033091706
140 112.454337891
142 113.868700399
144 115.276115246
146 116.676518447
148 118.06984602
150 119.456033982
152 120.835018348
154 122.206735135
156 123.57112036
158 124.928110039
160 126.277640189
162 127.619646826
164 128.954065966
166 130.280833627
168 131.599885824
170 132.911158575
172 134.214587895
174 135.510109801
176 136.79766031
178 138.077175438
180 139.348591202
182 140.611843618
184 141.866868702
186 143.113602472
188 144.351980943
190 145.581940132
192 146.803416056
194 148.016344731
196 149.220662174
198 150.4163044
200 151.603207428
202 152.781307272
204 153.95053995
206 155.110841479
208 156.262147874
210 157.404395151
212 158.537519329
214 159.661456423
216 160.776142449
218 161.881513424
220 162.977505365
222 164.064054288
224 165.14109621
226 166.208567146
228 167.266403115
230 168.314540131
232 169.352914212
234 170.381461374
236 171.400117634
238 172.408819007
240 173.407501511
242 174.396101163
244 175.374553978
246 176.342795973
248 177.300763164
250 178.248391569
252 179.185617203
254 180.112376083
256 181.028604226
258 181.934237648
260 182.829212365
262 183.713464394
264 184.586929752
266 185.449544455
268 186.30124452
270 187.141965963
272 187.9716448
274 188.790217048
276 189.597618724
278 190.393785843
280 191.178654424
282 191.952160481
284 192.714240032
286 193.464829093
288 194.20386368
290 194.931279811
292 195.647013501
294 196.351000767
296 197.043177625
298 197.723480093
300 198.391844186
};
\addplot [semithick, color2]
table {%
-300 -243.97295317
-298 -242.636997144
-296 -241.294163108
-294 -239.944511433
-292 -238.58810249
-290 -237.224996648
-288 -235.855254278
-286 -234.478935751
-284 -233.096101437
-282 -231.706811706
-280 -230.311126929
-278 -228.909107476
-276 -227.500813718
-274 -226.086306025
-272 -224.665644767
-270 -223.238890315
-268 -221.806103039
-266 -220.36734331
-264 -218.922671497
-262 -217.472147973
-260 -216.015833106
-258 -214.553787267
-256 -213.086070827
-254 -211.612744156
-252 -210.133867625
-250 -208.649501604
-248 -207.159706462
-246 -205.664542572
-244 -204.164070303
-242 -202.658350025
-240 -201.147442109
-238 -199.631406926
-236 -198.110304845
-234 -196.584196238
-232 -195.053141474
-230 -193.517200925
-228 -191.976434959
-226 -190.430903949
-224 -188.880668263
-222 -187.325788274
-220 -185.76632435
-218 -184.202336863
-216 -182.633886183
-214 -181.06103268
-212 -179.483836725
-210 -177.902358688
-208 -176.31665894
-206 -174.72679785
-204 -173.13283579
-202 -171.534833129
-200 -169.932850239
-198 -168.326947489
-196 -166.71718525
-194 -165.103623893
-192 -163.486323788
-190 -161.865345304
-188 -160.240748814
-186 -158.612594686
-184 -156.980943292
-182 -155.345855001
-180 -153.707390185
-178 -152.065609214
-176 -150.420572458
-174 -148.772340287
-172 -147.120973072
-170 -145.466531184
-168 -143.809074992
-166 -142.148664867
-164 -140.48536118
-162 -138.819224301
-160 -137.1503146
-158 -135.478692448
-156 -133.804418215
-154 -132.127552272
-152 -130.448154989
-150 -128.766286736
-148 -127.082007884
-146 -125.395378803
-144 -123.706459864
-142 -122.015311437
-140 -120.321993893
-138 -118.626567601
-136 -116.929092933
-134 -115.229630259
-132 -113.528239948
-130 -111.824982372
-128 -110.119917901
-126 -108.413106905
-124 -106.704609756
-122 -104.994486822
-120 -103.282798475
-118 -101.569605084
-116 -99.8549670217
-114 -98.1389446567
-112 -96.42159836
-110 -94.7029885019
-108 -92.9831754528
-106 -91.2622195832
-104 -89.5401812634
-102 -87.817120864
-100 -86.0930987552
-98 -84.3681753074
-96 -82.6424108912
-94 -80.9158658769
-92 -79.1886006349
-90 -77.4606755357
-88 -75.7321509496
-86 -74.003087247
-84 -72.2735447985
-82 -70.5435839742
-80 -68.8132651448
-78 -67.0826486806
-76 -65.3517949519
-74 -63.6207643293
-72 -61.8896171831
-70 -60.1584138838
-68 -58.4272148017
-66 -56.6960803072
-64 -54.9650707708
-62 -53.2342465629
-60 -51.5036680539
-58 -49.7733956141
-56 -48.0434896141
-54 -46.3140104242
-52 -44.5850184148
-50 -42.8565739564
-48 -41.1287374193
-46 -39.401569174
-44 -37.6751295908
-42 -35.9494790402
-40 -34.2246778926
-38 -32.5007865184
-36 -30.777865288
-34 -29.0559745718
-32 -27.3351747403
-30 -25.6155261638
-28 -23.8970892127
-26 -22.1799242575
-24 -20.4640916686
-22 -18.7496518164
-20 -17.0366650712
-18 -15.3251918036
-16 -13.6152923838
-14 -11.9070271824
-12 -10.2004565697
-10 -8.4956409162
-8 -6.79264059221
-6 -5.09151596819
-4 -3.39232741454
-2 -1.69513530167
0 0
2 1.69301812007
4 3.38385868812
6 5.07246133375
8 6.75876568654
10 8.44271137609
12 10.124238032
14 11.8032852838
16 13.4797927612
18 15.1537000936
20 16.8249469108
22 18.4934728423
24 20.1592175176
26 21.8221205664
28 23.4821216183
30 25.1391603028
32 26.7931762496
34 28.4441090882
36 30.0918984482
38 31.7364839593
40 33.3778052509
42 35.0158019528
44 36.6504136944
46 38.2815801054
48 39.9092408153
50 41.5333354538
52 43.1538036504
54 44.7705850348
56 46.3836192365
58 47.992845885
60 49.5982046101
62 51.1996350413
64 52.7970768082
66 54.3904695403
68 55.9797528673
70 57.5648664187
72 59.1457498241
74 60.7223427132
76 62.2945847155
78 63.8624154607
80 65.4257745782
82 66.9846016977
84 68.5388364487
86 70.088418461
88 71.6332873639
90 73.1733827873
92 74.7086443605
94 76.2390117133
96 77.7644244752
98 79.2848222759
100 80.8001447448
102 82.3103315115
104 83.8153222058
106 85.3150564571
108 86.8094738951
110 88.2985141493
112 89.7821168494
114 91.2602216248
116 92.7327681053
118 94.1996959204
120 95.6609446997
122 97.1164540727
124 98.5661636691
126 100.010013119
128 101.44794205
130 102.879890095
132 104.30579688
134 105.725602037
136 107.139245195
138 108.546665984
140 109.947804033
142 111.342598971
144 112.730990429
146 114.112918035
148 115.48832142
150 116.857140213
152 118.219314043
154 119.574782541
156 120.923485336
158 122.265362056
160 123.600352333
162 124.928395796
164 126.249432074
166 127.563400796
168 128.870241593
170 130.169894094
172 131.462297928
174 132.747392725
176 134.025118115
178 135.295413728
180 136.558219192
182 137.813474137
184 139.061118194
186 140.301090992
188 141.533332159
190 142.757781327
192 143.974378124
194 145.18306218
196 146.383773124
198 147.576450587
200 148.761034197
202 149.937463585
204 151.10567838
206 152.265618211
208 153.417222709
210 154.560431502
212 155.695184221
214 156.821420494
216 157.939079952
218 159.048102224
220 160.14842694
222 161.239993729
224 162.322742221
226 163.396612045
228 164.461542832
230 165.51747421
232 166.564345809
234 167.602097259
236 168.630668189
238 169.64999823
240 170.66002701
242 171.660694159
244 172.651939307
246 173.633702083
248 174.605922117
250 175.568539039
252 176.521492477
254 177.464722063
256 178.398167425
258 179.321768192
260 180.235463996
262 181.139194464
264 182.032899227
266 182.916517914
268 183.789990155
270 184.653255579
272 185.506253816
274 186.348924496
276 187.181207249
278 188.003041702
280 188.814367488
282 189.615124234
284 190.405251571
286 191.184689128
288 191.953376535
290 192.711253421
292 193.458259416
294 194.194334149
296 194.91941725
298 195.63344835
300 196.336367076
};
\addplot [semithick, color3]
table {%
-300 -246.021402181
-298 -244.703944934
-296 -243.378926307
-294 -242.046411775
-292 -240.70646681
-290 -239.359156887
-288 -238.004547478
-286 -236.642704059
-284 -235.273692102
-282 -233.89757708
-280 -232.514424468
-278 -231.124299739
-276 -229.727268367
-274 -228.323395825
-272 -226.912747586
-270 -225.495389126
-268 -224.071385916
-266 -222.640803431
-264 -221.203707145
-262 -219.76016253
-260 -218.310235061
-258 -216.853990211
-256 -215.391493453
-254 -213.922810262
-252 -212.448006111
-250 -210.967146474
-248 -209.480296824
-246 -207.987522634
-244 -206.488889379
-242 -204.984462532
-240 -203.474307566
-238 -201.958489956
-236 -200.437075175
-234 -198.910128696
-232 -197.377715992
-230 -195.839902539
-228 -194.296753809
-226 -192.748335275
-224 -191.194712413
-222 -189.635950694
-220 -188.072115593
-218 -186.503272583
-216 -184.929487138
-214 -183.350824731
-212 -181.767350837
-210 -180.179130928
-208 -178.586230479
-206 -176.988714963
-204 -175.386649853
-202 -173.780100623
-200 -172.169132747
-198 -170.553811698
-196 -168.934202951
-194 -167.310371978
-192 -165.682384253
-190 -164.05030525
-188 -162.414200442
-186 -160.774135303
-184 -159.130175307
-182 -157.482385927
-180 -155.830832637
-178 -154.17558091
-176 -152.516696221
-174 -150.854244042
-172 -149.188289847
-170 -147.51889911
-168 -145.846137305
-166 -144.170069904
-164 -142.490762383
-162 -140.808280214
-160 -139.12268887
-158 -137.434053826
-156 -135.742440555
-154 -134.047914531
-152 -132.350541228
-150 -130.650386118
-148 -128.947514676
-146 -127.241992374
-144 -125.533884688
-142 -123.82325709
-140 -122.110175054
-138 -120.394704054
-136 -118.676909563
-134 -116.956857054
-132 -115.234612002
-130 -113.51023988
-128 -111.783806162
-126 -110.055376321
-124 -108.32501583
-122 -106.592790164
-120 -104.858764797
-118 -103.1230052
-116 -101.385576849
-114 -99.6465452174
-112 -97.9059757779
-110 -96.1639340044
-108 -94.4204853707
-106 -92.6756953502
-104 -90.9296294167
-102 -89.1823530437
-100 -87.4339317048
-98 -85.6844308736
-96 -83.9339160238
-94 -82.182452629
-92 -80.4301061627
-90 -78.6769420985
-88 -76.9230259102
-86 -75.1684230712
-84 -73.4131990552
-82 -71.6574193358
-80 -69.9011493866
-78 -68.1444546812
-76 -66.3874006932
-74 -64.6300528963
-72 -62.872476764
-70 -61.1147377699
-68 -59.3569013877
-66 -57.5990330909
-64 -55.8411983532
-62 -54.0834626482
-60 -52.3258914494
-58 -50.5685502306
-56 -48.8115044652
-54 -47.0548196269
-52 -45.2985611894
-50 -43.5427946261
-48 -41.7875854108
-46 -40.032999017
-44 -38.2791009184
-42 -36.5259565884
-40 -34.7736315009
-38 -33.0221911293
-36 -31.2717009473
-34 -29.5222264285
-32 -27.7738330464
-30 -26.0265862747
-28 -24.2805515871
-26 -22.535794457
-24 -20.7923803582
-22 -19.0503747642
-20 -17.3098431486
-18 -15.570850985
-16 -13.8334637471
-14 -12.0977469085
-12 -10.3637659427
-10 -8.63158632334
-8 -6.90127352409
-6 -5.17289301853
-4 -3.44651028025
-2 -1.72219078287
0 0
2 1.71999659476
4 3.4377335278
6 5.15314532551
8 6.86616651428
10 8.57673162051
12 10.2847751706
14 11.9902316909
16 13.6930357079
18 15.3931217478
20 17.0904243372
22 18.7848780025
24 20.4764172699
26 22.1649766659
28 23.8504907169
30 25.5328939493
32 27.2121208894
34 28.8881060637
36 30.5607839986
38 32.2300892204
40 33.8959562556
42 35.5583196305
44 37.2171138715
46 38.8722735051
48 40.5237330575
50 42.1714270553
52 43.8152900248
54 45.4552564923
56 47.0912609844
58 48.7232380273
60 50.3511221475
62 51.9748478713
64 53.5943497252
66 55.2095622356
68 56.8204199287
70 58.4268573311
72 60.0288089692
74 61.6262093692
76 63.2189930577
78 64.8070945609
80 66.3904484053
82 67.9689891174
84 69.5426512234
86 71.1113692497
88 72.6750777229
90 74.2337111692
92 75.787204115
94 77.3354910868
96 78.8785066109
98 80.4161852137
100 81.9484614216
102 83.4752697611
104 84.9965447584
106 86.5122209401
108 88.0222328324
110 89.5265149618
112 91.0250018547
114 92.5176280374
116 94.0043280364
118 95.485036378
120 96.9596875887
122 98.4282161948
124 99.8905567228
126 101.346643699
128 102.79641165
130 104.239795101
132 105.676728581
134 107.107146614
136 108.530983727
138 109.948174446
140 111.358653299
142 112.762354811
144 114.159213509
146 115.549163919
148 116.932140567
150 118.308077981
152 119.676910685
154 121.038573208
156 122.393000074
158 123.740125811
160 125.079884945
162 126.412212002
164 127.737041509
166 129.054307992
168 130.363945978
170 131.665889992
172 132.960074561
174 134.246434212
176 135.524903472
178 136.795416865
180 138.057908919
182 139.312314161
184 140.558567116
186 141.796602311
188 143.026354273
190 144.247757527
192 145.460746601
194 146.66525602
196 147.861220311
198 149.048574
200 150.227251614
202 151.397187679
204 152.558316722
206 153.710573269
208 154.853891846
210 155.98820698
212 157.113453196
214 158.229565023
216 159.336476985
218 160.434123609
220 161.522439422
222 162.60135895
224 163.67081672
226 164.730747257
228 165.781085089
230 166.821764741
232 167.85272074
234 168.873887613
236 169.885199885
238 170.886592084
240 171.877998735
242 172.859354366
244 173.830593501
246 174.791650669
248 175.742460394
250 176.682957204
252 177.613075625
254 178.532750183
256 179.441915406
258 180.340505818
260 181.228455946
262 182.105700318
264 182.972173459
266 183.827809896
268 184.672544154
270 185.506310761
272 186.329044243
274 187.140679127
276 187.941149938
278 188.730391203
280 189.508337448
282 190.2749232
284 191.030082986
286 191.773751331
288 192.505862762
290 193.226351805
292 193.935152987
294 194.632200835
296 195.317429874
298 195.990774631
300 196.652169632
};
\addplot [semithick, color4]
table {%
-300 -218.882602596
-298 -218.132814018
-296 -217.367792108
-294 -216.587652625
-292 -215.792511329
-290 -214.982483979
-288 -214.157686335
-286 -213.318234155
-284 -212.4642432
-282 -211.595829229
-280 -210.713108001
-278 -209.816195275
-276 -208.905206812
-274 -207.98025837
-272 -207.041465709
-270 -206.088944588
-268 -205.122810767
-266 -204.143180006
-264 -203.150168063
-262 -202.143890698
-260 -201.12446367
-258 -200.092002739
-256 -199.046623665
-254 -197.988442206
-252 -196.917574123
-250 -195.834135174
-248 -194.738241119
-246 -193.630007718
-244 -192.50955073
-242 -191.376985913
-240 -190.232429029
-238 -189.075995836
-236 -187.907802093
-234 -186.72796356
-232 -185.536595997
-230 -184.333815163
-228 -183.119736817
-226 -181.894476718
-224 -180.658150627
-222 -179.410874302
-220 -178.152763504
-218 -176.88393399
-216 -175.604501522
-214 -174.314581857
-212 -173.014290757
-210 -171.703743979
-208 -170.383057285
-206 -169.052346432
-204 -167.71172718
-202 -166.361315289
-200 -165.001226519
-198 -163.631576628
-196 -162.252481376
-194 -160.864056523
-192 -159.466417827
-190 -158.059681049
-188 -156.643961948
-186 -155.219376283
-184 -153.786039814
-182 -152.3440683
-180 -150.8935775
-178 -149.434683174
-176 -147.967501082
-174 -146.492146982
-172 -145.008736634
-170 -143.517385798
-168 -142.018210233
-166 -140.511325699
-164 -138.996847955
-162 -137.474892759
-160 -135.945575873
-158 -134.409013055
-156 -132.865320064
-154 -131.31461266
-152 -129.757006603
-150 -128.192617652
-148 -126.621561565
-146 -125.043954104
-144 -123.459911027
-142 -121.869548093
-140 -120.272981063
-138 -118.670325694
-136 -117.061697748
-134 -115.447212983
-132 -113.826987159
-130 -112.201136034
-128 -110.56977537
-126 -108.933020924
-124 -107.290988457
-122 -105.643793727
-120 -103.991552495
-118 -102.334380519
-116 -100.67239356
-114 -99.0057073758
-112 -97.3344377268
-110 -95.6587003722
-108 -93.9786110714
-106 -92.2942855838
-104 -90.6058396689
-102 -88.913389086
-100 -87.2170495947
-98 -85.5169369543
-96 -83.8131669242
-94 -82.105855264
-92 -80.3951177329
-90 -78.6810700905
-88 -76.9638280961
-86 -75.2435075093
-84 -73.5202240893
-82 -71.7940935956
-80 -70.0652317878
-78 -68.3337544251
-76 -66.599777267
-74 -64.863416073
-72 -63.1247866024
-70 -61.3840046147
-68 -59.6411858693
-66 -57.8964461257
-64 -56.1499011432
-62 -54.4016666813
-60 -52.6518584994
-58 -50.900592357
-56 -49.1479840134
-54 -47.3941492282
-52 -45.6392037606
-50 -43.8832633702
-48 -42.1264438163
-46 -40.3688608585
-44 -38.610630256
-42 -36.8518677684
-40 -35.0926891551
-38 -33.3332101755
-36 -31.5735465889
-34 -29.813814155
-32 -28.054128633
-30 -26.2946057824
-28 -24.5353613626
-26 -22.776511133
-24 -21.0181708532
-22 -19.2604562824
-20 -17.5034831801
-18 -15.7473673058
-16 -13.9922244189
-14 -12.2381702788
-12 -10.4853206449
-10 -8.73379127663
-8 -6.98369793346
-6 -5.23515637478
-4 -3.48828236003
-2 -1.74319164862
0 0
2 1.74117682642
4 3.48022307122
6 5.21702297496
8 6.95146077822
10 8.68342072158
12 10.4127870456
14 12.1394439909
16 13.863275798
18 15.5841667075
20 17.3020009599
22 19.0166627959
24 20.7280364561
26 22.4360061809
28 24.140456211
30 25.8412707869
32 27.5383341492
34 29.2315305386
36 30.9207441955
38 32.6058593605
40 34.2867602742
42 35.9633311773
44 37.6354563102
46 39.3030199135
48 40.9659062279
50 42.6239994939
52 44.277183952
54 45.9253438428
56 47.568363407
58 49.206126885
60 50.8385185175
62 52.4654225451
64 54.0867232082
66 55.7023047476
68 57.3120514037
70 58.9158474171
72 60.5135770284
74 62.1051244783
76 63.6903740071
78 65.2692098557
80 66.8415162644
82 68.4071774739
84 69.9660777247
86 71.5181012575
88 73.0631323128
90 74.6010551312
92 76.1317539532
94 77.6551130195
96 79.1710165706
98 80.679348847
100 82.1799940894
102 83.6728365383
104 85.1577604343
106 86.634650018
108 88.10338953
110 89.5638632108
112 91.0159553009
114 92.4595500411
116 93.8945316718
118 95.3207844336
120 96.7381925671
122 98.1466403129
124 99.5460119116
126 100.936191604
128 102.31706363
130 103.68851223
132 105.050421646
134 106.402676118
136 107.745159885
138 109.07775719
140 110.400352272
142 111.712829372
144 113.015072731
146 114.306966589
148 115.588395187
150 116.859242765
152 118.119393563
154 119.368731824
156 120.607141786
158 121.834507691
160 123.050713779
162 124.255644291
164 125.449183467
166 126.631215549
168 127.801624775
170 128.960295388
172 130.107111627
174 131.241957734
176 132.364717948
178 133.475276511
180 134.573517663
182 135.659325644
184 136.732584695
186 137.793179057
188 138.84099297
190 139.875910675
192 140.897816413
194 141.906594423
196 142.902128947
198 143.884304225
200 144.853004497
202 145.808114005
204 146.749516989
206 147.677097689
208 148.590740346
210 149.490329201
212 150.375748494
214 151.246882465
216 152.103615356
218 152.945831407
220 153.773414858
222 154.58624995
224 155.384220924
226 156.167212019
228 156.935107478
230 157.68779154
232 158.425148445
234 159.147062436
236 159.853417751
238 160.544098632
240 161.218989319
242 161.877974052
244 162.520937073
246 163.147762622
248 163.75833494
250 164.352538266
252 164.930256842
254 165.491374908
256 166.035776705
258 166.563346474
260 167.073968454
262 167.567526887
264 168.043906013
266 168.502990072
268 168.944663306
270 169.368809955
272 169.775314258
274 170.164060458
276 170.534932795
278 170.887815508
280 171.222592839
282 171.539149028
284 171.837368317
286 172.117134944
288 172.378333152
290 172.62084718
292 172.844561269
294 173.04935966
296 173.235126593
298 173.401746309
300 173.549103048
};
\end{axis}

\end{tikzpicture}
\subcaption{Kennlinie}
	\label{fig:opt.kleinerBereich.K}
\end{subfigure}
\begin{subfigure}{0.5 \textwidth}
	\setlength\figureheight{7.5cm}
	\setlength\figurewidth{7.5cm}
    % This file was created by matplotlib2tikz v0.6.17.
\begin{tikzpicture}

\definecolor{color0}{rgb}{0.12156862745098,0.466666666666667,0.705882352941177}

\begin{axis}[
xlabel={Iteration},
ylabel={QGesamt1},
xmin=0.85, xmax=4.15,
ymin=1.23267561823032, ymax=1.28694353103391,
width=\figurewidth,
height=\figureheight,
tick align=outside,
tick pos=left,
x grid style={white!69.01960784313725!black},
y grid style={white!69.01960784313725!black},
legend style={at={(0.03,0.97)}, anchor=north west, draw=white!80.0!black},
legend entries={{Güte}},
legend cell align={left}
]
\addlegendimage{no markers, color0}
\addplot [semithick, color0, mark=x, mark size=3, mark options={solid}, only marks]
table {%
1 1.28447680772466
2 1.24194667673614
3 1.23514234153958
4 1.2499561642434
};
\end{axis}

\end{tikzpicture}
\subcaption{Qualität}
	\label{fig:opt.kleinerBereich.Q}
\end{subfigure}
\label{fig:opt.Kennlinie}
\caption{Zweite Ansatz zur Anpassung von $K$}
\end{figure}

\end{document}
