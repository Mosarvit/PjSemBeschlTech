\documentclass[../Report.tex]{subfiles}


\begin{document}


\chapter{Fazit}
\label{chap:fazit}
---- In diesem Kapitel wird eine kurze Evaluierung vorgenommen. Welche Aspekte der Problemstellung wurden erfüllt (welche nicht), welche Hindernisse genommen? Einordnung der eigenen Arbeit in Kontext der in der Einleitung geführten Rahmenbedingung? Welche Erkenntnisse sind besonders erwähnenswert? Hier können Erfahrungen mit den gedachten Vorteilen (siehe \ref{sec:code.mot} ) des Codes oder eine Bewertung der Sinnhaftigkeit der Optimierung nochmals geführt werden.--- 
\\
Erfahrung: ähnlich viel Zeitaufwand in Aufräumen des gegebenen Codes wie notwendig war, um (komplexe) Funktionalität zu erweitern. -> Erweiterbarkeit viel leichter, weitere Möglichkeiten im Idealfall schneller möglich einzubetten und zu testen. 
\\
\\
\\
%%% Zur Optimierung:
Auf Basis der im Rahmen dieser Arbeit vorliegenden Werte lässt sich keine Aussage darüber treffen, ob das verwendete Modell für jedes Signal (Signalform und Amplitude) gleichermaßen genau das Verhalten des Messaufbaus simuliert.
Deshalb könnte es ausgehend von initialen Werten für die Bausteine des Hammerstein-Modells möglich sein, über eine iterative Optimierung eine Anpassung an $\Hcompl$ und $K$ derart vorzunehmen, dass sie für das gewünschte $\Uout_{, \mathrm{ideal}}$ hinreichend gute Werte liefert. Dies könnte etwa zur Kalibrierung der Kavität auf Grundlage der Charakteristik des Vortages genutzt werden. Vorteilhaft ist, dass die iterative Optimierung unabhängig der zur Berechnung der initialen Charakteristik genutzten Signale arbeiten könnte und damit etwa eine Verbesserung der Kennlinie für die momentan gewünschte Amplitude der Gapspannung möglich wäre.
Es ließen sich mit einer solchen Optimierung beim Versuch, hohe Amplituden zu erreichen, auch Aussagen über die - falls vorhandenen - Grenzen des Hammerstein-Modells für den Messaufbau treffen.
\\
Bewertung der Zielführung der Optimierung


\end{document}