\documentclass[../Report.tex]{subfiles}


\begin{document}


\chapter*{Abstract}
\label{chap:abstract}

Für die sinnvolle Nutzung der Barrier Bucket Kavität an den Speicherringen der GSI Darmstadt ist ein qualitativ hochwertiger Einzelsinus als Ausgangsspannung am Gap unabdingbar. 
Modelliert man den Messaufbau durch ein Hammerstein-Modell mit einer nichtlinearen Kennlinie im Zeitbereich und einer linearen Übertragungsfunktion im Frequenzbereich, so verlagert sich die Schwierigkeit auf das hinreichend genaue Bestimmen dieser Komponenten.
Diese lassen sich zwar einzeln über automatisierte Messungen erfassen, sind dabei aber nicht unabhängig voneinander. Zur Verbesserung der Modellierung liegt eine iterative Anpassung der beiden Bausteine nahe.
\\
\\
In einem ersten Ansatz zur Optimierung werden für Kennlinie und Übertragungsfunktion getrennte Algorithmen genutzt. Es zeigt sich am simulierten idealen Hammerstein-Modell, dass dieser Ansatz prinzipiell in die richtige Richtung läuft.
Für eine merkliche Verbesserung der Modellierung auch am realen System sind jedoch eine Reihe von Hürden zu nehmen. Diese sind teilweise auf Diskretisierung und Interpolation zurückzuführen. Jedoch liegen gerade bei der nichtlinearen Kennlinie auch konzeptionelle Probleme bezüglich des Definitionsbereichs vor, die eine genauere Betrachtung nicht nur für die Optimierung erfordern.
\\
\\
Unverzichtbare Grundlage der hier verfolgten Ansätze sowie Ausgangspunkt für die weitere Arbeit an einer Optimierung ist das Design einer Programmstruktur, die an den Kontext eines Messaufbaus angepasst ist. Als außerordentlich hilfreich erweist sich ein modularer Aufbau von Funktionalitäten. Zusätzlich bietet sich ein Programmier-Konzept an, das auf periodische Tests der Funktionalitäten ausgelegt ist. Ergänzend hierzu vereinfacht sich die Erweiterung der Funktionalität durch die Möglichkeit der Simulation des mathematischen Modells erheblich.

\end{document}